
\section{THE STATE OF THE AIRCRAFT\label{sec:INS}}

The primary sources of information on the location and motion of the
aircraft are inertial navigation systems\index{INS}\index{inertial reference unit|see {IRU}}\index{inertial navigation system|see {INS}}
and global positioning systems\index{GPS}. Both are described in
this section, and combined results that merge the best features of
each into composite variables for location and motion are also discussed.
Useful references for material in this section are \href{http://nldr.library.ucar.edu/repository/assets/technotes/TECH-NOTE-000-000-000-064.pdf}{Lenschow (1972)}
and \href{http://www.eol.ucar.edu/raf/Bulletins/bulletin23.html}{RAF Bulletin 23}.\index{Bulletin 23}

\subsection{Inertial Reference Systems}

An Inertial Navigation System (INS)\index{INS} or Inertial Reference
Unit\index{IRU} (IRU) provides measurements of aircraft position,
velocity relative to the Earth, acceleration and attitude or orientation.
The IRU provides basic measurements of acceleration and angular rotation
rate, while the INS integrates those measurements to track the position,
altitude, velocity, and orientation of the aircraft. For the GV, the
system is a\index{INS!GV} Honeywell\index{INS!Honeywell Laseref}
Laseref IV HG2001 GD03 Inertial Reference System; for the C-130, it
is a Honeywell Model HG1095-AC03 Laseref V SM Inertial Reference System.\index{INS!C130}
These systems are described on the EOL web site, at \href{http://www.eol.ucar.edu/instruments/inertial-navigation-system-aka-inertial-reference-unit}{this URL}.
Data from the IRS come via a serial digital bit stream (the ARINC
digital bus) to the ADS\index{ADS=aircraft data system} (Aircraft
Data System). Because there is some delay in transmission and recording
of these variables, adjustments for this delay are made when the measurements
are merged into the processed data files, as documented in the NetCDF\index{NetCDF!header}
header files and as discussed in Section \ref{subsec:Synchronization-of-Measurements}.
Typical delays are about 80 ms for variables including ACINS, PITCH,
ROLL, and THDG.

Some variables are recorded only on the original ``raw'' data tapes
and are not usually included in final archived data files; these are
discussed at the end of this subsection. See also the discussion in
Section \ref{sec:OBSOLETE-VARIABLES}, \vpageref{LTN51}, for information
on results from inertial systems that were used prior to installation
of the present Honeywell systems.\label{punch3.1}

An Inertial Navigation System\index{INS!alignment} ``aligns'' while
the aircraft is stationary by measurement of the variations in its
reference frame caused by the rotation of the Earth. Small inaccuracy
in that alignment leads to a ``Schuler oscillation''\index{Schuler oscillation}\index{oscillation, Schuler}
that produces oscillatory errors in position and other measurements,
with a period $\tau_{Sch}$ of about 84 minutes ($\tau_{Sch}=2\pi\sqrt{R_{E}/g}$).\sindex[lis]{tauSch@$\tau_{Sch}$=period of a Schuler oscillation}
Position errors of less than 1.0 n mi/hr are within normal operating
specifications. See Section \ref{subsec:IRS/GPS} for discussion of
additional variables, similar to the following, for which corrections
are made for these errors via reference to data from a Global Positioning
System.

Some projects have used smaller Systron Donner C-MIGITS\index{INS!C-MIGITS}
Inertial Navigation Systems with GPS coupling, usually in connection
with special instruments like a wing-mounted wind-sensing system.
For these units, variable names usually begin with the letter C but
otherwise have names matching the following variables (e.g., CLAT).
GPS coupling via a Kalman filter is incorporated in the measurements
from these units. They are described at \href{https://www.eol.ucar.edu/instruments/c-migits-insgps-system}{this web address}.

Uncertainties associated with measurements from the IRS are discussed
in a Technical Note,\index{wind!uncertainty!Tech Note} available
at \href{http://dx.doi.org/10.5065/D60G3HJ8}{this URL}. See page
7 of that document and the tables on pages 41 and 49.\\

\begin{hangparagraphs}
\textbf{Latitude (}$\text{�}$\textbf{): }\textbf{\uline{LAT}}\sindex[var]{LAT}\index{LAT}\nop{LAT}{}\\
\emph{The aircraft latitude} \emph{or angular distance north of the
equator in an Earth reference frame}.\index{latitude}\emph{ }Positive
values are north of the equator; negative values are south. The resolution
is 0.00017$^{\circ}$ and the accuracy is reported by the manufacturer
to be 0.164$^{\circ}$ after 6 h of flight. Values are provided by
the INS at a frequency of 10 Hz.

\textbf{Longitude (}$\text{�}$\textbf{): }\textbf{\uline{LON}}\sindex[var]{LON}\textbf{\uline{\index{LON}}}\nop{LON}{}\textbf{\uline{}}\\
\emph{The aircraft longitude} or angular distance east of the prime
meridian in an Earth reference frame.\index{longitude} Positive values
are east of the prime meridian; negative values are west. The resolution
is 0.00017$^{\circ}$ and the accuracy is reported by the manufacturer
to be 0.164$^{\circ}$ after 6 h of flight. Values are provided by
the INS at a frequency of 10 Hz.

\textbf{Aircraft True Heading (}$^{\circ}$\textbf{): }\textbf{\uline{THDG}}\index{THDG}\nop{THDG}{}\\
\emph{The azimuthal angle between the centerline of the aircraft (pointing
ahead, toward the nose) and a line of meridian. }This azimuthal angle
is measured in a polar coordinate system oriented relative to the
Earth with polar axis upward and azimuthal angle measured relative
to true north.\index{heading!true} The heading thus indicates the
orientation of the aircraft, not necessarily the direction in which
the aircraft is traveling. The resolution is 0.00017$\text{�}$ and
the accuracy is quoted by the manufacturer as 0.2$\text{�}$ after
6 h of flight. Values are provided by the INS at a frequency of 25
Hz. ``True'' distinguishes the heading from the magnetic heading\index{heading!magnetic},
the heading that would be measured by a magnetic compass. For more
information on the coordinate system used, see \href{http://www.eol.ucar.edu/raf/Bulletins/bulletin23.html}{RAF Bulletin 23}\index{Bulletin 23}.\label{punch3.2}

\textbf{Aircraft Pitch Attitude Angle (}$\text{�}$\textbf{): }\textbf{\uline{PITCH}}\sindex[var]{PITCH}\index{PITCH}\index{INS!specifications}\nop{PITCH}{}\\
\emph{The angle between the centerline of the aircraft (pointing ahead,
toward the nose) and the horizontal plane in a reference frame relative
to the Earth with polar axis upward.}\index{pitch} Positive values
correspond to the nose of the aircraft pointing above the horizon.
The resolution is 0.00017$\text{�}$ and the accuracy is quoted by
the manufacturer as 0.05$\text{�}$ after 6 h of flight. Values are
provided by the INS at a frequency of 50 Hz.

\textbf{Aircraft Roll Attitude Angle (}$\text{�}$\textbf{): }\textbf{\uline{ROLL}}\sindex[var]{ROLL}\index{ROLL}\nop{ROLL}{}\\
\emph{The angle of rotation about the longitudinal axis of the aircraft
required to bring the lateral axis (along the wings) to the horizontal
plane.}\index{roll} Positive angles indicate that the starboard (right)
wing is down ((i.e., a clockwise rotation has occurred from level
when facing forward in the aircraft). The resolution is 0.00017$\text{�}$
and the accuracy is quoted by the manufacturer as 0.05$\text{�}$
after 6 h of flight. Values are provided by the INS at a frequency
of 50 Hz.

\textbf{Aircraft Vertical Acceleration (m\,s$^{-2}$): }\textbf{\uline{ACINS}}\sindex[var]{ACINS}\index{ACINS}\nop{ACINS}{}\\
\index{acceleration!vertical}\emph{The acceleration upward (relative
to the Earth) as measured by an inertial reference unit.\index{IRU}}
With INSs now in use, the internal drift is removed by the INS via
pressure damping through reference to the pressure altitude\index{pressure damping}.\footnote{For earlier projects using the Litton LTN-51 INS, this is a direct
measurement without adjustment for changes in gravity during flight
and without pressure-damping. Previous use employed a baro-inertial
loop to compensate for drift in the integrated measurement. See the
discussion of WP3 below.} Positive values are upward. The sample rate is 50 Hz and the resolution
is 0.0024 m\,s$^{-2}$. 

\textbf{Computed Aircraft Vertical Velocity (m/s): }\textbf{\uline{VSPD}}\sindex[var]{VSPD}\index{VSPD}\nop{VSPD}{}\\
\emph{The upward velocity of the aircraft, or rate-of-climb relative
to the Earth, as measured by the INS.}\index{velocity!aircraft vertical}
VSPD is determined within the INS by integration of the vertical acceleration,
with damping based on measured pressure to correct for accumulated
errors in the integration of acceleration. The sample rate is 50 Hz
with a resolution of 0.00016 m/s. The Honeywell Laseref INS employs
a baro-inertial loop\index{baro-inertial loop}, similar to that described
below for WP3\index{WP3} and the Litton LTN-51\index{INS!Litton LTN-51},
to update the value of the acceleration. This variable is also filtered
within the INS so that there is little variance with frequency higher
than 0.1 Hz. 

\textbf{Pressure-Damped Aircraft Vertical Velocity (m/s): }\textbf{\uline{WP3}}\sindex[var]{WP3}\index{WP3}
(obsolete)\\
\label{WP3 algorithm}This was a derived variable incorporating a
third-order damping feedback loop\index{baro-loop} to remove the
drift from the inertial system's vertical accelerometer (ACINS or
VZI) using pressure altitude (PALT) as a long-term, stable reference.
Positive values are up. The Honeywell INS now in use provides its
own version of this measurement, VSPD\index{VSPD}, and WP3 is now
considered obsolete (and in any case should not be calculated from
ACINS as provided by the Honeywell Laseref IRS\index{INS!Honeywell}
because that ACINS already incorporates pressure damping). Documentation
is included here because many old data files include this variable.
Note that ``pressure altitude\index{altitude!pressure}'' is not
a true altitude but an altitude equivalent to the ambient pressure
in a standard atmosphere, so updating a variable integrated from inertial
measurements to this value can introduce errors vs.~the true altitude..
WP3 was calculated by the data-processing software as follows (with
coefficients in historical use and not updated to the recommendations
elsewhere in this technical note):\footnote{Regarding signs, note that ACINS is a number near zero, not near g,
and so already has the estimated acceleration of gravity removed.
The assumption made in the following is that the INS will report values
adjusted for the gravitational acceleration \emph{at the point of
alignment}, which would be $G_{L}$. If $g_{F}$, the estimate for
gravity at the flight altitude (palt) and latitude (lat), is \emph{smaller}
than $G_{L}$ then the difference ($G_{L}-g_{f}$) will be positive;
this will correct for the reference value for ACINS being the gravity
measured at alignment ($G_{L}$) when it should actually be the sensed
gravity ($g_{f}$) at the measurement point, so to obtain (sensed
acceleration - $g_{f}$) it is necessary to add ($G_{L}-g_{f}$) to
ACINS, \emph{increasing }``acz'' in this case. However, the situation
with ``vcorac'' is reversed: ``vcorac'' is a positive term for
all eastward flight, for example, but in that case the motion of the
aircraft makes objects seem lighter (i.e., they experience less acceleration
of gravity) than without such flight. ACINS is positive upward so
it represents a net acceleration of the aircraft upward (as imposed
by the combination of gravity and the lift force of the aircraft).
To accomplish level flight in these circumstances, the aircraft must
actually accelerate downward so the accelerometer will experience
a negative excursion relative to slower flight. To compensate, ``vcorac''
must make a positive contribution to remove that negative excursion
from ``acz''. In the conceptual extreme that the aircraft flies
fast enough for the interior to appear weightless, ACINS would reduce
to -1{*}$G_{L}$ and vcorac would increase to +$G_{L}$, leaving acz
near zero as required if the aircraft were to remain in level flight
in the rotating frame}\\
\\
\\
\\
\noindent\doublebox{\begin{minipage}[t]{1\columnwidth - 2\fboxsep - 7.5\fboxrule - 1pt}%
\begin{center}
{[}See next page{]}
\par\end{center}%
\end{minipage}}\\
\fbox{\begin{minipage}[t]{0.95\textwidth}%
$g_{1}$ = 9.780356 m\,s$^{-2}$ \\
$a_{1}$ = 0.31391116$\times$10$^{-6}$ m$^{-1}$\\
$a_{2}$ = .0052885 (dimensionless)\\
VEW (VNS) = eastward (northward) groundspeed of the aircraft (see
below)\\
LAT = latitude measured by the IRS {[}$\text{�}${]}\\
$C_{dr}=\pi/180^{\circ}$ = conversion factor, degrees to radians\\
PALT = pressure altitude of the aircraft\\
$\Omega$ = angular rotation of the earth$^{\dagger}$ = 7.292116$\times10^{6}$
radians/s\\
$R_{E}$ = radius of the Earth$^{\dagger}$ = 6.371229$\times10^{6}$
m\\
$g_{f}$ = local gravity corrected for latitude and altitude\\
$V_{c}$ = correction to gravity for the motion of the aircraft\\
$G_{L}$ = local gravity at the location of INS alignment, corrected
to zero altitude\\
$\{C[0],C[1],C[2]\}$ = feedback coefficients, \{0.15, 0.0075, 0.000125\}
for 125-s response

\rule[0.5ex]{1\linewidth}{1pt}
\begin{enumerate}
\item From the pressure altitude PALT (in m) and the latitude LAT, estimate
the acceleration of gravity:\\
\[
g_{f}=g_{1}\left(1+a_{2}\sin^{2}(\mathrm{C_{dr}\{LAT\})}+a_{1}\mathrm{\{PALT\}}\right)
\]
\item Determine corrections for Coriolis acceleration\index{Coriolis acceleration}
and centrifugal acceleration\index{centrifugal acceleration}: \\
\[
a_{c}=2\Omega\mathrm{\{VEW\}}\cos(C_{r}\mathrm{\{LAT\}})+\frac{\mathrm{\{VEW\}}^{2}+\mathrm{\{VNS\}}^{2}}{R_{E}}
\]
Estimate the acceleration $a_{z}$ (code variable 'acz') experienced
by the aircraft as follows:\\
\[
\mathrm{\{acz\}}=a_{z}=\mathrm{\{ACINS\}}+G_{L}-g_{f}+a_{c}
\]
Use a feedback loop to update the integrated value of the acceleration.
The following code segment uses \emph{acz}=$a_{z}$, \emph{deltaT}
to represent the time between updates, and \emph{hi3, hx,} and \emph{hxx}
to store the feedback terms:\\
\fbox{\parbox[t]{0.95\textwidth}{%
\begin{lyxcode}
wp3{[}FeedBack{]}~+=~(acz~-~C{[}1{]}~{*}~hx{[}FeedBack{]}~~\\
~~~-~C{[}2{]}~{*}~hxx{[}FeedBack{]})~{*}~deltaT{[}FeedBack{]};
\end{lyxcode}
%
}}
\item Update the feedback terms:\\
\fbox{\parbox[t]{0.95\textwidth}{%
\begin{lyxcode}
hi3{[}FeedBack{]}~=~hi3{[}FeedBack{]}~+~(wp3{[}FeedBack{]}~~\\
~~~~-~C{[}0{]}~{*}~hx{[}FeedBack{]})~{*}~deltaT{[}FeedBack{]};~\\
hx{[}FeedBack{]}~~=~hi3{[}FeedBack{]}~-~palt;~\\
hxx{[}FeedBack{]}~=~hxx{[}FeedBack{]}~~\\
~~~~~~~~~~~~~~+~hx{[}FeedBack{]}~{*}~deltaT{[}FeedBack{]};~
\end{lyxcode}
%
}}
\item Set WP3 to the average of the last wp3 result and the current wp3
result.
\end{enumerate}
%
\end{minipage}}

\textbf{Inertial Altitude (m): }\textbf{\uline{ALT}}\sindex[var]{ALT}\index{ALT}\nop{ALT}{}\\
\emph{The altitude of the aircraft as provided by an INS}, with pressure
damping applied within the INS to the integrated aircraft vertical
velocity to avoid the accumulation of errors.\index{altitude!inertial}
The value therefore is updated to the pressure altitude, not the geometric
altitude, and should be regarded as a measurement of pressure altitude
that has short-term variations as provided by the INS. The sample
rate is 25 Hz with a resolution of 0.038 m. In some projects ALT also
referred to the altitude from the avionics GPS system; the preferred
and current variable name for that is ALT\_G.\label{punch3.10}

\textbf{Aircraft Ground Speed (m/s): }\textbf{\uline{GSF}}\sindex[var]{GSF}\index{GSF}\nop{GSF}{}\\
\emph{The ground speed of the aircraft as provided by an INS.}\index{ground speed}\index{velocity!aircraft}
\index{speed!ground}The resolution is 0.0020 m/s, and the INS provides
this measurement at a frequency of 10 Hz.

\textbf{Aircraft Ground Speed East Component (m/s): }\textbf{\uline{VEW}}\sindex[var]{VEW}\index{VEW}\nop{VEW}{}\\
\emph{The east-directed component of ground speed as provided by an
INS. }The resolution is 0.0020 m/s, and the INS provides this measurement
at a frequency of 10 Hz.

\textbf{Aircraft Ground Speed North Component (m/s) }\textbf{\uline{VNS}}\sindex[var]{VNS}\index{VNS}\nop{VNS}{}\\
\emph{The north-directed component of ground speed as provided by
an INS. }The resolution is 0.0020 m/s, and the INS provides this measurement
at a frequency of 10 Hz.

\textbf{Distance East/North of a Reference (km): }\textbf{\uline{DEI}}\textbf{\sindex[var]{DEI}\index{DEI@DE\textbf{I}}/}\textbf{\uline{DNI}}\sindex[var]{DNI}\index{DNI}\nop{DEI}{}\nop{DNI}{}\\
\emph{Distance east or north of a project-dependent reference point.
}These are derived outputs obtained by subtracting a fixed reference
position from the current position. The values are determined from
measurements of latitude and longitude and converted from degrees
to distance in a rectilinear coordinate system. The reference position
can be either the starting location of the flight or a user-defined
reference point (e.g., the location of a project radar). The accuracy
of these values is dependent on the accuracy of the source of latitude
and longitude measurements (see LAT and LON), and the calculations
are only appropriate for short distances because they do not take
into account the spherical geometry of the Earth.\label{punch:3.11}\\
\fbox{\begin{minipage}[t]{0.9\textwidth}%
LON$_{ref}$ = reference longitude ($\text{�}$)\\
LAT$_{ref}$ = reference latitude ($\text{�}$)\\
$C_{deg2km}=$ conversion factor, degrees latitude to km\sindex[con]{conversion factor, degrees latitude to km}
$\equiv$ 111.12 km /$^{\circ}$\\
\\
\rule[0.5ex]{1\linewidth}{1pt}

\[
\mathrm{DEI}=(\mathrm{\{LON\}}-\{\mathrm{LON}_{ref}\})\mathrm{C_{deg2km}}\cos(\mathrm{\{LAT\}})
\]
\[
\mathrm{DNI=(\mathrm{\{LAT\}}=\{\mathrm{LAT}_{ref}\})C_{deg2km}}
\]
%
\end{minipage}}

\textbf{Radial Azimuth/Distance from Fixed Reference: }\textbf{\uline{FXAZIM}}\textbf{\sindex[var]{FXAZIM}\index{FXAZIM},
}\textbf{\uline{FXDIST}}\sindex[var]{FXDIST}\index{FXDIST}\nop{FXAZIM}{}\nop{FXDIST}{}\\
\emph{Azimuth and distance from a project-dependent reference point.
}The units of the azimuthal angle are degrees (relative to true north)
and the distance is in kilometers. These are calculated by rectangular-to-polar
conversion of DEI and DNI, described in the preceding paragraph.
\end{hangparagraphs}


\subsubsection*{\noun{\textendash Raw IRS Variables Not Included In Normal Data Files}:\textendash{}}

The following INS and IRU variables\index{IRU!raw variables}\index{INS!measurements not in normal data files}
are not normally included in archived data files, but their values
are recorded by the ADS and can be obtained from the original ``raw''
data files: \\

\begin{hangparagraphs}
\textbf{Raw Lateral Body Acceleration (m/s$^{2}$): }\textbf{\uline{BLATA}}\sindex[var]{BLATA}\index{BLATA}\nop{BLATA}{}\\
The raw output from the IRU lateral accelerometer. Positive values
are toward the starboard, normal to the aircraft center line. The
sample rate is 50 Hz with a resolution of 0.0024 m\,s$^{-2}$.

\textbf{Raw Longitudinal Body Acceleration (m/s$^{2}$): }\textbf{\uline{BLONA}}\sindex[var]{BLONA}\index{BLONA}\nop{BLONA}{}\\
The raw output from the IRU longitudinal accelerometer. Positive values
are in the direction of the nose of the aircraft and parallel to the
aircraft center line. The sample rate is 50 Hz with a resolution of
0.0024 m\,s$^{-2}$.

\textbf{Raw Normal Body Acceleration (m/s$^{2}$): }\textbf{\uline{BNORMA}}\sindex[var]{BNORMA}\index{BNORMA}\nop{BNORMA}{}\\
The raw output from the IRU vertical accelerometer. Positive values
are upward in the reference frame of the aircraft, normal to the aircraft
center line and lateral axis. The sample rate is 50 Hz with a resolution
of 0.0024 m\,s$^{-2}$.

\textbf{Raw Body Pitch Rate (}$\text{�}$\textbf{/s): }\textbf{\uline{BPITCHR}}\sindex[var]{BPITCHR}\index{BPITCHR}\nop{BPITCHR}{}\\
The raw output of the IRU pitch rate gyro. Positive values indicate
the nose moving upward and refer to rotation about the aircraft's
lateral axis. The sample rate is 50 Hz with a resolution of 0.0039$\text{�}$/s.

\textbf{Raw Body Roll Rate (}$\text{�}$\textbf{/s): }\textbf{\uline{BROLLR}}\sindex[var]{BROLLR}\index{BROLLR}\nop{BROLLR}{}\\
The raw output of the IRU roll rate gyro. Positive values indicate
starboard wing moving down and refer to rotation about the aircraft
center line. The sample rate is 50 Hz with a resolution of 0.0039$\text{�}$/s.

\textbf{Raw Body Yaw Rate (}$\text{�}$\textbf{/s): }\textbf{\uline{BYAWR}}\sindex[var]{BYAWR}\index{BYAWR}\nop{BYAWR}{}\\
The raw output of the IRU yaw rate. Positive values represent the
nose turning to the starboard and refer to rotation about the aircraft's
vertical axis. The sample rate is 50 Hz with a resolution of 0.0039$\text{�}$/s. 
\end{hangparagraphs}


\subsection{Global Positioning Systems}

Primary GPS\index{global positioning system|see{GPS}}\index{GPS}
variables specifying the position and velocity of the aircraft are
provided by \index{GPS receivers} GPS receivers, currently a NovAtel
Model OEM 5 \index{GPS!NovAtel}unit on the GV and a NovAtel Model
OEM-4 receiver on the C-130 (to be replaced by an OEM-6 after January
2014). See \href{http://www.eol.ucar.edu/instruments/research-global-positioning-system}{this link}
for a description of these systems. The coordinate system used for
all GPS measurements is the World Geodetic System WGS-84; \index{altitude!pressure}\index{altitude!geopotential}\index{altitude!geometric}\index{gravity!standard}
for details, see \href{http://earth-info.nga.mil/GandG/publications/tr8350.2/wgs84fin.pdf}{this link}.\footnote{There are four measures of height or altitude discussed in this technical
note, height relative to the WGS-84 reference surface, geometric height,
geopotential height and pressure height. The WGS-84 height (measured
by GPS instruments) is height relative to a reference system in which
zero is defined by a reference ellipsoid, a surface having constant
gravitational equipotential that approximates that at mean sea level.
The true mean sea level, represented by the ``geoid,'' is more structured
and departs significantly from the WGS-84 reference surface, often
by several 10s of meters. The geometric height is the true height
above a reference surface, often taken to be mean sea level or the
geoid; this may therefore differ significantly from the height measured
by a GPS unit. There is a variable included below, GGEOIDHT, that
provides a measure of the difference. Geopotential height is the height
that would give a particular value of the geopotential, or gravitational
potential energy per unit mass, if that mass were raised against standard
gravity (not varying, e.g., with latitude or height) to that altitude.
For the purpose of this definition, standard gravity is defined to
be 9.80665$\,\mathrm{m\,\mathrm{s^{-2}}}$. Finally, pressure altitude,
defined in detail below, is the altitude in the ISA Standard Atmosphere
where the pressure matches a specified value; it is not a geometric
coordinate but rather a measure of pressure.} The uncertainty of the position measurements is specified by the
manufacturer to be 1.5 m CEP horizontal (3.3 m 95\% CEP).\footnote{CEP is the Circular Error Probability, the radius of a circle that
contains 50\% of the measurements; 95\% CEP contains 95\% of the measurements.
When OnmiSTAR corrections are available (involving extra cost and
not available in all areas of the globe, so not available for all
projects) the uncertainty decreases to 0.15 m 95\% CEP. The vertical
uncertainty is about twice as great as the horizontal uncertainty.
Because variables are stored as 4-byte single-precision floating point
numbers, the inherent storage precision can limit the precision of
the recorded position to about 1 m.} The accuracy of velocity measurements is 0.03 m/s RMS for all axes.
All variables are provided by the GPS receivers at 5 Hz. Latitude
and longitude are recorded in a special log file (called the GPGGA
log) with a resolution of 0.0001 degree, while Earth-relative velocity
is recorded in the GPRMC log with resolution of 0.1 m/s. Starting
in January 2014 new logs (named BESTPOS and BESTVEL) will also be
recorded to preserve more significant digits in the measurements.
The BESTPOS log will have a position resolution of $10^{-11}\deg$,
while the BESTVEL log will be recorded with 0.0001 m/s resolution. 

Some of the following variables are also available from alternate
Garmin GPS16 receivers, for which the variable name is qualified by
the name of that unit; e.g., GGLAT\_GMN for GGLAT as measured by a
Garmin GPS\index{GPS!Garmin} unit. In addition, some of the measurements
from the GPS\index{GPS!aircraft avionics unit} units that are part
of the aircraft avionics systems are recorded; these are denoted by
a suffix ``\_G'' or ``\_A''. Measurements from before about 2000
used Trimble TANS-III receivers,\index{GPS!Trimble TANS-III} \label{punch3.3}with
the ability to track up to 6 satellites at a time but needing only
4 to provide 3-dimensional position and velocity data (3 satellites
for 2-dimensions). The accuracy of the position measurements for that
unit was stated to be 25 meters (horizontal) and 35 meters (vertical)
under ``steady-state conditions.''\footnote{Note: The GPS signals at one time suffered from ``selective availability,''
a US DOD term for a perturbed signal that degraded GPS absolute accuracy
to 100 meters. This was especially noticeable in the altitude measurement,
so GALT normally was not useful. As of 1 May 2000, selective availability
was deactivated to allow everyone to obtain better position measurements.
See the Interagency GPS Executive Board web site for more information
on selective availability and GPS measurements prior to 2000.} Likewise, velocity measurements are within 0.2 m/s for all axes.
Measurement resolution is that of 4-byte IEEE format (about 6 significant
digits). All variables were provided by the Trimble receivers at 1
Hz.

A special correction is needed for variables GGVEW, GGVNS, and GGVSPD,
which measure the motion \emph{at the GPS antenna} relative to the
Earth. The conventional wind calculation addresses the difference
between the motion at the radome (where the relative wind is measured)
and the INS (where variables VEW, VNS, VSPD are measured) arising
from rotation of the aircraft. However, if GGVSPD is used instead
of VSPD for vertical wind or GGVEW and GGVNS are used (perhaps via
the complementary filter) for the horizontal wind, an additional correction
is needed for the displacement between the GPS antenna and the INS
receiver. On the GV, this distance is -4.30~m. A correction for aircraft
rotation is therefore applied to GGVSPD, GGVEW, and GGVNS, as described
below.
\begin{hangparagraphs}
\textbf{GPS Latitude (}$\text{�}$\textbf{): }\textbf{\uline{GGLAT}}\sindex[var]{GGLAT}\index{GGLAT},
\textbf{\uline{LAT\_G\sindex[var]{LAT_G@LAT\_G}\index{LAT_G@LAT\_G}}};
also formerly \textbf{\uline{GLAT\sindex[var]{GLAT} \index{GLAT}}}\nop{GGLAT}{}\\
\emph{The aircraft latitude measured by a global positioning system.}
Positive values are north of the equator; negative values are south.
Because these variables are recorded in netCDF files as single-precision
GGLAT is provided by the data-system GPS; LAT\_G and LATF\_G are from
the avionics system GPS. LATF\_G is a fine-resolution measurement
that requires special processing. 

\textbf{GPS Longitude (}$\text{�}$\textbf{): }\textbf{\uline{GGLON}}\sindex[var]{GGLON}\index{GGLON},
\textbf{\uline{LON\_G\sindex[var]{LON_G@LON\_G}\index{LON_G@LON\_G}}};
also formerly \textbf{\uline{GLON\sindex[var]{GLON}\index{GLON}}}\nop{GGLON}{}\\
\emph{The aircraft longitude measured by a global positioning system.}
Positive values are east of the prime meridian; negative are west.
GGLON is provided by the (or a) data-system GPS; LON\_G and LONF\_G
are from the avionics system GPS. LONF\_G is a fine-resolution measurement
that requires special processing. \label{punch:3-12}

\textbf{GPS Ground Speed (m/s): }\textbf{\uline{GGSPD}}\sindex[var]{GGSPD}\index{GGSPD},
\textbf{\uline{GSF}}\textbf{\_}\textbf{\uline{G}}\textbf{\sindex[var]{GSF_G@GSF\_G}\index{GSF_G@GSF\_G}}\nop{GGSPD}{}\\
\emph{The aircraft ground speed measured by a global positioning system.\index{velocity!aircraft}
}GGSPD originates from a data-system GPS; GSF\_G originates from an
avionics-system GPS.

\textbf{GPS Ground Speed Vector East Component (m/s): }\textbf{\uline{GGVEW}}\sindex[var]{GGVEW}\index{GGVEW},
\textbf{\uline{VEW\_G}}\sindex[var]{VEW_G@VEW\_G}\index{VEW_G@VEW\_G}\nop{GGVEW}{}\\
\emph{The east component of ground speed measured by a global positioning
system.} GGVEW originates from a data-system GPS; VEW\_G originates
from an avionics-system GPS. In the case of GGVEW, when this is used
in the calculation of horizontal wind, the following correction (omitted
before 2016) is applied:\\
\[
{\rm GGVEWA}={\rm GGVEW}-L_{G}\dot{\psi}\frac{\cos\psi}{\cos\phi}
\]
where GGVEWA is the corrected value used in the wind calculation,
$L_{G}=-4.30$~m for the GV, $\dot{\psi}$ is the rate-of-change
of heading (in radians), and $\psi$ and $\phi$ are respectively
the heading and roll angles. The variable BYAWR transmitted from the
INS gives the rate-of-change of heading $\dot{\psi}$ after conversion
from $^{\circ}\thinspace s^{-1}$ to radians~$s^{-1}$. 

\textbf{GPS Ground Speed Vector North Component (m/s): }\textbf{\uline{GGVNS}}\sindex[var]{GGVNS}\index{GGVNS},
\textbf{\uline{VNS\_G}}\uline{\sindex[var]{VNS_G@VNS\_G}\index{VNS_G@VNS\_G}}\nop{GGVNS}{}\\
\emph{The northward component of ground speed as measured by a global
positioning system.} GGVNS originates from a data-system GPS; VNS\_G
originates from an avionics-system GPS. In the case of GGVNS, when
this is used in the calculation of horizontal wind, the following
correction (omitted before 2016) is applied:\\
\[
{\rm GGVNSA}={\rm GGVNS}+L_{G}\dot{\psi}\frac{\sin\psi}{\cos\phi}
\]
where GGVNSA is the corrected value used in the wind calculation,
$L_{G}=-4.30$~m for the GV, $\dot{\psi}$ is the rate-of-change
of heading (in radians), and $\psi$ and $\phi$ are respectively
the heading and roll angles. The variable BYAWR transmitted from the
INS gives the rate-of-change of heading $\dot{\psi}$ after conversion
from $^{\circ}\thinspace s^{-1}$ to radians~$s^{-1}$. 

\textbf{GPS-Computed Aircraft Vertical Velocity (m/s):}\nop{GGVSPD}{}\textbf{
}\textbf{\uline{GGVSPD, VSPD\_G\sindex[var]{VSPD_G@VSPD\_G}\index{VSPD_G@VSPD\_G}\sindex[var]{GGVSPD}\index{GGVSPD}}}\textbf{,}
\textbf{\uline{GVZI}}\sindex[var]{GVZI (obsolete)}\index{GVZI}
(obsolete)\\
\emph{The aircraft vertical velocity provided by an avionics GPS unit.}
\index{velocity!aircraft vertical}Positive values are upward.\label{punch3.9}
When GGVSPD is used in the calculation of vertical wind, the following
correction\index{correction!vertical wind!rotation in pitch} (omitted
before 2016) is applied:\label{punch:3-13}\\
\[
{\rm GGVSPDA}={\rm GGVSPD}-L_{G}\dot{\theta}
\]
where $L_{G}=-4.30$~m for the GV and $\dot{\theta}$ is the rate-of-change
of the pitch angle, corresponding to the IRU variable BPITCHR{*}$\pi$/180.
The variable GGVSPDA is used internally but not recorded in the data
archives.

\textbf{GPS Altitude (m): }\textbf{\uline{GGALT}}\sindex[var]{GGALT}\index{GGALT},
\textbf{\uline{GALT\_A\sindex[var]{GALT_A@GALT\_A}\index{GALT_A@GALT\_A}}}\nop{GGALT}{}\\
\emph{The aircraft altitude}\index{altitude!aircraft!GPS}\emph{ measured
by a global positioning system.} The measurement is with respect to
a geopotential surface (MSL) defined by the Earth model of the GPS,
which is WGS84.\index{World Geodetic System}\index{WGS-84} Positive
values are above the reference surface. GGALT originates from a data-system
GPS; GALT\_A originates from an avionics-system GPS.

\textbf{GPS Aircraft Track Angle ($\text{�}$):}\nop{GGTRK}{}\textbf{
}\textbf{\uline{GGTRK\sindex[var]{GGTRK}\index{GGTRK}}}\textbf{,}\textbf{\uline{
TKAT\_G}}\sindex[var]{TKAT_G@TKAT\_G}\textbf{\uline{\index{TKAT_G@TKAT\_G}}}\\
\emph{The direction of the aircraft track (degrees clockwise from
true north)} as measured by a data-system global positioning system
(GGTRK) or an avionics-system GPS (TKAT\_G)\emph{.}

\textbf{GPS Height of the Geoid (m):}\nop{GGOIDHT}{}\textbf{ }\textbf{\uline{GGEOIDHT\sindex[var]{GGEOIDHT}\index{GGEOIDHT}}}\\
\emph{Height of geoid, approximating mean sea level, above the WGS84
ellipsoid.} \label{punch3.4}

\textbf{GPS Satellites Tracked:}\nop{GGNSAT}{}\textbf{ }\textbf{\uline{GGNSAT}}\textbf{\sindex[var]{GGNSAT}\index{GGNSAT}}\\
\emph{The number of satellites tracked by a GPS unit.}

\textbf{GPS Mode:}\nop{GGWUAL}{}\textbf{ }\textbf{\uline{GGQUAL}}\sindex[var]{GGQUAL}\index{GGQUAL}\\
\emph{GPS quality flag:}\\
\begin{minipage}[t]{0.95\columnwidth}%
\begin{lyxlist}{00.00.0000}
\item [{~~~~~0}] \setlength{\itemsep}{-1\parsep}Invalid
\item [{~~~~~1}] Valid measurement but without quality enhancement
\item [{~~~~~2}] Measurement enhanced by the Satellite-Based Augmentation
System, a means of improving GPS accuracy and integrity by broadcasting
from geostationary satellites wide area corrections for GPS satellite
orbits and ionospheric delays. In the US, this uses the Wide-Area
Augmentation System or WAAS. This is described in some data files
as a differential-GPS measurement.
\item [{~~~~~5}] Fully locked-in OmniSTAR XP, usually starting after
about 20 minutes of tracking the GPS satellites and receiving the
OmniSTAR data feed. This mode tracks the carrier phases of the L1
and L2 GPS carrier frequencies and provides about 15 cm accuracy in
position.
\end{lyxlist}
%
\end{minipage}

\textbf{GPS Mode:}\nop{GMODE}{}\textbf{ }\textbf{\uline{GMODE\sindex[var]{GMODE (obsolete)}\index{GMODE}
}}\emph{(obsolete)}\\
This is the former output from the Trimble GPS indicating the mode
of operation. The normal value is 4, indicating automatic (not manual)
mode and that the receiver is operating in 4-satellite (as opposed
to fewer) mode.\label{punch3.5}

\textbf{GPS Status:}\nop{GGSTATUS}{}\textbf{ }\textbf{\uline{GGSTATUS}}\sindex[var]{GGSTATUS}\index{GGSTATUS},
\textbf{\uline{GSTAT\_G}}\textbf{\sindex[var]{GSTAT_G@GSTAT\_G}}\textbf{\uline{\index{GSTAT_G@GSTAT\_G}}},
\textbf{\uline{GSTAT}}\sindex[var]{GSTAT}\index{GSTAT} (obsolete)\\
The status of the GPS receiver. A value of 1 indicates that the receiver
is operating normally; a value of 0 indicates a warning regarding
data quality. GGSTATUS indicates the status of the data-system GPS;
GSTAT\_G indicates the status of the avionics-system GPS. The obsolete
variable GSTAT, formerly used for the same purpose, has the reverse
meaning: A value of 0 indicates normal operation and any other code
indicates a malfunction or warning regarding poor data accuracy. 

\label{punch:3-14}
\end{hangparagraphs}


\subsection{Other Measurements of Aircraft Altitude}
\begin{hangparagraphs}
\textbf{Geometric Radio Altitude (m):}\nop{HGM}{}\textbf{ }\textbf{\uline{HGM}}\sindex[var]{HGM}\index{HGM}
\emph{- (obsolete)}\\
\emph{The distance to the surface below the aircraft,} measured by
a radar altimeter\index{altimeter!radar}. The maximum range is 762m
(2,500 ft). The instrument changes in accuracy at an altitude of 152
m: The estimated error from 152 m to 762 m is 7\%, while the estimated
error for altitudes below 152 m is 1.5 m or 5\%, whichever is greater.

\textbf{Geometric Radar Altitude (Extended Range) (APN-159) (m):}\nop{HGME}{}\textbf{
}\textbf{\uline{\label{HGME}HGME}}\sindex[var]{HGME}\index{HGME}\\
\emph{The distance to the surface below the aircraft, measured by
a radar altimeter. }There are two outputs from an APN-159 radar altimeter\index{altimeter!radar}\index{APN-159},
one with coarse resolution (CHGME) and one with fine resolution (HGME).
Both raw outputs cycle through the range 0-360 degrees, where one
cycle corresponds to 4,000 feet for HGME and to 100,000 feet for CHGME.
To resolve the ambiguity arising from these cycles, 4,000-foot increments
are added to HGME to maintain agreement with CHGME. This preserves
the fine resolution of HGME (1.86 m) throughout the altitude range
of the APN-159.

\textbf{Geometric Radar Altitude (Extended Range) (APN-232) (m):}\nop{HGM232}{}\textbf{
}\textbf{\uline{HGM232}}\sindex[var]{HGM232}\index{HGM232}\\
\emph{Altitude above the ground} as measured by an APN-232 radar altimeter.

\textbf{Pressure-Damped Inertial Altitude (m):}\nop{HI3}{}\textbf{
}\textbf{\uline{HI3}}\sindex[var]{HI3 (obsolete)}\index{HI3}
(obsolete)\\
\emph{The aircraft altitude obtained from the twice-integrated IRU
acceleration (ACINS), pressure-adjusted to obtain long-term agreement
with PALT. }Note that this variable has mixed character, producing
short-term variations that accurately track the inertial system changes
but with adjustment to the pressure altitude, which is not a true
altitude. The variable is not appropriate for estimates of true altitude,
but proves useful in the updating algorithm used with the LTN-51 INS
for vertical wind. See the discussion of WP3 \vpageref{WP3 algorithm}.
This variable is now obsolete.

\textbf{ISA Pressure Altitude (m):}\nop{PALT}{}\textbf{ }\textbf{\uline{PALT}}\sindex[var]{PALT}\index{PALT}\\
\emph{The altitude in the International Standard Atmosphere where
the pressure is equal to the reference barometric (ambient) pressure
(PSXC).}\index{altitude!pressure}\index{International Standard Atmosphere}
\index{ISA|see {International Standard Atmosphere}}\footnote{See ``U.S. Standard Atmosphere, 1976'', NASA-TM-A-74335, available
for download at \href{http://ntrs.nasa.gov/archive/nasa/casi.ntrs.nasa.gov/19770009539_1977009539.pdf}{this URL}.} The pressure altitude is best interpreted as a variable equivalent
to the measured pressure, not as a geometric altitude. In the following
description of the algorithm, some constants (identified by the symbol
$^{\ddagger}$) are specified as part of the ISA and so should not
be ``improved'' to more modern values such as those given in the
table in section \ref{ConstantsBox} (e.g., $R_{0}^{\ddagger}).$\footnote{Prior to and including some projects in 2010, processing used slightly
different coefficients: for aircraft other than the GV, $T_{0}/\lambda$
was represented by -43308.83, the reference pressure $p_{0}$ was
taken to be 1013.246, and the exponent $x$ was represented numerically
by 0.190284. For the GV, the value of $T_{0}/\lambda$ was taken to
be 44308.0, the transition pressure $p_{T}$ was 226.1551 hPa, $x$
= 0.190284, and coefficient $\frac{R_{0}^{\prime}T_{T}}{gM_{d}}$
was taken to be 6340.70 m instead of 6341.620 m as obtained below.
The difference between these older values and the ones recommended
below is everywhere less than 10 m and so is small compared to the
expected uncertainty in pressure measurements, because 1 hPa change
in pressure leads to a change in pressure altitude that varies from
about 8\textendash 40 m over the altitude range of the GV.} A note at \href{https://www.eol.ucar.edu/system/files/PressureAltitude.pdf}{this URL}
describes the pressure altitude in more detail and documents the change
that was made in November 2010.\\
\\
\fbox{\begin{minipage}[t]{0.9\textwidth}%
$T_{0}^{\ddagger}$= 288.15 K, reference temperature\\
$\lambda_{a}^{\ddagger}$ \sindex[lis]{lambdaa@$\lambda_{a}$= tropospheric lapse rate, standard atmosphere}=
-0.0065 K/m = the lapse rate\index{International Standard Atmosphere!lapse rate}
for the troposphere$^{\ddagger}$\\
$p$ = measured static (ambient) pressure, hPa, usually from PSXC\\
$p_{0}^{\ddagger}$\sindex[lis]{P0star@$p_{0}^{\ddagger}$= reference pressure for zero altitude,
ISA} = 1013.25 hPa, reference pressure for PALT=0 $^{\ddagger}$\\
$M_{d}^{\ddagger}$ = 28.9644 $kg/kmol$= molecular weight of dry
air, ISA definition $^{\ddagger}$\\
$g^{\ddagger}$ = 9.80665 m\,s$^{-2}$, acceleration of gravity $^{\ddagger}$\\
$R_{0}^{\ddagger}$ = universal gas constant, defined$^{\ddagger}$
as 8.31432$\times10^{3}$ $J\,\mathrm{kmol}^{-1}\,\mathrm{K}^{-1}$\\
$z_{T}^{\ddagger}$ = altitude of the ISA tropopause = 11,000 m $^{\ddagger}$
\\
$x=-R_{0}^{\ddagger}\lambda_{a}^{\ddagger}/(M_{d}^{\ddagger}g^{\ddagger})$
$\approx$ 0.1902632 (dimensionless)\footnote{This is the value, rounded to seven significant figures, that is used
for data processing.}\\
\\
\rule[0.5ex]{1\linewidth}{1pt}

For pressure > 226.3206 hPa (equivalent to a pressure altitude < $z_{T}$):\\
\[
\mathrm{PALT}=-\left(\frac{T_{0}^{\ddagger}}{\lambda^{\ddagger}}\right)\left(1-\left(\frac{p}{p_{0}^{\ddagger}}\right)^{x}\right)
\]
otherwise, if $T_{T}$\sindex[lis]{Tt@$T_{T}$= temperature at the ISA tropopause}
and $p_{T}$\sindex[lis]{pT@$p_{T}$= pressure at the ISA tropopause}
are respectively the temperature and pressure at the altitude\index{International Standard Atmosphere!tropopause}
$z_{T}$:\\
\[
T_{T}=T_{0}+\lambda^{\ddagger}z_{T}^{\ddagger}=216.65\,\mathrm{K}
\]
\[
p_{T}=p_{0}^{\ddagger}\Bigl(\frac{T_{0}^{\ddagger}}{T_{T}}\Bigr)^{\frac{g^{\ddagger}M_{d}^{\ddagger}}{\lambda^{\ddagger}R_{0}^{\ddagger}}}=226.3206\,\mathrm{hPa}
\]
\[
\mathrm{PALT}=z_{T}^{\ddagger}+\frac{R_{0}^{\ddagger}T_{T}}{g^{\ddagger}M_{d}^{\ddagger}}\ln\left(\frac{p_{T}}{p}\right)
\]
which, after conversion from natural to base-10 logarithm, is coded
to be equivalent to the following:\\

\begin{lyxcode}
//~transition~pressure~at~the~assumed~ISA~tropopause:

\#define~ISAP1~226.3206

//~reference~pressure~for~standard~atmosphere:

\#define~ISAP0~1013.25

if~(psxc~>~ISAP1)

~~~~palt~=~44330.77~{*}~(1.0~-~pow(psxc/ISAP0,~0.1902632));

else

~~~~palt~=~11000.0~+~14602.12~{*}~log10(ISAP1/psxc);
\end{lyxcode}
%
\end{minipage}}\\
\\

\textbf{Altitude, Reference (MSL): }\textbf{\uline{ALTX}}\textbf{\sindex[var]{ALTX}\index{ALTX}
}\emph{(Obsolete)}\textbf{, }\textbf{\uline{GGALTC}}\sindex[var]{GGALTC}\index{GGALTC}
\emph{(Obsolete) }\label{punch3.6}\\
\emph{Derived altitude above the geopotential surface, }obtained
by combining information from a GPS receiver and an inertial reference
system. This variable was intended to compensate for times when GPS
reception was lost by incorporating information from the IRS measurement
of altitude. GPS status measurements were used to detect signal loss,
although sometimes this signal was delayed for a few seconds after
the signal was lost. A 10-second running average was calculated of
the difference between the GPS altitude and the reference altitude.
When the sample-to-sample altitude difference changed more than 50
meters or when the GPS status detected a degraded signal, the derived
variable (ALTX or GGALTC) became the alternate reference altitude
adjusted by the latest running-average difference between that reference
altitude and GGALT. When reception was recovered, to avoid a sudden
discontinuity in altitude, the derived variable was adjusted back
to the GPS altitude gradually over the next 10 seconds. \\
\\
This obsolete variable should be used with caution because the reference
altitude used in past calculations was the IRS altitude updated to
the pressure altitude of the aircraft. To account for the difference
between pressure and geometric altitude, a regression equation was
used, normally $z=a_{0}+a_{1}\mathrm{*{PALT}}$ where $a_{0}=-46.3$\,m
and $a_{1}=0.97866$ but often adjusted dependent on project conditions.
This introduced problems in early applications with the GV because
it did not account for the pressure-altitude transition at the ISA
tropopause. Use of a pressure altitude as reference introduces additional
errors in altitude in regions that are not barotropic.\\
\end{hangparagraphs}


\subsection{\label{subsec:IRS/GPS}Combining IRS and GPS Measurements}

Measurements from the global positioning\index{GPS} and inertial
navigation systems\index{INS} are combined to produce new variables
that take advantage of the strengths of each, so that the resulting
variables have the long-term stability of the GPS and the short-term
resolution of the INS. This section describes some variables that
result from this blending of variables. These corrected variables
are usually the best available when the GPS and IRS are both functioning. 

One can determine if the GPS is functioning by examining the GPS status
variables described in the previous section or by looking for spikes
or ``flat-lines'' in the data. If the GPS data are missing for a
short time (a few seconds to a minute), accuracy is not affected.
However, longer dropouts will result in uncertainties degrading toward
those of the INS. Without the GPS or another ground reference, the
IRS error cannot be determined empirically, and one should assume
that it is within the manufacturer's specification (1 nautical mile
of error per hour of flight, 90\% CEP). When the GPS is active, RAF
estimates that the correction algorithm produces a position with an
error less than 1.5 m. \label{punch3.7}Due to the nature of the algorithm,
the error will increase from about 1.5 meters to the INS specification
in about one-half hour after GPS information is lost.\medskip{}

\begin{hangparagraphs}
\textbf{GPS-Corrected Inertial Ground Speed Vector, (m/s):}\nop{VEWC}{}\nop{VNSC}{}\textbf{
}\textbf{\uline{VEWC}}\textbf{\sindex[var]{VEWC}\index{VEWC},
}\textbf{\uline{VNSC}}\sindex[var]{VNSC}\index{VNSC}\\
\index{velocity!aircraft}These variables result from combining GPS
and INS output of the east and north components of ground speed from
a complementary-filter\index{filter!complementary (for wind)} algorithm.
Positive values are toward the east and north, respectively. The smooth,
high-resolution, continuous measurements from the inertial navigation
system, \{VNS, VEW\}, which can slowly accumulate errors over time,
are combined with the measurements from the GPS, \{GVNS, GVEW\}, which
have good long-term stability, via an approach based on a complementary
filter. A low-pass filter, $F_{L}(\{\mathrm{GVNS,GVEW\}})$\sindex[lis]{FL@$F_{L}$= digital low-pass filter},
is applied to the GPS measurements of groundspeed, which are assumed
to be valid for frequencies at or lower than the cutoff frequency
$f_{c}$\sindex[lis]{fc@$f_{c}$= cutoff frequency for the filter $F_{L}$}
of the filter. Then the complementary high-pass filter, denoted ($1-F_{L}$)($\{\mathrm{VNS,VEW\}}$),
is applied to the IRS measurements of groundspeed, which are assumed
valid for frequencies at or higher than $f_{c}$. Ideally, the transition
frequency would be selected where the GPS errors (increasing with
frequency) are equal to the IRS errors (decreasing with frequency).
\\
\\
The procedure is use now is documented in the \href{http://dx.doi.org/10.5065/D60G3HJ8}{Technical Note on Wind Uncertainty},
beginning on p.~125. It is a three-pole Butterworth low-pass filter\index{filter!Butterworth},
originally coded following the algorithm described in Bosic, S.~M.,
1980: \emph{Digital and Kalman filtering : An Introduction to Discrete-Time
Filtering and Optimum Linear Estimation, }p. 49. As described in \href{https://drive.google.com/open?id=0B1kIUH45ca5AY0k0cTB6QlBEQUU}{this memo},
it has been revised (2014) to use coefficients generated by the R
routine ``butter().'' The digital filter used is recursive, not
centered, to permit calculation during a single pass through the data.
If the cutoff frequency lies where both the GPS and INS measurements
are almost the same, then the detailed characteristics of the filter
(e.g., phase shift) in the transition region do not matter because
the complementary filters have canceling effects when applied to the
same signal. The transition frequency $f_{c}$ was chosen to be (1/600)
Hz. The Butterworth filter was chosen because it provides flat response
away from the transition.\footnote{For historical reasons, the details of the now obsolete filter as
originally coded and used for many years are described here. For the
current version with coefficients, see the memo referenced above.\\
 %
\noindent\fbox{\begin{minipage}[t]{1\textwidth - 2\fboxsep - 2\fboxrule}%
\textbf{CONSTANTS} (dependent on time constant $\tau):$\footnote{For processing prior to the time of this review, the factor $\sqrt{\frac{3}{2}}$
was erroneously $\frac{\sqrt{3}}{2}$.}\\
$a=\frac{2\pi}{\tau}$, $a_{2}$=$a\,e^{-a/2}(\cos(a{\color{red}\sqrt{\frac{3}{2}}})+\sqrt{\frac{1}{3}}\sin(a{\color{red}\sqrt{\frac{3}{2}}}))$,
$a_{3}$=2$e^{-a/2}$$\cos(a{\color{red}\sqrt{\frac{3}{2}}})$, $a_{4}$=$e^{-a}$\\
\rule[0.5ex]{1\linewidth}{1pt}
\begin{lyxcode}
//~input~x~=~unfiltered~signal

//~output~returned~is~low-pass-filtered~input

//~tau~determines~the~cutoff

//~zf{[}{]}~saves~values~~for~recursion

zf{[}2{]}~=~-a{*}x~+~a2{*}zf{[}5{]}~+~a3{*}zf{[}3{]}~-~a4{*}zf{[}4{]};~~~

zf{[}1{]}~=~a{*}x~+~a4{*}zf{[}1{]};~~~

zf{[}4{]}~=~zf{[}3{]};~~~

zf{[}3{]}~=~zf{[}2{]};~~~

zf{[}5{]}~=~x;

return(zf{[}1{]}~+~zf{[}2{]});~
\end{lyxcode}
%
\end{minipage}}} The net result then is the sum of these two filtered signals, calculated
as described in the box \vpageref{CompFilterBox}:\\
\\
\fbox{\begin{minipage}[t]{0.95\textwidth}%
\label{CompFilterBox}VEW\index{VEW} = IRS-measured east component
of the aircraft ground speed\\
VNS\index{VNS} = IRS-measured north component of the aircraft ground
speed\\
GGVEW\index{GGVEW} = GPS-measured east component of the aircraft
ground speed\\
GGVNS\index{GGVNS} = GPS-measured north component of the aircraft
ground speed\\
$F_{L}()$ = three-pole Butterworth low-pass recursive digital filter\\
\\
\rule[0.5ex]{1\linewidth}{1pt}

\[
\{\mathrm{VNSC}\}=F_{L}(\mathrm{\{GGVNS\})}+(1-F_{L})(\{\mathrm{VNS\}})
\]
\[
\{\mathrm{VEWC}\}=F_{L}(\mathrm{\{GGVEW\})}+(1-F_{L})(\{\mathrm{VEW\}})
\]
%
\end{minipage}}\\
\\
This result is used as long as the GPS signals are continuous and
flagged as being valid. When that is not the case, some means is needed
to avoid sudden discontinuities in velocity (and hence wind speed),
which would introduce spurious effects into variance spectra and other
properties dependent on a continuously valid measurement of wind.
To extrapolate measurements through periods when the GPS signals are
lost (as sometimes occurs, for example, in turns) a fit is determined
to the difference between the best-estimate variables \{VNSC,VEWC\}
and the IRS variables \{VNS,VEW\} for the period before GPS reception
was lost, and that fit is used to extrapolate through periods when
GPS reception is not available. The procedure is as described on the
next page.\\
\label{punch:3-15}%
\begin{minipage}[t]{0.94\columnwidth}%
\begin{enumerate}
\item \setlength{\itemsep}{-1\parsep}If GPS reception has never been valid
earlier in the flight, use the INS values without correction. 
\item Whenever both GPS\index{GPS} and INS are good, update the low-pass-filtered
estimate of the difference between them. This is added to the INS
measurement to obtain the corrected variable. Also update a least-squares
fit to the difference between the GPS and INS groundspeeds, for each
component. The errors are assumed to result primarily from a Schuler
oscillation\index{Schuler oscillation}, so the three-term fit is
of the form $\Delta=a_{1}+a_{2}\sin(\Omega_{Sch}t)+a_{3}\cos(\Omega_{Sch}t)$
, where $\Omega_{Sch}$ is the angular frequency of the Schuler oscillation
(taken to be $2\pi/(5067\,s))$ and $t$ is the time since the start
of the flight. A separate fit is used for each component of the velocity
and each component of the position (discussed below under LATC and
LONC). The fit matrix\index{fit matrix, wind correction} used to
determine these coefficients is updated each time step but the accumulated
fit factors decay exponentially with a 30-min decay constant, so the
terms used to determine the fit are exponentially weighted over the
period of valid data with a time constant that decays exponentially
into the past with a characteristic time of 30 min. This is long enough
to determine a significant portion of the Schuler oscillation but
short enough to emphasize recent measurements of the correction.
\item When GPS data become invalid, if sufficient data (spanning 30 min)
have been accumulated, invert the accumulated fit matrices to determine
the coefficients $\{a_{1},a_{2},a_{3}\}$ and then use the formula
for $\Delta$ in the preceding step to extrapolate the correction
to the IRS measurements while the GPS measurements remain invalid.
Doing so immediately would introduce a discontinuity in \{VNSC,VEWC\},
however, so the correction $\Delta$ is introduced smoothly by adjusting
\{VNSC, VEWC\} as follows: if dvy is the adjustment added to the INS
measurement, adjust it according to $\mathrm{dvy}^{\prime}$=$\eta$\,dvy$+(1-\eta)\Delta$
where $\mathrm{dvy}{}^{\prime}$ is the sequentially adjusted correction
and\sindex[lis]{eta@$\eta=$update constant for exponential updating}
$\eta=0.995\,s^{-1}$ is chosen to give a decaying transition with
a time constant of about 5.5 min. This has the potential to introduce
some artificial variance at this scale and so should be considered
in cases where variance spectra are analyzed in detail, but it has
much less influence on such spectra than a discontinuous transition
would. Ideally, the current fit and the last filtered discrepancy
(VNSC$_{0}$-GVNS$_{0}$) should be about equal, so transitioning
between them should not introduce a significant change.\label{punch3.8}
\item To avoid transients that would result from switching abruptly to the
complementary-filter solution when GPS again becomes valid, the correction
factors (e.g., dvy) are also updated smoothly toward the complementary-filter
solution, using for example $\mathrm{dvy}^{\prime}$=$\eta$\,dvy$+(1-\eta)F_{L}(v_{y}^{GPS}-v_{y}^{IRU})$
where $F_{L}$ is the low-pass filter and $v_{y}$ the northward component
of aircraft velocity.
\end{enumerate}
%
\end{minipage}

\medskip{}

\textbf{GPS-Corrected Inertial Latitude and Longitude (}$\text{�}$\textbf{):}\nop{}{LATC}\nop{LONC}{}\textbf{
}\textbf{\uline{LATC}}\textbf{\sindex[var]{LATC}\index{LATC},
}\textbf{\uline{LONC}}\sindex[var]{LONC}\index{LONC}\\
\emph{Combined GPS and IRS output of latitude and longitude.} Positive
values are north and east, respectively.\label{punch:3-16} These
variables\index{latitude}\index{longitude} are the best estimate
of position, obtained by the following approach:\\
\fbox{\begin{minipage}[t]{0.95\textwidth}%
LAT\index{LAT} = latitude measured by the IRS\\
LON\index{LON} = longitude measured by the IRS\\
GGLAT\index{GGLAT} = latitude measured by the GPS\\
GGLON\index{GGLON} = longitude measured by the GPS\\
VNSC\index{VNSC} = aircraft ground speed, north component, corrected
\\
VEWC\index{VEWC} = aircraft ground speed, east component, corrected\\
\\
\rule[0.5ex]{1\linewidth}{1pt}
\begin{enumerate}
\item \setlength{\itemsep}{-1\parsep}Initialize the corrected position
at the IRS position at the start of the flight or after any large
change (>5$\text{�}$) in the IRS position.
\item Integrate forward from that position using the aircraft groundspeed
with components \{VNSC,VEWC\}. Note that in the absence of GPS information
this will introduce long-term errors because it does not account for
the Earth's spherical geometry. It provides good short-term accuracy,
but the GPS updating in the next step is needed to compensate for
the difference between a rectilinear frame and the Earth's spherical
coordinate frame and provides a smooth yet accurate track.
\item Use an exponential adjustment to the GPS position, with time constant
that is typically about 100 s.\footnote{specifically, LATC += $\eta$(GLAT-LATC) with $\eta=2\pi/(600\,\mathrm{s})$}
\item To handle periods when the GPS becomes invalid, use an approach analogous
to that for groundspeed, whereby a Schuler-oscillation fit to the
difference between the GPS and IRS measurements is accumulated and
used to extrapolate through periods when the GPS is invalid. 
\end{enumerate}
%
\end{minipage}}

\medskip{}

\medskip{}

\textbf{\uline{}}\\
\end{hangparagraphs}


