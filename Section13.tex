
\section{AIR CHEMISTRY MEASUREMENTS}

\subsection{Variables in Standard Data Files\label{punch:6-1}}
\begin{hangparagraphs}
\textbf{Carbon Monoxide Preliminary Mixing Ratio (ppbv): }\textbf{\uline{CORAW\_AL}}\textbf{\sindex[var]{CORAW_AL@CORAW\_AL}\index{CORAW_AL@CORAW\_AL}}\\
\emph{The preliminary measurement of CO mixing ratio from the Aero-Laser
model AL-5002 CO analyzer, before final calibrations are applied.}
This instrument measures CO\index{CO} by vacuum ultraviolet resonance
fluorescence. It is a commercial version of the instrument described
by Gerbig et al.\footnote{Journal of Geophysical Research, Vol. 104, No. D1, 1699-1704, 1999}
The instrument is described further at \href{https://www.eol.ucar.edu/instruments/aero-laser-vuv-resonance-fluorescence-carbon-monoxide-instrument}{this URL}.
The time resolution is 1 second. This variable is sometimes present
in flight and in preliminary ground processing, but normally it is
replaced by COMR\_AL in final processing.

\textbf{Carbon Monoxide Mixing Ratio (ppbv): }\textbf{\uline{COMR\_AL}}\textbf{\sindex[var]{COMR_AL@COMR\_AL}\index{COMR_AL@COMR\_AL}}\\
\emph{The mixing ratio measured by the Aero-Laser model AL-5002 CO
analyzer.} See also CORAW\_AL above. The calculation of COMR\_AL is
based on in-flight calibrations conducted 1-2 times per hour, when
a gas of known concentration\index{concentration!calibration gas}
is supplied to the instrument and then a catalyst trap removes CO
to provide a zero reference. The calibration\index{calibration!gas}
results in a sensitivity and zero that are then used to convert the
measurements from the instrument (recorded as counts per second) to
a mixing ratio in units of ppbv. Time-dependent sensitivity and zero
coefficients are computed post-flight as a linear interpolation between
flight calibrations. This variable normally appears in final data
sets for a project.\footnote{In isolated cases XCOMR\sindex[var]{XCOMR} or XCOMR\_AL\sindex[var]{AL}
was used for this variable name.} The algorithm is described in the following box:\\
\\
\fbox{\begin{minipage}[t]{0.95\columnwidth}%
CPS\index{CPS=counts per second} = counts per second from the instrument\\
S(t) = sensitivity\sindex[lis]{S(t)=sensitivity function, calibration}
at time t = (CPS when exposed to cal gas) / concentration of cal gas

Z(t) = zero\sindex[lis]{Z(t)=zero function, calibration} at time
t = CPS when exposed to air passing through the catalyst trap

\rule[0.5ex]{1\columnwidth}{1pt}

\begin{equation}
\{\mathrm{COMR\_AL}\}=(\mathrm{\{\mathrm{CPS}\}-Z(t))/S(t)}\label{eq:COMR}
\end{equation}
%
\end{minipage}}\\
See also the obsolete variables in Section \ref{sec:OBSOLETE-VARIABLES},
where variables from an earlier TECO Model 48 CO analyzer, in use
before 2000, are described.

\textbf{}%
\noindent\begin{minipage}[t]{1\columnwidth}%
\textbf{Carbon Dioxide Mixing Ratio (ppmv): }\textbf{\uline{CO2\_PIC}}\uline{x}\index{CO2_PICx@CO2\_PICx}\sindex[var]{CO2_PICx@CO2\_PICx}\\
\textbf{Methane Mixing Ratio (ppmv): }\textbf{\uline{CH4\_PICx}}\index{CH4_PICx@CH4\_PICx}\sindex[var]{CH4_PICx@CH4\_PICx}%
\end{minipage}\textbf{}\\
\emph{Respectively, the carbon dioxide and methane mixing ratio measured
by a Picarro CO2/CH4 instrument.} The letter 'x' may be replaced by
the model number of the instrument (e.g., 1301) or it may be blank.
The Picarro CO2/CH4 G1301-f flight analyzer is a fast response trace
gas monitor that measures CO$_{2}$ and CH$_{4}$ using wavelength-scanned
cavity ring-down spectroscopy. The time resolution is 0.2 \textendash{}
1 seconds. Additional information characterizing the instrument can
be found at \href{https://www.eol.ucar.edu/instruments/picarro-instrument-airborne-measurement-co2-and-ch4}{this URL}.
During flight, both measurements are calibrated 1-2 times per hour
via sampling of a working standard, and linear calibration coefficients
are applied based on multi-point lab calibration data and in-flight
calibration checks. The procedure is analogous to that used for COMR\_AL,
as described immediately above. When water vapor is not removed from
the ambient sample stream (the normal case), a correction factor for
water present in the sensing cell must be applied following the approach
of Richardson et al.,\footnote{Richardson, S.~J., N.~L.~Miles, K.~J.~Davis, E.~R.~Crosson,
C.~W.~Rella, and A.~E.~Andrews, 2012\emph{: }Field testing of
cavity ring-down spectroscopy analyzers measuring carbon dioxide and
water vapor.\emph{ J.~Atmos.~Oceanic\_Technol, }\textbf{\emph{29,}}\emph{
397\textendash 406.}} as follows: \\
\\
\noindent\fbox{\begin{minipage}[t]{1\columnwidth - 2\fboxsep - 2\fboxrule}%
{[}CO$_{2}${]}$_{wet}$ = carbon dioxide mixing ratio as measured
in the sensing cell (with water)

{[}CO$_{2}${]}$_{dry}$ = carbon dioxide mixing ratio in dry air,
corrected for the effects of water vapor

{[}CH$_{4}${]}$_{wet}$ = methane mixing ratio as measured in the
sensing cell (with water)

{[}CH$_{4}${]}$_{dry}$ = methane mixing ratio in dry air, corrected
for the effects of water vapor

$W$ = water vapor mixing ratio measured in the instrument cell (percent
by volume)

\{$c_{0}$, $c_{1}$\} = \{-0.01200, -2.674$\times10^{-4}$\} (dimensionless)

\{$d_{0}$, $d_{1}$\} = \{-0.00982, -2.393$\times10^{-4}$\} (dimensionless)

\rule[0.5ex]{1\columnwidth}{1pt}

\begin{equation}
\{\mathrm{CO2\_PICX\}}=[\mathrm{CO_{2}]_{dry}=}\frac{[\mathrm{CO_{2}]_{wet}}}{1+c_{0}W+c_{1}W^{2}}\label{eq:CO2pic}
\end{equation}
\begin{equation}
\{\mathrm{CH4\_PICX\}}=[\mathrm{CH_{4}]_{dry}=}\frac{[\mathrm{CH_{4}]_{wet}}}{1+d_{0}W+d_{1}W^{2}}\label{eq:CH4pic}
\end{equation}
%
\end{minipage}}

\noindent\begin{minipage}[t]{1\columnwidth}%
\textbf{Chemiluminescent Ozone Sample Flow Rate (sccm): }\textbf{\uline{XFO3FS}}\textbf{\index{XFO3FS}\sindex[var]{XFO3FS}}\\
\textbf{Chemiluminescent Ozone Nitric Oxide Flow Rate (sccm): }\textbf{\uline{XFO3FN}}\textbf{O\index{XFO3FNO}\sindex[var]{XFO3FNO}}\\
\textbf{Chemiluminescent Ozone Sample Pressure (mb): }\textbf{\uline{XFO3P\index{XFO3P}\sindex[var]{XFO3P}}}%
\end{minipage}\\
\emph{Flows and pressure within the chemiluminescence ozone sensor.}
The sample rate, in standard $cm^{3}/s$, is XFO3FS, while XFO3FNO
gives the NO flow rate in the same units and XFO3P is the pressure
in the ozone sample cell. These variables apply to measurements made
by an earlier version of the fast ozone instrument. They have not
been present in projects since 2006.

\textbf{Fast response NO chemiluminescence ozone mixing ratio (ppbv):
}\textbf{\uline{FO3\_ACD}}\textbf{, }\textbf{\uline{FO3\_CL}},
\textbf{XO3, O3MR\_CL\sindex[var]{O3MR_CL@O3MR\_CL}\index{O3MR_CL@O3MR\_CL}\sindex[var]{XO3}}\index{XO3}\index{FO3_x@FO3\_x}\sindex[var]{FO3_x@FO3\_x}\index{FO3_ACD@FO3\_ACD}\index{FO3_CL@FO3\_CL}\sindex[var]{FO3_ACD@FO3\_ACD}\sindex[var]{FO3_CL@FO3\_CL}\\
\emph{The ozone mixing ratio (by volume) measured by an NO chemiluminescence
instrument. } The instrument detects chemiluminescence from the reaction
of nitric oxide (NO) with ambient ozone, using a dry-ice cooled, red-sensitive
photomultiplier employing photon-counting electronics. The measurement
principle is described by Ridley et al.~(1992)\footnote{Ridley, B.~A., F.~E.~Grahek, and J.~G.~Walega, 1992: A small,
high-sensitivity, medium-response ozone detector suitable for measurements
from light aircraft. \emph{J.~Atmos.~Oceanic Technol., }\textbf{9,
}142\textendash 148. }, and there is additional information describing the instrument at
\href{https://www.eol.ucar.edu/instruments/nitric-oxide-chemiluminescence-ozone-instrument}{this URL}.
The time resolution is 0.2 seconds, and typical uncertainty is 5\%.
The background signal is measured 1-2 times hourly during flights.
Linear calibration coefficients are applied to the photon count rate
to produce mixing ratios, and a correction is applied for water vapor
during final processing, as follows: \\
\noindent\fbox{\begin{minipage}[t]{1\columnwidth - 2\fboxsep - 2\fboxrule}%
CPS\index{CPS=counts per second} = counts per second from the instrument\\
{[}O$_{3}${]}$_{wet}$ = ozone mixing ratio as measured in the sensing
cell (with water)

{[}O$_{3}${]}$_{dry}$ = ozone mixing ratio in dry air, corrected
for the effects of water vapor\\
$S(t)$ = sensitivity at time t = (\{CPS\} when exposed to cal gas)
/ concentration of cal gas

$Z(t)$ = background at time t = \{CPS\} when exposed to zero-ozone
air

$W^{\prime}$ = water vapor mixing ratio by volume\sindex[lis]{rv@$r_{v}$=water vapor mixing ratio by volume}
(expressed as a fraction; dimensionless)

$\kappa$ = correction factor for water vapor = 4.3 (dimensionless)

\rule[0.5ex]{1\columnwidth}{1pt}

\begin{equation}
[\mathrm{O}_{3}]_{wet}=\frac{\{\mathrm{CPS}\}-Z(t)}{S(t)}\label{eq:FO3}
\end{equation}
\begin{equation}
\{\mathrm{F03\_ACD}\}=[\mathrm{O_{3}]}_{dry}=[\mathrm{O_{3}]}_{wet}\times(1+\kappa r_{v})\label{eq:FO3WaterCorr}
\end{equation}
%
\end{minipage}}

\textbf{Uncorrected TECO Ozone Mixing Ratio (ppb): }\textbf{\uline{TEO3}}\index{TEO3}\sindex[var]{TEO3}\\
\emph{The uncorrected ozone mixing ratio output from the TECO model
49c UV ozone analyzer.} See TEO3C. 

\noindent\begin{minipage}[t]{1\columnwidth}%
\textbf{Internal TECO Ozone Sampling Pressure (hPa): }\textbf{\uline{TEP}}\textbf{\index{TEP}\sindex[var]{TEP},
}\textbf{\uline{TEO3P}}\index{TEO3P}\sindex[var]{TEO3P}\\
\textbf{Internal TECO Ozone Sampling Temperature ($^{\circ}C$): }\textbf{\uline{TET\index{TET}}}\sindex[var]{TET}%
\end{minipage}\\
\emph{The pressure (TEP) or temperature (TET) inside the detection
cell of the TECO 49 UV ozone analyzer.} These are used to convert
the measurements from the instrument to units of ppbv. In many projects,
the cell temperature was not recorded so an expected cell temperature
in the aircraft cabin must be used in processing.

\textbf{Corrected TECO Ozone Mixing Ratio (ppbv): }\textbf{\uline{TEO3C}}\index{TEO3C}\sindex[var]{TEO3C}\\
\emph{The ozone mixing ratio (by volume) determined by the TECO model
49c UV ozone analyzer (cf.~\href{https://www.eol.ucar.edu/instruments/thermo-environmental-instruments-model-49-ozone-analyzer}{this description})
after correction for the pressure and temperature in the cell by application
of the ideal gas law.} Because the basic measurement is ozone density
in the chamber, this measurement must be converted to a mixing ratio
by dividing by the air density, calculated from the pressure and temperature
measured in the chamber (TEP and TET respectively).\label{punch:6-3}
The instrument provides output only each ten seconds, and measurements
are collected in the 3 s preceding the update. The measurements may
be artificially high or low when rapid changes in humidity are present,
as may occur when crossing the top of the boundary layer or when going
through clouds. In operation on the ground prior to takeoff or immediately
after landing, a high concentration of hydrocarbons can cause spuriously
high measurements. The detection limit is 1 ppbv with an uncertainty
of $\pm$5\%. This instrument is seldom used as of 2014 and may soon
be classified as obsolete.

\textbf{}%
\noindent\begin{minipage}[t]{1\columnwidth}%
\textbf{NO Raw Counts (counts per sample interval): }\textbf{\uline{XNO}}\textbf{\index{XNO}}\sindex[var]{XNO}\textbf{}\\
\textbf{NOy Raw Counts (counts per sample interval): }\textbf{\uline{XNOY\index{XNOY}}}\sindex[var]{XNOY}\textbf{\uline{}}\\
\textbf{NO Calibration Flow (SLPM): }\textbf{\uline{XNOCF}}\textbf{\index{XNOCF}}\sindex[var]{XNOCF}\textbf{}\\
\textbf{NOy Calibration Flow (SLPM): }\textbf{\uline{XNCLF}}\textbf{\index{XNCLF}}\sindex[var]{XNCLF}\textbf{}\\
\textbf{NO, NOy Measurement Status (dimensionless): }\textbf{\uline{XNST}}\textbf{\index{XNST}}\sindex[var]{XNST}\textbf{}\\
\textbf{NO Zero Air Flow (SLPM): }\textbf{\uline{XNOZA}}\textbf{\index{XNOZA}}\sindex[var]{XNOZA}\textbf{}\\
\textbf{NOy Zero Air Flow (SLPM): }\textbf{\uline{XNZAF}}\textbf{\index{XNZAF}}\sindex[var]{XNZAF}\textbf{}\\
\textbf{NO Sample Flow (SLPM): }\textbf{\uline{XNOSF}}\textbf{\index{XNOSF}}\sindex[var]{XNOSF}\textbf{}\\
\textbf{NOy Sample Flow (SLPM): }\textbf{\uline{XNSAF}}\textbf{\index{XNSAF}}\sindex[var]{XNSAF}\textbf{}\\
\textbf{NOy Reaction Chamber Pressure (mb): }\textbf{\uline{XNOYP}}\textbf{\index{XNOYP}}\sindex[var]{XNOYP}\textbf{}\\
\textbf{Gold NOy Converter Temperature ($\text{�}$C): }\textbf{\uline{XNMBT\index{XNMBT}}}\sindex[var]{XNMBT}%
\end{minipage}\\
\emph{The measurements provided by the NO+NO$_{2}$ instrument, }which
is described at \href{https://www.eol.ucar.edu/instruments/no-no2-instrument}{this link}\emph{.}
XNO and XNOY are the raw data counts from the NO and NO$_{2}$ instruments,
respectively, and XNCLF and XNOCF are the respective calibration flows
for these instruments. \label{punch:6-2} XNST records the status
for both instruments: In measurement mode, XNST is 0, while XNST is
5 when the instruments are in zero mode and 10 when the instruments
are in calibration mode. the NOy and NO instruments. The instrument
is in the measure mode for XNST of 0. For a XNST reading of 5 the
instruments are in the zero mode. XNST value of 10 is the calibration
mode. XNOZA and XNZAF are flow rates for zero air used to back flush
inlets, typically at takeoff and landing, and for calibration using
``zero'' air. Even if the status, XNST, is 0, indicating the instrument
is in the measurement mode, when XNOZA and XNZAF are approximately
1 SLPM the instrument is measuring zero air and not ambient air. \label{punch:6-4}
XNOSF and XNSAF are the sample flow rates through the NO and NO$_{2}$
instruments respectively. These values are typically about 1 SLPM.
XNMBT is the temperature of the gold NO$_{2}$ converter.

\noindent\begin{minipage}[t]{1\columnwidth}%
C\textbf{orrected NO Mixing Ratio (ppbv): }\textbf{\uline{XNOCAL}}\textbf{\index{XNOCAL}}\\
\textbf{Corrected NO$_{2}$ Mixing Ratio (ppbv): }\textbf{\uline{XNYCAL\index{XNYCAL}}}%
\end{minipage}\\
\emph{The calibrated NO and NO$_{2}$ volumetric mixing ratio, respectively,
measured by the NO-NO$_{2}$ instrument.} See \href{https://www.eol.ucar.edu/instruments/nitric-oxide-chemiluminescence-ozone-instrument}{this link}
for a description of the instrument.\label{punch:6-5} The NO and
NO$_{2}$ data are represented by a cubic spline for baseline subtraction,
and then the calibration coefficients are applied and the measurements
are converted to units of ppbv. The quality of the data can be assessed
by examining the accuracy of the zero correction. This instrument
\label{punch:6-6} adds water vapor to the sample stream to reduce
the effect of ambient water on the final signal. The water vapor addition
is not sufficient to saturate the sample stream, but enough to remove
much of the interference. The detection limits of the NO,NO$_{2}$
instruments are 50 ppbv for a one-second averaging time. The uncertainty
is $\pm$ 5\%.
\end{hangparagraphs}


\subsection{Variables in Special Data Sets}

Research projects often incorporate user-supplied instruments into
payloads, and those instruments produce data files that are either
recorded independently or merged into the standard netCDF data files
for the projects. In addition, NCAR offers a set of instruments that
require additional data processing and analysis, often because the
measurements require special interpretation to obtain the desired
measurements. The following instruments can provide such air-chemistry
measurements: 

\fbox{\begin{minipage}[t]{0.98\columnwidth}%
\begin{itemize}
\item \setlength{\itemsep}{-1\parsep}Advanced Whole Air Sampler (\href{http://www.eol.ucar.edu/instruments/advanced-whole-air-sampler}{AWAS})\index{AWAS}
\item Chemical Ionization Mass Spectrometer (\href{http://www.eol.ucar.edu/instruments/georgia-tech-chemical-ionization-mass-spectrometer}{CIMS})\index{CIMS}
\item Quantum Cascade Laser Spectrometer (\href{http://www.eol.ucar.edu/instruments/quantum-cascade-laser-spectrometer}{QCLS})\index{QCLS}
\item Trace Organic Gas Analyzer (\href{http://www.eol.ucar.edu/instruments/trace-organic-gas-analyzer}{TOGA})\index{TOGA}
\end{itemize}
%
\end{minipage}}\\
Follow the links in the box to descriptions of these instruments on
the EOL web site. Those descriptions include brief explanations of
how data are acquired and handled. The process varies with instrument;
The CIMS and QCLS instruments produce variables that are often merged
into the standard netCDF archived data files for projects, the AWAS
collects samples that are later analyzed using ground-based instruments
but result in a special dataset dependent on analysis technique and
sample location and duration, while the TOGA is usually analyzed to
produce dozens of trace-gas measurements, some of which can be merged
into standard netCDF files. 

Users interested in using these measurements should contact \href{mailto:mailto:raf-dm@eol.ucar.edu}{EOL/RAF data management}
for data access and assistance.
