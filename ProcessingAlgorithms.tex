% Options for packages loaded elsewhere
\PassOptionsToPackage{unicode}{hyperref}
\PassOptionsToPackage{hyphens}{url}
%
\documentclass[
  english,
]{book}
\usepackage{amsmath,amssymb}
\usepackage{lmodern}
\usepackage{ifxetex,ifluatex}
\ifnum 0\ifxetex 1\fi\ifluatex 1\fi=0 % if pdftex
  \usepackage[T1]{fontenc}
  \usepackage[utf8]{inputenc}
  \usepackage{textcomp} % provide euro and other symbols
\else % if luatex or xetex
  \usepackage{unicode-math}
  \defaultfontfeatures{Scale=MatchLowercase}
  \defaultfontfeatures[\rmfamily]{Ligatures=TeX,Scale=1}
\fi
% Use upquote if available, for straight quotes in verbatim environments
\IfFileExists{upquote.sty}{\usepackage{upquote}}{}
\IfFileExists{microtype.sty}{% use microtype if available
  \usepackage[]{microtype}
  \UseMicrotypeSet[protrusion]{basicmath} % disable protrusion for tt fonts
}{}
\makeatletter
\@ifundefined{KOMAClassName}{% if non-KOMA class
  \IfFileExists{parskip.sty}{%
    \usepackage{parskip}
  }{% else
    \setlength{\parindent}{0pt}
    \setlength{\parskip}{6pt plus 2pt minus 1pt}}
}{% if KOMA class
  \KOMAoptions{parskip=half}}
\makeatother
\usepackage{xcolor}
\IfFileExists{xurl.sty}{\usepackage{xurl}}{} % add URL line breaks if available
\IfFileExists{bookmark.sty}{\usepackage{bookmark}}{\usepackage{hyperref}}
\hypersetup{
  pdftitle={RAF Technical Note: Processing Algorithms},
  pdfauthor={Author: NCAR Research Aviation Facility Science and Instrumentation Group},
  pdflang={en},
  hidelinks,
  pdfcreator={LaTeX via pandoc}}
\urlstyle{same} % disable monospaced font for URLs
\usepackage{longtable,booktabs,array}
\usepackage{calc} % for calculating minipage widths
% Correct order of tables after \paragraph or \subparagraph
\usepackage{etoolbox}
\makeatletter
\patchcmd\longtable{\par}{\if@noskipsec\mbox{}\fi\par}{}{}
\makeatother
% Allow footnotes in longtable head/foot
\IfFileExists{footnotehyper.sty}{\usepackage{footnotehyper}}{\usepackage{footnote}}
\makesavenoteenv{longtable}
\usepackage{graphicx}
\makeatletter
\def\maxwidth{\ifdim\Gin@nat@width>\linewidth\linewidth\else\Gin@nat@width\fi}
\def\maxheight{\ifdim\Gin@nat@height>\textheight\textheight\else\Gin@nat@height\fi}
\makeatother
% Scale images if necessary, so that they will not overflow the page
% margins by default, and it is still possible to overwrite the defaults
% using explicit options in \includegraphics[width, height, ...]{}
\setkeys{Gin}{width=\maxwidth,height=\maxheight,keepaspectratio}
% Set default figure placement to htbp
\makeatletter
\def\fps@figure{htbp}
\makeatother
\setlength{\emergencystretch}{3em} % prevent overfull lines
\providecommand{\tightlist}{%
  \setlength{\itemsep}{0pt}\setlength{\parskip}{0pt}}
\setcounter{secnumdepth}{-\maxdimen} % remove section numbering
\usepackage{booktabs}
\usepackage{longtable}
\usepackage{array}
\usepackage{multirow}
\usepackage{wrapfig}
\usepackage{float}
\usepackage{colortbl}
\usepackage{pdflscape}
\usepackage{tabu}
\usepackage{threeparttable}
\usepackage{threeparttablex}
\usepackage[normalem]{ulem}
\usepackage{makecell}
\usepackage{xcolor}
\ifxetex
  % Load polyglossia as late as possible: uses bidi with RTL langages (e.g. Hebrew, Arabic)
  \usepackage{polyglossia}
  \setmainlanguage[]{english}
  \setotherlanguage[]{greek}
\else
  \usepackage[greek,main=english]{babel}
% get rid of language-specific shorthands (see #6817):
\let\LanguageShortHands\languageshorthands
\def\languageshorthands#1{}
  \newcommand{\textgreek}[2][]{\foreignlanguage{greek}{#2}}
  \newenvironment{greek}[2][]{\begin{otherlanguage}{greek}}{\end{otherlanguage}}
\fi
\ifluatex
  \usepackage{selnolig}  % disable illegal ligatures
\fi

\title{RAF Technical Note: Processing Algorithms}
\author{Author: NCAR Research Aviation Facility Science and
Instrumentation Group}
\date{Date: June, 2022}

\begin{document}
\frontmatter
\maketitle

{
\setcounter{tocdepth}{1}
\tableofcontents
}
\mainmatter
\hypertarget{index}{%
\chapter*{Preface}\label{index}}
\addcontentsline{toc}{chapter}{Preface}

This site serves as the RAF Technical Note on processing algorithms that
are used for processing data collected on the RAF research aircraft. The
variables that appear in archived data are defined here, and the
algorithms used in their calculation are described. This document does
not describe the instrumentation used, but there are many links to such
descriptions. The goal of this document is to describe both current and
past algorithms, so it is expected that this document will be modified
often so that it can remain current.

\hypertarget{document-change-log}{%
\section{Document Change Log}\label{document-change-log}}

\begin{table}
\centering
\begin{tabular}{c|c|c|l}
\hline
Revision & GitHub\_Tag & PDF & Summary\\
\hline
June, 2022 (Current) & Release V2022.0 & V2022.0 PDF & Updates to constants based on CODATA published values (2018) (SB). Numerous updates to 3. The State of the Aircraft and 4. The State of the Atmosphere (SB, CW). Link to manuscript on Total Air Temperature (TAT) data quality analysis below Eq. 4.14 (JC). Correction to Eq. 7.9 to have UPRESS multiplied by 10 (MR). Other cosmetic changes have been made to the PDF-format version of June 2019, which was generated using the program LyX (WC).\\
\hline
June, 2019 & Release V2019.1 & V2019.1 PDF & License has been changed from GPL to BSD-3.\\
\hline
February, 2019 & Release V2019.0 & V2019.0 PDF & Original release.\\
\hline
\end{tabular}
\end{table}

\hypertarget{citation-information}{%
\section{Citation Information}\label{citation-information}}

When referencing this RAF Technical Note on processing algorithms,
please use the DOI 10.26023/zc3gpm25 - for example as a citation:
UCAR/NCAR - Earth Observing Laboratory (2022) RAF Technical Note:
Processing Algorithms. UCAR/NCAR - Earth Observing Laboratory.
\url{https://doi.org/10.26023/zc3gpm25} V2022.0 Retrieved enter date
here.

\hypertarget{introduction}{%
\chapter{Introduction}\label{introduction}}

\hypertarget{background-information}{%
\section{Background Information}\label{background-information}}

This technical report defines the variables used in data sets that are
collected by the research aircraft operated by the Research Aviation
Facility (RAF) of the National Center for Atmospheric Research. Where
appropriate, it also documents the equations that are used by the
processing software (currently ``nimbus'') to calculate the derived
measurements that result from the use of one or more other basic
measurements (e.g., potential temperature). Since 1993, data from
research flights have been archived in NetCDF format
(cf.~http://www.unidata.ucar.edu/software/netcdf/docs/), and the NetCDF
header for recent projects includes detailed information on the
measurements present in the file, how they depend on other measurements,
units, etc. The conventions that the RAF uses for NetCDF data files are
documented at http://www.eol.ucar.edu/raf/Software/netCDF.html .

This document should change as changes in processing algorithms are
implemented, but the intent is also to provide a history of algorithms
that have been used, so there is an effort to document how historical
data files were processed along with the descriptions of current
algorithms. Unlike some technical reports, this document is likely to
change over time and should provide a history extending back to RAF
Bulletin 9, which documented the processing algorithms as they existed
before about 2003.

Currently, the data acquisition process on the research aircraft of the
Research Aviation Facility, Earth Observing Laboratory, proceeds as
follows:

\begin{enumerate}
\def\labelenumi{\arabic{enumi}.}
\tightlist
\item
  Analog or digital outputs from instruments are sampled at regular
  intervals, typically 50 Hz when possible. Analog outputs are converted
  to digital values via analog-to-digital converters. The investigators'
  handbooks for each aircraft describe this process in detail, including
  resolution of the sampling and handling of the results. Often, signals
  from user-supplied instruments are also included in the measured
  values that are handled by the data system.\\
\item
  The digital outputs are then recorded by the data system on the
  aircraft. Currently, this is a task of the
  \href{http://www.eol.ucar.edu/data/software/nidas}{``NIDAS''} system
  described below. That system also controls the sampling, time stamps,
  and other aspects of data recording.\\
\item
  In flight, the data are processed by the \emph{``nimbus''} data
  processing program, which makes them available for display via aeros
  for real-time monitoring of measurements.\\
\item
  Following the flight, \emph{nimbus} again processes the data. At this
  stage, measurements can be re-sampled with averaging and/or
  interpolation to produce various data rates, usually 1 Hz or 25 Hz,
  and known lags in measurements can be introduced to adjust
  measurements to a common time basis. As part of this processing,
  \emph{nimbus} applies calibration coefficients where appropriate to
  convert recorded values (e.g., voltage) to engineering units (e.g.,
  \(^\circ\)C). Determining or checking these calibration coefficients
  is part of the pre-flight and post-flight procedures for each
  project.\\
\item
  The output from \emph{nimbus} is the data file that is the permanent
  archive from the experiment, often after merging in additional data
  sets from users that are not recorded in the original data file
  produced by \emph{NIDAS.} These files, in NetCDF format, have headers
  that contain metadata on each measurement (such as the calibration
  coefficients, the instrument that produced the measurement, etc.).
  Many of the variables in these files are discussed in this technical
  note, but the files may also include additional project-specific
  measurements for which the NetCDF header and the project reports will
  be the only documentation.
\end{enumerate}

For assistance accessing data from RAF-supported projects, contact the
RAF data management group via this email address.

The data system has changed several times over the history of RAF. For a
discussion of the history of the data systems, see
\href{www/ADSHistory.pdf}{this note} written by Richard Friesen. The
versions of data systems that produced most of the data still available
were, approximately, as given in the following table:

\begin{table}
\centering
\begin{tabular}{c|c|c|l}
\hline
Data.System & start & end & aircraft\\
\hline
ADS I & 1984 & 1992 & King Air 200T, Sabreliner (1987), Electra (1991)\\
\hline
ADS II & 1992 & 2007 & C-130\\
\hline
ADS III (NIDAS)\textasciicircum{}[ADS III is the name given to the full data system, which includes these components: NIDAS (for data acquisition and recording); NIMBUS (for data processing, both in flight and after the flight); AEROS (for data display in flight); and the Mission Coordinator Station and satellite communications system (for transmission of data to and from the aircraft, display of such data for mission decisions, and support for written “chat” communications among project participants both on the aircraft and on the ground).] & 2005 &  & GV, C-130 after 2007\\
\hline
\end{tabular}
\end{table}

Before 1993, data were processed by a different program, ``GENPRO,'' and
a different output format (also named GENPRO) was used for archived
datasets. Appendix E in
\href{https://opensky.ucar.edu/islandora/object/archives\%3A8729}{RAF
Bulletin 9}, the previous description of RAF data products that is now
superseded by this technical note, describes that format. Some variable
names in this document, esp.~in section @ref(obsolete-variables), refer
to obsolete variable names, some used with GENPRO and others referring
to instruments that are now retired. These names are included here so
that this report can be a reference for older archived data as well as
for current data files.

\hypertarget{alphabetical-list-of-variables}{%
\section{Alphabetical List of
Variables}\label{alphabetical-list-of-variables}}

At the end of this document, there is a list of all the variable names
that appear in standard data files along with links to the primary
discussion of those variables; see
\href{./appendix-b-variable-names.html\#variable-names}{Appendix B}. In
some cases redundant measurements are present, often for key
measurements like pressure or temperature. When these are used in
subsequent calculation of derived variables like potential temperature,
some choice is usually made regarding which measurement is considered
most reliable for a particular project or flight, and a single derived
variable is calculated on the basis of the chosen input variable(s). To
record which measurements were so designated, a reference measurement
chosen from a group of redundant measurements usually has a variable
name ending with the letter(s) X or XC.\footnote{Some that do not follow
  this convention are ATTACK and SSLIP; see the individual descriptions
  that follow.} To see the variables in a particular netCDF data file,
use the command ``ncdump -h file.nc''.

\hypertarget{constants-and-symbols}{%
\section{Constants and Symbols}\label{constants-and-symbols}}

The following table contains values used for some constants in this
document. For reference, the symbols used here and elsewhere in this
document are defined in the List of Symbols near the end of the document
(cf.~\href{./appendix-a-list-of-symbols.html\#list-of-symbols}{Appendix
A}). Where references are to the ``NIST Chemistry WebBook'', the
associated URL is http://webbook.nist.gov. References to the CODATA
Internationally recommended values of the Fundamental Physical Constants
are available at https://physics.nist.gov/cuu/Constants/index.html. The
optimization involved in adjustment of these coefficients is documented
in \href{https://aip.scitation.org/doi/10.1063/5.0064853}{E. Tiesinga,
P. J. Mohr, D. B. Newell, and B. N. Taylor, J. Phys. Chem. Ref. Data 50,
033105 (2021)}. In this technical note, references to these symbols will
often have these symbols or definitions marked by the symbol
\(^\dagger\) to indicate that the values used are the standard ones in
the following table.\footnote{This table does not account for the 2017
  revision of the basic SI units. The changes are not significant, but
  eventually processing should account for them. New values: Avogadro
  constant: \(N_A=6.02214076\times 10^{23}\) particles per mole;
  Boltzmann constant: \(k=1.380649 × 10^{-23}\) joules/kelvin;
  \(R_0=8.314463\times 10^{3}\) J kmol\(^{-1}\)K\(^{-1}\);
  Stefan-Boltzmann constant
  \(5.670374419... x 10^{-8} W m^{-2} K^{-4}\). These definitions are
  exact. Some other constants in the table below will have minor
  associated changes related to new reference values for basic units
  including the kilogram, meter, and kelvin. See
  \url{https://www.nist.gov/si-redefinition/definitions-si-base-units}
  for the 2017 values of the SI units.}

\begin{center}\rule{0.5\linewidth}{0.5pt}\end{center}

\textbf{\protect\hypertarget{constants-table}{}{Values Used for
``Constants''}}

\(T_{0}\) = temperature in kelvin corresponding to \(0^{\circ}\)C =
273.15~K\\
\(T_{3}\) = temperature corresponding to the triple point of
water\index{triple point of water} = 273.16~K\\
\(M_{d}\) = molecular weight of dry
air\(^{(a)}\)\index{molecular weight!dry air}, 28.9637
kg\(\,\)kmol\(^{-1}\)~~\(^{(c)}\)\\
\(M_{w}\) = molecular weight of water\index{molecular weight!water},
18.0153 kg~kmol\(^{-1}\)~~\(^{(d)}\)\footnote{}\\
\(R_{0}\) = universal gas
constant\(^{a}\)\index{universal gas constant}\index{gas constant!universal}
= 8.314462618\(\times 10^{3}\) J~kmol\(^{-1}\)K\(^{-1}\)~~
\(^{(e)}\)\footnote{\label{fn:2006-CODATA}}\\
\(N_{A}\) = Avogadro constant = 6.02214076\(\times 10^{26}\)
molecules~kmol\(^{-1}\)\index{Avogadro constant}\\
\(k=R_{0}/(\mathrm{N_{A})}=1.380649\times10^{-23}\mathrm{J}\,\mathrm{K}^{-1}\)\index{Boltzmann constant}\\
\(R_{d}=(R_{0}/M_{d}\)) = gas constant for dry
air\index{gas constant!dry air}\\
\(R_{w}\) = (\(R_{0}/M_{w})\) = gas constant for water
vapor\index{gas constant!water vapor}\\
\(R_{E}\) = radius of the Earth\index{radius of the Earth} =
6.371229\(\times 10^{6}\)m~~\(^{(f)}\)\footnote{}\\
\(c_{p}\) = specific heat of dry air at constant
pressure\index{specific heat!dry air!constant pressure} =
\(\frac{7}{2}R_{d}\) = 1.00473\(\times10^{3}\)
J~kg\(^{-1}\)K\(^{-1}\)~~\(^{(g)}\)\footnote{}\\
\(c_{v}\) = specific heat of dry air at constant
volume\index{specific heat!dry air!constant volume} =
\(\frac{5}{2}R_{d}\) = 0.71766\(\times 10^{3}\)
J~kg\(^{-1}\)K\(^{-1}\)\\
\hspace*{0.333em}\hspace*{0.333em}\hspace*{0.333em}\hspace*{0.333em}\hspace*{0.333em}\emph{(specific
heat values are at 0\(^{\circ}\)C; small variations with temperature are
not included here)}\\
\(\gamma\) = ratio of specific
heats\index{specific heat ratio, dry air}, \(c_{p}/c_{v}\), taken to be
1.4 (dimensionless) for dry air\\
\(\Omega\) = angular rotation rate of the
Earth\index{Earth, angular rotation rate} = 7.292115\(\times10^{-5}\)
radians/s\\
\(\Omega_{Sch}\) = angular frequency of the Schuler
oscillation\index{Schuler oscillation} = \(\sqrt{\frac{g}{R_{E}}}\)\\
\(\sigma\) = Stephan-Boltzmann
Constant\index{Stephan-Boltzmann Constant} =
5.670374419\(\times10^{-8}\)W~m\(^{-2}\)K\(^{-4}\)~~\(^{(e)}\)

\_\_\_\_\_\_\_\_\_\_ \textsuperscript{(a)} The International Standard
Atmosphere specifies \(g=9.80665\) m s\textsuperscript{-2},
\(M_{w}\)=28.9644 and \(R_{0}\) = 8.31432x10\textsuperscript{3} J
kmol\textsuperscript{-1}K\textsuperscript{-1}, so these values are used
to calculate pressure altitude.\\
\textsuperscript{(b)} cf.~Moritz, H., 1988: Geodetic Reference System
1980, Bulletin Geodesique, Vol. 62 , no. 3, and
\href{http://earth-info.nga.mil/GandG/publications/tr8350.2/wgs84fin.pdf}{this
link}.\\
\textsuperscript{(c)} Jones, F. E., 1978: J. Res. Natl. Bur. Stand.,
83(5), 419, as quoted by Lemmon, E. W., R. T. Jacobsen, S. G.
Penoncello, and D. G., Friend, J. Phys. Chem. Ref. Data, Vol. 29, No.~3,
2000, pp.~331-385. The quoted values of mole fraction are from Jones
(1978) and the calculation of mean molecular weight is shown in the
following table using values of molecular weights taken from the NIST
Standard Reference Database 69: NIST Chemistry WebBook as of March 2011.
With CO\textsubscript{2} about 0.00041 (as in 2020) and others decreased
proportionately, the mean is 28.9640.

Calculation of the mean molecular weight of air

Gas

mole fraction x

molecular weight M

x*M

N\textsubscript{2}

0.78102

28.0134

21.87903

O\textsubscript{2}

0.20946

31.9988

6.70247

Ar

0.00916

39.948

0.36592

CO\textsubscript{2}

0.00033

44.0095

0.01452

~~~~~~~~~~The mean value is 28.9619, but see above for a more modern
adjustment. \textsuperscript{(d)} NIST Standard Reference Database 69:
NIST Chemistry WebBook as of March 2011. \textsuperscript{(e)} 2018
CODATA\\
\textsuperscript{(f)} matching the value used by the inertial reference
systems, as discussed in the ``INS'' section.\\
\textsuperscript{(g)} The specific heat of dry air at 1013 hPa and
250--280 K as given by Lemmon et al.~(2000) is 29.13 J/(mol-K), which
translates to \(1005.8\pm 0.3\) J/(kg-K). However, the uncertainty limit
associated with values of specific heat is quoted as 1\%, and the
experimental data cited in that paper show scatter that is at least
comparable to several tenths percent, so the ideal-gas value cited here
is well within the range of uncertainty. For this reason, and because
this value is in widespread use, the ideal-gas value is used throughout
the algorithms described here.

\begin{center}\rule{0.5\linewidth}{0.5pt}\end{center}

\hypertarget{general-information-about-data-files}{%
\chapter{General Information About Data
Files}\label{general-information-about-data-files}}

\hypertarget{units-and-abbreviations}{%
\section{Units and Abbreviations}\label{units-and-abbreviations}}

This report uses the SI system of units, but with many exceptions. Among
them are the following:

\begin{enumerate}
\def\labelenumi{\arabic{enumi}.}
\tightlist
\item
  The millibar (mb), equal to one hectopascal (hPa), was used for
  pressure with some older variables.\\
\item
  Many variables are presented in the units most often used for that
  variable, even when they involve CGS units or mixed CGS-MKS units, as
  for example {[}g m\textsuperscript{-3}{]} for liquid water content or
  {[}cm\textsuperscript{-3}{]} for droplet concentration.\\
\item
  Flow rates are often quoted in liters per minute (LPM) or standard
  liters per minute (SLPM) because those terms are linked to properties
  of commercially available instruments with flow control. One liter is
  10\textsuperscript{-3}m\textsuperscript{3}. Standard temperature and
  pressure are respectively 273.15~K~and 1013.25~hPa. However, there is
  considerable ambiguity in the definition of ``standard'' conditions
  (mostly regarding the choice of the reference temperature) because
  some flow controllers and flowmeters specify a different ``standard''
  temperature, so the particular usage will be documented when this term
  is used. Mass flow meters provide a measure of the flow of mass but
  usually report the measurement in terms of the volume flow that would
  be present under standard conditions (e.g.., SLPM). Therefore, to
  convert to volumetric flow at other conditions, if the fluid density
  is \(\rho\) and the mass flow rate in units of mass per time is
  denoted by \(\dot{m}^\prime\), the volumetric flow is
  \(Q=\dot{m}^\prime/\rho\). Then the mass flow rate in units of
  standard volume per time is \(\dot{m}=\dot{m}^\prime/\rho_s\) where
  \(\rho_s\) is the density of the fluid under standard conditions. To
  convert to volumetric flow under other conditions,
  \(Q(p,T)=\dot{m}^\prime/\rho\) = \(\dot{m}\rho_s/\rho\) =
  \(\dot{m}p_{s}T/(p\thinspace T_{s})\) where \(p\) and \(T\) are the
  pressure and absolute temperature for the desired measurement and
  \(p_s\) amd \(T_s\) are the corresponding values for standard
  conditions.\\
\item
  The International Bureau of Weights and Measures recommends against
  use of units like percent or parts per million, but these are in
  common use in atmospheric chemistry and elsewhere so data files
  continue to use those units for relative humidity or the concentration
  of chemical species. Proper SI units for a volumetric mixing ratio
  would be, e.g., \(\mu\)mol~mol\textsuperscript{-1},
  nmol~mol\textsuperscript{-1}, or pmol~mol\textsuperscript{-1}, but
  variables are instead often assigned the respective units of ppmv,
  ppbv or pptv for parts per million, billion or trillion by volume.
  Care must be taken to interpret ppbv especially, because ``billion''
  has different meaning in different languages and different countries;
  herein, 1 ppbv means a volumetric ratio of 1:10\textsuperscript{9}.
  Many measurements produce native results in terms of a mass ratio,
  often described as a mixing ratio \(r_m\) that specifies the mass of
  the measured gas per unit mass of ``air'' (where the mass of the
  ``air'' does not include the variable constituents, usually only
  significant for water vapor). The ideal gas law relates the density
  ratio of two gases \((\rho_1:\rho_2)\) to the ratio of their partial
  pressures \((p_1:p_2)\) or number densities \((n_1:n_2)\) as
  follows:\\
  \begin{align}
  r_{m}=\frac{\rho_{1}}{\rho_{2}}=\frac{p_{1}M_{1}}{p_{2}M_{2}}=\frac{n_{1}M_{1}}{n_{2}M_{2}}
  (\#eq:rm)
  \end{align} where \(M_1\) and \(M_2\) are respective molecular weights
  for the two gases. The ratio of number densities or, equivalently,
  partial pressures, denoted here as \(r_v\) because it is also the
  volumetric mixing ratio, is related to the mass mixing ratio as
  follows:\\
  \begin{equation}
  r_{v}=\frac{n_{1}}{n_{2}}=\left(\frac{M_{2}}{M_{1}}\right)r_{m}
  (\#eq:rv)
  \end{equation} When concentrations are recorded with units of
  ``ppmv'', ``ppbv'' or ``pptv'', these units refer respectively to
  \(10^6r_v\), \(10^9r_v\), and \(10^{12}r_v\) with \(r_v\) given by the
  above equation.\\
\item
  The unit ``hertz'' (abbreviation Hz) is the proper unit for a periodic
  sampling frequency and will be used here in place of the more awkward
  ``samples per second.''
\item
  In some cases, particularly for older data files, speed has been
  recorded in units of knots (= 0.514444 m/s) and distance in nautical
  miles {≡} 1852 m).
\end{enumerate}

In Appendix A there is a list of symbols.\footnote{Some symbols used
  only once and defined where they are used are omitted from this list}
The next table defines some abbreviations and additional symbols used
for units in this report, in addition to the standard abbreviations for
the mks system of units:

\begin{table}
\centering
\begin{tabular}{c|c}
\hline
abbreviation or symbol & definition\$\textasciicircum{}a\$\\
\hline
º & degree, angle measurement ≡ \$\textbackslash{}pi/180\$\\
\hline
ft & foot ≡ 0.3048 m\\
\hline
mb & millibar ≡ 100 Pa ≡ 1 hPa\\
\hline
ppmv & parts per million by volume (see item 4 above)\\
\hline
ppbv & parts per billion (\$10\textasciicircum{}9\$) by volume (see item 4 above)\\
\hline
pptv & parts per trillion by volume (see item 4 above)\\
\hline
n mi & nautical mile ≡1852 m\\
\hline
kt & knot (n mi/hour) ≡ (1852/3600) m/s = 0.514444… m/s\\
\hline
\multicolumn{2}{l}{\textsuperscript{a} where ≡ is used, the relation is exact by definition}\\
\end{tabular}
\end{table}

\hypertarget{variables-used-to-denote-time}{%
\section{Variables Used to Denote
Time}\label{variables-used-to-denote-time}}

Although there are some exceptions in old archived data files, the data
in all modern output files are referenced to Coordinated Universal Time
(UTC). The time and date of the data acquisition system are synchronized
to time from the Global Positioning System (GPS) at the beginning of
each flight, and for data acquired by the present ADS-3 (NIDAS) data
acquisition system time is synchronized continuously with the GPS time.
Time variables vary for older archived data files; some of the following
are obsolete, but are included here for reference because they are
important to those wanting to use those archives.

\hypertarget{time}{%
\subsubsection*{Time (s): Time}\label{time}}
\addcontentsline{toc}{subsubsection}{Time (s): Time}

The reference-time counter for the output data files, used by data
system versions beginning with ADS-3. It is an integer output at 1 Hz
and has an initial value of zero at the start of the flight. Add this to
the ``Time:units'' attribute found in the NETCDF header section to
obtain the UTC time.

Example attribute: ~ Time:units = ``seconds since 2006-04-26 12:55:00
+0000'' ;

For code examples that show how to use ``Time'' see:
http://www.eol.ucar.edu/raf/Software/TimeExamp.html

\hypertarget{base-time}{%
\subsubsection*{Reference Start Time (s): base\_time (Obsolete; versions
before ADS-3 only)}\label{base-time}}
\addcontentsline{toc}{subsubsection}{Reference Start Time (s):
base\_time (Obsolete; versions before ADS-3 only)}

The reference time for the netCDF output data files for data system
versions before ADS-3. It represents the time of the first data record.
Its format is Unix time (elapsed seconds after midnight 1 January 1970).
Add time\_offset (below) to obtain the time for each data record. (Note:
base\_time is a single scalar, not a ``record'' variable, so it occurs
just once in the output file.)

\hypertarget{time-offset}{%
\subsubsection*{Time Offset from Reference Start Time (s): time\_offset
(Obsolete)}\label{time-offset}}
\addcontentsline{toc}{subsubsection}{Time Offset from Reference Start
Time (s): time\_offset (Obsolete)}

\emph{The time offset from base\_time of each data record used for the
NETCDF output files produced by data system versions before ADS-3.} It
starts at zero (0) and increments each second, so it can also be thought
of as a record counter. Use this measurement and add base\_time to
obtain the time for each data record.

\hypertarget{hms}{%
\subsubsection*{\texorpdfstring{Raw Tape Time (h, min, s): HOUR, MINUTE,
SECOND
(\emph{Obsolete})}{Raw Tape Time (h, min, s): HOUR, MINUTE, SECOND (Obsolete)}}\label{hms}}
\addcontentsline{toc}{subsubsection}{Raw Tape Time (h, min, s): HOUR,
MINUTE, SECOND (\emph{Obsolete})}

These three time variables are recorded directly from the aircraft's
data system. Since ADS-3, this information is replaced by the ``Time''
variable and the ``Time:units'' attribute of that variable.

\hypertarget{mdy}{%
\subsubsection*{\texorpdfstring{Date (m, d, y): MONTH, DAY, YEAR
(\emph{Obsolete})}{Date (m, d, y): MONTH, DAY, YEAR (Obsolete)}}\label{mdy}}
\addcontentsline{toc}{subsubsection}{Date (m, d, y): MONTH, DAY, YEAR
(\emph{Obsolete})}

These three variables represent the date when the aircraft's data system
began recording data. They are repeated as 1 Hz variables but are NOT
incremented if the time rolls over to the next day. Use base\_time and
time\_offset for reference timing. Since ADS-3, this information is
replaced by the ``Time'' variable and the ``Time:units'' attribute of
that variable.

\hypertarget{synchronization-of-measurements}{%
\section{Synchronization of
Measurements}\label{synchronization-of-measurements}}

Measurements sampled under control of the ``NIDAS'' sampling system are
acquired at 50 Hz. However, the standard archive files are produced at a
rate of 1 Hz, and each sample is the average of 50 samples. Therefore,
the time associated with measurements reported at 1~Hz is actually an
average over the specified second, so the reference time for the
averaged measurement is actually 0.5~s past the reported time. Analogous
offsets apply to variables reported at other rates different from 50~Hz.
Where it applies, electronic filters with cutoff frequency of 25 Hz are
used with analog measurements. Higher-rate files are sometimes produced,
standardized to 25 Hz but sometimes at other frequencies.

There are time shifts inherent in many of the measurements, and in some
cases (e.g., those produced by inertial reference units) these time
shifts arise because the information is transmitted from the measuring
system at a time later than when it was sampled. In these cases, shifts
(``lags'') are applied to the measurements. The lags may be either
static or dynamic. Static lags are specified in a configuration file,
saved for each project; dynamic lags provided as part of data sampling
by specific instruments are recorded by NIDAS for use in processing.
Dynamic lags are usually a difference in time from a gridded time value
to the time it was actually acquired. e.g.~for a 5-Hz parameter the
expected or gridded millisecond offset into each second would be 0, 200,
400, 600, and 800. If the data actually were sampled or acquired at 50,
250, 450, 650, and 850 ms then the dynamic lag for this particular
second would be -50 ms. Corrections for time lags are applied to
measurements before conversion to one of the standard data rates.

\protect\hypertarget{elecFilter}{}{Where data rates for particular
measurements do not match the basic 50 Hz sampling rate, linear
interpolation is used to obtain higher-rate values. For 1 Hz data files,
measurements are then averaged within each second. For 25 Hz files, 50
Hz measurements are digitally filtered using a finite impulse response
(FIR) filter, while data acquired at less than 25 Hz are linearly
interpolated to 25 Hz and then FIR-filtered for smoothing.}

\hypertarget{other-comments-on-terminology}{%
\section{Other Comments on
Terminology}\label{other-comments-on-terminology}}

\hypertarget{variable-names-in-equations}{%
\subsection{Variable Names in
Equations}\label{variable-names-in-equations}}

This report often uses variable names in equations, and sometimes there
is potential for confusion because the variable names consist of
multiple characters. In most cases, to denote that the variable name is
the variable in the equation (as opposed to each of the letters in the
variable name representing quantities to be multiplied together), the
variable name has been enclosed in brackets, as in \{TASX\}. In
addition, variable names are displayed with upright Roman character
sets, while other symbols in equations are shown using slanted (script)
character sets as is conventional for mathematical equations. In cases
where code segments (usually expressed in C code) are included to
document how calculations are performed, monospaced character sets
indicate that the segment is a representation of how the processing
could be coded. Such a code segment is not always a direct copy of the
code in use, but such code is sometimes the most convenient way to
express the algorithm in use.

\hypertarget{distinction-between-original-measurements-and-derived-variables}{%
\subsection{Distinction Between Original Measurements and Derived
Variables}\label{distinction-between-original-measurements-and-derived-variables}}

Many of the variables in the data files and in this report are derived
from combinations of measurements. The terms ``raw'' or ``original''
measurement are sometimes used for a minimally processed output received
directly from a sensor or instrument. Such measurements may be converted
to engineering units via calibration coefficients, but otherwise they
are a direct representation of the output from a sensor.\footnote{Calibration
  coefficients, e.g.~those used to convert from voltage output from an
  analog sensor to a measured quantity with physical units like {º}C,
  are not included or discussed in this report. They are normally
  included in project reports and, in recent years, many are included in
  the header of the NETCDF file.} In contrast, derived variables (e.g.,
potential temperature) depend on one or more ``raw'' measurements and
are not direct results of output from an instrument. For most derived
measurements, a box that follows an introductory comment is used in this
report to document the processing algorithm. The box has two parts; in
the top are definitions used and explanations regarding variables that
enter the calculation, while the bottom portion contains the equation,
algorithm, or code segment that documents how the variable is
calculated.

\hypertarget{dimensions-in-equations}{%
\subsection{Dimensions in Equations}\label{dimensions-in-equations}}

An effort has been made to avoid dimensions in equations except where it
would be awkward otherwise. Some scale factors are introduced for only
this purpose (e.g., to avoid dimensions in arguments to logarithmic or
exponential functions), and some effort was made to isolate dimensions
to defined constants rather than requiring that variables in equations
be used with specific units. However, some exceptions remain to be
consistent with historical usage.

\hypertarget{the-state-of-the-aircraft}{%
\chapter{The State of the Aircraft}\label{the-state-of-the-aircraft}}

The primary sources of information on the location and motion of the
aircraft are inertial navigation systems and global positioning systems.
Both are described in this section, and combined results that merge the
best features of each into composite variables for location and motion
are also discussed. Useful references for material in this section are
Lenschow (1972) and RAF Bulletin 23.

\hypertarget{inertial-reference-systems}{%
\section{Inertial Reference Systems}\label{inertial-reference-systems}}

An Inertial Navigation System (INS) or Inertial Reference Unit (IRU)
provides measurements of aircraft position, velocity relative to the
Earth, acceleration and attitude or orientation. The IRU provides basic
measurements of acceleration and angular rotation rate, while the INS
integrates those measurements to track the position, altitude, velocity,
and orientation of the aircraft. For the GV, the system is a Honeywell
Laseref IV HG2001 GD03 Inertial Reference System; for the C-130, it is a
Honeywell Model HG1095-AC03 Laseref V SM Inertial Reference System.
These systems are described on the EOL web site, at this URL. Data from
the IRS come via a serial digital bit stream (the ARINC digital bus) to
the ADS (Aircraft Data System). Because there is some delay in
transmission and recording of these variables, adjustments for this
delay are made when the measurements are merged into the processed data
files, as documented in the NetCDF header files and as discussed in
Section @ref(synchronization-of-measurements). Typical delays are about
80 ms for variables including ACINS, PITCH, ROLL, and THDG.

Some variables are recorded only on the original ``raw'' data files and
are not usually included in final archived data files; these are
discussed at the end of this subsection. See also the discussion in
Section @ref(obsolete-variables) for information on results from
inertial systems that were used prior to installation of the present
Honeywell systems.

An Inertial Navigation System ``aligns'' while the aircraft is
stationary by measurement of the variations in its reference frame
caused by the rotation of the Earth. Small inaccuracy in that alignment
leads to a ``Schuler oscillation'' that produces oscillatory errors in
position and other measurements, with a period \(\tau_{Sch}\) of about
84 minutes (\(\tau_{Sch}=2\pi\sqrt{R_{E}/g}\)). Position errors of less
than \(1.0\,\)n mi/h are within normal operating specifications. See
Section @ref(combining-irs-and-gps-measurements) for discussion of
additional variables, similar to the following, for which corrections
are made for these errors via reference to data from a Global
Positioning System.

Some projects have used smaller Systron Donner C-MIGITS Inertial
Navigation Systems with GPS coupling, usually in connection with special
instruments like a wing-mounted wind-sensing system. For these units,
variable names usually begin with the letter C but otherwise have names
matching the following variables (e.g., CLAT). GPS coupling via a Kalman
filter is incorporated in the measurements from these units. They are
described at this web address.

Uncertainties associated with measurements from the IRS are discussed in
a Technical Note, available at this URL. See page 7 of that document and
the tables on pages 41 and 49.

\hypertarget{standard-variables}{%
\subsection{Standard Variables}\label{standard-variables}}

\hypertarget{latitude}{%
\subsubsection*{\texorpdfstring{Latitude ({º}):
LAT}{Latitude (º): LAT}}\label{latitude}}
\addcontentsline{toc}{subsubsection}{Latitude ({º}): LAT}

\emph{The aircraft latitude or angular distance north of the equator in
an Earth reference frame.} Positive values are north of the equator;
negative values are south. The resolution is 0.00017\(^{\circ}\) and the
accuracy is reported by the manufacturer to be 0.164\(^{\circ}\) after 6
h of flight. Values are provided by the INS at a frequency of 10 Hz.

\hypertarget{longitude}{%
\subsubsection*{\texorpdfstring{Longitude ({º}):
LON}{Longitude (º): LON}}\label{longitude}}
\addcontentsline{toc}{subsubsection}{Longitude ({º}): LON}

\emph{The aircraft longitude or angular distance east of the prime
meridian in an Earth reference frame.} Positive values are east of the
prime meridian; negative values are west. The resolution is
0.00017\(^{\circ}\) and the accuracy is reported by the manufacturer to
be 0.164\(^{\circ}\) after 6 h of flight. Values are provided by the INS
at a frequency of 10 Hz.

\hypertarget{thdg}{%
\subsubsection*{\texorpdfstring{Aircraft True Heading (\(^{\circ}\):
THDG}{Aircraft True Heading (\^{}\{\textbackslash circ\}: THDG}}\label{thdg}}
\addcontentsline{toc}{subsubsection}{Aircraft True Heading
(\(^{\circ}\): THDG}

\emph{The azimuthal angle between the center-line of the aircraft
(pointing ahead, toward the nose) and a line of meridian.} This
azimuthal angle is measured in a polar coordinate system oriented
relative to the Earth with polar axis upward and azimuthal angle
measured relative to true north. The heading thus indicates the
orientation of the aircraft, not necessarily the direction in which the
aircraft is traveling. The resolution is 0.00017\(^{\circ}\) and the
uncertainty is quoted by the manufacturer as 0.2\(^{\circ}\) after 6 h
of flight. Values are provided by the INS at a frequency of 25 Hz.
``True'' distinguishes the heading from the magnetic heading, the
heading that would be measured by a magnetic compass. For more
information on the coordinate system used, see RAF Bulletin 23.

\hypertarget{pitch}{%
\subsubsection*{\texorpdfstring{Aircraft Pitch Attitude Angle
(\(^{\circ}\)):
PITCH}{Aircraft Pitch Attitude Angle (\^{}\{\textbackslash circ\}): PITCH}}\label{pitch}}
\addcontentsline{toc}{subsubsection}{Aircraft Pitch Attitude Angle
(\(^{\circ}\)): PITCH}

\emph{The angle between the center-line of the aircraft (pointing ahead,
toward the nose) and the horizontal plane in a reference frame relative
to the Earth with polar axis upward.} Positive values correspond to the
nose of the aircraft pointing above the horizon. The resolution is
0.00017\(^{\circ}\) and the uncertainty is quoted by the manufacturer as
0.05\(^{\circ}\) after 6 h of flight. Values are provided by the INS at
a frequency of 50 Hz.

\hypertarget{roll}{%
\subsubsection*{\texorpdfstring{Aircraft Roll Attitude Angle
(\(^{\circ}\)):
ROLL}{Aircraft Roll Attitude Angle (\^{}\{\textbackslash circ\}): ROLL}}\label{roll}}
\addcontentsline{toc}{subsubsection}{Aircraft Roll Attitude Angle
(\(^{\circ}\)): ROLL}

\emph{The angle of rotation about the longitudinal axis of the aircraft
required to bring the lateral axis (along the wings) to the horizontal
plane.} Positive angles indicate that the starboard (right) wing is down
((i.e., a clockwise rotation has occurred from level when facing forward
in the aircraft). The resolution is 0.00017\(^{\circ}\) and the
uncertainty is quoted by the manufacturer as 0.05\(^{\circ}\) after 6 h
of flight. Values are provided by the INS at a frequency of 50 Hz.

\hypertarget{ACINS}{%
\subsubsection*{\texorpdfstring{Aircraft Vertical Acceleration
(m s\textsuperscript{-2}):
ACINS}{Aircraft Vertical Acceleration (m s-2): ACINS}}\label{ACINS}}
\addcontentsline{toc}{subsubsection}{Aircraft Vertical Acceleration
(m s\textsuperscript{-2}): ACINS}

\emph{The acceleration upward (relative to the Earth) as measured by an
inertial reference unit.} With INSs now in use, the internal drift that
arises when this measurement is integrated to get aircraft vertical
speed and then altitude is removed by the INS via pressure damping
through reference to the pressure altitude.\footnote{For earlier
  projects using the Litton LTN-51 INS, this is a direct measurement
  without adjustment for changes in gravity during flight and without
  pressure-damping. Previous use employed a baro-inertial loop to
  compensate for drift in the integrated measurement. See the discussion
  of WP3 below.} Positive values are upward. The sample rate is 50 Hz
and the resolution is 0.0024 m s\(^{-2}\).

\hypertarget{vspd}{%
\subsubsection*{Computed Aircraft Vertical Velocity (m/s):
VSPD}\label{vspd}}
\addcontentsline{toc}{subsubsection}{Computed Aircraft Vertical Velocity
(m/s): VSPD}

\emph{The upward velocity of the aircraft, or rate-of-climb relative to
the Earth, as measured by the INS.} VSPD is determined within the INS by
integration of the vertical acceleration, with damping based on measured
pressure to correct for accumulated errors in the integration of
acceleration. The sample rate is 50 Hz with a resolution of 0.00016 m/s.
The Honeywell Laseref INS employs a baro-inertial loop, similar to that
described below for WP3 and the Litton LTN-51, to update the value of
the acceleration. This variable is also filtered within the INS so that
there is little variance with frequency higher than 0.1 Hz.

\hypertarget{roc}{%
\subsubsection*{Aircraft Rate of Climb (m/s): ROC (new
2017)}\label{roc}}
\addcontentsline{toc}{subsubsection}{Aircraft Rate of Climb (m/s): ROC
(new 2017)}

\emph{The rate of climb or upward speed of the aircraft,} as measured by
the INS with correction so as to represent the derivative of the
geometric height. This variable is calculated by integration of the
variable ACINS and then addition of the low-pass-filtered difference
between that integral and the climb rate determined from the hydrostatic
equation. The result retains the high-frequency response from the INS
while matching the low-frequency average value determined from the
hydrostatic equation, and so represents change in geometric height. This
memo contains additional background information on this variable.

\(g\) = \label{-constant-g}acceleration of
gravity~~\(^{(a)}\)\footnote{ } at latitude \(\lambda\) and altitude
\(z\) above the WGS-84
\index{WGS-84 geoid}geoid,~~\(^{(b)}\)\footnote{}\\
\begin{align}
g(z,\lambda)=g_{e}\left(\frac{1+g_{1}\sin^{2}(\lambda)}{(1-g_{2}\sin^{2}\lambda)^{1/2}}\right)(1-(k_{1}-k_{2}\sin^{2}(\lambda))z+k_{3}z^{2})
(\#eq:gsublambda)
\end{align}\\
\hspace*{0.333em}\hspace*{0.333em}\hspace*{0.333em}\hspace*{0.333em}\hspace*{0.333em}\hspace*{0.333em}\hspace*{0.333em}where
\(g_{e}=9.780327\)~m~s\(^{-2}\), \(g_{1}=0.00193185\),
\(g_{2}=0.00669438\), and\\
\hspace*{0.333em}\hspace*{0.333em}\hspace*{0.333em}\hspace*{0.333em}\hspace*{0.333em}\hspace*{0.333em}\hspace*{0.333em}
\{\(k_{1},k_{2},k_{3}\)\} = \{3.15704\(\times 10^{-7}\mathrm{m}^{-1}\),
2.10269\(\times 10^{-9}\mathrm{m}^{-1}\),
7.37452\(\times 10^{-14}\mathrm{m}^{-2}\)\}

\(R_{d}{}^{\dagger}\) = gas constant for dry air\\
\(T_{K}\) = absolute temperature = (ATX + 273.15)\\
\(a\) = \protect\hyperlink{acins}{ACINS} = upward acceleration as
measured by the INS {[}m~s\textsuperscript{-2}{]}\\
\(p\) = \protect\hyperlink{psx}{PSXC} = measured ambient pressure
{[}hPa{]}\\
\(\Delta p\) = difference between current and last value of PSXC\\
\(\Delta t\) = time between samples (1/\(f\) where \(f\) is the sample
frequency)\\
\(F_{L}\) = low-pass Butterworth filter (cf.~p.~\pageref{compFilter}).

~1. From consecutive measurements of pressure, estimate the rate of
climb from the hydrostatic equation:\\
\begin{equation}
w_{p}=-\frac{R_{d}T_{k}}{gp}\frac{\Delta p}{\Delta t}
(\#eq:wphe)  
\end{equation}\\
\hspace*{0.333em}2. Add the current measurement of acceleration to the
cumulative sum: \begin{equation}
w_{p}^{*}\leftarrow w_{p}^{*}+a\Delta t
(\#eq:wpstar)
\end{equation}\\
\hspace*{0.333em}3. Define ROC as the sum of \(w_{p}^{*}\) and the
low-pass filtered value of (\(w_{p}-w_{p}^{*}\)):\\
\begin{align}
\mathrm{ROC} & =w_{p}^{*}+F_{L}(w_{p}-w_{p}^{*})
(\#eq:ROC)  
\end{align}

\hypertarget{wp3}{%
\subsubsection*{Pressure-Damped Aircraft Vertical Velocity (m/s):
WP3}\label{wp3}}
\addcontentsline{toc}{subsubsection}{Pressure-Damped Aircraft Vertical
Velocity (m/s): WP3}

\protect\hypertarget{WP3ux20algorithm}{}{}This was a derived variable
incorporating a third-order damping feedback loop to remove the drift
from the inertial system's vertical accelerometer (ACINS or VZI) using
pressure altitude (PALT) as a long-term, stable reference. Positive
values are up. The Honeywell INS now in use provides its own version of
this measurement, VSPD, and WP3 is now considered obsolete (and in any
case should not be calculated from ACINS as provided by the Honeywell
Laseref IRS because that ACINS already incorporates pressure damping).
Documentation is included here because many old data files include this
variable. Note that ``pressure altitude'' is not a true altitude but an
altitude equivalent to the ambient pressure in a standard atmosphere, so
updating a variable integrated from inertial measurements to this value
can introduce errors vs.~the true altitude. WP3 was calculated by the
data-processing software as follows (with coefficients in historical use
and not updated to the recommendations elsewhere in this technical
note):\footnote{Regarding signs, note that ACINS is a number near zero,
  not near g, and so already has the estimated acceleration of gravity
  removed. The assumption made in the following is that the INS will
  report values adjusted for the gravitational acceleration \emph{at the
  point of alignment}, which would be {\emph{G}\emph{L}}. If
  {\emph{g}\emph{F}}, the estimate for gravity at the flight altitude
  (palt) and latitude (lat), is \emph{smaller} than {\emph{G}\emph{L}}
  then the difference ({\emph{G}\emph{L} − \emph{g}\emph{f}}) will be
  positive; this will correct for the reference value for ACINS being
  the gravity measured at alignment ({\emph{G}\emph{L}}) when it should
  actually be the sensed gravity ({\emph{g}\emph{f}}) at the measurement
  point, so to obtain (sensed acceleration - {\emph{g}\emph{f}}) it is
  necessary to add ({\emph{G}\emph{L} − \emph{g}\emph{f}}) to ACINS,
  \emph{increasing} ``acz'' in this case. However, the situation with
  ``vcorac'' is reversed: ``vcorac'' is a positive term for all eastward
  flight, for example, but in that case the motion of the aircraft makes
  objects seem lighter (i.e., they experience less acceleration of
  gravity) than without such flight. ACINS is positive upward so it
  represents a net acceleration of the aircraft upward (as imposed by
  the combination of gravity and the lift force of the aircraft). To
  accomplish level flight in these circumstances, the aircraft must
  actually accelerate downward so the accelerometer will experience a
  negative excursion relative to slower flight. To compensate,
  ``vcorac'' must make a positive contribution to remove that negative
  excursion from ``acz''. In the conceptual extreme that the aircraft
  flies fast enough for the interior to appear weightless, ACINS would
  reduce to -1*{\emph{G}\emph{L}} and vcorac would increase to
  +{\emph{G}\emph{L}}, leaving acz near zero as required if the aircraft
  were to remain in level flight in the rotating frame.}

\(g_{1}\) = 9.780356 m~s\textsuperscript{-2}\\
\(a_{1}\) =
0.31391116\(\times 10\)\textsuperscript{-6}m\textsuperscript{-1}\\
\(a_{2}\) = .0052885 (dimensionless)\\
\protect\hyperlink{vew}{VEW} (\protect\hyperlink{vns}{VNS}) = eastward
(northward) groundspeed of the aircraft (see below)\\
\protect\hyperlink{latitude}{LAT} = latitude measured by the IRS
{[}º{]}\\
\(C_{dr}=\pi/180^{\circ}\) = conversion factor, degrees to radians\\
\protect\hyperlink{palt}{PALT} = pressure altitude of the aircraft\\
\(\Omega\) = angular rotation of the earth\(^{\dagger}\) =
7.292116\(\times10^{6}\) radians/s\\
\(R_{E}\) = radius of the Earth\(^{\dagger}\) = 6.371229\(\times10^{6}\)
m\\
\(g_{f}\) = local gravity corrected for latitude and altitude\\
\(V_{c}\) = correction to gravity for the motion of the aircraft\\
\(G_{L}\) = local gravity at the location of INS alignment, corrected to
zero altitude\\
\(\{C[0],C[1],C[2]\}\) = feedback coefficients, \{0.15, 0.0075,
0.000125\} for 125-s response

~1. From the pressure altitude PALT (in m) and the latitude LAT,
estimate the acceleration of gravity:\\
\begin{equation}
g_{f}=g_{1}\left(1+a_{2}\sin^{2}(\mathrm{C_{dr}\{LAT\})}+a_{1}\mathrm{\{PALT\}}\right)
(\#eq:gf)
\end{equation} ~2. Determine corrections for Coriolis
acceleration\index{Coriolis acceleration} and centrifugal
acceleration\index{centrifugal acceleration}:\\
\begin{equation}
a_{c}=2\Omega\mathrm{\{VEW\}}\cos(C_{r}\mathrm{\{LAT\}})+\frac{\mathrm{\{VEW\}}^{2}+\mathrm{\{VNS\}}^{2}}{R_{E}}
(\#eq:ac)
\end{equation} ~~~Estimate the acceleration \(a_{z}\) (code variable
`acz') experienced by the aircraft as follows:\\
\begin{equation}
\mathrm{\{acz\}}=a_{z}=\mathrm{\{ACINS\}}+G_{L}-g_{f}+a_{c}
(\#eq:acz)
\end{equation} ~~~Use a feedback loop to update the integrated value of
the acceleration.\\
\hspace*{0.333em}\hspace*{0.333em}\hspace*{0.333em}The following code
segment uses \emph{acz} to represent acceleration \(a_{z}\),\\
\hspace*{0.333em}\hspace*{0.333em}\hspace*{0.333em}``deltaT'' to
represent the time between updates, and ``hx'' and ``hxx'' to store the
feedback terms:\\
\hspace*{0.333em}\hspace*{0.333em}\hspace*{0.333em}\hspace*{0.333em}\hspace*{0.333em}wp3{[}FeedBack{]}
+= (acz - C{[}1{]} * hx{[}FeedBack{]} -- C{[}2{]} * hxx{[}FeedBack{]}) *
deltaT{[}FeedBack{]}\\
\hspace*{0.333em}3. Update the feedback terms (using ``hi3'' for
storage):\\
\hspace*{0.333em}\hspace*{0.333em}\hspace*{0.333em}\hspace*{0.333em}\hspace*{0.333em}hi3{[}FeedBack{]}
= hi3{[}FeedBack{]} + (wp3{[}FeedBack{]} -- C{[}0{]} * hx{[}FeedBack{]})
* deltaT{[}FeedBack{]};\\
\hspace*{0.333em}\hspace*{0.333em}\hspace*{0.333em}\hspace*{0.333em}\hspace*{0.333em}hx{[}FeedBack{]}
= hi3{[}FeedBack{]} - palt;\\
\hspace*{0.333em}\hspace*{0.333em}\hspace*{0.333em}\hspace*{0.333em}\hspace*{0.333em}hxx{[}FeedBack{]}
= hxx{[}FeedBack{]} + hx{[}FeedBack{]} * deltaT{[}FeedBack{]};\\
\hspace*{0.333em}4. Set WP3 to the average of the last wp3 result and
the current wp3 result.

\hypertarget{alt}{%
\subsubsection*{Inertial Altitude (m): ALT}\label{alt}}
\addcontentsline{toc}{subsubsection}{Inertial Altitude (m): ALT}

\emph{The altitude of the aircraft as provided by an INS}, with pressure
damping applied within the INS to the integrated aircraft vertical
velocity to avoid the accumulation of errors. The value therefore is
updated to the pressure altitude, not the geometric altitude, and should
be regarded as a measurement of pressure altitude that has short-term
variations as provided by the INS. The sample rate is 25 Hz with a
resolution of 0.038 m. In some projects ALT also referred to the
altitude from the avionics GPS system; the preferred and current
variable name for that is ALT\_G.

\hypertarget{gsf}{%
\subsubsection*{Aircraft Ground Speed (m/s): GSPD}\label{gsf}}
\addcontentsline{toc}{subsubsection}{Aircraft Ground Speed (m/s): GSPD}

\emph{The ground speed of the aircraft as provided by an INS.} The
resolution is 0.0020 m/s, and the INS provides this measurement at a
frequency of 10 Hz. Formerly GSF. Update to GSPD occurred in 2014.

\hypertarget{vew}{%
\subsubsection*{Aircraft Ground Speed East Component (m/s):
VEW}\label{vew}}
\addcontentsline{toc}{subsubsection}{Aircraft Ground Speed East
Component (m/s): VEW}

\emph{The east-directed component of ground speed as provided by an
INS.} The resolution is 0.0020 m/s, and the INS provides this
measurement at a frequency of 10 Hz.

\hypertarget{vns}{%
\subsubsection*{Aircraft Ground Speed North Component (m/s):
VNS}\label{vns}}
\addcontentsline{toc}{subsubsection}{Aircraft Ground Speed North
Component (m/s): VNS}

\emph{The north-directed component of ground speed as provided by an
INS.} The resolution is 0.0020 m/s, and the INS provides this
measurement at a frequency of 10 Hz.

\hypertarget{dei-dni}{%
\subsubsection*{Distance East/North of a Reference (km): DEI,
DNI}\label{dei-dni}}
\addcontentsline{toc}{subsubsection}{Distance East/North of a Reference
(km): DEI, DNI}

\emph{Distance east or north of a project-dependent reference point.}
These are derived outputs obtained by subtracting a fixed reference
position from the current position. The values are determined from
measurements of latitude and longitude and converted from degrees to
distance in a rectilinear coordinate system. The reference position can
be either the starting location of the flight or a user-defined
reference point (e.g., the location of a project radar). The accuracy of
these values is dependent on the accuracy of the source of latitude and
longitude measurements (see LAT and LON), and the calculations are only
appropriate for short distances because they do not take into account
the spherical geometry of the Earth.

\protect\hyperlink{longitude}{LON}\(_{ref}\) = reference longitude (º)\\
\protect\hyperlink{latitude}{LAT}\(_{ref}\) = reference latitude (º)\\
\(C_{deg2km}=\) conversion factor, degrees latitude to
km\sindex[con]{conversion factor, degrees latitude to km} \(\equiv\)
111.12 km / \(^{\circ}\)

\begin{equation}
\begin{split}
\mathrm{DEI} = & \mathrm{C_{deg2km}}(\mathrm{\{LON\}}-\{\mathrm{LON}_{ref}\})\cos(\mathrm{\{LAT\}}) \notag \\ 
\mathrm{DNI} = & \mathrm{C_{deg2km}}(\mathrm{\{LAT\}}-\{\mathrm{LAT}_{ref}\})  
\end{split}
(\#eq:DNSEW)
\end{equation}

\hypertarget{fxaxim}{%
\subsubsection*{Radial Azimuth/Distance from Fixed Reference FXAZIM,
FXDIST}\label{fxaxim}}
\addcontentsline{toc}{subsubsection}{Radial Azimuth/Distance from Fixed
Reference FXAZIM, FXDIST}

\emph{Azimuth and distance from a project-dependent reference point.}
The units of the azimuthal angle are degrees (relative to true north)
and the distance is in kilometers. These are calculated by
rectangular-to-polar conversion of DEI and DNI, described in the
preceding paragraph.

\hypertarget{special-use-irs}{%
\subsection{Additional Special-Use Variables}\label{special-use-irs}}

The following INS and IRU variables are not normally included in
archived data files, but their values are recorded by the ADS and can be
obtained from the original ``raw'' data files:

\hypertarget{blata}{%
\subsubsection*{\texorpdfstring{Raw Lateral Body Acceleration
(m/s\textsuperscript{2}):
BLATA}{Raw Lateral Body Acceleration (m/s2): BLATA}}\label{blata}}
\addcontentsline{toc}{subsubsection}{Raw Lateral Body Acceleration
(m/s\textsuperscript{2}): BLATA}

The raw output from the IRU lateral accelerometer. Positive values are
toward the starboard, normal to the aircraft center line. The sample
rate is 50 Hz with a resolution of 0.0024 m s\textsuperscript{-2}.

\hypertarget{blona}{%
\subsubsection*{\texorpdfstring{Raw Longitudinal Body Acceleration
(m/s\textsuperscript{2}):
BLONA}{Raw Longitudinal Body Acceleration (m/s2): BLONA}}\label{blona}}
\addcontentsline{toc}{subsubsection}{Raw Longitudinal Body Acceleration
(m/s\textsuperscript{2}): BLONA}

The raw output from the IRU longitudinal accelerometer. Positive values
are in the direction of the nose of the aircraft and parallel to the
aircraft center line. The sample rate is 50 Hz with a resolution of
0.0024 m s\textsuperscript{-2}.

\hypertarget{bnorma}{%
\subsubsection*{\texorpdfstring{Raw Normal Body Acceleration
(m/s\textsuperscript{2}):
BNORMA}{Raw Normal Body Acceleration (m/s2): BNORMA}}\label{bnorma}}
\addcontentsline{toc}{subsubsection}{Raw Normal Body Acceleration
(m/s\textsuperscript{2}): BNORMA}

The raw output from the IRU vertical accelerometer. Positive values are
upward in the reference frame of the aircraft, normal to the aircraft
center line and lateral axis. The sample rate is 50 Hz with a resolution
of 0.0024 m s\textsuperscript{-2}.

\hypertarget{bpitchr}{%
\subsubsection*{\texorpdfstring{Raw Body Pitch Rate ({º}/s):
BPITCHR}{Raw Body Pitch Rate (º/s): BPITCHR}}\label{bpitchr}}
\addcontentsline{toc}{subsubsection}{Raw Body Pitch Rate ({º}/s):
BPITCHR}

The raw output of the IRU pitch rate gyro. Positive values indicate the
nose moving upward and refer to rotation about the aircraft's lateral
axis. The sample rate is 50 Hz with a resolution of 0.0039{º}/s.

\hypertarget{brollr}{%
\subsubsection*{\texorpdfstring{Raw Body Roll Rate ({º}/s):
BROLLR}{Raw Body Roll Rate (º/s): BROLLR}}\label{brollr}}
\addcontentsline{toc}{subsubsection}{Raw Body Roll Rate ({º}/s): BROLLR}

The raw output of the IRU roll rate gyro. Positive values indicate
starboard wing moving down and refer to rotation about the aircraft
center line. The sample rate is 50 Hz with a resolution of 0.0039{º}/s.

\hypertarget{byawr}{%
\subsubsection*{\texorpdfstring{Raw Body Yaw Rate ({º}/s):
BYAWR}{Raw Body Yaw Rate (º/s): BYAWR}}\label{byawr}}
\addcontentsline{toc}{subsubsection}{Raw Body Yaw Rate ({º}/s): BYAWR}

The raw output of the IRU yaw rate. Positive values represent the nose
turning to the starboard and refer to rotation about the aircraft's
vertical axis. The sample rate is 50 Hz with a resolution of
0.0039{º}/s.

\hypertarget{global-positioning-systems}{%
\section{Global Positioning Systems}\label{global-positioning-systems}}

Primary GPS variables specifying the position and velocity of the
aircraft are provided by GPS receivers, currently a NovAtel Model
OEM-628 receiver. Prior to c.~2014 a NovAtel OEM4 was on the C-130 and a
OEMV on the GV. See this link for a description of these systems. The
coordinate system used for all GPS measurements is the World Geodetic
System WGS-84.\footnote{There are four measures of height or altitude
  discussed in this technical note, height relative to the WGS-84
  reference surface, geometric height relative to mean sea level,
  geopotential height and pressure height. The WGS-84 height (measured
  by GPS instruments) is height relative to a reference system in which
  zero is defined by a specified reference ellipsoid representing the
  shape of the Earth. This is not defined to be a level surface in the
  sense of being an equipotential surface. A better approximation of
  mean sea level is the ``geoid'', a surface having constant
  gravitational equipotential that approximates mean sea level. The
  geoid is more structured than the WGS-84 reference ellipsoid and
  departs significantly from it, often by several 10s of meters. Even
  the geoid does not represent mean sea level exactly because local mean
  sea level can be influenced by variations in water density, mean wind,
  or ocean circulation, but the geoid is usually the reference used for
  measurements labeled ``MSL'' except when fine-scale local effects must
  be considered. The geometric height is the true height above a
  reference surface, often taken to be mean sea level or the geoid; this
  may therefore differ significantly from the height measured directly
  by a GPS unit. There is a variable included below, GGEOIDHT, that
  provides a measure of the difference. Modern GPS receivers typically
  incorporate a model geoid and report altitude as height above the
  geoid. The Novatel receivers use the EGM96 geoid. Geopotential height
  is the height above mean sea level that would give the geopotential,
  or gravitational potential energy per unit mass, of the actual parcel
  if that mass were raised against standard gravity (not varying, e.g.,
  with latitude or height) to that altitude. For the purpose of this
  definition, standard gravity is defined to be
  9.80665{ \emph{m} \emph{s} − 2}. Finally, pressure altitude, defined
  in detail below, is the altitude in the ISA Standard Atmosphere where
  the pressure matches a specified value; it is not a geometric
  coordinate but rather a measure of pressure.} The uncertainty of the
horizontal position measurements for the Novatel OEM628 receiver
currently in use is specified by NovAtel to be typically 1.2 m RMS, 0.6
m, or 0.1 m for single-point L1/L2 measurements,WAAS mode, or OmniSTAR /
TerraStar mode, respectively. This specification is subject to
ionospheric and tropospheric conditions, satellite geometry,
interference, etc. Vertical uncertainty is approximately twice the
horizontal uncertainty. Because variables are stored as 4-byte
single-precision floating point numbers, the inherent storage precision
can limit the precision of the recorded position to about 1 m, depending
on latitude. The full-precision data is available from the original
aircraft data files upon request.The accuracy of velocity measurements
is 0.03 m/s RMS. Prior to {[}DATE NEEDED{]} all variables were provided
by the GPS receivers at 5 Hz. Prior to November 2014 all variables
provided by the NovAtel GPS receivers were recorded at 5 Hz. Between
then and May 2021 the output rate was 10 Hz, and since June 2021 the
output rate is 20 Hz.

Prior to January 2014, latitude and longitude were recorded from the
NMEA GPGGA log with a resolution of 0.0001 arcminutes (1.6\^{}-6
degrees),while altitude was recorded with 0.01 m resolution.
Speed-over-ground was recorded from the NMEA GPRMC log with resolution
of 0.001 knots (5x10\^{}-4 m/s). Starting in January 2014 new output
logs (named BESTPOS and BESTVEL) have also been recorded to preserve
more significant digits in the measurements. The BESTPOS log has a
horizontal position resolution of 10\^{}−11 degrees and altitude
resolution of 10\^{}-4 m, while in the BESTVEL log the horizontal
speed-over-ground is recorded with 1x10\^{}-4 m/s resolution. The
BESTVEL log also reports the aircraft vertical speed with 1x10\^{}-4 m/s
resolution.In addition, the BESTPOS log also reports the estimated
uncertainty (in meters) of the horizontal position and altitude.

Some of the following variables are also available from alternate Garmin
GPS16 receivers, for which the variable name is qualified by the name of
that unit; e.g., GGLAT\_GMN for GGLAT as measured by a Garmin GPS unit.
In addition, some of the measurements from the GPS units that are part
of the aircraft avionics systems are recorded; these are denoted by a
suffix ``\_G'' or ``\_A''. Measurements from before about 2000 used
Trimble TANS-III receivers, with the ability to track up to 6 satellites
at a time but needing only 4 to provide 3-dimensional position and
velocity data (3 satellites for 2-dimensions). The accuracy of the
position measurements for that unit was stated to be 25 meters
(horizontal) and 35 meters (vertical) under ``steady-state
conditions.''\footnote{The GPS signals at one time suffered from
  ``selective availability,'' a US DOD term for a perturbed signal that
  degraded GPS absolute accuracy to 100 meters. This was especially
  noticeable in the altitude measurement, so GALT normally was not
  useful. As of 1 May 2000, selective availability was deactivated to
  allow everyone to obtain better position measurements. See the
  Interagency GPS Executive Board web site for more information on
  selective availability and GPS measurements prior to 2000.} Likewise,
velocity measurements are within 0.2 m/s for all axes. Measurement
resolution is that of 4-byte IEEE format (about 6 significant digits).
All variables were provided by the Trimble receivers at 1 Hz.

A special correction is needed for variables GGVEW, GGVNS, and GGVSPD,
which measure the motion \emph{at the GPS antenna} relative to the
Earth. The conventional wind calculation addresses the difference
between the motion at the radome (where the relative wind is measured)
and the INS (where variables VEW, VNS, VSPD are measured) arising from
rotation of the aircraft. However, if GGVSPD is used instead of VSPD for
vertical wind or GGVEW and GGVNS are used (perhaps via the complementary
filter) for the horizontal wind, an additional correction is needed for
the displacement between the GPS antenna and the INS receiver. On the
GV, this distance is -4.30~m. A correction for aircraft rotation is
therefore applied to GGVSPD, as described below.

\hypertarget{gglat}{%
\subsubsection*{\texorpdfstring{GPS Latitude ({º}): GGLAT, LAT\_G; also
formerly
GLAT}{GPS Latitude (º): GGLAT, LAT\_G; also formerly GLAT}}\label{gglat}}
\addcontentsline{toc}{subsubsection}{GPS Latitude ({º}): GGLAT, LAT\_G;
also formerly GLAT}

\emph{The aircraft latitude measured by a global positioning system.}
Positive values are north of the equator; negative values are south.
These variables are recorded in netCDF files as single-precision values.
GGLAT is provided by the data-system GPS; LAT\_G and LATF\_G are from
the avionics system GPS. LATF\_G is a fine-resolution measurement that
requires special processing.

\hypertarget{gglon}{%
\subsubsection*{\texorpdfstring{GPS Longitude ({º}): GGLON, LON\_G; also
formerly
GLON}{GPS Longitude (º): GGLON, LON\_G; also formerly GLON}}\label{gglon}}
\addcontentsline{toc}{subsubsection}{GPS Longitude ({º}): GGLON, LON\_G;
also formerly GLON}

\emph{The aircraft longitude measured by a global positioning system.}
Positive values are east of the prime meridian; negative are west. GGLON
is provided by the (or a) data-system GPS; LON\_G and LONF\_G are from
the avionics system GPS. LONF\_G is a fine-resolution measurement that
requires special processing.

\hypertarget{ggspd}{%
\subsubsection*{GPS Ground Speed (m/s): GGSPD, GSPD\_G}\label{ggspd}}
\addcontentsline{toc}{subsubsection}{GPS Ground Speed (m/s): GGSPD,
GSPD\_G}

\emph{The aircraft ground speed measured by a global positioning
system.} GGSPD originates from a data-system GPS; GSF\_G originates from
an avionics-system GPS.

\hypertarget{ggvew}{%
\subsubsection*{GPS Ground Speed Vector East Component (m/s): GGVEW,
VEW\_G}\label{ggvew}}
\addcontentsline{toc}{subsubsection}{GPS Ground Speed Vector East
Component (m/s): GGVEW, VEW\_G}

\emph{The eastward component of ground speed measured by a global
positioning system.} GGVEW originates from a data-system GPS; VEW\_G
originates from an avionics-system GPS. In the case of GGVEW, when this
is used in the calculation of horizontal wind, the following correction
would be needed: \begin{equation}
\mathrm{\{GGVEWA\}}=\mathrm{\{GGVEW\}-L_G}\dot{\psi}\frac{\cos\psi}{\cos\phi}
(\#eq:GGVEWA)
\end{equation}\\
where GGVEWA is the corrected value used in the wind calculation, , and
\(\psi\) and \(\phi\) are respectively the heading and roll angles,
{\emph{L}\emph{G} =  − 4.30}~m for the GV, and \(\dot{\psi}\) is the
rate-of-change of heading (in radians). The variable BYAWR transmitted
from the INS gives the rate-of-change of heading after conversion from
{\(^{\circ}\thinspace s^{-1}\)} to radians~{\emph{s} − 1}. This
correction is not applied in normal processing because the use of the
complementary filter, discussed in
Sect.~@ref(combining-irs-and-gps-measurements), makes it of negligible
importance. More information is contained in this memo.

\hypertarget{ggvns}{%
\subsubsection*{GPS Ground Speed Vector North Component (m/s): GGVNS,
VNS\_G}\label{ggvns}}
\addcontentsline{toc}{subsubsection}{GPS Ground Speed Vector North
Component (m/s): GGVNS, VNS\_G}

\emph{The northward component of ground speed as measured by a global
positioning system.} GGVNS originates from a data-system GPS; VNS\_G
originates from an avionics-system GPS. In the case of GGVNS, when this
is used in the calculation of horizontal wind, the following correction
would be needed: \begin{equation}
\mathrm{\{GGVNSA\}} = \mathrm{\{GGVNS\}}+L_{G}\dot{\psi}\frac{\sin\psi}{\cos\phi}
(\#eq:GGVNSA)
\end{equation}\\
where GGVNSA is the corrected value used in the wind calculation,
\(\psi\) and \(\phi\) are respectively the heading and roll angles,
{\emph{L}\emph{G} =  − 4.30}~m for the GV, and \(\dot{\psi}\) is the
rate-of-change of heading (in radians). The variable BYAWR transmitted
from the INS gives the rate-of-change of heading {\emph{ψ̇}} after
conversion from {\(^{\circ}\thinspace s^{-1}\)} to
radians~{\emph{s} − 1}. This correction is not applied in normal
processing because the use of the complementary filter, discussed in
Sect.~@ref(combining-irs-and-gps-measurements), makes it of negligible
importance, as discussed in the note referenced for GGVEW.

\hypertarget{ggvspd}{%
\subsubsection*{GPS-Measured Aircraft Vertical Velocity (m/s): GGVSPD;
also (obsolete) VSPD\_G and GVZI}\label{ggvspd}}
\addcontentsline{toc}{subsubsection}{GPS-Measured Aircraft Vertical
Velocity (m/s): GGVSPD; also (obsolete) VSPD\_G and GVZI}

\emph{The aircraft vertical velocity provided by a GPS unit.} Positive
values are upward. When GGVSPD is used in the calculation of vertical
wind, the following correction (omitted before 2017) is applied:
\begin{equation}
\mathrm{\{GGVSPDA\}} = \mathrm{\{GGVSPD\}}-L_{G}\dot{\theta}
(\#eq:GGVSPDA)
\end{equation} where {\emph{L}\emph{G} =  − 4.30}~m for the GV and
\(\dot{\theta}\) is the rate-of-change of the pitch angle, corresponding
to the IRU variable BPITCHR*{\emph{π}}/180. The variable GGVSPDA is used
internally but not recorded in the data archives. See this memo for
additional justification.

\hypertarget{ggalt}{%
\subsubsection*{GPS Altitude (m MSL): GGALT, GALT\_A}\label{ggalt}}
\addcontentsline{toc}{subsubsection}{GPS Altitude (m MSL): GGALT,
GALT\_A}

\emph{The aircraft altitude} \emph{measured by a global positioning
system.} The measurement is with respect to the geoid as represented
internally by the GPS receiver and is determined by adding the
adjustment --GGEOIDHT to the direct measurement relative to the
ellipsoidal Earth model of the GPS, which is defined by WGS-84. Positive
values are above the reference surface. GGALT originates from a
data-system GPS; GALT\_A originates from an avionics-system GPS. See the
discussion of height at the beginning of this subsection and the
variable GGEOIDHT below for interpretation of these GPS-based
measurements.

\hypertarget{geoph}{%
\subsubsection*{GPS Geopotential Altitude (m): GEOPTH}\label{geoph}}
\addcontentsline{toc}{subsubsection}{GPS Geopotential Altitude (m):
GEOPTH}

\emph{The aircraft geopotential altitude above mean sea level.} If
{\emph{g}(\emph{z}, \emph{λ})} is the acceleration of gravity as
represented by the formula in the Table of Constants in Section
@ref(constants-and-symbols), then the formula used for calculation of
GEOPTH is obtained by integrating that formula from the reference
surface for MSL (the geoid, {\emph{Δ}} above the WGS84 reference
ellipse) to the geometric altitude {\emph{H}}, which is
{\emph{H} + \emph{Δ}} above the reference ellipse. The result is
normally close (within about 0.5~m) to that obtained with
{\emph{Δ} = 0}. There are additional details in this memo.

\(H\) = aircraft altitude above mean sea level, {[}m{]}
(\protect\hyperlink{ggalt}{GGALT})\\
\(\lambda\) = latitude (\protect\hyperlink{gglat}{GGLAT}) converted to
radians\\
\(Z(H,\ \lambda)\) = aircraft geopotential height {[}m{]} (GEOPTH)\\
\(g_{0}\) = constant acceleration of gravity as defined for the
International Standard Atmosphere\\
\(g_{e},\,g_{1},\,g_{2},\,k_{i}\) as defined in the Table of Constants
in Sect.~@ref(constants-and-symbols)\\
\(\Delta\)= height of the geoid above the WGS-84 reference ellipse
(GGEOIDHT)

\begin{equation}
\mathrm{\{GEOPHT\}}=Z(H,\lambda)=\frac{1}{g_{0}}\Biggl\{ g_{e}\left(\frac{1+g_{1}\sin^{2}\lambda}{(1-g_{2}\sin^{2}\lambda)^{1/2}}\right) \notag \\   
\times \left(H-\frac{1}{2}\left((H+\Delta)^{2}-\Delta^{2}\right)(k_{1}-k_{2}\sin^{2}\lambda)+\frac{1}{3}\left((H+\Delta)^{3}-\Delta^{3}\right)k_{3}\right)\Biggr\}  
(\#eq:GEOPHT)
\end{equation}

\hypertarget{ggtrk}{%
\subsubsection*{\texorpdfstring{GPS Aircraft Track Angle ({º}): GGTRK,
TKAT\_G}{GPS Aircraft Track Angle (º): GGTRK, TKAT\_G}}\label{ggtrk}}
\addcontentsline{toc}{subsubsection}{GPS Aircraft Track Angle ({º}):
GGTRK, TKAT\_G}

\emph{The direction of the aircraft track (degrees clockwise from true
north)} as measured by a data-system global positioning system (GGTRK)
or an avionics-system GPS (TKAT\_G)\emph{.}

\hypertarget{ggeoidht}{%
\subsubsection*{GPS Height of the Geoid (m): GGEOIDHT}\label{ggeoidht}}
\addcontentsline{toc}{subsubsection}{GPS Height of the Geoid (m):
GGEOIDHT}

\emph{Height of geoid, approximating mean sea level, above the WGS-84
ellipsoid. Also commonly called the geoid undulation} The height above
mean sea level is found by subtracting this value from the height above
the WGS-84 reference ellipse as provided by GPS-based measurements.The
NovAtel OEM628 receivers use the EGM96 model which provides geoid height
on a 0.5° x 0.5° grid and 0.1 m height resolution.

\hypertarget{ggnsat}{%
\subsubsection*{GPS Satellites Tracked: GGNSAT}\label{ggnsat}}
\addcontentsline{toc}{subsubsection}{GPS Satellites Tracked: GGNSAT}

\emph{The number of satellites tracked by a GPS receiver. The receiver
may not use all tracked satellites when calculating the position and
velocities.}

\hypertarget{ggqual}{%
\subsubsection*{GPS Quality Flag: GGQUAL}\label{ggqual}}
\addcontentsline{toc}{subsubsection}{GPS Quality Flag: GGQUAL}

\emph{GPS quality flag:}

\begin{table}

\caption{\label{tab:unnamed-chunk-34}**Meaning of GGQUAL:**}
\centering
\begin{tabular}[t]{c|l}
\hline
GGQUAL & description\\
\hline
0 & invalid\\
\hline
1 & valid but without quality enhancement. Approximately 1.2 m RMS horizontal accuracy.\\
\hline
2 & Receiving OmniSTAR/TerraStar corrections but not fully converged to the OmniSTAR/Terrastar position accuracy specification. Horizontal accuracy will be between 1.2 m and 0.1 m RMS.\\
\hline
5 & Fully locked-in OmniSTAR XP or TerraStar C, usually starting after about 20 minutes of tracking the GPS satellites and receiving the OmniSTAR or TerraStar data feed. This mode provides 0.1 m RMS or better horizontal position accuracy. This is described in some documents as differential GPS.\\
\hline
9 & Measurement enhanced by the Satellite-Based Augmentation System, a means of improving GPS accuracy and integrity by broadcasting from geostationary satellites wide area corrections for GPS satellite orbits and ionospheric delays. In the US, this uses the Wide-Area Augmentation System or WAAS. This is described in some data files as a differential-GPS measurement. Horizontal accuracy is approximately 0.6 m RMS.\\
\hline
\end{tabular}
\end{table}

\hypertarget{gmode}{%
\subsubsection*{GPS Mode: GMODE (obsolete)}\label{gmode}}
\addcontentsline{toc}{subsubsection}{GPS Mode: GMODE (obsolete)}

This is the former output from the Trimble GPS indicating the mode of
operation. The normal value is 4, indicating automatic (not manual) mode
and that the receiver is operating in 4-satellite (as opposed to fewer)
mode.

\hypertarget{ggstatus}{%
\subsubsection*{GPS Status: GGSTATUS, GSTAT\_G, GSTAT}\label{ggstatus}}
\addcontentsline{toc}{subsubsection}{GPS Status: GGSTATUS, GSTAT\_G,
GSTAT}

The status of the GPS receiver. A value of 1 indicates that the receiver
is operating normally; a value of 0 indicates a warning regarding data
quality. GGSTATUS indicates the status of the data-system GPS; GSTAT\_G
indicates the status of the avionics-system GPS. The obsolete variable
GSTAT, formerly used for the same purpose, has the reverse meaning: A
value of 0 indicates normal operation and any other code indicates a
malfunction or warning regarding poor data accuracy.

\hypertarget{altitude-latitude-and-longitude-standard-deviation-m-ggaltsd-gglatsd-gglonsd}{%
\subsubsection{Altitude, Latitude, and Longitude Standard Deviation (m):
GGALTSD, GGLATSD,
GGLONSD}\label{altitude-latitude-and-longitude-standard-deviation-m-ggaltsd-gglatsd-gglonsd}}

The estimated standard deviation in meters (1-sigma) of the altitude,
latitude, and longitude measurements. These values are reported by the
BESTPOS log in use since 2014.

\hypertarget{differential-age-s-ggdage}{%
\subsubsection{Differential Age (s):
GGDAGE}\label{differential-age-s-ggdage}}

The time since the last OmniSTAR/TerraStar corrections data was
received. The corrections data typically update approximately every 15
seconds, and the estimated position uncertainty increases during the
time between updates. Once the age exceeds the timeout (300 s by
default) the receiver will exit OmniSTAR/TerraStar mode.

\hypertarget{horizontal-dilution-of-precision-gghdop}{%
\subsubsection{Horizontal Dilution of Precision:
GGHDOP}\label{horizontal-dilution-of-precision-gghdop}}

A measure of the impact the spatial geometry of the observed satellites
has on the calculated horizontal position uncertainty. Values less then
2 are considered good. This is superceded by the reported measurement
standard deviation recorded in the GGALTSD, GGLATSD, and GGLONSD
variables.

\hypertarget{other-measurements-of-aircraft-altitude}{%
\section{Other Measurements of Aircraft
Altitude}\label{other-measurements-of-aircraft-altitude}}

\hypertarget{hgm}{%
\subsubsection*{Geometric Radio Altitude (m): HGM -
(obsolete)}\label{hgm}}
\addcontentsline{toc}{subsubsection}{Geometric Radio Altitude (m): HGM -
(obsolete)}

\emph{The distance to the surface below the aircraft,} measured by a
radar altimeter. The maximum range is 762m (2,500 ft). The instrument
changes in accuracy at an altitude of 152 m: The estimated error from
152 m to 762 m is 7\%, while the estimated error for altitudes below 152
m is 1.5 m or 5\%, whichever is greater.

\hypertarget{hgme-159}{%
\subsubsection*{Geometric Radar Altitude (Extended Range) APN-159
(APN-159) (m): HGME}\label{hgme-159}}
\addcontentsline{toc}{subsubsection}{Geometric Radar Altitude (Extended
Range) APN-159 (APN-159) (m): HGME}

\emph{The distance to the surface below the aircraft, measured by a
radar altimeter.} There are two outputs from an APN-159 radar altimeter,
one with coarse resolution (CHGME) and one with fine resolution (HGME).
Both raw outputs cycle through the range 0-360 degrees, where one cycle
corresponds to 4,000 feet for HGME and to 100,000 feet for CHGME. To
resolve the ambiguity arising from these cycles, 4,000-foot increments
are added to HGME to maintain agreement with CHGME. This preserves the
fine resolution of HGME (1.86 m) throughout the altitude range of the
APN-159.

\hypertarget{hgm-232}{%
\subsubsection*{Geometric Radar Altitude (Extended Range) APN-232 (m):
HGM232}\label{hgm-232}}
\addcontentsline{toc}{subsubsection}{Geometric Radar Altitude (Extended
Range) APN-232 (m): HGM232}

\emph{Altitude above the ground} as measured by an APN-232 radar
altimeter.

\hypertarget{altg}{%
\subsubsection*{Height Above Terrain (m): ALTG}\label{altg}}
\addcontentsline{toc}{subsubsection}{Height Above Terrain (m): ALTG}

\emph{The aircraft altitude above the Earth's surface} as represented by
the next variable. If GGALT is the altitude above mean sea level,
ALTG=GGALT{−}SFC.

\hypertarget{sfc}{%
\subsubsection*{Height of the Earth's Surface (m MSL): SFC}\label{sfc}}
\addcontentsline{toc}{subsubsection}{Height of the Earth's Surface (m
MSL): SFC}

\emph{The altitude of the Earth's surface} at a location directly below
the aircraft. The data source is the Shuttle Radar Topography Mission of
2000. The height estimate is described in this memo.

\hypertarget{hi3}{%
\subsubsection*{Pressure-Damped Inertial Altitude (m): HI3
(obsolete)}\label{hi3}}
\addcontentsline{toc}{subsubsection}{Pressure-Damped Inertial Altitude
(m): HI3 (obsolete)}

\emph{The aircraft altitude obtained from the twice-integrated IRU
acceleration (ACINS), pressure-adjusted to obtain long-term agreement
with PALT.} Note that this variable has mixed character, producing
short-term variations that accurately track the inertial system changes
but with adjustment to the pressure altitude, which is not a true
altitude. The variable is not appropriate for estimates of true
altitude, but proves useful in the updating algorithm used with the
LTN-51 INS for vertical wind. See the discussion of WP3. This variable
is now obsolete.

\hypertarget{palt}{%
\subsubsection*{ISA Pressure Altitude (m): PALT}\label{palt}}
\addcontentsline{toc}{subsubsection}{ISA Pressure Altitude (m): PALT}

\emph{The geopotential altitude in the International Standard Atmosphere
where the pressure is equal to the reference barometric (ambient)
pressure (PSXC).}\footnote{See ``U.S. Standard Atmosphere, 1976'',
  NASA-TM-A-74335, available for download at this URL.} The pressure
altitude is best interpreted as a variable equivalent to the measured
pressure, not as a geometric altitude. In the following description of
the algorithm, some constants (identified by the symbol {‡}) are
specified as part of the ISA and so should not be ``improved'' to more
modern values such as those given in the table of constants in Section
@ref(constants-and-symbols) (e.g., {\emph{R}0‡).}\footnote{Prior to and
  including some projects in 2010, processing used slightly different
  coefficients: for aircraft other than the GV, {\emph{T}0/\emph{λ}} was
  represented by -43308.83, the reference pressure {\emph{p}0} was taken
  to be 1013.246, and the exponent {\emph{x}} was represented
  numerically by 0.190284. For the GV, the value of {\emph{T}0/\emph{λ}}
  was taken to be 44308.0, the transition pressure {\emph{p}\emph{T}}
  was 226.1551 hPa, {\emph{x}} = 0.190284, and coefficient
  {\(\frac{R_{0}^{\prime}T_{T}}{gM_{d}}\)} was taken to be 6340.70 m
  instead of 6341.620 m as obtained below. The difference between these
  older values and the ones recommended below is everywhere less than 10
  m and so is small compared to the expected uncertainty in pressure
  measurements, because 1 hPa change in pressure leads to a change in
  pressure altitude that varies from about 8--40 m over the altitude
  range of the GV.} A note at this link describes the pressure altitude
in more detail and documents the change that was made in November 2010.

\(T_{0}^{\ddagger}\)= 288.15 K, reference temperature\\
\(\lambda_{a}^{\ddagger}\)
\sindex[lis]{lambdaa@$\lambda_{a}$= tropospheric lapse rate, standard atmosphere}=
-0.0065 \(^{\circ}\)C per geopotential meter = the lapse
rate\index{International Standard Atmosphere!lapse rate} for the
troposphere\(^{\ddagger}\)\\
\(p\) = measured static (ambient) pressure, hPa, usually from
\protect\hyperlink{psx}{PSXC}\\
\(p_{0}^{\ddagger}\)\sindex[lis]{P0star@$p_{0}^{\ddagger}$= reference pressure for zero altitude,
ISA} = 1013.25 hPa, reference pressure for PALT=0 \(^{\ddagger}\)\\
\(M_{d}^{\ddagger}\) = 28.9644 kg/kmol = molecular weight of dry air,
ISA definition \(^{\ddagger}\)\\
\(g^{\ddagger}\) = 9.80665 m~s\(^{-2}\), acceleration of gravity
\(^{\ddagger}\)\\
\(R_{0}^{\ddagger}\) = universal gas constant, defined\(^{\ddagger}\) as
8.31432\(\times10^{3}\)
J~kmol\textsuperscript{-1}~K\textsuperscript{-1}\\
\(z_{T}^{\ddagger}\) = altitude of the ISA tropopause = 11,000 m
\(^{\ddagger}\)\\
\(x=-R_{0}^{\ddagger}\lambda_{a}^{\ddagger}/(M_{d}^{\ddagger}g^{\ddagger})\)
\(\approx\) 0.1902632
(dimensionless)\footnote{This is the value, rounded to seven significant figures, that is used for data processing.}

For pressure \textgreater{} 226.3206 hPa (equivalent to a pressure
altitude \textless{} \(z_{T}\)): \begin{equation}
\mathrm{PALT}=-\left(\frac{T_{0}^{\ddagger}}{\lambda^{\ddagger}}\right)\left(1-\left(\frac{p}{p_{0}^{\ddagger}}\right)^{x}\right)
(\#eq:PALT1)
\end{equation}\\
otherwise, if
\(T_{T}\)\sindex[lis]{Tt@$T_{T}$= temperature at the ISA tropopause} and
\(p_{T}\)\sindex[lis]{pT@$p_{T}$= pressure at the ISA tropopause} are
respectively the temperature and pressure at the
altitude\index{International Standard Atmosphere!tropopause}
\(z_{T}\):\\
\[T_{T}=T_{0}+\lambda^{\ddagger}z_{T}^{\ddagger}=216.65\,\mathrm{K}\]
\[p_{T}=p_{0}^{\ddagger}\Bigl(\frac{T_{0}^{\ddagger}}{T_{T}}\Bigr)^{\frac{g^{\ddagger}M_{d}^{\ddagger}}{\lambda^{\ddagger}R_{0}^{\ddagger}}}=226.3206\,\mathrm{hPa}\]
\begin{equation}
\mathrm{PALT}=z_{T}^{\ddagger}+\frac{R_{0}^{\ddagger}T_{T}}{g^{\ddagger}M_{d}^{\ddagger}}\ln\left(\frac{p_{T}}{p}\right)
(\#eq:PALT2)
\end{equation}\\
which, after conversion from natural to base-10 logarithm, is coded to
be equivalent to the following:

\begin{verbatim}
       ## transition pressure at the assumed ISA tropopause:
       #define ISAP1 226.3206
       ## reference pressure for standard atmosphere:
       #define ISAP0 1013.25
       if (psxc > ISAP1) {
         palt = 44330.77 * (1.0 - pow(psxc / ISAP0, 0.1902632))
       } else {
         palt = 11000.0 + 14602.12 * log10(ISAP1 / psxc)
       }
\end{verbatim}

\hypertarget{altx}{%
\subsubsection*{Altitude, Reference (m MSL): ALTX (Obsolete), GGALTC
(Obsolete)}\label{altx}}
\addcontentsline{toc}{subsubsection}{Altitude, Reference (m MSL): ALTX
(Obsolete), GGALTC (Obsolete)}

\emph{Derived altitude above the geopotential surface,} obtained by
combining information from a GPS receiver and an inertial reference
system. This variable was intended to compensate for times when GPS
reception was lost by incorporating information from the IRS measurement
of altitude. GPS status measurements were used to detect signal loss,
although sometimes this signal was delayed for a few seconds after the
signal was lost. A 10-second running average was calculated of the
difference between the GPS altitude and the reference altitude. When the
sample-to-sample altitude difference changed more than 50 meters or when
the GPS status detected a degraded signal, the derived variable (ALTX or
GGALTC) became the alternate reference altitude adjusted by the latest
running-average difference between that reference altitude and GGALT.
When reception was recovered, to avoid a sudden discontinuity in
altitude, the derived variable was adjusted back to the GPS altitude
gradually over the next 10 seconds. This obsolete variable should be
used with caution because the reference altitude used in past
calculations was the IRS altitude updated to the pressure altitude of
the aircraft. To account for the difference between pressure and
geometric altitude, a regression equation was used, normally
\emph{z} = \emph{a}0 + \emph{a}1 \emph{ }PALT* where
{\emph{a}0 =  − 46.3} m and {\emph{a}1 = 0.97866} but often adjusted
dependent on project conditions. This introduced problems in early
applications with the GV because it did not account for the
pressure-altitude transition at the ISA tropopause. Use of a pressure
altitude as reference introduces additional errors in altitude in
regions that are not barotropic.

\hypertarget{combining-irs-and-gps-measurements}{%
\section{Combining IRS and GPS
Measurements}\label{combining-irs-and-gps-measurements}}

Measurements from the global positioning and inertial navigation systems
are combined to produce new variables that take advantage of the
strengths of each, so that the resulting variables have the long-term
stability of the GPS and the short-term resolution of the INS. This
section describes some variables that result from this blending of
variables. These corrected variables are usually the best available when
the GPS and IRS are both functioning.

One can determine if the GPS is functioning by examining the GPS status
variables described in the previous section or by looking for spikes or
``flat-lines'' in the data. If the GPS data are missing for a short time
(a few seconds to a minute), accuracy is not affected. However, longer
dropouts will result in uncertainties degrading toward those of the INS.
Without the GPS or another ground reference, the IRS error cannot be
determined empirically, and one should assume that it is within the
manufacturer's specification (1 nautical mile of error per hour of
flight, 90\% CEP). When the GPS is active, RAF estimates that the
correction algorithm produces a position with an error less than 1.5 m.
Due to the nature of the algorithm, the error will increase from about
1.5 meters to the INS specification in about one-half hour after GPS
information is lost.

\hypertarget{vewc-vnsc}{%
\subsubsection*{GPS-Corrected Inertial Ground Speed Vector (m/s): VEWC,
VNSC}\label{vewc-vnsc}}
\addcontentsline{toc}{subsubsection}{GPS-Corrected Inertial Ground Speed
Vector (m/s): VEWC, VNSC}

These variables result from combining GPS and INS output of the east and
north components of ground speed from a
\protect\hypertarget{compFilter}{}{complementary-filter} algorithm.
Positive values are toward the east and north, respectively. The smooth,
high-resolution, continuous measurements from the inertial navigation
system, \{VNS, VEW\}, which can slowly accumulate errors over time, are
combined with the measurements from the GPS, \{GVNS, GVEW\}, which have
good long-term stability, via an approach based on a complementary
filter. A low-pass filter,
{\emph{F}\emph{L}(\{\emph{GVNS}, \emph{GVEW}\})}, is applied to the GPS
measurements of groundspeed, which are assumed to be valid for
frequencies at or lower than the cutoff frequency {\emph{f}\emph{c}} of
the filter. Then the complementary high-pass filter, denoted
({1 − \emph{F}\emph{L}})({\{\emph{VNS}, \emph{VEW}\}}), is applied to
the IRS measurements of groundspeed, which are assumed valid for
frequencies at or higher than {\emph{f}\emph{c}}. Ideally, the
transition frequency would be selected where the GPS errors (increasing
with frequency) are equal to the IRS errors (decreasing with frequency).
The procedure is use now is documented in the Technical Note on Wind
Uncertainty, beginning on p.~125. It is a three-pole Butterworth
low-pass filter, originally coded following the algorithm described in
Bosic, S.~M., 1980: \emph{Digital and Kalman filtering : An Introduction
to Discrete-Time Filtering and Optimum Linear Estimation,} p.~49. As
described in this memo, it has been revised (2014) to use coefficients
generated by the R routine ``butter().'' The digital filter used is
recursive, not centered, to permit calculation during a single pass
through the data. If the cutoff frequency lies where both the GPS and
INS measurements are almost the same, then the detailed characteristics
of the filter (e.g., phase shift) in the transition region do not matter
because the complementary filters have canceling effects when applied to
the same signal. The transition frequency {\emph{f}\emph{c}} was chosen
to be (1/600) Hz. The Butterworth filter was chosen because it provides
flat response away from the transition.\footnote{For historical reasons,
  the details of the now obsolete filter as originally coded and used
  for many years are described here. For the current version with
  coefficients, see the memo referenced above.}

\textbf{CONSTANTS} (dependent on time constant
\(\tau\)):\textsuperscript{(a)}\\
\footnote{}\\
\(a=\frac{2\pi}{\tau}\),
\(a_{2}\)=\(a\ e^{-a/2}(\cos(a\sqrt{\frac{3}{2}})+\sqrt{\frac{1}{3}}\sin(a\sqrt{\frac{3}{2}}))\),
\(a_{3}\)=2\(e^{-a/2}\)\(\cos(a\sqrt{\frac{3}{2}})\),
\(a_{4}\)=\(e^{-a}\)\\
\_\_\_\_\_\_\_\_\_\_\\
\textsuperscript{(a)} For processing prior to the 2014, the factor
\(\sqrt{\frac{3}{2}}\) was erroneously \(\frac{\sqrt{3}}{2}\).

\begin{verbatim}
// input x = unfiltered signal  
// output returned is low-pass-filtered input  
// tau determines the cutoff  
// zf[] saves values for recursion  
zf[2] = -a * x + a2 * zf[5] + a3 * zf[3] - a4 * zf[4];  
zf[1] = a{*}x + a4{*}zf[1];     
zf[4] = zf[3];  
zf[3] = zf[2];  
zf[5] = x;  
return(zf[1] + zf[2]); 
\end{verbatim}

The net result then is the sum of these two filtered signals, calculated
as described in the following boxes:

\protect\hyperlink{vew}{VEW} = IRS-measured east component of the
aircraft ground speed\\
\protect\hyperlink{vns}{VNS} = IRS-measured north component of the
aircraft ground speed\\
\protect\hyperlink{ggvew}{GGVEW} = GPS-measured east component of the
aircraft ground speed\\
\protect\hyperlink{ggvns}{GGVNS} = GPS-measured north component of the
aircraft ground speed\\
\(F_{L}()\) = three-pole Butterworth low-pass recursive digital filter

\begin{align}\begin{split}
\{\mathrm{VNSC}\} & = F_{L}(\mathrm{\{GGVNS\})}+(1-F_{L})(\{\mathrm{VNS\}})\\
\{\mathrm{VEWC}\} & = F_{L}(\mathrm{\{GGVEW\})}+(1-F_{L})(\{\mathrm{VEW\}})
\end{split}(\#eq:VC)
\end{align}

This result is used as long as the GPS signals are continuous and
flagged as being valid. When that is not the case, some means is needed
to avoid sudden discontinuities in velocity (and hence wind speed),
which would introduce spurious effects into variance spectra and other
properties dependent on a continuously valid measurement of wind. To
extrapolate measurements through periods when the GPS signals are lost
(as sometimes occurs, for example, in turns) a fit is determined to the
difference between the best-estimate variables \{VNSC,VEWC\} and the IRS
variables \{VNS,VEW\} for the period before GPS reception was lost, and
that fit is used to extrapolate through periods when GPS reception is
not available. The procedure is as described below.

~1. If GPS reception has never been valid earlier in the flight, use the
INS values without correction.\\
\hspace*{0.333em}2. Whenever both GPS and INS are good, update the
low-pass-filtered estimate of the difference between them. This is added
to the INS measurement to obtain the corrected variable. Also update a
least-squares fit to the difference between the GPS and INS
groundspeeds, for each component. The errors are assumed to result
primarily from a Schuler oscillation, so the three-term fit is of the
form \(\Delta=a_1+a_2\sin(\Omega_{Sch}t)+a_3\cos(\Omega_{Sch}t)\), where
\(\Omega_{Sch}\) is the angular frequency of the Schuler oscillation
(taken to be \(2\pi/5067\)~s, and \(t\) is the time since the start of
the flight. A separate fit is used for each component of the velocity
and each component of the position (discussed below under LATC and
LONC). The \protect\hypertarget{fit-matrix}{}{fit matrix} used to
determine these coefficients is updated each time step but the
accumulated fit factors decay exponentially with a 30-min decay
constant, so the terms used to determine the fit are exponentially
weighted over the period of valid data with a time constant that decays
exponentially into the past with a characteristic time of 30 min. This
is long enough to determine a significant portion of the Schuler
oscillation but short enough to emphasize recent measurements of the
correction. ~3. When GPS data become invalid, if sufficient data
(spanning 30 min) have been accumulated, invert the accumulated fit
matrices to determine the coefficients \{\(a_1,a_2,a_3\)\} and then use
the formula for \(\Delta\) in the preceding step to extrapolate the
correction to the IRS measurements while the GPS measurements remain
invalid. Doing so immediately would introduce a discontinuity in
\{VNSC,VEWC\}, however, so the correction \(\Delta\) is introduced
smoothly by adjusting \{VNSC, VEWC\} as follows: If \(dvy\) is the
adjustment added to the INS measurement, adjust it according to
\(dvy^\prime=\eta\ dvy+(1-\eta)\Delta\) where \(dvy^\prime\) is the
sequentially adjusted correction and \(\eta=0.995\) s\(^{-1}\) is chosen
to give a decaying transition with a time constant of about 5.5 min.
This has the potential to introduce some artificial variance at this
scale and so should be considered in cases where variance spectra are
analyzed in detail, but it has much less influence on such spectra than
a discontinuous transition would. Ideally, the current fit and the last
filtered discrepancy (VNSC\textsubscript{0}\(-\)GVNS\textsubscript{0}
should be about equal, so transitioning between them should not
introduce a significant change. ~4. To avoid transients that would
result from switching abruptly to the complementary-filter solution when
the GPS measurements again become valid, the correction factors (e.g.,
\(dvy\)) are also updated smoothly toward the complementary-filter
solution, using for example
\(dvy^\prime=\eta\ dvy+(1-\eta)F_L(v_y^{GPS}-v_y^{IRS})\) where \(F_L\)
is the low-pass filter and \(v_y\) is the northward component of
aircraft groundspeed.

\hypertarget{latc-lonc}{%
\subsubsection*{\texorpdfstring{GPS-Corrected Inertial Latitude and
Longitude ({º}): LATC,
LONC}{GPS-Corrected Inertial Latitude and Longitude (º): LATC, LONC}}\label{latc-lonc}}
\addcontentsline{toc}{subsubsection}{GPS-Corrected Inertial Latitude and
Longitude ({º}): LATC, LONC}

\emph{Combined GPS and IRS output of latitude and longitude.} Positive
values are north and east, respectively.These variables are the best
estimate of position, obtained by the following approach:

\protect\hyperlink{latitude}{LAT} = latitude measured by the IRS\\
\protect\hyperlink{longitude}{LON} = longitude measured by the IRS\\
\protect\hyperlink{gglat}{GGLAT} = latitude measured by the GPS\\
\protect\hyperlink{gglon}{GGLON} = longitude measured by the GPS\\
\protect\hyperlink{vewc-vnsc}{VNSC} = aircraft ground speed, north
component, corrected\\
\protect\hyperlink{vewc-vnsc}{VEWC} = aircraft ground speed, east
component, corrected

\textbf{1.} Initialize the corrected position at the IRS position at the
start of the flight or after any large change (\textgreater5\(^\circ\))
in the IRS position. \textbf{2.} Integrate forward from that position
using the aircraft groundspeed with components \{VNSC,VEWC\}. Note that
in the absence of GPS information this will introduce long-term errors
because it does not account for the Earth's spherical geometry. It
provides good short-term accuracy, but the GPS updating in the next step
is needed to compensate for the difference between a rectilinear frame
and the Earth's spherical coordinate frame and provides a smooth yet
accurate track. \textbf{3.} Use an exponential adjustment to the GPS
position, with time constant that is typi- cally about 100
s.\textsuperscript{(a)} \textbf{4.} To handle periods when the GPS
becomes invalid, use an approach analogous to that for groundspeed,
whereby a Schuler-oscillation fit to the difference between the GPS and
IRS measurements is accumulated and used to extrapolate through periods
when the GPS is invalid. \_\_\_\_\_\_\_\_\_\_\\
\textsuperscript{(a)} \emph{specifically, LATC += η(GLAT-LATC) with η =
2π/(600 s)}

\hypertarget{the-state-of-the-atmosphere}{%
\chapter{The State of the
Atmosphere}\label{the-state-of-the-atmosphere}}

\hypertarget{information-on-instruments-and-calibrations}{%
\section{Information on Instruments and
Calibrations}\label{information-on-instruments-and-calibrations}}

The instruments used to collect the measurements that lead to the
variables in this section are described on the EOL web site, in the
``State Parameters'' section at this URL. The data acquisition and
processing for these variables and the calibration coefficients used
where applicable are described.

\hypertarget{variable-names}{%
\section{Variable Names}\label{variable-names}}

Measurements of some meteorological state variables like pressure,
temperature, and water vapor pressure may originate from multiple
sensors mounted at various locations on an aircraft. To distinguish
among similar measurements, many variable names incorporate an
indication of where the measurement was made. In this document,
locations in variable names are represented by ``x'', where ``x'' may be
one of the following:

\begin{table}
\centering
\begin{tabular}{c|l}
\hline
Character & location\\
\hline
B & bottom (or bottom-most)\\
\hline
B & (obsolete) boom\\
\hline
F & fuselage\\
\hline
G & (obsolete) gust probe\\
\hline
R & radome\\
\hline
T & top (or top-most)\\
\hline
W & wing\\
\hline
\end{tabular}
\end{table}

In addition, a true letter 'X' (not replaced by the above letters) may
be appended to a measurement to indicate that it is the preferred choice
among similar measurements and is therefore used to calculate derived
variables that depend on the measured quantity. Other suffixes sometimes
used to distinguish among measurements are these: 'D' for a digital
sensor; 'H' for a heated (usually, anti-iced) sensor, 'L' for port-side
sensors, and 'R' for starboard-side sensors.

\hypertarget{pressure}{%
\section{Pressure}\label{pressure}}

\hypertarget{standard-pressure-measurements}{%
\subsection{Standard Pressure
Measurements}\label{standard-pressure-measurements}}

\hypertarget{psx}{%
\subsubsection*{Static or Ambient Pressure (hPa): PSx, PSxC, PS\_A, PSX,
PSXC, PSFD, PSFRD, PSTF}\label{psx}}
\addcontentsline{toc}{subsubsection}{Static or Ambient Pressure (hPa):
PSx, PSxC, PS\_A, PSX, PSXC, PSFD, PSFRD, PSTF}

\emph{The atmospheric pressure at the flight level of the aircraft,
measured by a calibrated absolute (barometric) transducer at location
x.} PSx is the measured static or ambient pressure before correction,
and it may be affected by local flow-field distortion. PS\_A is the
pressure measurement taken from the avionics system on the aircraft,
processed via unknown algorithms in the avionics system that may smooth,
correct, and perhaps delay the result. PSxC is PSx corrected for local
flow-field distortion (see RAF Bulletin \#21 and the discussion in this
memo and this supplement), and PSXC is the preferred corrected
measurement used for derived calculations. These measurements have been
made using various sensors, so it is best to consult the project
documentation for the transducer used. Recent measurements from both the
C-130 and the GV have been made using a Paroscientific Model 1000
Digiquartz Transducer.

Corrections to the pressures have been determined by reference to some
standard, including a ``trailing cone'' sensor, the pressure PS\_A from
the cockpit avionics system, or (since 2012) the Laser Air Motion
Sensing System (LAMS). The latter correction is discussed in the memo
referenced above, where corrections used prior to 2011 are also
discussed. Beginning in 2012, the deduced corrections {\emph{Δp}} to the
measured pressures as functions of dynamic pressure {\emph{q},} angle of
attack {\emph{α},}\footnote{A weakness is this form for the pressure
  correction is that occasionally the radome ports become plugged with
  ice and the measurement of angle of attack is not available. When the
  variable ATTACK representing angle of attack is invalid, the angle of
  attack is instead calculated from PITCH{−}VSPD/TASX, which
  approximates the angle of attack if the vertical wind is zero.} and
the Mach number {\emph{M}} are described by the following equations and
coefficients:

\textbf{For the C-130,}\footnote{For C-130 measurements prior to 2012
  but after September 2003, the correction applied to PSF was
  \emph{Δp} = \emph{p} + max ((3.29 + \{\emph{QCX}\} * 0.0273),4.7915)
  using units of hPa. Prior to Sept 2003, the correction was
  {\emph{Δp} = max ((4.66 + 11.4405\emph{Δp}\emph{α}/\emph{Δq}\emph{r}}),
  1.113). For both PSFD and PSFRD, the correction prior to (2012?) was
  {\emph{Δp} = \emph{p} + max ((3.29 + \{\emph{QCX}\} \emph{ 0.0273),4.7915).
  For GV measurements Aug 2006 to 2012, {\emph{Δp}=} (-1.02 + 0.1565}}q)
  + q1{\emph{(0.008 + q1}}(7.1979e-09{\emph{q1 - 1.4072e-05). Before Aug
  2006: {\emph{Δp}=}(3.08 - 0.0894}}\{PSF\}) + \{QCF\}{\emph{(-0.007474
  + \{QCF\}}}4.0161e-06).} \begin{equation}
\frac{\Delta p}{p_{m}}=d_{0}+\frac{q_{m}}{p_{m}}(d_{1}+d_{4}\frac{\alpha^{2}}{a_{r}^{2}})+d_{2}\frac{\alpha}{a_{r}}+d_{3}\thinspace M
(\#eq:PCORC130)
\end{equation}

where, for {\emph{p}\emph{m}} = PSFD , {\emph{q}\emph{m}} = dynamic
pressure (QCF), {\emph{α} = \emph{ATTACK}} and {\emph{a}\emph{r} = 1∘}
(included to keep the equation and coefficients dimensionless), and
\{{\emph{d}0, \emph{d}1, \emph{d}2, \emph{d}3, \emph{d}4}\} =
\{-4.389e-03, -2.966e-02, -6.831e-05, 2.672e-02, 2.4466e-03\}. For
PSFRD, the corresponding coefficients are \{0.007372, 0.12774,
{−}6.8776e-4, {−}0.02994, 0.001630\}. The latter coefficients are
significantly different from the coefficients for PSFD, but the static
ports where PSFRD is measured are at a different location on the
fuselage so different flow-distortion effects are expected.

\textbf{For the GV,}\footnote{See this memo and this revisionfor details
  regarding implementation of this representation of {\emph{Δp}} for the
  GV.} \begin{equation}
\frac{\Delta p}{p}=a_{0}+a_{1}\frac{q}{p}+a_{2}M^{3}+a_{3}\frac{\alpha}{a_{r}}
(\#eq:PCORforGV)
\end{equation}\\
where, for {\emph{p}} = PSF, {\emph{q}} = QCF,
{\emph{α} = \emph{ATTACK}}, and {\emph{a}\emph{r} = 1∘} (included to
keep the equation and coefficients dimensionless)
\{{\emph{a}0, \emph{a}1, \emph{a}2, \emph{a}3}\} = \{{ − 0.012255},
{0.075372}, { − 0.087508}, {0.002148}\}\}.

In equations @ref(eq:PCORC130) and @ref(eq:PCORforGV) the Mach number is
calculated from the uncorrected measurements of {\emph{p}} and
{\emph{q}}, using dry-air values for {\emph{R}}, {\emph{c}\emph{v}} and
{\emph{c}\emph{p}}, via\\
\begin{equation}
M=\left\{ \left(\frac{2c_{v}}{R}\right)\left[\left(\frac{p+q}{p}\right)^{R/c_{p}}-1\right]\right\}^{1/2}\,\,\,.
(\#eq:MachEquation)
\end{equation}

The pressure PSTF is measured at the static-pressure port of a
pitot-static tube mounted on the top of the fuselage. The correction to
this pressure (leading to PSTFC) differs from the pressure corrections
listed above in that it is based on an empirical fit to other
measurements of static pressure. The airflow at that sensor is distorted
by the fuselage, so that measurement is usually less reliable than other
measures of static pressure. See this note for additional discussion of
this measurement.

For additional information on these correction coefficients, see this
note and Cooper et al.~(2014).\footnote{Atmos. Meas. Tech., 7,
  3215-3231, 2014 \url{doi:10.5194/amt-7-3215-2014}.}

\hypertarget{qcx}{%
\subsubsection*{Dynamic Pressure (hPa): QCx, QCxC, QCX, QCXC, QCTF,
QCTFC, QC\_A}\label{qcx}}
\addcontentsline{toc}{subsubsection}{Dynamic Pressure (hPa): QCx, QCxC,
QCX, QCXC, QCTF, QCTFC, QC\_A}

\emph{The pressure excess caused by bringing the airflow to rest
relative to the aircraft.} These quantities represent the difference
between the total pressure {\emph{p}\emph{t}} as measured at the inlet
of a pitot tube or other forward-pointing port and the ambient pressure
that would be present in the absence of motion through the air. The
variable QC\_A is provided by the avionic data system and is subject to
an unknown delay and smoothing. The variables ending in ``C'' have been
corrected for flow-distortion effects, mostly arising from errors in the
measurement of static pressure. Since 2012, the corrections are based on
measurements from the LAMS system as described for PSxC, and they have
the same functional form as in @ref(eq:PCORC130) and @ref(eq:PCORforGV)
except that the correction applied to {\emph{q}} is { − \emph{Δp}} with
reversed sign because {\emph{q} = \emph{p}\emph{t} − \emph{p}\emph{a}}
and the error arises primarily from the error in {\emph{p}\emph{a}}. The
same correction is applied to QCR because it is also measured relative
to the static pressure ports so errors in the pressure sensed at those
ports affect QCR in the same way that QCF is affected. See the notes
referenced in the preceding section, and also RAF Bulletin 21 for the
corrections applied to earlier data files.\footnote{\emph{C-130}, prior
  to 2012: For QCFC: subtract max(4.66+11.4405 * \(\mathrm{\{ADIFR\}}\)
  / \(\mathrm{\{QCR\}}\), 1.113); For QCFRC prior to Sept 2003: same as
  for QCFC; after/including Sept 2003, subtract
  max(3.29+\(\mathrm{\{QCX\}}\) \emph{ 0.0273, 4.7915); For QCRC:
  subtract max((3.29+\(\mathrm{\{QCX\}}\) } 0.0273), 4.7915). \emph{GV}
  Aug 2006 to 2012: For QCF, subtract (1.02+\(\mathrm{\{PSF\}}\)(0.215 -
  0.04 * \(\mathrm{\{QCF\}}\)/1000.) + \{QCF\} * (\(-0.003266\) +
  \(\mathrm{\{QCF\}}\) * 1.613e-06))}

A Rosemount Model 1221 differential pressure transducer is used for
current measurements of dynamic pressure on the C-130, and a Honeywell
PPT transducer is used on the GV. This measurement enters the
calculation of true airspeed and Mach number and so is needed to
calculate many derived variables. In the case of QCRC from the GV, one
additional correction is applied (beginning 2017). The uncorrected
measurement QCR is affected by flow angles, while QCF is not (for modest
angle of attack or sideslip), so an additional adjustment is needed. The
needed correction can be found by using an empirical relationship
matching QCR to QCF, which leads to the following equation:\\
\begin{equation}
\mathrm{\{QCR\}}=b_0+b_1\mathrm{\{QCR\}}+b_2\mathrm{\{AKRD\}}^2+b_3\mathrm{\{SSRD\}}^2-\Delta p
(\#eq:PCORforQCR)
\end{equation}

where \(\Delta p\) is given by @ref(eq:PCORforGV) and the coefficients
are \{{\emph{b}0 − 3}\}=\{{ − 0.5635}, {0.9982}, 0.0273, {0.0562}\}.
Some justification for this correction is contained in this note. A
similar correction is not made for measurements from the C-130 radome
because they do not appear to be necessary, as discussed in that note.

The measurement QCTF is made at the dynamic-pressure port of a
pitot-static sensor mounted on the top of the fuselage. Because this is
located in a region of airflow distortion around the fuselage of the GV,
special processing is required. The method proposed in this note is to
calculate a corrected dynamic pressure from QCTFC=QCTF+PSTF-PSXC. For
high-rate files, PSXC is low-pass-filtered with a cut-off frequency of
0.5 Hz to eliminate high-frequency noise in this measurement.

\hypertarget{dvalue}{%
\subsubsection*{D-Value (m): DVALUE}\label{dvalue}}
\addcontentsline{toc}{subsubsection}{D-Value (m): DVALUE}

The difference between geopotential altitude and pressure altitude (m).
This variable is calculated from \{GEOPHT\}{−}\{PALT\} and, for
appropriate flight segments, can be used to measure height gradients on
a constant-pressure surface. Prior to 2018, this was calculated from
\{GGALT\} -- \{PALT\}.

\hypertarget{p-special}{%
\subsection*{Special Pressure Measurements (hPa): PSDPx, CAVP, PCAB,
PS\_VXL, PSURF}\label{p-special}}
\addcontentsline{toc}{subsection}{Special Pressure Measurements (hPa):
PSDPx, CAVP, PCAB, PS\_VXL, PSURF}

\emph{PSDPx and CAVP\_x are measurements of the pressure in the housing
of the dew-point sensors,} as discussed in connection with DPxC.
\emph{PCAB is a measurement of the pressure in the cabin of the
aircraft. PS\_VXL is the pressure measured by the VCSEL hygrometer.
PSURF is the estimated surface pressure} calculated from HGME (a
radar-altimeter measurement of height), TVIR, PSXC, and MR using the
thickness equation as shown in the box below.
\protect\hyperlink{TVIR}{TVIR} and \protect\hyperlink{MR}{MR} are
described later in this section, and \protect\hyperlink{HGME}{HGME} was
described in Section @ref(other-measurements-of-aircraft-altitude). The
average temperature for the layer is obtained by using HGME and assuming
a dry-adiabatic lapse rate from the flight level to the surface. Because
of this assumption, the result is only valid for flight in a well-mixed
surface layer or in other conditions in which the temperature lapse rate
matches the dry-adiabatic lapse rate.\footnote{The symbol {†} indicates
  that values are included in the table of constants in
  Sect.~@ref(constants-and-symbols).}

\protect\hyperlink{psx}{PSXC} = ambient pressure {[}hPa{]}\\
\href{./3-the-state-of-the-aircraft.html\#hgme-159}{HGME} = (radar)
altitude above the surface {[}m{]}\\
\protect\hyperlink{TVIR}{TVIR} = virtual temperature
{[}\(^{\circ}\mathrm{C}\){]}\\
PSURF = estimated surface pressure {[}hPa{]}\\
\(g\) = acceleration of gravity\(^{\dagger}\)\\
\(R_{d}\) = gas constant for dry air\(^{\dagger}\)\\
\(c_{pd}\) = specific heat of dry air at constant pressure\(^{\dagger}\)

\begin{equation}
T_{m}=(\mathrm{\{TVIR\}}+T_{0})+0.5\mathrm{\{HGM\}}\frac{g}{c_{pd}}
(\#eq:Tm4PSURF)
\end{equation} \begin{equation}
\mathrm{PSURF}=\mathrm{\{PSXC\}}\,\exp\left\{ \frac{g\,\{\mathrm{HGME}\}}{R_{d}T_{m}}\right\}
(\#eq:PSURF)
\end{equation}

\hypertarget{temperature-section}{%
\section{Temperature}\label{temperature-section}}

\hypertarget{recovery-t}{%
\subsubsection*{\texorpdfstring{Recovery Temperature ({º}C): RTx, RTHx,
RTHRx,
RTX}{Recovery Temperature (ºC): RTx, RTHx, RTHRx, RTX}}\label{recovery-t}}
\addcontentsline{toc}{subsubsection}{Recovery Temperature ({º}C): RTx,
RTHx, RTHRx, RTX}

\emph{The recovery temperature is the temperature sensed by a
temperature probe that is exposed to the atmosphere.} In flight, the
temperature is heated above the ambient temperature because it senses
the temperature of air near the sensor that has been heated
adiabatically during compression as it is brought near the airspeed of
the aircraft. These variables are the measurements of that recovery
temperature from calibrated temperature sensors at location x, for
processing prior to about 2012; more recently, the names are simply
RTF\# or RTH\# where \# is a number starting with 1.\footnote{Prior to
  2012, these variables were called ``total temperature'' and symbols
  starting with 'TT' instead of 'RT' were used. That name was misleading
  because these values are not true total-temperature measurements, for
  which the air would be at the same speed as the aircraft, but instead
  recovery-temperature measurements. The name has been changed to
  correct this mis-labeling, although this was a long-standing
  convention in past datasets.} For Rosemount or HARCO temperature
probes in current use, the recovery temperature is near the total
temperature, but all probes must be corrected to obtain either true
total temperature or true ambient temperature. In the standard output,
the variable name also conveys the sensor type: RTF\# or RTx is a
measurement from a Rosemount Model 102 non-deiced temperature sensor,
RT\#H or RTHx is the measurement from a Rosemount Model 102 anti-iced
(heated) temperature sensor, and RTH\# or RTHx is the measurement from a
HARCO heated sensor. Some past experiments also used a reverse-flow
temperature housing and a fast-response ``K'' housing; the associated
variable names for these probes were TTRF and TTKP.\footnote{See the
  related obsolete variables TTx, which are previously used names for
  these variables. The names were changed to clarify that the quantity
  represented is the recovery temperature, not the total temperature.}

\hypertarget{ambient-t}{%
\subsubsection*{\texorpdfstring{Ambient Temperature ({∘}C): ATx, ATX,
ATHx,
ATxD}{Ambient Temperature (∘C): ATx, ATX, ATHx, ATxD}}\label{ambient-t}}
\addcontentsline{toc}{subsubsection}{Ambient Temperature ({∘}C): ATx,
ATX, ATHx, ATxD}

\emph{The temperature of the atmosphere at the location of the aircraft,
as it would be measured by a sensor at rest relative to the air.}
\{\#ATX\} The 'x' in the name of the variable used for ambient
temperature, ATx, conveys the same information regarding sensor type and
location as the variable name used with total (recovery) temperature.
See the discussion above regarding RTx. The ambient temperature (also
known as the static air temperature) is calculated from the measured
recovery temperature, which is increased above the ambient temperature
by dynamic heating caused by the airspeed of the aircraft. The
calculated temperature therefore depends on the recovery temperature RTx
as well as the dynamic and ambient pressure, usually respectively QCXC
and PSXC. The ambient and dynamic pressures are first corrected from the
raw measurements QCX and PSX to obtain variables that account for
deviations caused by airflow around the aircraft and/or
position-dependent systematic errors, as discussed in the section
describing PSxC. The following basic equations are developed on the
basis of conservation of energy for a perfect gas undergoing an
adiabatic compression. This section combines discussion of the
calculations of temperature and airspeed, to reflect the linkage between
these derived measurements. To provide accuracy in the equations, this
discussion considers effects of the humidity of the air on
characteristics like the gas constant and the specific heats. Most
archived data before 2012 used values for dry air, although a special
variable TASHC has been used to represent the true airspeed in cases
where the correction was significant. That variable is based on a good
approximation to the results from the following equations; see the
discussion of TASHC later in this section. TASHC is now considered an
obsolete variable. New variables ATxD and TASxD have been introduced
that neglect the humidity corrections and perform all calculations as if
the humidity is negligible. \\
As discussed above, temperature sensors on aircraft that are exposed to
the airflow do not measure the total temperature but rather the
temperature of the air immediately in contact with the sensing element.
This air will not have undergone an adiabatic deceleration completely to
zero velocity and hence will have a temperature {\emph{T}\emph{r}}
somewhat less than the total temperature {\emph{T}\emph{t}} that would
require the air to reach zero velocity. {\emph{T}\emph{r}} is the
measured or ``recovery'' temperature., The ratio of the actual
temperature difference attained to the temperature difference relative
to the total temperature is defined to be the ``recovery factor''
\(\alpha_r\): \begin{equation}
\alpha_{r}=\frac{T_{r}-T_{a}}{T_{t}-T_{a}}
(\#eq:alphar)
\end{equation} where \(T_a\) is the ambient air temperature. From
conservation of energy:\\
\begin{equation}
\frac{U_{a}^{2}}{2}+c_{p}^{\prime}T_{a}=\frac{U_{r}^{2}}{2}+c_{p}^{\prime}T_{r}=\frac{U_{t}^{2}}{2}+c_{p}^{\prime}T_{t}
(\#eq:Ua)
\end{equation} where primes on quantities like {\emph{c}\emph{p}′}, or
(below) {\emph{c}\emph{v}′} and {\emph{R}′} denote properties of moist
air, respectively the specific heat at constant pressure, specific heat
at constant volume, and gas constant.\\

Moist-Air Considerations Primes on the symbols denote that these values
should be moist-air values, appropriately weighted averages of the
dry-air and water-vapor contributions. The practice prior to 2014 was to
use the dry-air values for specific heats and the gas constant, except
as described in connection with TASHC below. Since 2014, calculations
use the appropriate values for moist air, except that to avoid errors
introduced by unrealistically high measurements of humidity the humidity
correction was limited to be less than or equal to the equilibrium value
at the measured temperature. The formulas used for the specific heats
and gas constant of moist air in terms of the water vapor pressure
\(e\), the specific heats for dry air
(\(c_{pd}=\frac{7}{2}R_{0},\,c_{vd}=\frac{5}{2}R_{0}\)) and water vapor
(\(c_{pw}=4R_{0},\,c_{vw}=3R_{0}\)), and the ratio of molecular weights
(\(\epsilon=M_{W}/M_{d}\)) are those of Khelif et al.~1999:\\
\begin{equation}
R^{\prime}=R_{d}/[1+(\epsilon-1)\frac{e}{p}]
\end{equation} \begin{equation}
c_{v}^{\prime} = \frac{(p-e)R^{\prime}}{pR_{d}}\frac{5R_{0}}{2M_{d}}+\frac{eR^{\prime}}{pR_{w}}\frac{3R_{0}}{M_{w}}=c_{vd}\frac{R^{\prime}}{R_{d}}\left(1+\frac{1}{5}\frac{e}{p}\right)
(\#eq:cvMoist)
\end{equation} \begin{equation}
c_{p}^{\prime} = c_{pd}\frac{R^{\prime}}{R_{d}}\left(1+\frac{1}{7}\frac{e}{p}\right)
(\#eq:cpMoist)
\end{equation} \begin{equation}
\gamma\,^{\prime} = \gamma_{d}\frac{1+\frac{1}{7}\frac{e}{p}}{1+\frac{1}{5}\frac{e}{p}}
(\#eq:gammaMoist)
\end{equation}\\
See also the discussion of TASHC \protect\hyperlink{tashc}{below} and
the reference there for Khelif et al.~1999.

In @ref(eq:Ua)
{\{\emph{U}\emph{a}, \emph{U}\emph{r}, \emph{U}\emph{t}\}} are
respectively the aircraft true airspeed, the airspeed relative to the
aircraft of the air in thermal contact with the sensor, and the airspeed
of air relative to the aircraft when fully brought to the motion of the
sensor (i.e., zero). Then, from @ref(eq:Ua) \begin{equation}
T_{a}=T_{r}-\alpha_{r}\frac{U_{a}^{2}}{2c_{p}^{\prime}}
(\#eq:Tair)
\end{equation}\\
The temperature sensors used on RAF aircraft are designed to decelerate
the air adiabatically to near zero velocity. Recovery factors determined
from wind tunnel testing for the Rosemount sensors are approximately
0.97 (unheated model) and 0.98 (heated models).\footnote{The recovery
  factor determined for the now-obsolete NCAR reverse-flow sensor was
  0.6. The recovery factor for the now retired NCAR fast-response
  (K-probe) temperature sensor was 0.8.} These values have also been
confirmed from flight maneuvers, often from ``speed runs'' where the
aircraft is flown level through its speed range and the variation of
recovery temperature with airspeed is used with @ref(eq:Tair), with the
assumption that {\emph{T}\emph{a}} remains constant, to determine the
recovery factor. Data files and project reports normally document what
recovery factor was used for calculating the true airspeed and ambient
temperature for a particular project. Because the values used in
processing have varied, the project reports should be consulted to find
what was used for particular projects. The Goodrich Technical Report
5755 documents wind-tunnel testing of the probes formerly made by
Rosemount. Their plot showed that, for heated sensors, there is a
significant variation with Mach number ({\emph{M}}); cf their Eq.~(38).
The dependence in their plot is represented well by the following
equations, where \(\alpha_r^{[h]}\) refers to heated probes and
\(\alpha_r^{[u]}\) to unheated probes:\\
\begin{equation}
\alpha_r^{[h]} = 0.988+0.053(\log_{10}M)+0.090(\log_{10}M)^2+0.091(\log_{10}M)^3
(\#eq:alphah)
\end{equation} \begin{equation}
\alpha_r^{[u]}=0.9959+0.0283(\log_{10}M)+0.0374(\log_{10}M)^2+0.0762(\log_{10}M)^3
(\#eq:alphau)
\end{equation}

Some studies of the recovery factor are discussed further in this memo
and this technical report.

The true airspeed \(U_a\) is used in @ref(eq:Tair) to calculate the
ambient temperature \(T_a\). However, the ambient temperature is also
needed to calculate the true airspeed. Therefore the constraints imposed
on ambient temperature and true airspeed by the measurements of recovery
temperature, total pressure (the pressure measured by a pitot tube
pointed into the airstream and assumed to be that obtained when the
incoming air is brought to rest relative to the aircraft), and ambient
pressure must be used to solve simultaneously for the two unknowns,
temperature and airspeed. The relationship is conveniently derived by
first calculating the dimensionless Mach number \(M\), which is the
ratio of the airspeed to the speed of sound
\(U_{s}=\sqrt{\gamma^{\prime}R^{\prime}T_{a}}\), where \(\gamma^\prime\)
is the ratio of specific heats of (moist) air,
\(c_p^\prime /c_v^\prime\) and \(R^\prime\) is the gas constant for
moist air. The Mach number is a function of air temperature only and can
be determined as follows:

a). Express energy conservation, as in @ref(eq:Tair), in the form\\
\begin{equation}
d\left(\frac{U^{2}}{2}\right)+c_{p}^{\prime}dT=0\,\,\,\,.
(\#eq:econ)
\end{equation}\\
where the total derivatives apply along a streamline as \(U\) changes
from \(U_a\) to \(U_t=0\) and \(T\) changes from \(T_a\) to \(T_t\).\\
b). Use the perfect gas law to replace \(dT\) with
\(\frac{pV}{nR}(\frac{dV}{V}+\frac{dp}{p})\) where \(V\) and \(p\) are
the volume and pressure of a parcel of air. Then use the expression for
adiabatic compression in the form \(pV^\gamma = \mathrm{constant}\) to
replace the derivative \(\frac{dV}{V}\) with
\(-\frac{1}{\gamma}\frac{dp}{p}\), leading to
\(dT=\frac{R^{\prime}T}{c_{p}^{\prime}}\frac{dp}{p}\) or, after
integration,
\(T(p)=T_{a}\left(\frac{p}{p_{a}}\right)^{R^{\prime}/c_{p}^{\prime}}\).
Using this expression for \(T\) in the formula for \(dT\) and then
integrating both total derivatives in @ref(eq:econ) along the streamline
leads to\\
\begin{equation}
\frac{U_{a}^{2}}{2}+c_{p}^{\prime}T_{a}=c_{p}^{\prime}T_{a}\left(\frac{p_{t}}{p_{a}}\right)^{\frac{R^{\prime}}{c_{p}^{\prime}}}
(\#eq:Ua2)
\end{equation}\\
where \(p_t\) is the total pressure (i.e., PSXC+QCXC) and \(p_a\) is the
ambient pressure (PSXC).\\
c). Use the above definition of the Mach number \(M\) (\(M=U_a/U_s\)) in
the form \(U_a^2=\gamma^\prime M^2 R^\prime T_a\) to obtain:\\
\begin{equation}
M^{2}=\left(\frac{2c_{v}^{\prime}}{R^{\prime}}\right)\left[\left(\frac{p_{t}}{p_{a}}\right)^{\frac{R^{\prime}}{c_{p}^{\prime}}}-1\right]
(\#eq:M2)
\end{equation}\\
which is the same as @ref(eq:MachEquation). This equation shows that
\(M\) can be found from measurements of \(p_t\) and \(p_a\) alone,
except for the moist-air corrections.

d). Use the expression for ambient temperature in terms of recovery
temperature and airspeed, @ref(eq:Tair), to obtain the temperature in
terms of the Mach number and the recovery temperature:\\
\begin{align}
T_{a} & = T_{r}-\alpha_{r}\frac{U_{a}^{2}}{2c_{p}^{\prime}}=T_{r}-\alpha_{r}\frac{M^{2}\gamma^{\prime}R^{\prime}T_{a}}{2c_{p}^{\prime}}\notag\\  
& = \frac{T_{r}}{1+\dfrac{\alpha_{r}M^{2}R^{\prime}}{2c_{v}^{\prime}}}  
(\#eq:TaEQ)
\end{align} \\
e). Express the true airspeed \(U_a\) as\\
\begin{equation}
U_{a}=M\sqrt{\gamma\,^{\prime}R^{\prime}T_{a}}
(\#eq:UaFinal)
\end{equation}

Then the temperature is found as described in the following
box:\footnote{A problem sometimes arises from use of the measured
  humidity, because that measurement might be obviously in error. For
  example, following descents the dew point determined from
  chilled-mirror hygrometers sometimes overshoots the correct value
  significantly, producing dew-point measurements well above the
  measured temperature. If such measurements are used, the result can
  produce a significant error in derived variables based on the
  humidity-corrected gas constant and specific heats. If the
  measurements are flagged as bad, there will be gaps in derived
  variables. To avoid these two errors, the corrections applied to the
  gas constant and specific heats are treated as follows: (i) The
  humidity correction is limited to not more than that given by the
  water-equilibrium humidity at the temperature ATXD, calculated using
  dry-air specific heats and gas constant. (ii) If the humidity from the
  primary sensor is flagged as a missing measurement (e.g., from a
  dew-point sensor), a secondary measurement is used (e.g., the VCSEL)
  in cases when the secondary sensor is almost always present in an
  experiment. (iii) As a backup, the variables TASxD and ATxD are always
  calculated omitting the humidity correction to the gas constant and
  the specific heats. These variables usually provide continuous
  measurements, although they will be offset from the humidity-corrected
  values. The offset indicates the magnitude of the correction when both
  are present, and one of the variables TASxD (ATxD) may be selected as
  TASX (ATX) in cases where missing values might cause a problem for
  derived variables.}

RTX\index{RTX} = recovery temperature (\(T_{r})\)\\
QCxC\index{QCxC} = dynamic pressure, corrected (\(q_{a}\))\\
PSXC\index{PSXC} = ambient pressure, after airflow/location correction
(\(p_{a}\))\\
MACHx\index{MACHx} = Mach number based on QCxC and PSXC;
cf.~@ref(eq:M2)\\
MACHX = best Mach number, based on QCXC and PSXC\\
\(\alpha_{r}\) = recovery factor for the particular temperature sensor\\
\(R^{\prime}\), \(c_{v}^{\prime}\) and \(c_{p}^{\prime}\) as defined
above and in the list of symbols

From @ref(eq:M2), \begin{equation}
\mathrm{MACHx}=\left\{ \left(\frac{2c_{v}^{\prime}}{R^{\prime}}\right)\left[\left(\frac{\mathrm{\{PSXC\}+\{QCxC\}}}{\mathrm{\{PSXC\}}}\right)^{\frac{R^{\prime}}{c_{p}^{\prime}}}-1\right]\right\} ^{1/2}  
(\#eq:MachBox)
\end{equation}\\
From @ref(eq:TaEQ), \begin{equation}
\mathrm{ATx}=\frac{\mathrm{\left(\{RTx\}+T_{0}\right)}}{\left(1+\dfrac{\alpha_{r}\mathrm{(\{MACHX\})}^{2}R^{\prime}}{2c_{v}^{\prime}}\right)}-T_{0}
(\#eq:ATbox)
\end{equation}

\hypertarget{AT-ITR}{%
\subsubsection*{\texorpdfstring{In-cloud Air Temperature, Radiometric
({∘}C):
AT\_ITR}{In-cloud Air Temperature, Radiometric (∘C): AT\_ITR}}\label{AT-ITR}}
\addcontentsline{toc}{subsubsection}{In-cloud Air Temperature,
Radiometric ({∘}C): AT\_ITR}

\emph{The radiometric ambient air temperature measured by the In-cloud
Air Temperature Radiometer,} which measures the radiometric temperature
in the 4.3 {\emph{μ}}m CO{2}
band.\protect\hypertarget{AT_ITR}{}{{[}AT\_ITR{]}} Its primary use is in
water cloud when the standard thermometers are affected by wetting. In
clear air the temperature is an average over an integrating range of up
to 100s of meters away from the aircraft, whereas in clouds the
integrating range is as little as 10 meters because of water droplets.
The calibration is by a polynomial fit of the internal reference
temperature and measured radiance to the ATX temperature outside of
clouds.\textgreater{}

\hypertarget{ophir-air-temperature-c-oat--oat-obsolete}{%
\subsubsection{\texorpdfstring{Ophir Air Temperature ({∘}C): OAT
\{-\#OAT\}
\emph{(obsolete)}}{Ophir Air Temperature (∘C): OAT \{-\#OAT\} (obsolete)}}\label{ophir-air-temperature-c-oat--oat-obsolete}}

\emph{The radiometric temperature reported by the Ophir III radiometer,}
which operates on the same principles as the
ITR,\protect\hypertarget{OAT}{}{{[}OAT{]}} with the same limitations.
The in-cloud air temperature radiometer is a later, improved version.
The Ophir III has been retired.

\hypertarget{humidity}{%
\section{Humidity}\label{humidity}}

\hypertarget{dew-point}{%
\subsubsection*{\texorpdfstring{Dew/Frost Point ({º}C): DPx, DP\_x,
DP\_DPx, MIRRTMP\_DPx See below for
DP\_VXL.}{Dew/Frost Point (ºC): DPx, DP\_x, DP\_DPx, MIRRTMP\_DPx See below for DP\_VXL.}}\label{dew-point}}
\addcontentsline{toc}{subsubsection}{Dew/Frost Point ({º}C): DPx, DP\_x,
DP\_DPx, MIRRTMP\_DPx See below for DP\_VXL.}

\emph{The mirror temperature measured directly by a dew-point sensor,
without correction.} The dew point or frost point is measured by either
an EG\&G Model 137, a General Eastern Model 1011B or a Buck Model 1011C
dew-point hygrometer. Below 0\textbf{{∘}C} the instrument is assumed to
be responding to the frost point, although occasionally in climbs there
is a short transition near the freezing level before the condensate on
the mirror of the instrument freezes and there may be a measurement
error before the condensate freezes. The measurements are usually made
within a housing where the pressure ({\emph{p}\emph{h})} may differ from
the ambient pressure, so the pressure in the housing affects the
measured dew point or frost point. The housing pressure is often
adjusted to be near the ambient pressure by appropriate orientation of
inlets, and recently the pressure in the housing is measured and a
correction is applied, as discussed in the next paragraph.

\hypertarget{dewpt-corrected}{%
\subsubsection*{\texorpdfstring{Corrected Dew Point ({º}C): DPXC,
DPxC\footnote{See also DP\_VXL and DP\_CR2C below.}}{Corrected Dew Point (ºC): DPXC, DPxC}}\label{dewpt-corrected}}
\addcontentsline{toc}{subsubsection}{Corrected Dew Point ({º}C): DPXC,
DPxC}

\emph{The dew point obtained from the original measurement after
correction for the housing pressure, the enhancement of the equilibrium
vapor pressure arising from the total pressure (discussed below), and
conversion from frost point if appropriate,} The result is the
temperature at which the equilibrium vapor pressure over a plane water
surface in the absence of other gases would match the actual water-vapor
pressure. Dew/frost-point hygrometers measure the equilibrium point in
the presence of air, and the presence of air affects the measurement in
a minor way that is represented by a small correction here named the
``enhancement factor.'' In the case where the dew-point or frost-point
sensor is exposed to ambient air directly, the enhancement factor is
defined so that the ambient water vapor pressure \(e_a\) is related to
\(T_p\), the \emph{measured} dew or frost point \emph{in the presence of
air} having total pressure \(p\), by \(e_a=f(p,T_P)e_s(T_p)\) where
\(e_s(T_p)\) is the vapor pressure in equilibrium with ice or water at
the dew or frost point \(T_p\) \emph{in the absence of air.} Calculation
of DPxC removes this dependence, so the vapor pressure obtained from
\(e_s(\{DPxC\})\) will be that vapor pressure corresponding to
equilibrium \emph{in the absence of air.} In addition, if the
measurement is below \(0^\circ\mathrm{C}\), it is assumed to be a
measurement of frost point and a corresponding dew point is calculated
from the measurement (also with correction for the influence of the
total pressure on the measurement). Some changes were made to these
calculations in 2011; for more information, see this memo. An additional
correction is needed in those cases where the pressure in the housing of
the instrument (measured as PSDPx or CAVP\_x) differs from the ambient
pressure, because the changed pressure affects the partial pressure of
water vapor in proportion to the change in total pressure and so changes
the measured dew point from the desired quantity (that in the ambient
air) to that in the housing. This is especially important in the case of
the GV because the potential effect increases with airspeed. If the
pressure in the housing is measured or otherwise known (e.g., from
correlations with other measurements), then this correction can be
introduced into the processing algorithm at the same time that the
correction for the presence of dry air is introduced, and the
enhancement factor should be evaluated at the pressure in the housing.

The relationship between water-vapor pressure and dew- or frost-point
temperature is based on the Murphy and Koop\footnote{Q. J. R. Meteorol.
  Soc. (2005), 131, pp.~1539--1565} (2005) equations.\footnote{Prior to
  2010, the vapor pressure relationship used was the Goff-Gratch formula
  as given in the Smithsonian Tables (List, 1980).} They express the
equilibrium vapor pressure as a function of frost point or dew point
\emph{and at a total air pressure \(p\)} via equations that are
equivalent to the following:\\
\begin{equation}
e_{s,i}(T_{FP})= b_{0}^{\prime}\exp(b_{1}\frac{(T_{0}-T_{FP})}{T_{0}T_{FP}}+b_{2}\ln(\frac{T_{FP}}{T_{0}})+b_{3}(T_{FP}-T_{0}))
(\#eq:MKi)
\end{equation} \begin{equation}
e_{s,w}(T_{DP})=c_{0}\exp\left((\alpha-1)c_{6}+d_{2}(\frac{T_{0}-T_{DP}}{T_{DP}T_{0}})\right)+d_{3}\ln(\frac{T_{DP}}{T_{0}})+d_{4}(T_{DP}-T_{0})
(\#eq:MKw)
\end{equation} \begin{equation}
f(p, T_P)= 1 + p(f_1 + f_2T_P + f_3T_P^2)
(\#eq:enhance)
\end{equation} where \(e\) is the water vapor pressure, \(T_{FP}\) or
\(T_{DP}\) is the frost or dew point, respectively, expressed in kelvin,
\(T_0\) = 273.15 K, \(e_{s,i}(T_{FP})\) is the equilibrium vapor
pressure over a plane ice surface at the temperature \(T_{FP}\),
\(e_{s,w}(T_{DP})\) is the equilibrium vapor pressure over a plane water
surface at the temperature \(T_{DP}\) (above or below \(T_0\)), and
\(f(p,T_P)\) is the enhancement factor at total air pressure \(p\) and
temperature \(T_P\), with \(T_P\) equal to \(T_{DP}−T_0\) when above
\(T_0\) and \(T_{FP}−T_0\) when below \(0^\circ\)C.\\
The coefficients used in the above formulas are given in the following
tables, with the additional definitions that
\(\alpha_T=\tanh(c_5(T-T_x))\), \(T_X=218.8\)~K, and
\(d_i=c_i+\alpha_Tc_{i+5}\) for i = \{2,3,4\}:

\begin{table}
\begin{wraptable}{l}{0pt}
\begin{tabular}[c]{c|c}
\hline
coefficient & value\\
\hline
\$b\_0\textasciicircum{}\textbackslash{}prime\$ & \$6.11536\$ hPa\\
\hline
\$b\_1\$ & \$-5723.265\$ K\\
\hline
\$b\_2\$ & \$3.53068\$\\
\hline
\$b\_3\$ & \$-0.00728332\$ K\$\textasciicircum{}\{-1\}\$\\
\hline
\$f\_1\$ & \$4.923\textbackslash{}times 10\textasciicircum{}\{-5\}\$ hPa\$\textasciicircum{}\{-1\}\$\\
\hline
\$f\_2\$ & \$-3.25\textbackslash{}times 10\textasciicircum{}\{-7\}\$hPa\$\textasciicircum{}\{-1\}\$K\$\textasciicircum{}\{-1\}\$\\
\hline
\$f\_3\$ & \$5.84\textbackslash{}times 10\textasciicircum{}\{-10\}\$hPa\$\textasciicircum{}\{-1\}\$K\$\textasciicircum{}\{-2\}\$\\
\hline
\end{tabular}\end{wraptable}\begin{wraptable}{r}{0pt}
\begin{tabular}[c]{c|c}
\hline
coefficient & value\\
\hline
\$c\_0\$ & \$6.091886\$ hPa\\
\hline
\$c\_1\$ & \$6.564725\$\\
\hline
\$c\_2\$ & \$-6763.22\$ K\\
\hline
\$c\_3\$ & \$-4.210\$\\
\hline
\$c\_4\$ & \$0.000367\$ K\$\textasciicircum{}\{-1\}\$\\
\hline
\$c\_5\$ & \$0.0415\$ K\$\textasciicircum{}\{-1\}\$\\
\hline
\$c\_6\$ & \$-0.1525967\$\\
\hline
\$c\_7\$ & \$-1331.22\$ K\\
\hline
\$c\_8\$ & \$-9.44523\$\\
\hline
\$c\_9\$ & \$0.014025\$ K\$\textasciicircum{}\{-1\}\$\\
\hline
\end{tabular}\end{wraptable}
\end{table}

The vapor pressure in the instrument housing, \(e_h\), is related to the
sensed dew or frost point according to equation @ref(eq:MKw) or
@ref(eq:MKi), but further corrections must also be made for the
enhancement factor and to account for possible difference between the
pressure in the sensor housing (\(p_h\)) and the ambient pressure
(\(p_a\)):\\
\begin{equation}
e_{a}=f(p_{a},T_{p})e_{h}\frac{p_{a}}{p_{h}}
(\#eq:housingCorr)
\end{equation} Because processing to obtain the corrected dew point DPxC
from the ambient vapor pressure \(e_a\) would require difficult
inversion of the above formulas, interpolation is used instead. A table
constructed from @ref(eq:MKw) and another constructed from @ref(eq:MKi),
giving water vapor pressure as a function of frost point or dew point
temperature in \(1^\circ\mathrm{C}\) increments from \(-100\) to
\(+50^\circ\mathrm{C}\), is then used with three-point Lagrange
interpolation (via a function described below as \(F_D(e)\)) to find the
dew point temperature from the vapor pressure.\footnote{prior to 2011
  the conversion was made using the formula
  {\emph{DPxC} = 0.009109 + \emph{DPx}(1.134055 + 0.001038\emph{DPx})}.
  For instruments producing measurements of vapor density (RHO), the
  previous Bulletin 9 section incorrectly gave the conversion formula as
  {\emph{DPxC} = 273.0\emph{Z}/(22.51 − \emph{Z})}, a conversion that
  would apply to frost point, not dew point. However, the code in use
  shows that the conversion was instead
  {237.3\emph{Z}/(17.27 − \emph{Z})}, where Z in both cases is
  {\emph{Z} = ln ((\emph{ATX} + 273.15)\emph{RHO}/1322.3)}.}

Tests of these interpolation formulas against high-accuracy numerical
inversion of formulas @ref(eq:MKw) and @ref(eq:MKi) showed that the
maximum error introduced by the interpolation formula was about
\(0.004^\circ\mathrm{C}\) and the standard error about
\(0.001^\circ\mathrm{C}\). This inversion then provides a corrected dew
point that incorporates the effects of the enhancement factor as well as
differences between the ambient pressure and that in the housing. The
algorithm is documented in the box below.

\(T_{p}\) = \protect\hyperlink{dew-point}{DPx} from instrument x
{[}\(^{\circ}\)C{]}, or alternately\\
\protect\hyperlink{rho}{RHO} = water vapor density measurement
{[}\(\mathrm{g\ }\mathrm{m}^{-3}\){]}; only one is used in any
calculation\\
\protect\hyperlink{ambient-t}{ATX} = reference ambient temperature
{[}\(^{\circ}C\){]}\\
\(T_{K}\)=ATX+\(T_{0}\)~\(^{\dagger}\) = ambient temperature {[}K{]}\\
\(p\) = \protect\hyperlink{psx}{PSXC} = reference ambient pressure
{[}hPa{]}\\
\(p_{h}\) = \protect\hyperlink{p-special}{CAVP}\_x = pressure in
instrument ``x'' housing {[}hPa{]}\\
\(e_{t}\) = intermediate vapor pressure used for calculation only\\
\(e\) = \protect\hyperlink{ewx}{EWx} = water vapor pressure from source
x {[}hPa{]}\\
\(M_{w}\) = molecular weight of water\(^{\dagger}\)\\
\(R_{0}\) = universal gas constant\(^{\dagger}\)\\
\(f(p_{h},T_{p})\) = enhancement factor (cf.~@ref(eq:enhance))\\
\(F_{d}(e)\) = interpolation formula giving dew point temperature from
water vapor pressure

For dew/frost point hygrometers, producing the measurement DPx:\\
\hspace*{0.333em}\hspace*{0.333em}\hspace*{0.333em}\hspace*{0.333em}if
DPx \textless{} 0\(^\circ\)C:\\
\hspace*{0.333em}\hspace*{0.333em}\hspace*{0.333em}\hspace*{0.333em}\hspace*{0.333em}\hspace*{0.333em}\hspace*{0.333em}\hspace*{0.333em}obtain
\(e_{t}\) from @ref(eq:MKi) using \(T_{FP}\)=DPx + \(T_{0}\)\\
\hspace*{0.333em}\hspace*{0.333em}\hspace*{0.333em}\hspace*{0.333em}else
(i.e., DPx \(\geq 0^\circ\)C):\\
\hspace*{0.333em}\hspace*{0.333em}\hspace*{0.333em}\hspace*{0.333em}\hspace*{0.333em}\hspace*{0.333em}\hspace*{0.333em}\hspace*{0.333em}obtain
\(e_{t}\) from @ref(eq:MKw) using \(T_{DP}=\mathrm{DPx}+T_{0}\)\\
\hspace*{0.333em}\hspace*{0.333em}\hspace*{0.333em}\hspace*{0.333em}correct
\(e_{t}\) for enhancement factor and internal pressure to get ambient
vapor pressure \(e\):\\
\begin{equation}
e=f(p_{h},T_{P})\,(\frac{p}{p_{h}})\,e_t
(\#eq:ebox)
\end{equation} ~~~~obtain DPxC by finding the dew point corresponding to
the vapor pressure \(e\):\\
\begin{equation}
\mathrm{\{DPxC\}} = F_{d}(e) 
(\#eq:DPxCbox)
\end{equation} - - - - - - - - - - - - - - - - - - - -\\
For other instruments producing measurements of vapor density (RHO
{[}g~m\(^{-3}\){]}:\textsuperscript{(a)}\\
\hspace*{0.333em}\hspace*{0.333em}\hspace*{0.333em}\hspace*{0.333em}find
the water vapor pressure in units of hPa: \begin{equation}
e = (\mathrm{\{RHO\}}\,R_{0}\,T_{K}\,/\,M_{w})\times 10^{-5}
(\#eq:RHObox)
\end{equation} ~~~~find the equivalent dew point:\\
\begin{equation}
\mathrm{\{DPxC\}} = F_{d}(e)
(\#eq:DPxC2)
\end{equation} \_\_\_\_\_\_\_\_\_\_\\
\textsuperscript{(a)} prior to 2011 the following formula was used:
\[Z=\frac{\ln((\mathrm{\{ATX\}}+273.15)\,\mathrm{\{RHO\}}}{1322.3}\]\\
\[\mathrm{\{DPxC\}}=\frac{273.0\,Z}{(22.51-Z)}\]

For other instruments that measure vapor density, such as a Lyman-alpha
or tunable diode laser hygrometers (including the Vertical Cavity
Surface Emitting Laser (VCSEL) hygrometer), a similar conversion is made
from vapor density to dew point, as described in the next paragraph.

\hypertarget{vcsel-dp}{%
\subsubsection*{\texorpdfstring{Dew Point Determined from the VCSEL
Hygrometer ({∘}C):
DP\_VXL}{Dew Point Determined from the VCSEL Hygrometer (∘C): DP\_VXL}}\label{vcsel-dp}}
\addcontentsline{toc}{subsubsection}{Dew Point Determined from the VCSEL
Hygrometer ({∘}C): DP\_VXL}

\emph{The dew point temperature determined from the measured water vapor
density from the VCSEL hygrometer.} The calculation is as described at
the bottom of the box immediately above this paragraph (above the
footnote). The water vapor density converted from a molecular density
{[}molecules\(\,\)cm\(^{-3}\){]} to a mass density
{[}g\(\,\)m\(^{-3}\){]} via\footnote{The conversion factor is given by
  this formula:
  {\[C^{\prime}=\frac{10^{6}\mathrm{cm}^{3}}{\mathrm{m}^{3}}\times\frac{M_{W}^{\dagger}}{N_{A}^{\dagger}}\]}
  where {\emph{N}\emph{A}} is the Avogadro constant, 6.022147{ × 1026}
  molecules~kmol{ − 1}.} \{CONCV\_VXL\} * \(2.9915\times 10^{17}\) is
used for \{RHO\}. DP\_VXL is given by DPxC on the last line of that
algorithm box. See CONCV\_VXL below.

\hypertarget{mirror-cr2}{%
\subsubsection*{\texorpdfstring{Frost Point Temperature from the CR2
Cryogenic Hygrometer ({∘}C): FP\_CR2,
MIRRORT\_CR2}{Frost Point Temperature from the CR2 Cryogenic Hygrometer (∘C): FP\_CR2, MIRRORT\_CR2}}\label{mirror-cr2}}
\addcontentsline{toc}{subsubsection}{Frost Point Temperature from the
CR2 Cryogenic Hygrometer ({∘}C): FP\_CR2, MIRRORT\_CR2}

\emph{The mirror temperature in the CR2 cryogenic hygrometer,} which is
normally the frost point inside the measuring chamber of the
instrument\emph{.} The measurement is often suspect when the value is
above about -15{∘}C; the measurement is intended for use below this
value. The CR2 is a cabin-mounted instrument, so the measured pressure
(P\_CR2) in the instrument must be used with the ambient pressure (PSXC)
to convert the measurement to ambient humidity measures like DP\_CR2 and
EW\_CR2.

\hypertarget{dp-cr2}{%
\subsubsection*{\texorpdfstring{Corrected Dew Point Temperature from the
CR2 Cryogenic Hygrometer ({∘}C):
DP\_CR2C}{Corrected Dew Point Temperature from the CR2 Cryogenic Hygrometer (∘C): DP\_CR2C}}\label{dp-cr2}}
\addcontentsline{toc}{subsubsection}{Corrected Dew Point Temperature
from the CR2 Cryogenic Hygrometer ({∘}C): DP\_CR2C}

\emph{The dew point temperature corresponding to equilibrium at the
ambient humidity,} as determined by the CR2 hygrometer. The measurement
of the mirror temperature inside the CR2, FP\_CR2, is converted to a
vapor pressure assuming equilibrium water vapor pressure relative to a
plane ice surface at that temperature, and the resulting vapor pressure
is converted to an ambient value via the assumption that the ratio of
vapor pressure internal to the instrument to ambient vapor pressure is
the same as the corresponding total pressure ratio. The resulting
ambient vapor pressure (EW\_CR2) is then converted to an equivalent
ambient dew point. The steps are the same as those in the algorithm box
for DPxC above, with these substitutions: FP\_CR2 is used for DPx and
P\_CR2 for \(p_h\).

\hypertarget{vcsel-uncor}{%
\subsubsection*{\texorpdfstring{Uncorrected Water Vapor Number Density
from the VCSEL Hygrometer (molecules cm{ − 3}):
\textgreater RAWCONC\_VXL}{Uncorrected Water Vapor Number Density from the VCSEL Hygrometer (molecules cm − 3): \textgreater RAWCONC\_VXL}}\label{vcsel-uncor}}
\addcontentsline{toc}{subsubsection}{Uncorrected Water Vapor Number
Density from the VCSEL Hygrometer (molecules cm{ − 3}):
\textgreater RAWCONC\_VXL}

\emph{The uncorrected water vapor number density reported by the VCSEL
hygrometer.} This is determined by comparing the measured absorption
peak height against a reference spectrum generated using the HITRAN
spectral parameters, the ambient temperature and the ambient
pressure.\footnote{For details see Zondlo, M. A., M. E. Paige, S. M.
  Massick, and J. A. Silver, 2010: Vertical cavity laser hygrometer for
  the National Science Foundation Gulfstream-V aircraft. \emph{J.
  Geophys. Res.,} \textbf{115,} D20309, \url{doi:10.1029/2010JD014445}.}

\hypertarget{vcsel-corr}{%
\subsubsection*{\texorpdfstring{Corrected Water Vapor Concentration from
the VCSEL Hygrometer (molecules cm{ − 3}):
CONCV\_VXL}{Corrected Water Vapor Concentration from the VCSEL Hygrometer (molecules cm − 3): CONCV\_VXL}}\label{vcsel-corr}}
\addcontentsline{toc}{subsubsection}{Corrected Water Vapor Concentration
from the VCSEL Hygrometer (molecules cm{ − 3}): CONCV\_VXL}

\emph{The corrected water vapor number density produced by the VCSEL
hygrometer,} after minor corrections for ambient temperature, pressure,
laser intensity and water vapor concentration. For more information on
calibration and data processing for this instrument, see the instrument
web page and additional documentation there.

\hypertarget{uvh-voltage}{%
\subsubsection*{Voltage Output from the UV Hygrometer (V):
XSIGV\_UVH}\label{uvh-voltage}}
\addcontentsline{toc}{subsubsection}{Voltage Output from the UV
Hygrometer (V): XSIGV\_UVH}

\emph{The voltage from a modern (as of 2012) version of the Lyman-alpha
hygrometer,} which provides a signal that represents water vapor
density. The instrument also provides measurements of pressure and
temperature inside the sensing cavity; they are, respectively,
XCELLPRES\_UVH and XCELLTEMP\_UVH. See the discussion of EW\_UVH below
for the data-processing algorithm that uses this variable.

\hypertarget{uvh-n}{%
\subsubsection*{\texorpdfstring{Water Vapor Number Density from the UV
Hygrometer (molecules cm{ − 3}):
CONCH\_UVH}{Water Vapor Number Density from the UV Hygrometer (molecules cm − 3): CONCH\_UVH}}\label{uvh-n}}
\addcontentsline{toc}{subsubsection}{Water Vapor Number Density from the
UV Hygrometer (molecules cm{ − 3}): CONCH\_UVH}

\emph{Water vapor number density (or concentration of molecules)
measured by the UV Hygrometer.} This is the direct measurement from the
instrument. Its calculation relies on a bench calibration that fits the
water vapor number density to the Beers-Lambert absorption law and
corrects for output offsets and the effect of UV absorption by
atmospheric constituents other than water vapor. See also the discussion
of EW\_UVH in the paragraph that immediately follows.

\hypertarget{ewx}{%
\subsubsection*{Water Vapor Pressure (hPa): EWx, EWX, EW\_VXL, EW\_UVH,
EW\_VXL, EDPC (obsolete)}\label{ewx}}
\addcontentsline{toc}{subsubsection}{Water Vapor Pressure (hPa): EWx,
EWX, EW\_VXL, EW\_UVH, EW\_VXL, EDPC (obsolete)}

\emph{The ambient vapor pressure of water,} also used in the calculation
of several derived variables. It is often obtained from an instrument
measuring dew point or water vapor density. In the case where it is
derived from a measurement of dew point (DPx), a correction is applied
for the enhancement factor that influences dew point or frost point
measurements.\footnote{prior to 2011, this variable was calculated using
  the Goff-Gratch formula. See the discussion of DPXC for more
  information on previous calculations.} The formula for obtaining the
ambient water vapor pressure as a function of dew point is given in the
discussion of DPxC above, Eqs.~@ref(eq:MKw) and @ref(eq:enhance), where
the calculation of the variables EWx and EWX are also discussed. EWX (or
previously EDPC) is the preferred variable that is selected from among
the possibilities \{EWx\} for subsequent calculation of derived
variables. For the case where water vapor pressure is determined by the
VCSEL hygrometer, EW\_VXL is determined from CONCV\_VXL:
EW\_VXL={\emph{C}}{\emph{k}}\{CONCV\_VXL\}\{ATX+273.15) where {\emph{k}}
is the Boltzmann constant and {\emph{C} = 10 − 4}(cm/m){3}(hPa/Pa)
converts units to hPa. In the case where the water vapor pressure is
determined from the UV Hygrometer data, this variable is calculated
using one of two methods:

\begin{enumerate}
\def\labelenumi{\arabic{enumi}.}
\tightlist
\item
  Using the ideal gas law to convert the water vapor number density from
  the UV Hygrometer to water vapor pressure, using XCELLTEMP\_UVH and
  XCELLPRES\_UVH, the measured temperature and pressure in the
  absorption cell, via the equation:\\
  \begin{equation}
  \mathrm{\{EW\_UVH\}=C\,\{CONC\_UVH\}\notag \\
  \times\frac{k\,(\mathrm{\{XCELLTEMP\_UVH\}+273.15)\,\mathrm{\{PSX\}}}}{\mathrm{{\{XCELLPRES\_UVH\}}}}}
  (\#eq:UVH1)
  \end{equation} or\\
\item
  Through use of a polynomial fit with coefficients fitted to \{EWX\}:\\
  \begin{equation}
  \mathrm{\{EWX\}}=c_0 + c_1(\mathrm{\{XSIGV\_UVH\}}) + c_2(\mathrm{\{XSIGV\_UVH\}}^2)
  (\#eq:UVH2)
  \end{equation}\\
  where \{EWX\} is a reference water vapor pressure provided by another
  instrument. This preserves the fast-response characteristics of the UV
  hygrometer while linking the absolute values to a baseline provided by
  a more stable instrument. This can be done on a flight-by-flight basis
  and largely eliminates drift.\footnote{For more details see Beaton, S.
    P. and M. Spowart, 2012: UV Absorption Hygrometer for Fast-Response
    Airborne Water Vapor Measurements. \emph{J. Atmos. Oceanic
    Technol.,} \textbf{\emph{29.}} DOI: 10.1175/JTECH-D-11-00141.1} See
  the project reports to determine which method was used for a
  particular project.
\end{enumerate}

\hypertarget{rhumw}{%
\subsubsection*{Relative Humidity (per cent or Pa/hPa):
RHUM}\label{rhumw}}
\addcontentsline{toc}{subsubsection}{Relative Humidity (per cent or
Pa/hPa): RHUM}

\emph{The ratio of the water vapor pressure to the water vapor pressure
in equilibrium over a plane} liquid\emph{-water surface,} scaled to
express the result in units of per cent or Pa/hPa:

\protect\hyperlink{ewx}{EWX} = atmospheric water vapor pressure (hPa)\\
\protect\hyperlink{ambient-t}{ATX} = ambient air temperature
{[}\(^{\circ}\mathrm{C}\){]}\\
\(T_{0}=273.15\) K\\
\(e_{s.w}(\mathrm{ATX+T_{0}})\) = equilibrium water vapor pressure at
\emph{dewpoint} ATX (hPa)\\
\hspace*{0.333em}\hspace*{0.333em}\hspace*{0.333em}\hspace*{0.333em}\hspace*{0.333em}\hspace*{0.333em}\hspace*{0.333em}\hspace*{0.333em}\hspace*{0.333em}\hspace*{0.333em}(see
Eq.~@ref(eq:MKw) for the formula used.)

\begin{equation}
\mathrm{\{RHUM\}}=100\%\,\times\,\frac{\mathrm{\{EWX\}}}{e_{s,w}(\mathrm{\{ATX\}+T_{0}})}
(\#eq:RHUM)
\end{equation}

To follow normal conventions, the change in equilibrium vapor pressure
that arises from the enhancement factor is not included in the
calculated relative humidity, even though the true relative humidity
should include the enhancement factor as specified in @ref(eq:enhance)
in the denominator of @ref(eq:RHUM).

\hypertarget{rhumi}{%
\subsubsection*{Relative Humidity with respect to Ice (per cent or
Pa/hPa): RHUMI}\label{rhumi}}
\addcontentsline{toc}{subsubsection}{Relative Humidity with respect to
Ice (per cent or Pa/hPa): RHUMI}

\emph{The ratio of the water vapor pressure to the water vapor pressure
in equilibrium over a plane} ice \emph{surface,} scaled to express the
result in units of per cent or Pa/hPa:

\protect\hyperlink{ewx}{EWX} = atmospheric water vapor pressure (hPa)\\
\protect\hyperlink{ambient-t}{ATX} = ambient air temperature
{[}\(^{\circ}\mathrm{C}\){]}\\
\(T_{0}=273.15\) K\\
\(e_{s.i}(\mathrm{ATX+T_{0}})\) = equilibrium water vapor pressure at
\emph{frostpoint} ATX (hPa)\\
\hspace*{0.333em}\hspace*{0.333em}\hspace*{0.333em}\hspace*{0.333em}\hspace*{0.333em}\hspace*{0.333em}\hspace*{0.333em}\hspace*{0.333em}\hspace*{0.333em}\hspace*{0.333em}(see
Eq.~@ref(eq:MKi) for the formula used.)

\begin{equation}
\mathrm{\{RHUMI\}}=100\%\,\times\,\frac{\mathrm{\{EWX\}}}{e_{s,i}(\mathrm{\{ATX\}+T_{0}})}
(\#eq:RHUMI)
\end{equation}

To follow normal conventions, the change in equilibrium vapor pressure
that arises from the enhancement factor is not included in the
calculated relative humidity, even though the true relative humidity
should include the enhancement factor as specified in @ref(eq:enhance)
in the denominator of @ref(eq:RHUMI).

\hypertarget{rho}{%
\subsubsection*{\texorpdfstring{Absolute Humidity, Water Vapor Density
(g/m{3}):RHOx}{Absolute Humidity, Water Vapor Density (g/m3):RHOx}}\label{rho}}
\addcontentsline{toc}{subsubsection}{Absolute Humidity, Water Vapor
Density (g/m{3}):RHOx}

\emph{The water vapor density computed from various measurements of
humidity as indicated by the 'x' suffix,} and conventionally expressed
in units of g kg{ − 1} or per mille. The calculation proceeds in
different ways for different sensors. For sensors that measure a
chilled-mirror temperature, the calculation is based on the equation of
state for a perfect gas and uses the water vapor pressure determined by
the instrument, as in the following box:

\protect\hyperlink{ambient-t}{ATX} = ambient temperature
(\(^{\circ}\mathrm{C}\))\\
\protect\hyperlink{ewx}{EWX} = water vapor pressure, hPa\\
\(C_{mb2Pa}\)= conversion factor, hPa to Pa\} = 100 Pa~hPa\(^{-1}\)
(conversion factor to MKS units)\\
\(C_{kg2g}=10^{3}\,\mathrm{g\,kg}^{-1}\) = (conversion factor to give
final units of g\(\,\)m\(^{-3}\))\\
\(T_{0}\) = 273.15,K

\begin{equation}
\mathrm{\{RHOx\}} = C_{kg2g}\frac{C_{mb2Pa}\mathrm{\{EWX\}}}{R_{w}\mathrm{(\{ATX\}+T_{0})}}
(\#eq:RHOx)
\end{equation}

For instruments measuring the vapor pressure density (including the
Lyman-alpha probes and the newer version called the UV hygrometer), the
basic measurement from the instrument is the water vapor density,
\textbf{\underline{RHOUV}} or **** \textbf{\underline{RHOLA}},
determined by applying calibration coefficients to the measured signals
(XUVI or VLA). In addition, a slow update to a dew-point measurement is
used to compensate for drift in the calibration. The processing used for
early projects with the Lyman-alpha instruments is similar but more
involved and won't be documented here because the instruments are
obsolete. See RAF Bulletin 9 for the processing previously used for
archived measurements from the Lyman-alpha hygrometers.\\

This algorithm for RHOUV is missing.

\hypertarget{sphum}{%
\subsubsection*{Specific Humidity (g/kg): SPHUM}\label{sphum}}
\addcontentsline{toc}{subsubsection}{Specific Humidity (g/kg): SPHUM}

\emph{The mass of water vapor per unit mass of (moist) air,
conventionally measured in units of g/kg or per mille.}

\protect\hyperlink{psx}{PSXC} = ambient total air pressure. hPa\\
\protect\hyperlink{ewx}{EWX} = ambient water vapor pressure, hPa\\
\(C_{kg2g}=10^{3}\,\)g\(\,\)kg\(^{-1}\) (conversion factor to give final
units of g\(\,\)kg\(^{-1}\))\\
\(M_{w}=\)molecular weight of water\(^{\dagger}\)\\
\(M_{d}=\)molecular weight of dry air\(^{\dagger}\)

\begin{equation}
\mathrm{\{SPHUM\}} = C_{kg2g}\frac{M_{w}}{M_{d}}(\mathrm{\frac{\{EWX\}}{\mathrm{\{PSXC\}-(1-\frac{M_{w}}{M_{d}})\{\mathrm{EWX}\}}}})
(\#eq:SPHUM)
\end{equation}

\hypertarget{MR}{%
\subsubsection*{Mixing Ratio (g/kg): MR, MRCR, MRLA, MRLA1,
MRLH}\label{MR}}
\addcontentsline{toc}{subsubsection}{Mixing Ratio (g/kg): MR, MRCR,
MRLA, MRLA1, MRLH}

\emph{The ratio of the mass of water to the mass of dry air in the same
volume of air,} conventionally expressed in units of g/kg or per mille.
Mixing ratios may be calculated for the various instruments measuring
humidity on the aircraft, and the variable names reflect the source: MR
from the dewpoint hygrometers, MRCR from the cryogenic hygrometer, MRLA
from the Lyman-alpha sensor, MRLA1 if there is a second Lyman-alpha
sensor, MRLH from a tunable-diode laser hygrometer, and MRVXL from the
VCSEL hygrometer (also a laser hygrometer). The example in the box below
is for the case of the dewpoint hygrometers; others are analogous.

\protect\hyperlink{psx}{PSXC} = ambient total air pressure. hPa\\
\protect\hyperlink{ewx}{EWX} = ambient water vapor pressure, hPa\\
\(C_{kg2g}=10^{3}\,\)g\(\,\)kg\(^{-1}\) (conversion factor to give final
units of g\(\,\)kg\(^{-1}\))\\
\(M_{w}=\)molecular weight of water\(^{\dagger}\)\\
\(M_{d}=\)molecular weight of dry air\(^{\dagger}\)

\begin{equation}
\mathrm{\{MR\}}=C_{kg2g}\frac{M_{w}}{M_{d}}\frac{\mathrm{\{EWX\}}}{(\mathrm{\{PSXC\}-\{EWX\})}}
(\#eq:MR)
\end{equation}

\hypertarget{derived-thermodynamic-variables}{%
\section{Derived Thermodynamic
Variables}\label{derived-thermodynamic-variables}}

\hypertarget{theta}{%
\subsubsection*{Potential Temperature (K): THETA}\label{theta}}
\addcontentsline{toc}{subsubsection}{Potential Temperature (K): THETA}

\emph{The absolute temperature reached if a dry parcel at the measured
pressure and temperature were to be compressed or expanded adiabatically
to a pressure of 1000 hPa}. It does not take into account the difference
in specific heats caused by the presence of water vapor, and water vapor
can change the exponent in the formula below enough to produce errors of
1 K or more.

\protect\hyperlink{ambient-t}{ATX} = ambient temperature,
\(^{\circ}\)C\\
\protect\hyperlink{psx}{PSXC} = ambient pressure (hPa)\\
\(p_{0}\) = reference pressure = 1000 hPa\\
\(R_{d}\) = gas constant for dry air\(^{\dagger}\)\\
\(c_{pd}\) = specific heat at constant pressure for dry
air\(^{\dagger}\) \(T_0=273.15\,\mathrm{K}\)

\begin{equation}
\mathrm{\{THETA\}}=\left(\mathrm{\{ATX\}}+T_{0}\right)\left(\frac{p_{0}}{\mathrm{\{PSXC\}}}\right)^{R_{d}/c_{pd}}
(\#eq:THETA)
\end{equation}

\hypertarget{thetae}{%
\subsubsection*{Pseudo-Adiabatic Equivalent Potential Temperature (K):
THETAP, THETAE}\label{thetae}}
\addcontentsline{toc}{subsubsection}{Pseudo-Adiabatic Equivalent
Potential Temperature (K): THETAP, THETAE}

\emph{The absolute temperature reached if a parcel of air were to be
expanded pseudo-adiabatically (i.e., with immediate removal of all
condensate) to a level where no water vapor remains, after which the dry
parcel would be compressed to 1000 hPa.} Beginning in 2011,
pseudo-adiabatic equivalent potential temperature is calculated using
the method developed by Davies-Jones (2009).\footnote{Davies-Jones, R.,
  2009: On formulas for equivalent potential temperature. \emph{Mon.
  Wea. Review,} \textbf{137,} 3137--3148.} This is discussed in the memo
available at this link. The following summarizes that study. The
Davies-Jones formula is:\\
\begin{equation}
\Theta_{P}=\Theta_{DL}\exp\{\frac{r(L_{0}^{*}-L_{1}^{*}(T_{L}-T_{0})+K_{2}r)}{c_{pd}T_{L}}\}
(\#eq:THETAP)
\end{equation}\\
\begin{equation}
\Theta_{DL}=T_{K}(\frac{p_{0}}{p_{d}})^{0.2854}\,(\frac{T_{k}}{T_{L}})^{0.28\times10^{-3}r}
(\#eq:THETADL)
\end{equation}\\
where \(T_K\) is the absolute temperature (in kelvin) at the measurement
level, \(p_d\) is the partial pressure of dry air at that level, \(p_0\)
is the reference pressure (conventionally 1000 hPa), \(r\) is the
(dimensionless) water vapor mixing ratio, \(c_p\) the specific heat of
dry air, \(T_L\) the temperature at the lifted condensation level (in
kelvin), and \(T_0=273.15\,\mathrm{K}\). The coefficients in this
formula are \(L_0^* = 2.56313\times 10^6\mathrm{J\,kg^{-1}}\),
\(L_1^* = 1754\,\mathrm{J\,kg^{-1}K^{-1}}\), and
\(K_2 = 1.137\times 10^6\mathrm{J\,kg^{-1}}\). The asterisks on
\(L_0^*\) and \(L_1^*\) indicate that these coefficients depart from the
best estimate of the coefficients that give the latent heat of
vaporization of water, but they have been adjusted to optimize the fit
to values obtained by exact integration. Note that, unlike the formula
discussed below that was used prior to 2011, the mixing ratio must be
used in dimensionless form (i.e., kg/kg), \emph{not} with units of g/kg.
The following empirical formula, developed by Bolton (1980),\footnote{Bolton,
  D., 1980: The computation of equivalent potential temperature.
  \emph{Mon. Wea. Rev.,} \textbf{108,} 1046--1053.} is used to calculate
\(T_L\):\\
\begin{equation}
T_{L}=\frac{\beta_{1}}{3.5\ln(T_{K}/\beta_{3})-\ln(\mathrm{e/\beta_{4}})+\beta_{5}}+\beta_{2}
(\#eq:TLCL)
\end{equation}\\
where \(e\) is the water vapor pressure, \(\beta_1 = 2840\,\mathrm{K}\),
\(\beta_2 = 55\,\mathrm{K}\), \(\beta_3 = 1\,\mathrm{K}\),
\(\beta_4 = 1\,\mathrm{hPa}\), and \(\beta_5 = -4.805\). (\(\beta_3\)
and \(\beta_4\) have been introduced into @ref(eq:TLCL) only to ensure
that arguments to logarithms are dimensionless and to specify the units
that must be used to achieve that.)

\(T_K\) = \protect\hyperlink{ambient-t}{ATX} + \(T_0\) = ambient
temperature {[}K{]}\\
\(e\) = \protect\hyperlink{ewx}{EWX} = water vapor pressure\\
\(p_d\) = \protect\hyperlink{psx}{PSXC} - \protect\hyperlink{ewx}{EWX} =
partial pressure of dry air {[}hPa{]}\\
\(p_{0}\) = reference pressure = 1000 hPa\\
\(r\) = \protect\hyperlink{MR}{MR} = water vapor mixing ratio\\
\(R_{d}\) = gas constant for dry air\(^{\dagger}\)\\
\(c_{pd}\) = specific heat at constant pressure for dry
air\(^{\dagger}\)\\
\(T_L\) = temperature at the lifted condensation level (LCL) {[}K{]}\\
\(L_0^*+L_1^*(T_L-T_0)\) = latent heat of vaporization at the LCL\\
\(L_0=2.56313 × 10^6\) J\(\,\)kg\(^{-1}\), \(L_1=1754\)
J\(\,\)kg\(^{-1}\)K\(^{-1}\)\\
\(K_2=1.137 × 10^6\) J\(\,\)kg\(^{-1}\)\\
\(\beta_{1-5}\) = \{2840 K, 55 K, 1 K, 1 hPa, −4.805\}

\begin{equation}
T_{L}=\frac{\beta_{1}}{3.5\ln(T_{K}/\beta_{3})-\ln(\mathrm{e/\beta_{4}})+\beta_{5}}+\beta_{2}
(\#eq:TLCLTP)
\end{equation} \begin{equation}
\Theta_{DL}=T_{K}(\frac{p_{0}}{p_{d}})^{0.2854}\,(\frac{T_{k}}{T_{L}})^{0.28\times10^{-3}r}
(\#eq:THETADLbox)
\end{equation} \begin{equation}
\Theta_{P}=\Theta_{DL}\exp\{\frac{r(L_{0}^{*}-L_{1}^{*}(T_{L}-T_{0})+K_{2}r)}{c_{pd}T_{L}}\}
(\#eq:THETAPbox)
\end{equation}

Prior to 2011, the variable called the equivalent potential
temperature\footnote{The AMS glossary defines equivalent potential
  temperature as applying to the adiabatic process, not the
  pseudo-adiabatic process; the name of this variable has therefore been
  changed.} and named THETAE in the output data files was that obtained
using the method of Bolton (1980), which used the same formula to obtain
the temperature at the lifted condensation level (\(T_L\)) and then used
that temperature to find the value of potential temperature of dry air
that would result if the parcel were lifted from that point until all
water vapor condensed and was removed from the air parcel. The formulas
used were as follows:

\(T_{L}\)= temperature at the lifted condensation level, K\\
\(T_0=273.15\,\mathrm{K}\) \protect\hyperlink{ambient-t}{ATX} = ambient
temperature {[}\(^{\circ}\mathrm{C}\){]}\\
EDPC = water vapor pressure {[}hPa{]} -- now superceded by
\protect\hyperlink{ewx}{EWX}\\
\protect\hyperlink{MR}{MR} = mixing ratio {[}g/kg{]}\\
\protect\hyperlink{theta}{THETA} = potential temperature {[}K{]}

\begin{equation}
T_{L}=\frac{2840.}{3.5\ln(\mathrm{\{ATX\}+T_{0}})-\ln(\mathrm{\{EDPC\}})-4.805}+55
(\#eq:TLCL2)
\end{equation} \begin{align}
\mathrm{\{THETAE\}} = & \mathrm{\{THETA\}}\left(\frac{3.376}{T_{L}}-0.00254\right)\notag \\
& \times (\mathrm{\{MR\}})(1+0.00081(\{MR\}))
(\#eq:THETAE)
\end{align}

Differences vs the new formula are usually minor but can be as much as
0.5 K.****

\hypertarget{TVIR}{%
\subsubsection*{\texorpdfstring{Virtual Temperature ({º}C):
TVIR}{Virtual Temperature (ºC): TVIR}}\label{TVIR}}
\addcontentsline{toc}{subsubsection}{Virtual Temperature ({º}C): TVIR}

\emph{The temperature of dry air having the same pressure and density as
the air being sampled.} The virtual temperature thus adjusts for the
buoyancy added by water vapor.

ATX = ambient temperature, \(^{\circ}\mathrm{C}\)\\
\(r\) = mixing ratio, dimensionless \{{[}\}kg/kg\{{]}\} = \{MR\}/(1000
g/kg)\\
\(T_{0}=273.15\),K

\begin{equation}
\mathrm{TVIR}=(\mathrm{\{ATX\}}+T_{0})\left(\frac{1+\frac{M_{d}}{M_{w}}r}{1+r}\right)-T_{0}
(\#eq:TVIR)
\end{equation}

\hypertarget{thetav}{%
\subsection*{Virtual Potential Temperature (K): THETAV}\label{thetav}}
\addcontentsline{toc}{subsection}{Virtual Potential Temperature (K):
THETAV}

\emph{A potential temperature analogous to the conventional potential
temperature except that it is based on virtual temperature instead of
ambient temperature.} Dry-adiabatic expansion or compression to the
reference level (1000 hPa) is assumed. As for THETA, use of dry-air
values for the gas constant and specific heat at constant pressure can
lead to significant errors in humid conditions. For further information,
see this note.

\protect\hyperlink{TVIR}{TVIR} = virtual temperature
{[}\(^{\circ}\mathrm{C}\){]}\\
\protect\hyperlink{psx}{PSXC} = ambient pressure {[}hPa{]}\\
\(R_{d}=\)gas constant for dry air\(^{\dagger}\)\\
\(c_{pd}=\)specific heat at constant pressure for dry
air\(^{\dagger}\)\\
\(T_{0}=273.15\,\)K\\
\(p_{0}\) = reference pressure, conventionally 1000 hPa

\begin{equation}
\mathrm{THETAV}=\left(\mathrm{\{TVIR\}}+T_{0}\right)\left(\frac{p_{o}}{\mathrm{\{PSXC\}}}\right)^{R_{d}/c_{pd}}
(\#eq:THETAV)
\end{equation}

\hypertarget{thetaq}{%
\subsubsection*{Wet-Equivalent Potential Temperature (K):
THETAQ}\label{thetaq}}
\addcontentsline{toc}{subsubsection}{Wet-Equivalent Potential
Temperature (K): THETAQ}

\emph{The absolute temperature reached if a parcel of air were to be
expanded adiabatically (i.e., retaining the condensed water in the
liquid phase and accounting for the specific heat of that condensate) to
a level where no water vapor remains, after which the condensate would
be removed and the resulting dry parcel compressed to 1000 hPa.} This
variable was not included in data archives prior to 2012. Emanuel (1994)
gives the following formula (his Eq. 4.5.11):\\
\begin{equation}
\Theta_{q}=T(\frac{p_{0}}{p_{d}})^{\frac{R_{d}}{c_{pt}}}\exp\left\{ \frac{L_{v}r}{c_{pt}T}\right\} \left(\frac{e}{e_{s,w}(T)}\right)^{-rR_{w}/c_{pt}}
(\#eq:THETAQ)
\end{equation}\\
where \(\Theta_q\) is the wet-equivalent potential temperature, \(L_v\)
the latent heat of vaporization, \(r\) the (dimensionless) water-vapor
mixing ratio, \(c_{pt} = c_{pd}+r_tc_w\) with \(r_t\) the total-water
mixing ratio including vapor and condensate, \(c_w\) the specific heat
of liquid water, and other symbols are as used previously. See this memo
for additional discussion of this variable, for values to use for the
latent heat and specific heat, and in particular for analysis indicating
that \(\Theta_Q\) evaluated with this formula can be expected to vary
from the true adiabatic value by a few tenths kelvin (in a worst case,
by about 1 K) because of variation in (and uncertainty in) the specific
heat of supercooled water at low temperature. The details of the
calculation are described in the following box. Note that this algorithm
only uses the liquid water content as measured by a King probe, PLWCC;
other similar calculations could be based on other measures of liquid
water such as that from a cloud-droplet spectrometer.

\(e=\)\{\protect\hyperlink{ewx}{EWX}\}\(*100\) = water vapor pressure
{[}Pa{]}\\
\protect\hyperlink{ambient-t}{ATX} = ambient temperature
(\(^{\circ}\mathrm{C}\))\\
\(r=\)\{\protect\hyperlink{MR}{MR}\}/1000.\index{MR} = mixing ratio
(dimensionless)\\
\(p_{d}=\)(\{\protect\hyperlink{psx}{PSXC}\}\(-\)\{\protect\hyperlink{ewx}{EWX}\})\(*100\)
= ambient dry-air pressure {[}Pa{]}\\
\(p_{0}=\)reference pressure for potential temperature, 10\(^{5}\)Pa\\
\(\chi=\)\{\href{./5-cloud-physics-variables.html\#plwcc}{PLWCC}\}/1000.=cloud
liquid water content {[}kg\(\,\)m\(^{-3}\){]}\\
\(R_{d}=\)gas constant for dry air\(^{\dagger}\)\\
\(\rho_{d}=\)density of dry air =
\(\frac{p_{d}}{R_{d}(\{ATX\}+T_{0})}\)\\
\(c_{pd}=\)specific heat of dry air\(^{\dagger}\)\\
\(c_{w}=\)specific heat of liquid water\(^{\dagger}\)\\
\(L_{v}=L_{0}+L_{1}\mathrm{\{ATX\}}\) where
\(L_{0}=2.501\times10^{6}\mathrm{J}\,\mathrm{kg^{-1}}\) and
\(L_{1}=-2370\,\mathrm{J\,\mathrm{kg^{-1}\,\mathrm{K^{-1}}}}\)

\begin{equation}
r_{t}=r+(\chi/\rho_{d})
(\#eq:rtotBox)
\end{equation} \begin{equation}
c_{pt}=c_{pd}+r_{t}c_{w}
(\#eq:cptBox)
\end{equation}\\
If outside cloud or below 100\% relative humidity, define\\
\begin{equation}
F_{1}=\left(\frac{e}{e_{s,w}(T)}\right)^{-\frac{rR_{w}}{c_{pt}}}
(\#eq:F1)
\end{equation}\\
otherwise set \(F_{1}=1\). Then\\
\begin{equation}
T_{1}=\mathrm{(\{ATX\}}+T_{0})\left\{ \frac{p_{0}}{(\mathrm{\{PSXC\}}-\mathrm{\{EDPC\})}}\right\} ^{\frac{R_{d}}{c_{pt}}}
(\#eq:T1box)
\end{equation} \begin{equation}
\mathrm{\{THETAQ\}}=T_{1}F_{1}\exp\left\{ \frac{L_{v}r}{c_{pt}(\{\mathrm{ATX\}}+T_{0})}\right\}
(\#eq:THETAQbox)
\end{equation}

\hypertarget{wind}{%
\section{Wind}\label{wind}}

RAF Bulletin 23 documents the calculation of wind components, both with
respect to the earth (UI, VI, WI, WS and WD) and with respect to the
aircraft (UX and VY). In data processing, a separate function (GUSTO in
GENPRO, gust.c in NIMBUS) is used to derive these wind components. That
function uses the measurements from an Inertial Navigation System (INS)
as well as aircraft true airspeed, aircraft angle of attack, and
aircraft sideslip angle. The wind components calculated in GUSTO/gust.c
are used to derive the wind direction (WD) and wind speed (WS).
Additional variables UIC, VIC, WSC, WDC, UXC, and VYC are also
calculated based on the variables VNSC, VEWC discussed in Section
@ref(combining-irs-and-gps-measurements), which combine INS and GPS
information to obtain improved measurements of the aircraft motion.
Those are usually the highest-quality measurements of wind because the
merged INS/GPS variables combine the high-frequency response of the INS
with the long-term accuracy of the GPS.

There is an extensive discussion of the wind-sensing system and the
uncertainties associated with measurements of wind in this Technical
Note. The details contained therein and in Bulletin 23 will not be
repeated here, so those documents should be consulted for additional
information. There are two exceptions that are discussed in more detail
here:

The calculation of vertical wind is described in more detail below for
the variables WI and WIC.

Because measurements obtained by a GPS receiver are often used, the
motion of the GPS receiving antenna relative to the IRU must be
considered. Standard processing corrects for the motion of the gust
system relative to the IRU arising from aircraft rotation, but a similar
correction is needed because the GPS antenna is displaced from the IRU.
The displacement is almost entirely along the longitudinal axis of the
aircraft, so GPS-measured velocities like GGVNS, GGVEW, and GGVSPD
(denoted here \(v_n\), \(v_e\), \(v_u\)) need correction as follows to
give measurements that apply at the location of the IRU. Then these
variables can be used in place of or to complement similar measurements
from the IRU in the processing algorithms. The equations are:\\
\begin{align}\begin{split}
\delta v_{u} = & -L_{G}\dot{\theta}\notag \\
\delta v_{e} = & -L_{G}\dot{\psi}\,{\cos\psi}\notag \\
\delta v_{n} = &  L_{G}\dot{\psi}\,{\sin\psi}
\end{split}
(\#eq:deltav)
\end{align}\\
where \(\theta\) and \(\psi\) respectively are the pitch and heading
angles and \(L_G\) is the distance forward along the longitudinal axis
from the IRU to the GPS antenna (\(−4.30\) m for the GV and \(-9.88\)~m
for the C-130 during and after 2015). The negative signs indicate that
the GPS antennas are behind the IRUs. The dots over the attitude-angle
symbols represent time derivatives, so for example \(\dot{\theta}\) is
the rate of change of the pitch angle. All angles are expressed in
radians. The correction terms should be added to the GPS-measured
velocity components so that they represent the motion of the IRU
relative to the Earth. This is done for the vertical wind, beginning in
2017, but for horizontal wind the complementary filter (discussed below)
removes high-frequency fluctuations from the GPS-derived measurements so
incorporation of these changes would have negligible effect. For more
information, see this note.

The variables pertaining to the relative wind are described in the next
subsection, and the variables characterizing the wind are then described
briefly in the last subsection. Some additional detail is included in
cases where procedures are not documented in that earlier bulletin.

\hypertarget{relative-wind}{%
\subsection{Relative Wind}\label{relative-wind}}

Wind is measured by adding two vectors, the measured air motion relative
to the aircraft (called the relative wind) and the motion of the
aircraft relative to the Earth. The following are the measurements used
to determine the relative wind. The motion of the aircraft relative to
the ground was discussed in Section @ref(inertial-reference-systems),
and the combination of these two vectors to measure the wind is
described in RAF Bulletin 23.

RAF uses the
radome\protect\hypertarget{radomeux20gust-sensingux20system}{}{}
gust-sensing technique\footnote{Brown, E.~N, C.~A.~Friehe, and
  D.~H.~Lenschow, 1983: \emph{Journal of Climate and Applied
  Meteorology,} \textbf{22,} 171--180} to measure incidence angles of
the relative wind (i.e., angles of attack and sideslip). The pressure
difference between sensing ports above and below the center line of the
radome is used, along with the dynamic pressure measured at a pitot tube
and referenced to the static pressure source, to determine the angle of
attack. The sideslip angle is determined similarly using the pressure
ports on the starboard and port sides of the radome. A Rosemount Model
858AJ gust probe has occasionally been used for specialized
measurements. The radome measurements are made by differential pressure
sensors located in the nose area of the aircraft and connected to the
radome by semi-rigid tubing.

\hypertarget{mach-number}{%
\subsubsection*{Mach Number (dimensionless): MACHx,
MACHX}\label{mach-number}}
\addcontentsline{toc}{subsubsection}{Mach Number (dimensionless): MACHx,
MACHX}

\emph{The Mach Number that characterizes the flight speed.} The Mach
number is defined as the ratio of the flight speed (or the magnitude of
the relative wind) to the speed of sound. See Eq.~@ref(eq:M2) for the
equation used. Many relatively old archived data files have instead a
variable XMACH2, which is the square of MACHx.

\hypertarget{true-airspeed}{%
\subsubsection*{Aircraft True Airspeed (m/s): TASx, TASxD,
TASX}\label{true-airspeed}}
\addcontentsline{toc}{subsubsection}{Aircraft True Airspeed (m/s): TASx,
TASxD, TASX}

\emph{The flight speed of the aircraft relative to the atmosphere.} This
derived measurement of the flight speed of the aircraft relative to the
atmosphere is based on the Mach number calculated from both the dynamic
pressure at location x and the static pressure. See the derivation for
ATx . The different variables for TASx (TASF, TASR, etc) use different
measurements of QCxC in the calculation of Mach number. The variable
TASxD is the result of calculations for which the Mach number, air
temperature, and true airspeed are determined for dry instead of humid
air. See the discussion of \protect\hyperlink{ATX}{ATX} for an
explanation of how humidity is handled in the calculation of true
airspeed.

(see @ref(eq:M2) and @ref(eq:TaEQ) for MACHx and ATX)\\
Note dependence of MACHx on choices for QCXC and PSXC\\
TASx depends on \protect\hyperlink{qcx}{QCXC},
\protect\hyperlink{psx}{PSXC}, \href{ambient-t}{ATX}\\
\hspace*{0.333em}\hspace*{0.333em}\hspace*{0.333em}\hspace*{0.333em}\hspace*{0.333em}where
PSXC and ATX are the preferred choices\\
\(\gamma^{\prime}\), \(R^{\prime}\), and \(T_{0}\): See the List of
Symbols

\begin{equation}
\mathrm{TASx}=\mathrm{\{MACHx\}}\sqrt{\gamma^{\prime}R^{\prime}\mathrm{\,(\{ATX\}}+T_{0})}
(\#eq:TASx)
\end{equation}

\hypertarget{tashc}{%
\subsubsection*{Aircraft True Airspeed (Humidity Corrected) (m/s):
TASHC}\label{tashc}}
\addcontentsline{toc}{subsubsection}{Aircraft True Airspeed (Humidity
Corrected) (m/s): TASHC}

This derived measurement of true airspeed accounted for deviations of
specific heats of moist air from those of dry air. See List, 1971, pp
295, 331-339, and Khelif, et al., 1999. This variable is no longer used
because the standard calculation of TASX (documented in the preceding
paragraph) now uses moist-air values of the specific heats and gas
constant. The equation previously used for this variable, given by
Khelif et al.~1999,\footnote{Khelif, D., S.P. Burns, and C.A. Friehe,
  1999: Improved wind measurements on research aircraft. \emph{Journal
  of Atmospheric and Oceanic Technology,} \textbf{16,} 860--875.} added
a moisture correction to the true airspeed derived for dry air, as
follows:

\(q\) = specific humidity (dimensionless) = \href{$sphum}{SPHUM}/1000.\\
\hspace*{0.333em}\hspace*{0.333em}\hspace*{0.333em}\hspace*{0.333em}\hspace*{0.333em}\hspace*{0.333em}
for SPHUM expressed in g/kg\\
\(c=0.000304\,\mathrm{kg\,g^{-1}}=0.304\) (dimensionless)

\begin{equation}
\mathrm{\{TASHC\}} = \mathrm{\{TASX\}} (1.0 + c\,q)  
(\#eq:TASHC)
\end{equation}

\hypertarget{adifr}{%
\subsubsection*{Attack Angle Differential Pressure (mb):
ADIFR}\label{adifr}}
\addcontentsline{toc}{subsubsection}{Attack Angle Differential Pressure
(mb): ADIFR}

\emph{The pressure difference between the top and bottom pressure ports
of a radome gust-sensing system.} This measurement is used to determine
the angle of attack; see AKRD below. **** Obsolete variable
\underline{ADIF} is a similar variable used for old gust-boom systems or
for Rosemount Model 858AJ flow-angle sensors.

\hypertarget{bdifr}{%
\subsubsection*{Sideslip Angle Differential Pressure (mb):
BDIFR}\label{bdifr}}
\addcontentsline{toc}{subsubsection}{Sideslip Angle Differential
Pressure (mb): BDIFR}

\emph{The pressure difference between starboard and port pressure inlets
of a radome gust-sensing system.} This measurement is used to determine
the sideslip angle; see SSRD below. Obsolete variable \underline{BDIF}
is a similar variable used for old gust-boom systems or for Rosemount
Model 858AJ flow-angle sensors.

\hypertarget{akrd}{%
\subsubsection*{\texorpdfstring{Attack Angle, Radome ({º}):
AKRD}{Attack Angle, Radome (º): AKRD}}\label{akrd}}
\addcontentsline{toc}{subsubsection}{Attack Angle, Radome ({º}): AKRD}

\emph{The angle of attack of the aircraft.} This derived measurement
represents the angle between the longitudinal axis of the aircraft and
the component of the relative wind vector in the plane of port-starboard
symmetry of the aircraft. The tangent of the angle of attack is the
ratio of the vertical to longitudinal component of the relative wind.
Positive values indicate flow moving upward (in the aircraft reference
frame) relative to the longitudinal axis. The calculation is based on
ADIFR and a measurement of dynamic pressure, and so is the measurement
produced by a radome gust-sensing system. Empirical sensitivity
coefficients for each aircraft, determined from special flight
maneuvers, are used; see RAF Bulletin 23 and this Technical Note for
more information. The sensitivity coefficients listed below have changed
when the radomes were changed or refurbished, so the project
documentation should be consulted for the values used in a particular
project. For more information on the latest C-130 calibration, see this
note. Prior to 2017, the procedure was based on the following algorithm:

\protect\hyperlink{adifr}{ADIFR} = attack differential pressure, radome
{[}hPa{]}\\
\protect\hyperlink{qcx}{QCF} = uncorrected dynamic pressure {[}hPa{]}\\
\protect\hyperlink{mach-number}{MACH} = uncorrected Mach number based on
QCF and PSF without humidity correction\\
\(e_{0},\,e{}_{1},\,e_{2}\) = sensitivity coefficients determined
empirically; typically:\\
\hspace*{0.333em}\hspace*{0.333em}\hspace*{0.333em}\hspace*{0.333em}\hspace*{0.333em}\{4.7532,
9.7908, 6.0781\} for the C-130\textsuperscript{(a)}\\
\hspace*{0.333em}\hspace*{0.333em}\hspace*{0.333em}\hspace*{0.333em}\hspace*{0.333em}\{4.605\(\,[^{\circ}]\),
\(18.44\,[^{\circ}]\), \(6.75\,[^{\circ}]\)\} for the GV\\
\_\_\_\_\_\_\_\_\_\_\\
\textsuperscript{(a)} Prior to Jan 2012, when the GV radome was changed:
\{5.516, 19.07, 2.08\}

\begin{equation}
\mathrm{\{AKRD\}}=e_{0}+\frac{\{\mathrm{ADIFR}\}}{\{\mathrm{QCF}\}}\left(e_{1}+e{}_{2}\mathrm{\{MACH\}}\right)
(\#eq:AKRD)
\end{equation}

See also this memo. Beginning in 2017, a different strategy was used, as
documented in more detail in this memo. Two variables were used to
represent the angle of attack, \(A\)=\{ADIFR\}/\{QCF\} and
\(q\)=\{QCF\}. However, each was filtered into complementary low-pass
and high-pass components, with the cutoff frequency at (1/600)~Hz, and
the separate components were used to represent the separate components
of angle of attack according to the following formula:

\protect\hyperlink{adifr}{ADIFR} = attack differential pressure, radome
{[}hPa{]}\\
\protect\hyperlink{qcx}{QCF} = uncorrected dynamic pressure {[}hPa{]}\\
\(A\) = (ADIFR/QCF) = \(A_{f}+A_{s}\) where \(A_{f}\) is the high-pass
and \(A_{s}\) the low-pass component\\
\(e_{1},\,d_{0},\,d{}_{1},\,d_{2}\) = sensitivity coefficients
determined empirically; typically, for the GV,\\
\hspace*{0.333em}\hspace*{0.333em}\hspace*{0.333em}\hspace*{0.333em}\hspace*{0.333em}\(e_{1}=21.481\,[^{\circ}]\)\\
\hspace*{0.333em}\hspace*{0.333em}\hspace*{0.333em}\hspace*{0.333em}\hspace*{0.333em}\(d_{1-3}\)
= \{\(4.5253\,[^{\circ}]\), \(19.9332\,[^{\circ}]\),
\(-0.00196099\,[^{\circ}\mathrm{hPa}^{-1}]\)\}

\begin{equation}
\mathrm{\{AKRD\}}=d_{0}+d_{1}A_{s}+d_{2}\mathrm{\{QCF\}_{s}+}e_{1}A_{f}
(\#eq:newAKRD)
\end{equation}

****

\hypertarget{attack}{%
\subsubsection*{\texorpdfstring{Reference Attack Angle ({º}):
ATTACK}{Reference Attack Angle (º): ATTACK}}\label{attack}}
\addcontentsline{toc}{subsubsection}{Reference Attack Angle ({º}):
ATTACK}

\emph{The reference angle of attack used to calculate derived
variables.} This variable is the reference selected from other
measurements of angle of attack in the data set. In most projects, it is
equal to AKRD. It is used where attack angle is needed for other derived
calculations (e.g., wind measurements).

\hypertarget{ssrd}{%
\subsubsection*{\texorpdfstring{Sideslip Angle (Differential Pressure)
({º}):
SSRD}{Sideslip Angle (Differential Pressure) (º): SSRD}}\label{ssrd}}
\addcontentsline{toc}{subsubsection}{Sideslip Angle (Differential
Pressure) ({º}): SSRD}

\emph{The angle of sideslip of the aircraft.} This derived measurement
represents the angle between the longitudinal axis of the aircraft and
the projection of the relative wind onto the plane determined by the
longitudinal and lateral axes. Positive values indicate airflow from the
starboard side. This variable is derived from BDIFR and a dynamic
pressure using a sensitivity function that has been determined
empirically for each aircraft.

\protect\hyperlink{bdifr}{BDIFR} = differential pressure between
sideslip pressure ports, radome {[}hPa{]}\\
\protect\hyperlink{qcx}{QCXC} = dynamic pressure {[}hPa{]}\\
\(s_{0},\,s{}_{1}\) = empirical coefficients dependent on the aircraft
and radome configuration\\
\hspace*{0.333em}\hspace*{0.333em}\hspace*{0.333em}\hspace*{0.333em}\hspace*{0.333em}
= \{-0.000983, (1/0.08189) \(^\circ\)\} for the C-130\\
\hspace*{0.333em}\hspace*{0.333em}\hspace*{0.333em}\hspace*{0.333em}\hspace*{0.333em}
= \{-0.0025, (1/0.04727) \(^\circ\)\} for the GV\textsuperscript{(a)}

\begin{equation}
\mathrm{\{SSRD\}} = s_{1}(\frac{\mathrm{\{BDIFR\}}}{\{\mathrm{QCXC}\}}+s_{0})
(\#eq:SSRD)
\end{equation} \_\_\_\_\_\_\_\_\_\_\\
\textsuperscript{(a)} The
\href{http://dx.doi.org/10.5065/D60G3HJ8}{technical note on wind
uncertainty} recommended using SSRD=\(e_{0}+e_{1}\)\{BDIFR\}/\{QCF\}
with \(e_{0}=0.008\) and \(e_{1}=22.302\). This has not yet been used in
processing as of May 2022.

\hypertarget{sslip}{%
\subsubsection*{\texorpdfstring{Reference Sideslip Angle ({º}):
SSLIP}{Reference Sideslip Angle (º): SSLIP}}\label{sslip}}
\addcontentsline{toc}{subsubsection}{Reference Sideslip Angle ({º}):
SSLIP}

\emph{The reference sideslip angle used to calculate derived variables}.
This variable is the reference selected from other measurements of
sideslip angle in the data set. In most projects, it is equal to SSRD.
It is used where sideslip angle is needed for other derived calculations
(e.g., wind measurements).

\hypertarget{wind-components-and-the-wind-vector}{%
\subsection{Wind Components and the Wind
Vector}\label{wind-components-and-the-wind-vector}}

\hypertarget{ui-vi-wi}{%
\subsubsection*{Wind Vector Components (m/s): UI, VI,
WI}\label{ui-vi-wi}}
\addcontentsline{toc}{subsubsection}{Wind Vector Components (m/s): UI,
VI, WI}

\emph{The three-dimensional wind vector with respect to the earth,} as
determined from the inertial reference systems. UI is the east-west
component with positive values \underline{toward} the east, VI is the
north-south component with positive values \underline{toward} the north,
and WI is the vertical component with positive values toward the zenith.
The calculation of WI differs from the description in Bulletin 23
because the output from the inertial reference system is different for
the modern units now in use. The vertical wind is the sum of the
vertical gust component (represented approximately by
TASX sin(ATTACK-PITCH)) and the motion of the aircraft as measured by
VSPD (discussed in Section XXX). Bulletin 23 describes the historical
calculation of the vertical motion of the aircraft via a
barometric-inertial feedback loop, but equivalent calculations
(including pressure damping to the pressure altitude) are incorporated
into current IRS units so VSPD already is the product of such a
calculation. To calculate WI, VSPD is therefore used in place of the
obsolete variable WP3 that was discussed in Bulletin 23. WIC should
usually be used instead of WI because VSPD, entering WI, is updated to
the pressure altitude and so can have false variations in baroclinic
conditions. WIC uses GGVSPD (or in some cases older GPS-based
rate-of-climb variables) in place of VSPD and so is more reliable. ****

\hypertarget{ws-wd}{%
\subsubsection*{\texorpdfstring{Wind Speed and Direction (m/s and {º}):
WS, WD}{Wind Speed and Direction (m/s and º): WS, WD}}\label{ws-wd}}
\addcontentsline{toc}{subsubsection}{Wind Speed and Direction (m/s and
{º}): WS, WD}

\emph{The magnitude and direction of the horizontal wind.} These
variables are obtained in a straightforward manner from UI and VI. The
resulting wind direction is relative to true north and represents the
direction \underline{from which} the wind blows. That is the reason that
180{∘} appears in the following algorithm.

\protect\hyperlink{ui-vi-wi}{UI} = easterly component of the horizontal
wind\\
\protect\hyperlink{ui-vi-wi}{VI} = northerly component of the horizontal
wind\\
atan2 = 4-quadrant arc-tangent function producing output in radians from
-\(\pi\) to \(\pi\)\\
\(C_{rd}\) = conversion factor, radians to degrees, = 180/\(\pi\)
{[}units: \(^{\circ}\),/,radian{]}

\begin{align}
\mathrm{WS} = & \sqrt{\mathrm{\{UI\}}^{2}+\{\mathrm{VI\}}^{2}}(\#eq:WS)\\
\mathrm{WD} = & C_{rd}\mathrm{\,atan2(\{UI\},\,\{VI\})}+180^{\circ}  
(\#eq:WD)
\end{align}

\hypertarget{ux-vy}{%
\subsubsection*{Wind Vector Longitudinal and Lateral Components (m/s):
UX and VY}\label{ux-vy}}
\addcontentsline{toc}{subsubsection}{Wind Vector Longitudinal and
Lateral Components (m/s): UX and VY}

\emph{The horizontal wind} \emph{vector relative to the frame of
reference attached to the aircraft.} UX is parallel to the longitudinal
axis and positive toward the nose. VY is along the lateral axis and
normal to the longitudinal axis; positive is toward the port (or left)
wing.

\hypertarget{uic-vic}{%
\subsubsection*{GPS-Corrected Wind Vector, East and North Components
(m/s): UIC, VIC}\label{uic-vic}}
\addcontentsline{toc}{subsubsection}{GPS-Corrected Wind Vector, East and
North Components (m/s): UIC, VIC}

\emph{The horizontal wind components} respectively
\emph{\underline{toward}} \emph{the east and} \emph{\underline{toward}}
\emph{the north.} They are derived from measurements from an inertial
reference unit (IRU) and a Global Positioning System (GPS), as described
in the discussion of VEW and VNS above. They are calculated just as for
UX and VY except that the GPS-corrected values for the aircraft
groundspeed are used in place of the IRU-based values. They are
considered ``corrected'' from the original measurements from the IRU or
GPS, as described in Section @ref(combining-irs-and-gps-measurements).

\hypertarget{wic}{%
\subsubsection*{Wind Vector, Vertical Component (m/s): WIC}\label{wic}}
\addcontentsline{toc}{subsubsection}{Wind Vector, Vertical Component
(m/s): WIC}

\emph{The component of the wind in the vertical direction.} This is the
standard calculation of vertical wind, obtained from the difference
between the measured vertical component of the relative wind and the
vertical motion of the aircraft (usually GGVSPD in recent projects).****
This should be used in preference to WI if the latter is present; see
the discussion of WP3 in Section @ref(inertial-reference-systems).
Positive values are toward the zenith.

\hypertarget{wsc-wdc}{%
\subsubsection*{\texorpdfstring{GPS-Corrected Wind Speed and Direction
(m/s and {∘}): WSC,
WDC}{GPS-Corrected Wind Speed and Direction (m/s and ∘): WSC, WDC}}\label{wsc-wdc}}
\addcontentsline{toc}{subsubsection}{GPS-Corrected Wind Speed and
Direction (m/s and {∘}): WSC, WDC}

\emph{The magnitude and direction of the wind vector,} obtained by
combining measurements from GPS and IRU units. These variables are
obtained in a straightforward manner from UIC and VIC, using equations
analogous to @ref(eq:WS) and @ref(eq:WD) but with UIC and VIC as input
measurements. They are expected to be the preferred measurements of wind
because they combine the best features of the IRU and GPS measurements.

\hypertarget{uxc-vyc}{%
\subsubsection*{GPS-Corrected Wind Vector, Longitudinal and Lateral
Components (m/s): UXC, VYC}\label{uxc-vyc}}
\addcontentsline{toc}{subsubsection}{GPS-Corrected Wind Vector,
Longitudinal and Lateral Components (m/s): UXC, VYC}

\emph{The longitudinal and lateral components of the three-dimensional
wind, similar to UX and VY, but corrected by the complementary-filter
algorithm that combines IRU and GPS measurements}. See the discussion in
Section @ref(combining-irs-and-gps-measurements). The components UXC and
VYC are toward the front of the aircraft and toward the port (left)
wing, respectively.

\hypertarget{special-use-remote}{%
\section{Special-Use Remote Sensors}\label{special-use-remote}}

The above variables are normally included in the archived netCDF files
from projects, but there are a few remote sensors that provide
additional state-parameter measurements in some projects. These
include:\protect\hypertarget{subsec:MTP}{}{}

\begin{itemize}
\tightlist
\item
  Microwave Temperature Profiler
  (\url{http://www.eol.ucar.edu/instruments/microwave-temperature-profiler})
  \{MTP\}) -- remotely sensed temperature profiles\\
\item
  Dropsonde System
  (\url{https://www.eol.ucar.edu/observing_facilities/avaps-dropsonde-system})
  \{AVAPS\}) -- profiles of temperature, humidity, and wind vs
  pressure.\\
\item
  GPS-Occultation Sensor
  (\url{http://www.eol.ucar.edu/instruments/gnss-instrument-system-multi-static-and-occultation-sensing})
  \{GISMOS\}) -- atmospheric soundings of refractivity via GPS
  occultation.
\end{itemize}

The links provided connect to descriptions of these instruments on the
EOL web site, and each provides a summary of how data are acquired and
processed. These measurements are not normally part of the archived
netCDF project files. Those interested in using these measurements
should contact EOL data management (mailto:raf-dm@eol.ucar.edu) for
access to the measurements and for information on how the measurements
are processed.

\hypertarget{cloud-physics-variables}{%
\chapter{Cloud Physics Variables}\label{cloud-physics-variables}}

\hypertarget{LWC}{%
\section{Measurements of Liquid Water Content}\label{LWC}}

\hypertarget{plwc}{%
\subsubsection*{Power, PMS/CSIRO (King) Liquid Water Content (W): PLWC,
PLWC1}\label{plwc}}
\addcontentsline{toc}{subsubsection}{Power, PMS/CSIRO (King) Liquid
Water Content (W): PLWC, PLWC1}

\emph{The power dissipated by the sensor of a PMS/CSIRO (King) liquid
water probe (in watts).} PLWC is the power required to maintain constant
temperature in a heated element as that element is cooled by convection
and evaporation of impinging liquid water. The convective heat losses
are determined by calibration in dry air over a range of airspeeds and
temperatures, so that the remaining power can be related to the liquid
water content. The instrument is described in RAF Bulletin 24 and at
this URL. See the following description for the algorithm used to obtain
liquid water content from this measurement of power.

\hypertarget{plwcc}{%
\subsubsection*{\texorpdfstring{PMS/CSIRO (King) Liquid Water Content
(g/m{3}): PLWCC,
PLWCC1}{PMS/CSIRO (King) Liquid Water Content (g/m3): PLWCC, PLWCC1}}\label{plwcc}}
\addcontentsline{toc}{subsubsection}{PMS/CSIRO (King) Liquid Water
Content (g/m{3}): PLWCC, PLWCC1}

\emph{The liquid water content} \emph{measured by a King probe.} This is
calculated by relating the power consumption required to maintain a
constant temperature to the liquid water content, taking into account
the effect of convective heat losses. The instrument and processing are
described by King et al.~(1978)\footnote{King, W. D., D. A. Parkin and
  R. J. Handsworth, 1978: A hot-wire liquid water device having fully
  calculable response characteristics. J. Appl. Meteorol., 17,
  1809--1813. See also Bradley, S. G., and W. D. King, 1979 Frequency
  response of the CSIRO Liquid Water Probe. J. Appl. Meteorol., 18,
  361--366.} and in a note available at this URL. Because the
temperature of the sensing wire is typically well above the boiling
point of water, the assumption made in processing is that the water
collected on the sensing wire is vaporized at the boiling point
{\emph{T}\emph{b}}. The boiling point is represented as a function of
pressure as described below.

\protect\hyperlink{plwc}{PLWC} = total power dissipated by the probe
{[}W{]}\\
\(P_{D}\) = power
dissipated\index{King probe!power dissipated}\sindex[lis]{Power@$P$=power}
by the cooling effect of dry air alone\\
\(P_{W}\) = power needed to heat and vaporize the liquid water that hits
the probe element\\
\(L\) = length\sindex[lis]{L@$L$ =length (of a King-probe element)} of
the probe sensitive element\index{King probe!element dimensions},
typically 0.021~m\\
\(d\)= diameter\sindex[lis]{d@$d$=diameter} of the probe sensitive
element, typically 1.805\(\times10^{-3}\)m\\
\(T_{s}\)= sensor
temperature\index{King probe!sensor temperature}\sindex[lis]{Ts@$T_{s}$=temperature of a sensor}
{[}\(^{\circ}\)C{]}\\
\(T_{a}\)= ambient temperature {[}\(^{\circ}\)C{]} =
\href{./4-the-state-of-the-atmosphere.html\#ambient-t}{ATX}\\
\(T_{b}\)\sindex[lis]{Tb@$T_{b}$= boiling temperature of water} =
boiling temperature of water (dependent on pressure):\\
\hspace*{0.333em}\hspace*{0.333em}\hspace*{0.333em}\hspace*{0.333em}\hspace*{0.333em}\hspace*{0.333em}\hspace*{0.333em}\hspace*{0.333em}\hspace*{0.333em}with
\(x=\log_{10}(p/(1\)hPa)), \(B=1^{\circ}\)C,\\
\hspace*{0.333em}\hspace*{0.333em}\hspace*{0.333em}\hspace*{0.333em}\hspace*{0.333em}\hspace*{0.333em}\hspace*{0.333em}\hspace*{0.333em}\hspace*{0.333em}and
\{\(b_{0},\) \(b_{1}\), \(b_{2}\), \(b_{3}\)\} = \{0.03366503,
1.34236135, -0.33479451, 0.0351934\}:\\
\hspace*{0.333em}\hspace*{0.333em}\hspace*{0.333em}\hspace*{0.333em}\hspace*{0.333em}\hspace*{0.333em}\hspace*{0.333em}\hspace*{0.333em}\hspace*{0.333em}\hspace*{0.333em}\(T_{b}=B\times10^{(b_{0}+b_{1}x+b_{2}x^{2}+b_{3}x^{3})}\)\\
\(T_{m}=(T_{a}+T_{s})/2\) = mean temperature for air properties\\
\(L_{v}(T_{b})\) = latent heat of vaporization of
water\sindex[lis]{Lv@$L_{v}$=latent heat of vaporization of water}\index{latent heat of vaporization}
= (2.501-0.00237\(T_{b}\))\(\times10^{6}\)J\(\,\)kg\(^{-1}\)\\
\(c_{w}\)\sindex[lis]{cw@$c_{w}$= specific heat of liquid water} =
specific heat of water = 4190~J\(\,\)kg\(^{-1}\)K\(^{-1}\) (mean value,
0--90\(^{\circ}\)C)\\
\(U_{a}\) = true airspeed {[}m/s{]} =
\href{./4-the-state-of-the-atmosphere.html\#true-airspeed}{TASX}\\
\(\lambda_{c}\)\sindex[lis]{lambdac@$\lambda_{c}$= thermal conductivity, dry air}
= thermal
conductivity\index{thermal conductivity|see {conductivity, thermal}}\index{conductivity!thermal}
of dry air
(2.38+0.0071\(T_{m}\))\(\times10^{-2}\)J\(\,\)m\(^{-1}\)s\(^{-1}\)K\(^{-1}\)\\
\(\mu\)\sindex[lis]{mu_{a}= dynamic viscosity of air@$\mu_{a}$= dynamic viscosity of air}
=viscosity of air =\index{viscosity}
(1.718+0.0049\(T_{m})\times10^{-5}\) kg\(\,\)m\(^{-1}\)s\(^{-1}\)\\
\(\rho_{a}\)\sindex[lis]{rhoa@$\rho_{a}$= density of air} = density of
air = \(p / (R_{d}(T_{a}+T_{0}))\)\\
Re\sindex[lis]{Re= Reynolds number} = Reynolds number =
\(\rho_{a}U_{a}d/\mu_{a}\)\\
Nu\sindex[lis]{Nu= Nusselt number} = Nusselt Number relating conduction
heat loss to the total heat loss for dry air:\\
~~~~~typically Nu=\(a_{0}\mathrm{Re}^{a_{1}}\) where\\
\hspace*{0.333em}\hspace*{0.333em}\hspace*{0.333em}\hspace*{0.333em}\hspace*{0.333em}for
the GV:\\
\hspace*{0.333em}\hspace*{0.333em}\hspace*{0.333em}\hspace*{0.333em}\hspace*{0.333em}\hspace*{0.333em}\hspace*{0.333em}\hspace*{0.333em}\(\{a_{0},\,a_{1}\}=\{1.868,\,0.343\}\)
for Re\textless7244\\
\hspace*{0.333em}\hspace*{0.333em}\hspace*{0.333em}\hspace*{0.333em}\hspace*{0.333em}\hspace*{0.333em}\hspace*{0.333em}\hspace*{0.333em}
and \(\{0.135,\,0.638\}\) otherwise,\\
\hspace*{0.333em}\hspace*{0.333em}\hspace*{0.333em}\hspace*{0.333em}\hspace*{0.333em}\hspace*{0.333em}\hspace*{0.333em}\hspace*{0.333em}except
when TASX \textless{} 150 m/s;\\
\hspace*{0.333em}\hspace*{0.333em}\hspace*{0.333em}\hspace*{0.333em}\hspace*{0.333em}\hspace*{0.333em}\hspace*{0.333em}\hspace*{0.333em}then
use \(\{0.133,\,0.382\}\).\\
\hspace*{0.333em}\hspace*{0.333em}\hspace*{0.333em}\hspace*{0.333em}\hspace*{0.333em}For
the C-130:\\
\hspace*{0.333em}\hspace*{0.333em}\hspace*{0.333em}\hspace*{0.333em}\hspace*{0.333em}\hspace*{0.333em}\hspace*{0.333em}\hspace*{0.333em}\(\{a_{0},\,a_{1}\}=\{0.118,\,0.675\}\).
\\
\(C_{kg2g}=1000\)\sindex[lis]{Ckg2g@$C_{kg2g}$= conversion factor, g to kg}
= grams per kilogram\\
\hspace*{0.333em}\hspace*{0.333em}\hspace*{0.333em}\hspace*{0.333em}\hspace*{0.333em}\hspace*{0.333em}\hspace*{0.333em}\hspace*{0.333em}\hspace*{0.333em}\hspace*{0.333em}(unit
conversion to conventional units for liquid water content)\\
\(\chi\) \sindex[lis]{chi@$\chi$=liquid water content}= liquid water
content {[}g/m\(^{3}\){]} = PLWCC

\begin{equation}
\mathrm{\{PLWC\}} = P_{D}+P_{W}
(\#eq:PLWCa)
\end{equation} where\\
\begin{equation}
P_{D}=\pi\mathrm{Nu}\,L\lambda_{c}(T_{s}-T_{a})
(\#eq:PLWCb)
\end{equation} \begin{equation}
P_{W}=L\,d[L_{v}(T_{b})+c_{w}(T_{b}-T_{a})]\,U_{a}\chi
(\#eq:PLWCc)
\end{equation} \emph{Result:}\\
\begin{equation}
\mathrm{\{PLWCC\}}=\chi=\frac{C_{kg2g}(\mathrm{\{PLWC\}}-P_{D})}{L\,d\,U_{a}[L_{v}(T_{b})+c_{w}(T_{b}-T_{a})]}
(\#eq:PLWCd)
\end{equation}

In addition, a processing step is used to remove drift by calculating
the offset required to zero measurements obtained outside cloud. This is
done by adjusting the coefficient {\emph{a}0} by nudging toward the
value required to give zero liquid water content outside cloud (as
indicated by another instrument, often a CDP showing droplet
concentration of \textless1 cm{ − 3}). Specifically, when out-of-cloud,
Nu{′} is calculated from
Nu{′ = \{\emph{PLWC}\}/(\emph{πLλ}\emph{c}(\emph{T}\emph{s} − \emph{T}\emph{a}))}.
Then the value of {\emph{a}0} is updated via {\emph{a}0} +=
(Nu{′/\emph{Re}\emph{a}1 − \emph{a}0)/\emph{τ}} (using, for the GV,
separate coefficients for each of the three branches). In this formula,
{\emph{τ}} should be the number of updates in a fixed period, e.g., for
a 100 s time constant and for 25-Hz processing, {\emph{τ} = 100 × 25}.
In addition, to avoid jumps when switching among the branches, the
linear coefficients \{{\emph{a}0}\} are adjusted with each transition
between branches to provide a continuous estimate of the zero value.

\hypertarget{plwcg}{%
\subsubsection*{\texorpdfstring{PVM-100 Liquid Water Content
(g/m\textsuperscript{3}):
PLWCG}{PVM-100 Liquid Water Content (g/m3): PLWCG}}\label{plwcg}}
\addcontentsline{toc}{subsubsection}{PVM-100 Liquid Water Content
(g/m\textsuperscript{3}): PLWCG}

\emph{Cloud liquid water content for cloud droplets in the approximate
size range from 3--50 {\emph{μ}}m.} The PVM produces a measure of the
liquid water content directly, but a baseline value is sometimes
subtracted by reference to another cloud droplet instrument such as an
FSSP or CDP, such that when the other instrument measures a very low
droplet concentration the baseline value for the PVM-100 is updated at
the corresponding time and that average is then subtracted from the
measurements directly produced by the PVM-100.

\hypertarget{rice}{%
\subsubsection*{Rosemount Icing Detector Signal (V): RICE}\label{rice}}
\addcontentsline{toc}{subsubsection}{Rosemount Icing Detector Signal
(V): RICE}

\emph{The voltage related to loading on the element of a Rosemount 871F
ice-accretion probe.} This instrument (see this URL) consists of a rod
set in vibration by a piezoelectric crystal. The oscillation frequency
of the probe changes with ice loading, so in supercooled cloud ice
accumulates on the sensor and the change in oscillation frequency is
transmitted as a DC voltage. When the rod loads to a trigger point, the
probe heats the rod to remove the ice. The rate of voltage change can be
converted to an estimate of the supercooled liquid water content, as
described in connection with the obsolete variable SCLWC. This
calculation is no longer provided routinely but can be duplicated by a
user on the basis of the SCLWC algorithm (see
\protect\hyperlink{SCLWC}{SCLWC} in Section @ref(obsolete-variables) for
one example; there are several other published algorithms.)

\hypertarget{sensors-1-D-probes}{%
\section{Sensors Detecting Individual Hydrometeors (1-D
Probes)}\label{sensors-1-D-probes}}

The RAF operates a set of hydrometeor detectors that provide
single-dimension measurements (i.e., not images) of individual particle
sizes. RAF Bulletin 24 contains extensive information on the operating
principles and characteristics of some of the older instruments. Here
the focus will be on the meanings of the variables in the archived data
files.

\protect\hypertarget{VariableNames1DProbes}{}{}Four- and five-character
variable names shown in this section are generic. The actual names
appearing in NIMBUS-generated production output data sets have appended
to them an underscore (\_) and three or four more characters that
indicate a probe's specific aircraft mounting location. For example,
AFSSP\_RPI refers to a variable from an FSSP-100 probe mounted on the
inboard, right-side pod. The codes presently in use are given in the
following table. For the GV, there are 12 locations available,
characterized by three letters. The first is the wing (\{L,R\} for
\{port,starboard\}), the second is the pylon (\{I,M,O\} for inboard,
middle, outboard\}), the third is which of the two possible canister
locations at the pylon is used (\{I,O\} for \{inboard, outboard\}).

\begin{table}
\centering
\begin{tabular}{c|c|c}
\hline
suffix & location & aircraft\\
\hline
OBL & Outboard Left & C-130Q\\
\hline
IBL & Inboard Left & C-130Q\\
\hline
OBR & Outboard Right & C-130Q\\
\hline
IBR & Inboard Right & C-130Q\\
\hline
LPO & Left Pod Outboard & C-130Q\\
\hline
LPI & Left Pod Inboard & C-130Q\\
\hline
LPC & Left Pod Center & C-130Q\\
\hline
RPO & Right Pod Outboard & C-130Q\\
\hline
RPI & Right Pod Inboard & C-130Q\\
\hline
RPC & Right Pod Center & C-130Q\\
\hline
OBL & Left Wing & Electra\\
\hline
IBL & Left Pylon & Electra\\
\hline
WDL & Window Left & Electra\\
\hline
OBR & Right Wing & Electra\\
\hline
IBR & Right Pylon & Electra\\
\hline
WDR & Window Right & Electra\\
\hline
\{L,R\}W\{I,M,O\}\{I,O\} & see discussion above & GV\\
\hline
\end{tabular}
\end{table}

The probe type also is coded into each variable's name, sometimes using
four characters, sometimes only one: FSSP-100 (FSSP or F), FSSP-300
(F300 or 3), CDP (CDP or D), UHSAS (UHSAS or U), PCASP (PCAS or P),
OAP-200X (200X or X), OAP-260X (260X or 6) and OAP-200Y (200Y or Y).
Prefix letters are used to identify the type of measurement
(A=accumulated particle counts per time interval per channel, C =
concentration per channel, CONC = Concentration from all channels, DBAR
= mean diameter, DISP = dispersion, PLWC =liquid water content, DBZ =
radar reflectivity factor).

Some of the probes discussed in this section are primarily aerosol
spectrometers but are described here rather than in Section
@ref(aerosol-particle-measurements) because they are similar to the
hydrometeor spectrometers and so are most economically discussed here.
However, see Section @ref(aerosol-particle-measurements) for the
processing algorithms that lead to concentrations from the UHSAS and
PCASP/SPP-200. The following table and discussion includes some obsolete
variables (for the 200X and 200Y) for the same reason. The table also
includes some variables derived from imaging spectrometers (the 2DC and
2DP probes) to highlight that the primary variables are similar to those
discussed in this sub-section. Those variables are discussed in the next
sub-section. In two cases, the FSSP and PCASP, two versions are listed,
an obsolete version and a current version with a revised processing
package (SPP-100 for the FSSP, SPP-200 for the PCASP). Both are included
for historical completeness, but algorithms in this document discuss the
current versions.

The archived data files sometimes have ``housekeeping'' variables
included that provide information on the operating state and data
quality from the probes. For example, the CDP provides information on
the average transit time, the voltage from the nominal 5-V source, the
control board temperature, the laser block temperature, the laser
current, the laser power monitor, the qualifier bandwidth, the qualifier
baseline, the qualifier threshold, the sizer baseline, the wing-board
temperature, an A-to-D overflow flag, and a count of particles rejected
as being outside the depth of field. The netCDF variables and attributes
should be consulted for this housekeeping information. The large number
of housekeeping variables will not be included in this document, so
appropriate manuals and the netCDF files should be consulted when
interpreting this information.

\textbf{Probes that produce size distributions of particles (with links
to descriptions):\protect\hypertarget{TableOfProbes}{}{}}

\textbf{Generic Name}

Probe\footnote{Probes without links are described at this URL.}

Channels

Usable\footnote{Channels may be unusable because the first channel is a
  historical carry-over and should be ignored, or because in the case of
  2D probes the entire-in sizing technique reduces the number of bins
  where particles can be sized. Also, when some channels have been
  considered unreliable the netCDF header may specify that the usable
  bins are smaller than indicated here.}

Diameter Range

Bin Width

{FSSP-100 original}

{F}

{FSSP-100}\footnote{Now obsolete but present in many archived data sets.}

{0--15}

{1--15}

FSSP/SPP-100

F

SPP-100

0--30

1--30

3 {\emph{μ}}m (typ.)

UHSAS

U

UHSAS

0--99

1--99

variable

CDP

D

CDP

0--30

1---30

variable

{F300}

{3}

{FSSP-300{\emph{b}}}

{0--30}

{1--30}

{0.3--20.0 {\emph{μm}}}

{variable}

PCASP/original

P

PCASP{\emph{b}}

0--15

1--15

0.1--3.0 {\emph{μ}}m

variable

PCASP/SPP-200

P

{SPP-200}

{0--30}

{1--30}

{0.1--3.0 {\emph{μm}}}

{variable}

{200X}

{X}

OAP-200X{{\emph{b}}}

{0--15}

{1--15}

{40--280 {\emph{μm}}}

{10 {\emph{μm}}}

{260X}

{6}

OAP-260X

{0-63}

{3--62}

{40-620 {\emph{μm}}}

{10 {\emph{μm}}}

{200Y}

{Y}

{OAP-200Y{\emph{b}}}

{0-15}

{1--15}

{300--4500 {\emph{μm}}}

{300 {\emph{μm}}}

1DC\footnote{See \protect\hyperlink{special-1d-nomenclature}{here} for
  an explanation of this name convention.}

2DC{\emph{b},}\footnote{Measurements from this and the next three 2D
  probes are discussed in Section~@ref(hydrometeor-imaging-probes).}
(old)

0-32

1-30{\emph{e}}

25--800 {\emph{μ}}m

25 {\emph{μ}}m

1DP

2DP{\emph{b}} (old)

0-32

1-30

200--6400 {\emph{μ}}m

200 {\emph{μ}}m

1DC

2DC (fast)

0-63

1-62\footnote{Some of the lowest channels are often considered
  unreliable and excluded in processing.}

25--1600 {\emph{μ}}m

25 {\emph{μ}}m

1DP

2DP (new)

0-63

1-62

100--6400 {\emph{μ}}m

100 {\emph{μ}}m

\hypertarget{CRPC}{%
\subsection*{Count Rate Per Channel: ACDP, AFSSP, AS100, AF300, AS200,
APCAS, A200X, A260X, A200Y, AUHSAS}\label{CRPC}}
\addcontentsline{toc}{subsection}{Count Rate Per Channel: ACDP, AFSSP,
AS100, AF300, AS200, APCAS, A200X, A260X, A200Y, AUHSAS}

\emph{The size distribution of the number of particles detected by a 1D
hydrometeor probe per unit time.} These measurements have ``vector''
character in the NetCDF output files, with dimension equal to the number
of channels in the table above and with one entry per channel. The first
element in the vector is a historical remnant from a time when
housekeeping information was stored here and should be ignored. For the
size limits of the channels, see the netCDF attributes of the following
variables for ``Size Distribution''.

\underline{}

\hypertarget{size-distribution}{%
\subsection*{Size Distribution:}\label{size-distribution}}
\addcontentsline{toc}{subsection}{Size Distribution:}

Size Distribution ({\emph{cm} − 3}channel{ − 1}): \underline{CFSSP},
\underline{CS100}, \underline{CF300}, \underline{CS200},
\underline{CPCAS}, \underline{CCDP}, \underline{CUHSAS} Size
Distribution (L{ − 1}channel{ − 1}): \underline{C200X},
\underline{C260X}, \underline{C200Y}

\protect\hypertarget{CUHSAS}{}{CUHSAS} \emph{The particle
concentrations} \emph{in each usable bin of the probe.} These netCDF
variables have ``vector'' character with dimension equal to the number
of channels in the table above. The first vector member should be
ignored. For some scattering spectrometer probes (FSSP-100, FSSP-300,
PCASP) the concentration value is modified by the probe activity (FACT,
PACT) as described below. The concentration is obtained from the total
number of particles detected and a calculated, probe-dependent sample
volume that is specified in recent projects by attributes (e.g., depth
of field and beam diameter) of this variable in the netCDF file. For
additional details, see the links in the table or, for older probes, RAF
Bulletin 24.

\hypertarget{concentration-cm-3-concd-concf-conc3-concp-concu}{%
\subsection{\texorpdfstring{Concentration (cm{ − 3}): CONCD, CONCF,
CONC3, CONCP,
CONCU;}{Concentration (cm − 3): CONCD, CONCF, CONC3, CONCP, CONCU;}}\label{concentration-cm-3-concd-concf-conc3-concp-concu}}

(L{ − 1}): CONCX, CONC6, CONCY \{\#concentration .unnumbered\}

\emph{The particle concentrations} \emph{summed over all channels to
give the total concentration in the size range of the probe.} For
example, \{CONCF\} = {∑\emph{i}\{\emph{CFSSP}\}\emph{i}}. For additional
details, see RAF Bulletin 24.

\hypertarget{mean-diameter}{%
\subsection*{\texorpdfstring{Mean Diameter ({\emph{μ}}m): DBARD, DBARF,
DBAR3, DBAR6, DBARP, DBARX, DBARY,
DBARU}{Mean Diameter (μm): DBARD, DBARF, DBAR3, DBAR6, DBARP, DBARX, DBARY, DBARU}}\label{mean-diameter}}
\addcontentsline{toc}{subsection}{Mean Diameter ({\emph{μ}}m): DBARD,
DBARF, DBAR3, DBAR6, DBARP, DBARX, DBARY, DBARU}

\emph{The arithmetic average of all measured particle diameters from a
particular probe.} This mean is calculated as follows:

\{Cy\(_{i}\)\} =
concentration\sindex[lis]{Cyi@$Cy_{i}$= concentration from hydrometeor probe y in channel i}
from probe y in channel i \hspace{0.6cm}(e.g.,
y=\protect\hyperlink{size-distribution}{FSSP} to calculate DBARF)\\
i1 = lowest usable channel for the probe\\
i2 = highest usable channel for the probe\\
\(d_{i}\)\sindex[lis]{di@$d_{i}$= diameter of hydrometeor in channel $i$}
= diameter of particles in channel i for this probe (\(\mu m\))\\
\hspace*{0.6cm}(calculated as the average of the lower and upper size
limits for the channel)

\begin{equation}
\mathrm{\{DBARx\}}=\frac{{\textstyle \sum_{i=i1}^{i2}}{\displaystyle {\displaystyle \left\{ \mathrm{Cy}_{i}\right\} d_{i}}}}{\sum_{i=i1}^{i2}\left\{ \mathrm{Cy}_{i}\right\} }
(\#eq:DBARbox)
\end{equation}

\hypertarget{dispersion}{%
\subsection*{Dispersion (dimensionless): DISPD, DISPF, DISP3, DISP6,
DISPP, DISPX, DISPY, DISPU}\label{dispersion}}
\addcontentsline{toc}{subsection}{Dispersion (dimensionless): DISPD,
DISPF, DISP3, DISP6, DISPP, DISPX, DISPY, DISPU}

\emph{The ratio of the standard deviation of particle diameters to the
mean particle diameter.}

\{\protect\hyperlink{mean-diameter}{DBARx}\} = mean particle diameter
{[}\(\mu m\){]}\\
\{Cy\(_{i}\)\}, i1, i2, \(d_{i}\) as for mean diameter above

\begin{equation}
\mathrm{\{DISPx\}}==\frac{1}{\{\mathrm{DBARx}\}}\,\left\{ \frac{{\textstyle \sum_{i=i1}^{i2}}{\displaystyle {\displaystyle \left\{ \mathrm{Cy}_{i}\right\} d_{i}^{2}}}}{\sum_{i=i1}^{i2}\left\{ \mathrm{Cy}_{i}\right\} }-\{\mathrm{DBARx}\}^{2}\right\} ^{1/2}
(\#eq:DISPbox)
\end{equation}

\hypertarget{PSD-LWC}{%
\subsection*{\texorpdfstring{Liquid Water Content (g m{ − 3}): PLWCD,
PLWCF, PLWCX, PLWC6,
PLWCY}{Liquid Water Content (g m − 3): PLWCD, PLWCF, PLWCX, PLWC6, PLWCY}}\label{PSD-LWC}}
\addcontentsline{toc}{subsection}{Liquid Water Content (g m{ − 3}):
PLWCD, PLWCF, PLWCX, PLWC6, PLWCY}

\emph{The density of liquid water represented by the size distribution
measured by a hydrometeor probe.} These variables are calculated from
the measured concentration (CONCx) and the third moment of the particle
diameter, with the assumption that the particle is a water drop. The
following box describes the calculation in terms of an equivalent
droplet diameter, the diameter that represents the mass in the detected
particle. The equivalent droplet diameter is normally the measured
diameter for liquid hydrometeors, but some processing has used other
assumptions and this is a choice that can be made based on project
needs. Using this definition allows for the approximate estimation of
ice water content in cases where it is known that all hydrometeors are
ice.

\(d_{e,i}\)\sindex[lis]{dei@$d_{e,i}$= equivalent melted diameter for channel i of a hydrometeor
spectrometer} = equivalent melted diameter for channel \(i\) of probe
x\\
\{Cy\(_{i}\)\}, i1, i2 as for mean diameter above\\
\(\varrho_{w}\)\sindex[lis]{rhow@$\rho_{w}$= density of liquid water} =
density of water {[}\(10^{3}kg/m^{3}\){]}

\begin{equation}
\mathrm{\{PLWCx\}}=\frac{\pi\varrho_{w}}{6}{\textstyle \sum_{i=i1}^{i2}}{\displaystyle {\displaystyle \left\{ \mathrm{Cy}_{i}\right\} d_{e,i}^{3}}}
(\#eq:LWCbox)
\end{equation} (units and a scale factor are selected so that the output
variable is in units of g\(\,\)m\(^{-3}\))

\hypertarget{DBZ}{%
\subsection*{Radar Reflectivity Factor (dbZ): DBZF, DBZX, DBZ6, DBZY,
DBZD}\label{DBZ}}
\addcontentsline{toc}{subsection}{Radar Reflectivity Factor (dbZ): DBZF,
DBZX, DBZ6, DBZY, DBZD}

\emph{The radar reflectivity factor} \emph{calculated from the measured
size distribution from a hydrometeor probe.} This is calculated from the
measured concentration and the sixth moment of the size distribution,
with the assumption that the particles are water drops. An equivalent
radar reflectivity factor can be calculated from the hydrometeor size
distribution if another assumption is made about composition of the
particles, but this variable is not part of normal data files. The radar
reflectivity factor is a characteristic only of the hydrometeor size
distribution; it is \emph{not} a measure of radar reflectivity, because
the latter also depends on wavelength, dielectric constant, and other
characteristics of the hydrometeors. The normally used radar
reflectivity factor is measured on a logarithmic scale that depends on a
particular choice of units, so (although it is not conventional) an
appropriate scale factor {\emph{Z}\emph{r}} is included in the following
equation to satisfy the convention that arguments of logarithms should
be dimensionless.

\(d_{i}\) = diameter for channel \(i\) of probe x\\
\{Cy\(_{i}\)\}, i1, and i2 as for mean diameter above\\
\(Z_{r}\)
\sindex[lis]{Zr@$Z_{r}$= scale factor for calculation of the radar reflectivity
factor}= reference factor for units = 1~mm\(^{6}\)m\(^{-3}\)

\begin{equation}
\mathrm{\{DBZx\}}=10\log_{10}\left({\textstyle \frac{1}{Z_{r}}\sum_{i=i1}^{i2}}{\displaystyle {\displaystyle \left\{ \mathrm{Cy}_{i}\right\} d_{i}^{6}}}\right)
(\#eq:DBZbox)
\end{equation}

\hypertarget{effective-radius}{%
\subsection*{\texorpdfstring{Effective Radius ({\emph{μ}}m): REFFD,
REFFF}{Effective Radius (μm): REFFD, REFFF}}\label{effective-radius}}
\addcontentsline{toc}{subsection}{Effective Radius ({\emph{μ}}m): REFFD,
REFFF}

\emph{One-half the ratio of the third moment of the diameter
measurements to the second moment.} This variable is useful in some
calculations that relate the liquid water content of a cloud layer to
its optical properties.

\(d_{i}\) = diameter for channel \(i\) of probe x\\
\{Cy\(_{i}\)\}, i1, and i2 as for mean diameter above

\begin{equation}
\mathrm{\{REFFx\}}=\frac{1}{2}\frac{\sum{\displaystyle {\displaystyle \left\{ \mathrm{Cy}_{i}\right\} d_{i}^{3}}}}{\sum{\displaystyle {\displaystyle \left\{ \mathrm{Cy}_{i}\right\} d_{i}^{2}}}}
(\#eq:REFFbox)
\end{equation}

\hypertarget{fssp-range}{%
\subsection*{FSSP-100 Range (dimensionless): FRNG,
FRANGE}\label{fssp-range}}
\addcontentsline{toc}{subsection}{FSSP-100 Range (dimensionless): FRNG,
FRANGE}

\emph{The size range in use for the FSSP-100} \emph{probe}.

Range

Nominal Size Range

Nominal Bin Width

0

2--47 {\emph{μm}}

3 {\emph{μm}}

1

2--32 {\emph{μm}}

2 {\emph{μm}}

2

1--15 {\emph{μm}}

1 {\emph{μm}}

3

0.5--7.5 {\emph{μm}}

0.5 {\emph{μm}}

In recent NETCDF data files, the actual bin boundaries used for
processing are recorded in the header. That header should be consulted
because processing sometimes uses non-standard sizes selected to adjust
for Mie scattering, which causes departures from the nominal linear
bins. Recent projects have all used range 0, but other choices have been
made in some older projects and other ranges are still available to
future projects.

\hypertarget{hydrometeor-imaging-probes}{%
\section{Hydrometeor Imaging Probes}\label{hydrometeor-imaging-probes}}

Instruments used to obtain hydrometeor images include the
two-dimensional imaging probes (2DC and 2DP) and some others that
require special processing and separate data records. The former are
described in this subsection. The latter include a three-view cloud
particle imager (3V-CPI), a small ice detector (SID-2H), and a
holographic imager (HOLODEC). For information regarding use of data from
the latter set of instruments, consult EOL/RAF data management via this
email address.

In addition to the standard processing that produces the variables in
this subsection, an alternate processor is available that provides some
additional options and capabilities, including the production of two
sets of variables that include either all particles or all particles
that pass a roundness test. Additional options include different ways of
defining the particle size (including circle fitting or sizing based on
the dimension along the direction of flight. Corrections to sizing are
made to account for diffraction, and a shattering correction can be
applied based on interarrival times. Some categories of spurious images
(e.g., ``streakers'') can be recognized and rejected. This processing is
described in this document and at this web page and is made available by
special arrangement.

Measurements based on the two 2D probes will be discussed together in
this section because the 2DC and 2DP probes function similarly,
differing primarily in the size resolution (typically 25 {\emph{μ}}m or
less for the 2DC and 100 or 200 {\emph{μ}}m for the 2DP). The following
variables have names like CONC1DC or CONC1DP to designate the two types
of hydrometeor imagers. In addition, variables normally have location
designations like '\_LWIO' as described at the beginning of section ;
see page . In the following 'y' is sometimes used to designate either
'C' or 'P'.

For the images from the 2D probes, separate data files need to be used.
RAF provides a routine ``XPMS2D'' that can be used to view the images
and calculate various properties of the hydrometeor population based on
these separate files.

\hypertarget{special-1d-nomenclature}{%
\subsubsection*{Special 1D Nomenclature}\label{special-1d-nomenclature}}
\addcontentsline{toc}{subsubsection}{Special 1D Nomenclature}

\protect\hypertarget{Despite-the-ux271Dux27}{}{}Despite the '1D'
nomenclature, the following variables are measured by 2D instruments;
the '1D' designation is used to indicate that this is the dimension that
would be sized by an equivalent 1D probe using a test that requires
unshadowed end diodes so that the full dimension of the particle can be
determined. As a consequence, the effective sample volume becomes
smaller as the measured dimension increases.

\hypertarget{a1dc-a1dp}{%
\subsubsection*{2D Count Rate Per Channel (count per time interval):
A1DC, A1DP}\label{a1dc-a1dp}}
\addcontentsline{toc}{subsubsection}{2D Count Rate Per Channel (count
per time interval): A1DC, A1DP}

\emph{The number of particles counted by a 2D probe in each of 62 size
bins in a specified time interval, usually 1 s.} These are used to
calculate the derived variables like CONC1DC, C1DC, and others that
follow, but are provided to allow re-calculation if a user wants to use
different sample volumes or sizing assumptions.

\hypertarget{c1dc-c1dp}{%
\subsubsection*{\texorpdfstring{2D Size Distribution
(L{ − 1}channel{ − 1}): C1DC,
C1DP}{2D Size Distribution (L − 1channel − 1): C1DC, C1DP}}\label{c1dc-c1dp}}
\addcontentsline{toc}{subsubsection}{2D Size Distribution
(L{ − 1}channel{ − 1}): C1DC, C1DP}

\emph{The concentration of particles measured by a 2D probe in each of
62 bins in a specified time interval, usually 1 s.} These are calculated
from A1DC by application of an assumed size-dependent sample volume
based on probe characteristics and the flight speed. These are provided
in a 64-element array for historical convention; the first element
should be ignored, and the technique requires that the end elements be
unshadowed and so precludes any measurement with width of 63 bins, so
the 64-element vector has valid information only in bins 1--63. The cell
boundaries are specified in the netCDF header as an attribute of C1DC or
C1DP, and they specify the end points of the bin; e.g,, in the
64-element array of provided cell boundaries, the first element is the
lower size limit of the first data cell which is the second element in
C1DC. For a typical 2DC with 25-{\emph{μ}}m size resolution, the cell
sizes increase by 25 {\emph{μ}}m per bin for each of the C1DC bins. Also
included as attributes with the netCDF variable C1DC or C1DP are the
size-dependent depth of field (mm) and effective sample area\footnote{commonly
  called ``EffectiveAreaWidth'' in the netCDF files} (mm), the latter
having values of zero for the first and last elements in the 64-value
vector.

\hypertarget{conc2d}{%
\subsubsection*{\texorpdfstring{2D Concentration (L{ − 1}): CONC1DC,
CONC1DC100, CONC1DC150,
CONC1DP}{2D Concentration (L − 1): CONC1DC, CONC1DC100, CONC1DC150, CONC1DP}}\label{conc2d}}
\addcontentsline{toc}{subsubsection}{2D Concentration (L{ − 1}):
CONC1DC, CONC1DC100, CONC1DC150, CONC1DP}

\emph{The total concentration of all particles detected by a 2D
hydrometeor imager,} or in the case of CONC1DC100 or CONC1DC150, the
concentration of all particles sized to be at least xxx {\emph{μ}}m in
the dimension perpendicular to the direction of flight, where xxx may be
100 150. These concentrations are the sum of the particle size
distribution given below (C1DC or C1DP), with appropriate channels
excluded for CONC1DC100 and CONC1DC150.

\hypertarget{dt1dc}{%
\subsubsection*{2D Dead Time (ms): DT1DC}\label{dt1dc}}
\addcontentsline{toc}{subsubsection}{2D Dead Time (ms): DT1DC}

\emph{The time in the sample interval during which the data rate
exceeded the recording capability of a 2DC probe.} This is used as a
correction factor when concentrations like CONC1DC or C1DC are
calculated. The variable does not apply to measurements from a 2DP
probe.

\hypertarget{dbar2d}{%
\subsubsection*{\texorpdfstring{2D Mean Diameter ({\emph{μ}}m): DBAR1DC,
DBAR1DP}{2D Mean Diameter (μm): DBAR1DC, DBAR1DP}}\label{dbar2d}}
\addcontentsline{toc}{subsubsection}{2D Mean Diameter ({\emph{μ}}m):
DBAR1DC, DBAR1DP}

\emph{The mean diameter calculated from the measured size distribution.}
In this calculation, the bin sizes are taken to be the averages of the
lower and upper limits of the size bins\emph{.} The calculation is as
described by @ref(eq:DBARbox).

\hypertarget{disp2d}{%
\subsubsection*{2D Dispersion (dimensionless): DISP1DC,
DISP1DP}\label{disp2d}}
\addcontentsline{toc}{subsubsection}{2D Dispersion (dimensionless):
DISP1DC, DISP1DP}

\emph{The standard deviation in particle diameter divided by the mean
diameter.} The formula used is given by @ref(eq:DISPbox).

\hypertarget{lwc2d}{%
\subsubsection*{\texorpdfstring{2D Liquid Water Content (g m{ − 3}):
PLWC1DC,
PLWC1DP}{2D Liquid Water Content (g m − 3): PLWC1DC, PLWC1DP}}\label{lwc2d}}
\addcontentsline{toc}{subsubsection}{2D Liquid Water Content
(g m{ − 3}): PLWC1DC, PLWC1DP}

\emph{The liquid water content (mass per volume) calculated from C1DC or
C1DP.} The calculation is as described by @ref(eq:LWCbox). To conform to
common usage, the liquid water content is expressed in non-MKS units of
g m{ − 3}.

\hypertarget{dbz2d}{%
\subsubsection*{2D Radar Reflectivity Factor (dBZ): DBZ1DC,
DBZ1DP}\label{dbz2d}}
\addcontentsline{toc}{subsubsection}{2D Radar Reflectivity Factor (dBZ):
DBZ1DC, DBZ1DP}

\emph{The radar reflectivity factor calculated from the measured size
distribution under the assumption that all particles are spherical water
drops.} The calculation is as described by @ref(eq:DBZbox).

\hypertarget{reff2d}{%
\subsubsection*{\texorpdfstring{2D Effective Radius ({\emph{μ}}m):
REFF2DC,
REFF2DP}{2D Effective Radius (μm): REFF2DC, REFF2DP}}\label{reff2d}}
\addcontentsline{toc}{subsubsection}{2D Effective Radius ({\emph{μ}}m):
REFF2DC, REFF2DP}

\emph{One-half the ratio of the third moment of the particle diameter to
the second moment.} The formula used is given by @ref(eq:REFFbox).

\hypertarget{air-chemistry-measurements}{%
\chapter{Air Chemistry Measurements}\label{air-chemistry-measurements}}

\hypertarget{variables-in-standard-data-files}{%
\section{Variables in Standard Data
Files}\label{variables-in-standard-data-files}}

\hypertarget{coraw-al}{%
\subsubsection*{Carbon Monoxide Preliminary Mixing Ratio (ppbv):
CORAW\_AL}\label{coraw-al}}
\addcontentsline{toc}{subsubsection}{Carbon Monoxide Preliminary Mixing
Ratio (ppbv): CORAW\_AL}

The preliminary measurement of CO mixing ratio from the Aero-Laser model
AL-5002 CO analyzer, before final calibrations are applied. This
instrument measures CO by vacuum ultraviolet resonance fluorescence. It
is a commercial version of the instrument described by Gerbig et
al.\footnote{Journal of Geophysical Research, Vol. 104, No.~D1,
  1699-1704, 1999.} The instrument is described further at this URL. The
time resolution is 1 second. This variable is sometimes present in
flight and in preliminary ground processing, but normally it is replaced
by COMR\_AL in final processing.

\hypertarget{comr-al}{%
\subsubsection*{Carbon Monoxide Mixing Ratio (ppbv):
COMR\_AL}\label{comr-al}}
\addcontentsline{toc}{subsubsection}{Carbon Monoxide Mixing Ratio
(ppbv): COMR\_AL}

The mixing ratio measured by the Aero-Laser model AL-5002 CO analyzer.
See also CORAW\_AL above. The calculation of COMR\_AL is based on
in-flight calibrations conducted 1-2 times per hour, when a gas of known
concentration is supplied to the instrument and then a catalyst trap
removes CO to provide a zero reference. The calibration results in a
sensitivity and zero that are then used to convert the measurements from
the instrument (recorded as counts per second) to a mixing ratio in
units of ppbv. Time-dependent sensitivity and zero coefficients are
computed post-flight as a linear interpolation between flight
calibrations. This variable normally appears in final data sets for a
project.\footnote{In isolated cases XCOMR or XCOMR\_AL was used for this
  variable name.} The algorithm is described in the following box:

CPS\index{CPS=counts per second} = counts per second from the
instrument\\
S(t) = sensitivity\sindex[lis]{S(t)=sensitivity function, calibration}
at time t\\
\hspace*{0.333em}\hspace*{0.333em}\hspace*{0.333em}\hspace*{0.333em}\hspace*{0.333em}\hspace*{0.333em}\hspace*{0.333em}=
(CPS when exposed to cal gas) / concentration of cal gas\\
Z(t) = zero\sindex[lis]{Z(t)=zero function, calibration} at time t\\
\hspace*{0.333em}\hspace*{0.333em}\hspace*{0.333em}\hspace*{0.333em}\hspace*{0.333em}\hspace*{0.333em}\hspace*{0.333em}=
CPS when exposed to air passing through the catalyst trap

\begin{equation}
\mathrm{\{COMR\_AL\}} = (\mathrm{\{CPS\}}-Z(t))/S(t)
(\#eq:COMRbox)
\end{equation}

See also the obsolete variables in Section @ref(obsolete-variables),
where variables from an earlier TECO Model 48 CO analyzer, in use before
2000, are described.

\hypertarget{co2-pic}{%
\subsubsection*{Carbon Dioxide and Methane Mixing Ratios (ppmv):
CO2\_PIC and CH4\_PICx}\label{co2-pic}}
\addcontentsline{toc}{subsubsection}{Carbon Dioxide and Methane Mixing
Ratios (ppmv): CO2\_PIC and CH4\_PICx}

Respectively, the carbon dioxide and methane mixing ratio measured by a
Picarro CO2/CH4 instrument. The letter 'x' may be replaced by the model
number of the instrument (e.g., 1301) or it may be blank. The Picarro
CO2/CH4 G1301-f flight analyzer is a fast response trace gas monitor
that measures CO{2} and CH{4} using wavelength-scanned cavity ring-down
spectroscopy. The time resolution is 0.2 -- 1 seconds. Additional
information characterizing the instrument can be found at this URL.
During flight, both measurements are calibrated 1-2 times per hour via
sampling of a working standard, and linear calibration coefficients are
applied based on multi-point lab calibration data and in-flight
calibration checks. The procedure is analogous to that used for
COMR\_AL, as described immediately above. When water vapor is not
removed from the ambient sample stream (the normal case), a correction
factor for water present in the sensing cell must be applied following
the approach of Richardson et al.,\footnote{Richardson, S.~J.,
  N.~L.~Miles, K.~J.~Davis, E.~R.~Crosson, C.~W.~Rella, and
  A.~E.~Andrews, 2012: Field testing of cavity ring-down spectroscopy
  analyzers measuring carbon dioxide and water vapor.
  J.~Atmos.~Oceanic\_Technol, 29, 397--406.} as follows:

{[}CO\(_{2}\){]}\(_{wet}\) = carbon dioxide mixing ratio as measured in
the sensing cell (with water)\\
{[}CO\(_{2}\){]}\(_{dry}\) = carbon dioxide mixing ratio in dry air,
corrected for the effects of water vapor\\
{[}CH\(_{4}\){]}\(_{wet}\) = methane mixing ratio as measured in the
sensing cell (with water)\\
{[}CH\(_{4}\){]}\(_{dry}\) = methane mixing ratio in dry air, corrected
for the effects of water vapor\\
\(W\) = water vapor mixing ratio measured in the instrument cell
{[}percent by volume{]}\\
\{\(c_{0}\), \(c_{1}\)\} = \{\(-0.01200,\,-2.674\times10^{-4}\)\}
{[}dimensionless{]}\\
\{\(d_{0}\), \(d_{1}\)\} = \{\(-0.00982,\ -2.393\times10^{-4}\)\}
{[}dimensionless{]}

\begin{equation}
\{\mathrm{CO2\_PICX\}}=[\mathrm{CO_{2}]_{dry}=}\frac{[\mathrm{CO_{2}]_{wet}}}{1+c_{0}W+c_{1}W^{2}}
(\#eq:CO2PICbox1)
\end{equation} \begin{equation}
\{\mathrm{CH4\_PICX\}}=[\mathrm{CH_{4}]_{dry}=}\frac{[\mathrm{CH_{4}]_{wet}}}{1+d_{0}W+d_{1}W^{2}}
(\#eq:CO2PICbox2)
\end{equation}

\hypertarget{xfo3fs}{%
\subsubsection*{Chemiluminescent Ozone Sample and Nitric Oxide Flow
Rates (sccm): XFO3FS, XF03FNO}\label{xfo3fs}}
\addcontentsline{toc}{subsubsection}{Chemiluminescent Ozone Sample and
Nitric Oxide Flow Rates (sccm): XFO3FS, XF03FNO}

Flows within the chemiluminescence ozone sensor. The sample rate, in
standard {cm3/s}, is XFO3FS, while XFO3FNO gives the NO flow rate in the
same units. These variables apply to measurements made by an earlier
version of the fast ozone instrument. They have not been present in
projects since 2006.

\hypertarget{xfo3p}{%
\subsubsection*{Chemiluminescent Ozone Sample Pressure (mb):
XFO3P}\label{xfo3p}}
\addcontentsline{toc}{subsubsection}{Chemiluminescent Ozone Sample
Pressure (mb): XFO3P}

Sample pressure in the chemiluminescence ozone sensor. This variable was
associated with measurements made by an earlier version of the fast
ozone instrument. It has not been present in projects since 2006.

\hypertarget{fo3-acd}{%
\subsubsection*{Fast response NO chemiluminescence ozone mixing ratio
(ppbv): FO3\_ACD, FO3\_CL, XO3, O3MR\_CL}\label{fo3-acd}}
\addcontentsline{toc}{subsubsection}{Fast response NO chemiluminescence
ozone mixing ratio (ppbv): FO3\_ACD, FO3\_CL, XO3, O3MR\_CL}

The ozone mixing ratio (by volume) measured by an NO chemiluminescence
instrument. The instrument detects chemiluminescence from the reaction
of nitric oxide (NO) with ambient ozone, using a dry-ice cooled,
red-sensitive photomultiplier employing photon-counting electronics. The
measurement principle is described by Ridley et al.~(1992),\footnote{Ridley,
  B.~A., F.~E.~Grahek, and J.~G.~Walega, 1992: A small,
  high-sensitivity, medium-response ozone detector suitable for
  measurements from light aircraft. J.~Atmos.~Oceanic Technol., 9,
  142--148.} and there is additional information describing the
instrument at this URL. The time resolution is 0.2 seconds, and typical
uncertainty is 5\%. The background signal is measured 1-2 times hourly
during flights. Linear calibration coefficients are applied to the
photon count rate to produce mixing ratios, and a correction is applied
for water vapor during final processing, as follows:

CPS\index{CPS=counts per second} = counts per second from the
instrument\\
{[}O\(_{3}\){]}\(_{wet}\) = ozone mixing ratio as measured in the
sensing cell (with water)\\
{[}O\(_{3}\){]}\(_{dry}\) = ozone mixing ratio in dry air, corrected for
the effects of water vapor\\
\(S(t)\) = sensitivity at time t = (CPS when exposed to cal gas) /
concentration of cal gas\\
\(Z(t)\) = background at time t = CPS when exposed to zero-ozone air\\
\(r_v\) = water vapor mixing ratio by
volume\sindex[lis]{rv@$r_{v}$=water vapor mixing ratio by volume}
{[}expressed as a fraction; dimensionless{]}\\
\(\kappa\) = correction factor for water vapor = 4.3 {[}dimensionless{]}

\begin{equation}
[\mathrm{O}_{3}]_{wet}=\frac{\mathrm{\{CPS\}}-Z(t)}{S(t)}
(\#eq:O3box1)
\end{equation} \begin{equation}
\mathrm{\{F03\_ACD\}}=[\mathrm{O_{3}}]_{dry} = \mathrm{[O_{3}]_{wet}}\times(1+\kappa r_{v})
(\#eq:O3box2)
\end{equation}

\hypertarget{te03}{%
\subsubsection*{Uncorrected TECO Ozone Mixing Ratio (ppb):
TEO3}\label{te03}}
\addcontentsline{toc}{subsubsection}{Uncorrected TECO Ozone Mixing Ratio
(ppb): TEO3}

The uncorrected ozone mixing ratio output from the TECO model 49c UV
ozone analyzer. See TEO3C.

\hypertarget{tep}{%
\subsubsection*{Internal TECO Ozone Sampling Pressure (hPa): TEP,
TEO3P}\label{tep}}
\addcontentsline{toc}{subsubsection}{Internal TECO Ozone Sampling
Pressure (hPa): TEP, TEO3P}

The pressure inside the detection cell of the TECO 49 UV ozone analyzer.
This and the following temperature are used to convert the measurements
from the instrument to units of ppbv.

\hypertarget{tet}{%
\subsubsection*{\texorpdfstring{Internal TECO Ozone Sampling Temperature
({∘C}):
TET}{Internal TECO Ozone Sampling Temperature (∘C): TET}}\label{tet}}
\addcontentsline{toc}{subsubsection}{Internal TECO Ozone Sampling
Temperature ({∘C}): TET}

The temperature inside the detection cell of the TECO 49 UV ozone
analyzer. This and the preceding pressure are used to convert the
measurements from the instrument to units of ppbv. In many projects, the
cell temperature was not recorded so an expected cell temperature in the
aircraft cabin must be used in processing.

\hypertarget{te03c}{%
\subsubsection*{Corrected TECO Ozone Mixing Ratio (ppbv):
TEO3C}\label{te03c}}
\addcontentsline{toc}{subsubsection}{Corrected TECO Ozone Mixing Ratio
(ppbv): TEO3C}

The ozone mixing ratio (by volume) determined by the TECO model 49c UV
ozone analyzer (cf.~this description) after correction for the pressure
and temperature in the cell by application of the ideal gas law. Because
the basic measurement is ozone density in the chamber, this measurement
must be converted to a mixing ratio by dividing by the air density,
calculated from the pressure and temperature measured in the chamber
(TEP and TET respectively). The instrument provides output only each ten
seconds, and measurements are collected in the 3 s preceding the update.
The measurements may be artificially high or low when rapid changes in
humidity are present, as may occur when crossing the top of the boundary
layer or when going through clouds. In operation on the ground prior to
takeoff or immediately after landing, a high concentration of
hydrocarbons can cause spuriously high measurements. The detection limit
is 1 ppbv with an uncertainty of {±}5\%. This instrument is seldom used
as of 2014 and may soon be classified as obsolete.

\hypertarget{no-noy}{%
\subsubsection*{\texorpdfstring{NO, NO\textsubscript{y}
Variables:}{NO, NOy Variables:}}\label{no-noy}}
\addcontentsline{toc}{subsubsection}{NO, NO\textsubscript{y} Variables:}

NO Raw Counts (counts per sample interval): \underline{XNO} NOy Raw
Counts (counts per sample interval): \underline{XNOY} NO Calibration
Flow (SLPM): \underline{XNOCF} NOy Calibration Flow (SLPM):
\underline{XNCLF} NO, NOy Measurement Status (dimensionless):
\underline{XNST} NO Zero Air Flow (SLPM): \underline{XNOZA} NOy Zero Air
Flow (SLPM): \underline{XNZAF} NO Sample Flow (SLPM): \underline{XNOSF}
NOy Sample Flow (SLPM): \underline{XNSAF} NOy Reaction Chamber Pressure
(mb): \underline{XNOYP} Gold NOy Converter Temperature ({º}C):
\underline{XNMBT}

The measurements provided by the NO+NO{2} instrument, which is described
at this link. XNO and XNOY are the raw data counts from the NO and NO{2}
instruments, respectively, and XNCLF and XNOCF are the respective
calibration flows for these instruments. XNST records the status for
both instruments: In measurement mode, XNST is 0, while XNST is 5 when
the instruments are in zero mode and 10 when the instruments are in
calibration mode. the NOy and NO instruments. The instrument is in the
measure mode for XNST of 0. For a XNST reading of 5 the instruments are
in the zero mode. XNST value of 10 is the calibration mode. XNOZA and
XNZAF are flow rates for zero air used to back flush inlets, typically
at takeoff and landing, and for calibration using ``zero'' air. Even if
the status, XNST, is 0, indicating the instrument is in the measurement
mode, when XNOZA and XNZAF are approximately 1 SLPM the instrument is
measuring zero air and not ambient air. XNOSF and XNSAF are the sample
flow rates through the NO and NO{2} instruments respectively. These
values are typically about 1 SLPM. XNMBT is the temperature of the gold
NO{2} converter.

\hypertarget{mr-no-no2}{%
\subsubsection*{\texorpdfstring{Corrected NO and NO\textsubscript{2}
Mixing Ratios (ppbv): XNOCAL,
XNYCAL}{Corrected NO and NO2 Mixing Ratios (ppbv): XNOCAL, XNYCAL}}\label{mr-no-no2}}
\addcontentsline{toc}{subsubsection}{Corrected NO and
NO\textsubscript{2} Mixing Ratios (ppbv): XNOCAL, XNYCAL}

The calibrated NO and NO\textsubscript{2} volumetric mixing ratio,
respectively, measured by the NO-NO\textsubscript{2} instrument. See
this link for a description of the instrument. The NO and
NO\textsubscript{2} data are represented by a cubic spline for baseline
subtraction, and then the calibration coefficients are applied and the
measurements are converted to units of ppbv. The quality of the data can
be assessed by examining the accuracy of the zero correction. This
instrument adds water vapor to the sample stream to reduce the effect of
ambient water on the final signal. The water vapor addition is not
sufficient to saturate the sample stream, but enough to remove much of
the interference. The detection limits of the NO, NO\textsubscript{2}
instruments are 50 ppbv for a one-second averaging time. The uncertainty
is {±} 5\%.

\hypertarget{awas-cims-qcls-toga}{%
\section{Variables in Special Data Sets}\label{awas-cims-qcls-toga}}

Research projects often incorporate user-supplied instruments into
payloads, and those instruments produce data files that are either
recorded independently or merged into the standard netCDF data files for
the projects. In addition, NCAR offers a set of instruments that require
additional data processing and analysis, often because the measurements
require special interpretation to obtain the desired measurements. The
following instruments can provide such air-chemistry measurements:

\begin{itemize}
\item
  \setlength{\itemsep}{-1\parsep}Advanced Whole Air Sampler
  \href{http://www.eol.ucar.edu/instruments/advanced-whole-air-sampler}{AWAS}
\item
  Chemical Ionization Mass Spectrometer
  \href{http://www.eol.ucar.edu/instruments/georgia-tech-chemical-ionization-mass-spectrometer}{CIMS}
\item
  Quantum Cascade Laser Spectrometer
  \href{http://www.eol.ucar.edu/instruments/quantum-cascade-laser-spectrometer}{QCLS}
\item
  Trace Organic Gas Analyzer
  \href{http://www.eol.ucar.edu/instruments/trace-organic-gas-analyzer}{TOGA}
\end{itemize}

Follow the links in the box to descriptions of these instruments on the
EOL web site. Those descriptions include brief explanations of how data
are acquired and handled. The process varies with instrument; The CIMS
and QCLS instruments produce variables that are often merged into the
standard netCDF archived data files for projects, the AWAS collects
samples that are later analyzed using ground-based instruments but
result in a special dataset dependent on analysis technique and sample
location and duration, while the TOGA is usually analyzed to produce
dozens of trace-gas measurements, some of which can be merged into
standard netCDF files.

Users interested in using these measurements should contact EOL/RAF data
management for data access and assistance.

\hypertarget{aerosol-particle-measurements}{%
\chapter{Aerosol Particle
Measurements}\label{aerosol-particle-measurements}}

\hypertarget{condensation-nucleus-counter}{%
\section{Condensation Nucleus
Counter}\label{condensation-nucleus-counter}}

RAF uses two modified TSI, Inc.~condensation nucleus counters to measure
the total concentration of ultrafine particles in the atmosphere, a
3760A using n-butyl alcohol and a water-based 3786 WCN (water
condensation nucleus) counter. Both are sensitive to particles in the
approximate diameter range from 0.010--3~\textgreek{m}m.

\hypertarget{pcn}{%
\subsubsection*{CN Counter Inlet Pressure (hPa): PCN}\label{pcn}}
\addcontentsline{toc}{subsubsection}{CN Counter Inlet Pressure (hPa):
PCN}

{]} The absolute pressure inside the inlet tube of the instrument. It as
measured by a Heise Model 623 pressure sensor for the 3760A, and
internally by the 3786 WCN.. The measurement is used to convert the
measured mass flow (FCN or XICN) to volumetric flow and to convert
measured particle concentration to equivalent ambient concentration.

\hypertarget{cntemp}{%
\subsubsection*{\texorpdfstring{CN Counter Inlet Temperature ({∘}C):
CNTEMP, TEMP1,
TEMP2}{CN Counter Inlet Temperature (∘C): CNTEMP, TEMP1, TEMP2}}\label{cntemp}}
\addcontentsline{toc}{subsubsection}{CN Counter Inlet Temperature
({∘}C): CNTEMP, TEMP1, TEMP2}

The sample air temperature measured at the intake of the 3760A or within
the 3786. The value is used to convert the measured mass flow (FCN or
XICN) to true volumetric flow and to convert measured particle
concentration to equivalent ambient concentration.

\hypertarget{fcnc}{%
\subsubsection*{Raw and Corrected CN Counter Sample Flow Rate (SLPM,
VLPM): FCN, FCNC}\label{fcnc}}
\addcontentsline{toc}{subsubsection}{Raw and Corrected CN Counter Sample
Flow Rate (SLPM, VLPM): FCN, FCNC}

The raw and corrected sample flows in the CN counters are treated
differently for the two models of CN counter. In the 3760A, FCN is
measured in standard liters per minute (SLPM) with a mass flow meter.
The flow meter gives the volumetric flow rate that would apply under
standard conditions of 1013.25~hPa and 21{∘}C. FCNC is the corrected
sample flow rate in volumetric liters per minute (VLPM) at instrument
pressure and temperature. For the 3760A:

\protect\hyperlink{pcn}{PCN} = pressure at the inlet to the CN counter
{[}hPa{]}\\
\protect\hyperlink{cntemp}{CNTEMP} = temperature at the inlet of the
sample tube {[}\(^{\circ}\)C{]}\\
\(p_{std}\) = \sindex[lis]{Pstd@$p_{std}$=standard pressure}standard
reference pressure, 1013.25~hPa\\
\(T_{std}\) = \sindex[lis]{Tstd@$T_{std}$=standard temperature}standard
reference temperature, 294.15 K\\
\(T_{0}\) = 273.15~K

\begin{equation}
\mathrm{\{FCNC\} = \{FCN\}}\frac{p_{std}}{\mathrm{\{PCN\}}}\frac{(\{\mathrm{CNTEMP\}}+T_{0})}{T_{std}}
(\#eq:FCNCbox)
\end{equation}

In the 3786, flows are determined in volumetric
cm{\(^{3}\thinspace\mathrm{min}^{-1}\)} from the pressure drop across an
orifice. The 3786 firmware makes density corrections internally, so its
reported sample flow is brought directly into the variable FCNC in units
of VLPM.

\hypertarget{xicnc}{%
\subsubsection*{Raw and Corrected CN Isokinetic Side Flow Rate (SLPM,
VLPM): XICN, XICNC}\label{xicnc}}
\addcontentsline{toc}{subsubsection}{Raw and Corrected CN Isokinetic
Side Flow Rate (SLPM, VLPM): XICN, XICNC}

XICN is the raw isokinetic side flow rate in standard liters per minute
(SLPM) measured with a mass flow meter, and XICNC is that flow corrected
for pressure and temperature to be the true volumetric flow. The side
flow is adjusted for isokinetic sampling at the inlet, but it is not
used further in processing.

\protect\hyperlink{xicnc}{XICN} = side-flow rate {[}SLPM{]}\\
\protect\hyperlink{pcn}{PCN} = pressure at the inlet to the CN counter
{[}hPa{]}\\
\protect\hyperlink{cntemp}{CNTEMP} = temperature at the inlet of the
sample tube {[}\(^{\circ}\)C{]}\\
\(p_{std}\) = standard reference pressure, 1013.25 mb\\
\(T_{std}\)\sindex[lis]{Tr@$T_{std}$= absolute reference temperature, STP}
= 294.15 K\\
\(T_{0}\) = 273.15~K

\begin{equation}
\mathrm{\{XICNC\} = \{XICN\}}\frac{p_{std}}{\mathrm{\{PCN\}}}\frac{(\{\mathrm{CNTEMP\}}+T_{0})}{T_{std}}
(\#eq:XICNCbox)
\end{equation}

\hypertarget{cnts}{%
\subsubsection*{CN Counter Output (counts per sample interval):
CNTS}\label{cnts}}
\addcontentsline{toc}{subsubsection}{CN Counter Output (counts per
sample interval): CNTS}

The raw output count from the condensation nucleus counter. For the
3760A condensation nucleus counter, the project-dependent sample rate
may be chosen in the range from 1--50~Hz but it is typically 10~Hz. In
some unusual cases the counts are divided by a selected power of two to
keep the counter from overflowing; see project documentation. The 3786
WCN may be programmed to report data at intervals from 0.1--3600~s.

\hypertarget{concn}{%
\subsubsection*{\texorpdfstring{Condensation Nucleus (CN) Concentration
(cm\textsuperscript{-3}):
CONCN}{Condensation Nucleus (CN) Concentration (cm-3): CONCN}}\label{concn}}
\addcontentsline{toc}{subsubsection}{Condensation Nucleus (CN)
Concentration (cm\textsuperscript{-3}): CONCN}

The number concentration of condensation nuclei in units of particles
per cm{3} in the ambient air at flight level. The calculation leading to
CONCN includes two corrections. The first accounts for coincidence of
particles in the viewing volume at high concentrations and is handled
differently in the two types of CN counter. For the 3760A, a statistical
adjustment is made based on the average time of a particle in the
viewing volume. This correction increases from about 1\% at a total
concentration of 10{3}~cm{ − 3} to nearly 11\% at 10{4}~cm{ − 3}, but
for concentrations above about 210{4}~cm{ − 3} significant uncertainty
remains. The 3786 instead measures the time each detected particle
occupies the viewing volume, and this accumulated ``dead time'' in each
sampling interval is subtracted from the elapsed time yielding a ``live
time'' for the determination of sample volume. With this correction an
accuracy of 12\%, not including statistical counting error, is specified
by the manufacturer at concentrations up to 10{5}~cm{ − 3}. The second
correction, applied to all CN counters, is a conversion from instrument
to ambient conditions.\footnote{Prior to Dec.~2007 the conversion to
  ambient concentration was not made and concentration was reported for
  instrument conditions.} In the following formulae, the corrected flow
FCNC in VLPM is explicitly converted to cm{3}s{ − 1} by the factor
(1000/60).

For the 3760A:

CNTS = particle counts per sample interval from the CN
counter\index{CNTS}\\
\(\Delta t\) = \sindex[lis]{Deltat@$\Delta t$=time interval}interval
between recorded samples {[}s{]}\\
\(D\) = scale factor (legacy; normally 1)\\
\(C_{flow}\) =
\sindex[lis]{Cflow@$C_{flow}$ = flow conversion factor}conversion
factor, (1000/60) cm\(^{3}\)L\(^{-1}\)min s\(^{-1}\)\\
FCNC = corrected sample flow rate (VLPM) for instrument
conditions\index{FCNC}\\
\(t_{vv}\) = average time a particle is in the view volume\\
\hspace*{0.333em}\hspace*{0.333em}\hspace*{0.333em}\hspace*{0.333em}\hspace*{0.333em}\hspace*{0.333em}\hspace*{0.333em}=
0.4\(\times10^{-6}\)~s\\
PCN = pressure at the inlet to the CN counter {[}hPa{]}\index{PCN}\\
CNTEMP = temperature at the inlet of the sample tube
{[}\(^{\circ}\)C{]}\index{CNTEMP}\\
PSXC = corrected ambient pressure {[}hPa{]}\index{PSXC}\\
ATX = ambient temperature {[}\(^{\circ}\)C{]}\index{ATX}\\
\(T_{0}\) = 273.15~K

\begin{equation}
\mathrm{A=\frac{\{CNTS\}}{\mathrm{(\{FCNC\}\times C_{flow})}\Delta t}\,D}
(\#eq:CONCNbox1)
\end{equation} The flow under instrument conditions, corrected for
coincidence, is then\\
\begin{equation}
B\mathrm{=A}\,e^{At_{vv}(\mathrm{\{FCNC\}\times C_{flow})}}
(\#eq:CONCNbox2)
\end{equation} and the concentration under ambient conditions
is\index{CONCN}\\
\begin{equation}
\mathrm{\{CONCN\}}=B\frac{\mathrm{\{PSXC\}}}{\mathrm{\{PCN\}}}\frac{\mathrm{(\{CNTEMP\}}+T_{0})}{(\mathrm{\{ATX\}}+T_{0})}
(\#eq:CONCNbox3)
\end{equation}

For the 3786 WCN:

\protect\hyperlink{cnts}{CNTS} = particle counts per sample interval
from the CN counter\\
\(\Delta t\) = \sindex[lis]{Deltat@$\Delta t$=time interval}interval
between recorded samples {[}s{]}\\
\(t_{d}\) = cumulative dead time during the sampling interval {[}s{]}\\
\(C_{flow}\) (see preceding box)\\
\protect\hyperlink{fcnc}{FCNC} = corrected sample flow rate (VLPM) for
instrument conditions\\
\protect\hyperlink{pcn}{PCN} = internal pressure of the CN counter
{[}hPa{]}\\
\protect\hyperlink{cntemp}{CNTEMP} = temperature of the optics block
{[}\(^{\circ}\)C{]}\\
\href{./4-the-state-of-the-atmosphere.html\#psx}{PSXC} = corrected
ambient pressure {[}hPa{]}\\
\href{./4-the-state-of-the-atmosphere.html\#ambient-t}{ATX} = ambient
temperature {[}\(^{\circ}\)C{]}\\
\(T_{0}\) = 273.15~K

\begin{equation}
\mathrm{A=\frac{\{CNTS\}}{\mathrm{(\{FCNC\}\times C_{flow})}(\Delta t-t_{d})}}
(\#eq:CONCN2box1)
\end{equation} \begin{equation}
\mathrm{\{CONCN\}}=A\frac{\mathrm{\{PSXC\}}}{\mathrm{\{PCN\}}}\frac{\mathrm{(\{CNTEMP\}}+T_{0})}{(\mathrm{\{ATX\}}+T_{0})}
(\#eq:cwCONCN2box2)
\end{equation}

\hypertarget{aerosol-spec}{%
\section{Aerosol Spectrometers}\label{aerosol-spec}}

For size-resolved measurements of the concentration of aerosol
particles, RAF deploys two instruments. The Ultra High Sensitivity
Aerosol Spectrometer (UHSAS) sizes particles in 99 bins from 0.06 to
1.0~\textgreek{m}m diameter, and the Passive Cavity Aerosol Spectrometer
Probe (PCASP) has 30 channels covering the diameter range 0.1 to
3~\textgreek{m}m. Flow and total concentration variables for these
instruments are described in this section, while additional variables
are covered along with other 1-D probes in
Sect.~@ref(sensors-1-D-probes), ``Sensors Detecting Individual
Hydrometeors (1-D Probes).''

\hypertarget{upress}{%
\subsubsection*{UHSAS Absolute Pressure in Optics Block (kPa):
UPRESS}\label{upress}}
\addcontentsline{toc}{subsubsection}{UHSAS Absolute Pressure in Optics
Block (kPa): UPRESS}

The pressure internal to the UHSAS instrument. This is an analog
measurement with calibration coefficients as recorded in the attributes
for the variable.

\hypertarget{pflw}{%
\subsubsection*{\texorpdfstring{Raw and Corrected Sample Flow Rate
(cm\textsuperscript{3}s\textsuperscript{-1}): USMPFLW or PFLW; USFLWC or
PFLWC}{Raw and Corrected Sample Flow Rate (cm3s-1): USMPFLW or PFLW; USFLWC or PFLWC}}\label{pflw}}
\addcontentsline{toc}{subsubsection}{Raw and Corrected Sample Flow Rate
(cm\textsuperscript{3}s\textsuperscript{-1}): USMPFLW or PFLW; USFLWC or
PFLWC}

Unlike the other 1-d probes, both UHSAS and PCASP have internal pumps so
their sample volumes are determined from the measured flows and do not
depend on true air speed. The UHSAS measures volumetric flow directly,
and it is adjusted to ambient conditions for the calculation of ambient
concentration. The PCASP returns a mass flow referenced to standard
conditions, and this also is converted to equivalent ambient volumetric
flow.

\protect\hyperlink{upress}{UPRESS} = internal UHSAS pressure
{[}kPa{]}\index{UPRESS}\\
\protect\hyperlink{pflw}{USMPFLW} = measured volumetric sample flow
{[}cm\(^{3}\)s\(^{-1}\){]}\\
\protect\hyperlink{pflw}{PFLW} = sample mass flow referenced to standard
conditions {[}cm\(^{3}\)s\(^{-1}\){]}\\
\(T_{blk}\) = UHSAS optical block temperature, 305 K\\
\(p_{std}\) = standard pressure, 1013.25 hPa\\
\(T_{std}\) = standard temperature, 298.15 K\index{UHSAS!STP}\\
\href{./4-the-state-of-the-atmosphere.html\#psx}{PSXC} = corrected
ambient pressure {[}hPa{]}\\
\href{./4-the-state-of-the-atmosphere.html\#ambient-t}{ATX} = ambient
temperature {[}\(^{\circ}\)C{]}\\
\(T_{0}\) = 273.15~K

\begin{equation}
\mathrm{\{PFLWC\}}=\mathrm{\{PFLW\}}\frac{p_{std}}{\mathrm{\{PSXC\}}}\frac{(\mathrm{\{ATX\}}+T_{0})}{T_{std}}  
(\#eq:PFLWCbox1)
\end{equation} \begin{equation}
\mathrm{\{USFLWC\}}=\mathrm{\{USMPFLW\}}\frac{(\mathrm{\{UPRESS\}}\times 10)} {\mathrm{\{PSXC\}}}\frac{\mathrm{(\{ATX\}}+T_{0})}{T_{blk}}  
(\#eq:PFLWCbox2)
\end{equation}

\hypertarget{tcntu-tcntp}{%
\subsubsection*{Total particle counts per sample interval, UHSAS or
PCASP: TCNTU, TCNTP}\label{tcntu-tcntp}}
\addcontentsline{toc}{subsubsection}{Total particle counts per sample
interval, UHSAS or PCASP: TCNTU, TCNTP}

The total particle counts in each sample interval for, respectively, the
UHSAS and PCASP instruments. These values are the sum of counts in all
cells of the spectrometers, as represented in the vector variables
CUHSAS or CS200. See the discussion of these variables in
Sect.~@ref(sensors-1-D-probes).

\hypertarget{concu-concp}{%
\subsubsection*{\texorpdfstring{Concentration, sum over all channels
(cm\textsuperscript{-3}s\textsuperscript{-1}): CONCU, CONCP, CONCU100,
CONCU500}{Concentration, sum over all channels (cm-3s-1): CONCU, CONCP, CONCU100, CONCU500}}\label{concu-concp}}
\addcontentsline{toc}{subsubsection}{Concentration, sum over all
channels (cm\textsuperscript{-3}s\textsuperscript{-1}): CONCU, CONCP,
CONCU100, CONCU500}

The particle concentrations summed over all or a subset of channels.
CONCU and CONCP are summed over all channels in the UHSAS and PCASP,
respectively, and are calculated as in the following boxed equations.
CONCU100 and CONCU500 are concentrations summed over channels in the
UHSAS giving particle concentrations for diameters greater than or equal
to 100~nm and 500~nm, respectively, and are calculated as for CONCU
except with TCNTU replaced by the sum over the appropriate channels.

\protect\hyperlink{tcntu-tcntp}{TCNTU} = total particle counts per
sample interval, UHSAS\\
\protect\hyperlink{tcntu-tcntp}{TCNTP} = total particle counts per
sample interval, PCASP\\
\(\Delta t\) = \sindex[lis]{Deltat@$\Delta t$=time interval}sample
interval {[}s{]}\\
\protect\hyperlink{pflw}{USFLWC} = corrected sample flow rate, UHSAS
{[}cm\(^{3}\)s\(^{-1}\){]}\\
\protect\hyperlink{pflw}{PFLWC} = corrected sample flow rate, PCASP
{[}cm\(^{3}\)s\(^{-1}\){]}

\begin{equation}
\mathrm{\{CONCU\}}=\frac{\mathrm{\{TCNTU\}}}{\mathrm{\{USFLWC\}}\Delta t}
(\#eq:CONCUbox1)
\end{equation} \begin{equation}
\mathrm{\{CONCP\}}=\frac{\mathrm{\{TCNTP\}}}{\mathrm{\{PFLWC\}}\Delta t}
(\#eq:CONCUbox2)
\end{equation}

\hypertarget{special-aerosol}{%
\section{Special Aerosol Measurements}\label{special-aerosol}}

Data from an aerosol mass spectrometer, a scanning mobility particle
spectrometer, and a giant nucleus impactor are recorded by these
instruments in separate data files and are not recorded by the aircraft
data system. The ancillary data sets are not merged into the netCDF
archives produced by EOL, so the special data files must be used for
these measurements. The data formats are described with the instruments
at the references given below:

\begin{itemize}
\tightlist
\item
  Aerosol Mass Spectrometer (AMS) data files contain size-segregated
  chemical composition of non-refractory, submicron aerosol particles.
  The instrument is described here:
  \href{https://www.eol.ucar.edu/instruments/time-flight-aerosol-mass-spectrometer}{AMS}.\\
\item
  Scanning Mobility Particle Spectrometer (SMPS) files contain fine
  particle differential size distributions. The number of channels and
  covered size range are variable. Diameter ranges from about 7.5~nm up
  to about 500~nm (pressure-dependent), and 15 size bins are typical.
  The instrument is described here:
  \href{https://www.eol.ucar.edu/instruments/scanning-mobility-particle-spectrometer}{SMPS}.\\
\item
  Auto-GNI, GNI Giant Nuclei Impactor (GNI) files contain dry
  differential particle size distributions. The instrument is described
  here:
  \href{https://www.eol.ucar.edu/instruments/giant-nuclei-impactor}{GNI}.
\end{itemize}

\hypertarget{radiation-variables}{%
\chapter{Radiation Variables}\label{radiation-variables}}

\hypertarget{measurements-of-irradiance-and-radiometric-temperature}{%
\section{Measurements of Irradiance and Radiometric
Temperature}\label{measurements-of-irradiance-and-radiometric-temperature}}

The following references, although in part obsolete now, have additional
information on radiation measurements from NCAR aircraft: RAF Bulletin
25, Bannehr and Glover, 1991, NCAR Technical Note NCAR/TN-364+STR, and
Albrecht and Cox, 1977.\footnote{Albrecht, B. and Cox, S.K.: 1977,
  Procedure for Improving Pyrgeometer Performance, J. Appl. Meteorol.,
  16, 188--197.} The instruments are described in the ``Radiation''
section on the EOL web site. Some other radiometric measurements appear
in Section~@ref(the-state-of-the-atmosphere) because the measurements
fit better there with measurements of state variables for the
atmosphere; these include two measurements of air temperature by
radiometric thermometers, \protect\hyperlink{AT_ITR}{AT\_ITR} and
\protect\hyperlink{OAT}{OAT}, and the Microwave Temperature Profiler
\href{http://www.eol.ucar.edu/instruments/microwave-temperature-profiler}{MTP}
that measures temperature profiles above and below the aircraft by
radiometric measurements.

\hypertarget{rstx}{%
\subsubsection*{\texorpdfstring{Radiometric (Surface or Sky/Cloud-Base)
Temperature ({∘C}):
RSTx}{Radiometric (Surface or Sky/Cloud-Base) Temperature (∘C): RSTx}}\label{rstx}}
\addcontentsline{toc}{subsubsection}{Radiometric (Surface or
Sky/Cloud-Base) Temperature ({∘C}): RSTx}

The equivalent black body temperature measured by an infrared
radiometer. The radiometers used on the GV and C-130 are Heimann Model
KT-19.85 precision radiation thermometers. The KT19.85 spectral band
extends from 9.6 to 11.5 m, and it has a 2̊ field of view. The x in the
variable name denotes the instrument location on either the bottom (B)
or top (T) of the aircraft. The KT-19.85 instruments are calibrated
using a black-body source manufactured by Eppley.\footnote{Some archived
  projects used this variable name for measurements from a narrow
  bandwidth, narrow field-of-view (2{º}) Barnes Engineering Model PRT-5
  precision radiation thermometer. This instrument is now retired. The
  spectral bandwidth available was either 8 to 14 {μm} or 9.5 to 11.5
  {μm}. Its cavity temperature was monitored and recorded as either
  TCAVB or TCAVT.}

\hypertarget{trstx}{%
\subsubsection*{\texorpdfstring{Radiometer Sensor Head Temperature
({∘C}):
TRSTx}{Radiometer Sensor Head Temperature (∘C): TRSTx}}\label{trstx}}
\addcontentsline{toc}{subsubsection}{Radiometer Sensor Head Temperature
({∘C}): TRSTx}

The temperature of the sensing head of the KT19.85 radiometer sensing
head, usually applying to RSTB, the primary down-looking instrument. The
down-looking instrument is normally heated to maintain a sensor-head
temperature near the scene temperature. Consult the archived netCDF
files or project reports for the calibration coefficients used, which
often varied among projects.

\hypertarget{irxv}{%
\subsubsection*{Pyrgeometer Output (V): IRxV}\label{irxv}}
\addcontentsline{toc}{subsubsection}{Pyrgeometer Output (V): IRxV}

The voltage representing long-wave irradiance, from a pyrgeometer
manufactured by Kipp \& Zonen. The CGR4 model used on the GV and C-130
includes a meniscus dome that provides a 180º field of view with
negligible directional response error over the spectral range of 4.2 to
45 m. The thermal stability of the dome construction and coupling to the
instrument body eliminates the need for dome temperature measurements or
dome shading. It is calibrated at the Naval Research Lab over a range of
temperatures encountered during flight according to procedures specified
by Bucholtz et al.~(2008).\footnote{Bucholtz , Anthony, Robert T. Bluth
  , Ben Kelly, Scott Taylor, Keir Batson, Anthony W. Sarto , Tim P.
  Tooman , Robert F. McCoy, 2008: The Stabilized Radiometer Platform
  (STRAP) --- An Actively Stabilized Horizontally Level Platform for
  Improved Aircraft Irradiance Measurements. J. Atmos. Oceanic Technol.
  , 25, 2161 -- 2175.} The pyrgeometers are usually flown in pairs, one
looking upward and one looking downward. The letter 'x' denotes location
on either bottom (B) or top (T) of the aircraft. The primary derived
variable from this instrument is IRxC, below.

\hypertarget{irxht}{%
\subsubsection*{\texorpdfstring{Pyrgeometer Housing Temperature ({∘}C):
IRxHT}{Pyrgeometer Housing Temperature (∘C): IRxHT}}\label{irxht}}
\addcontentsline{toc}{subsubsection}{Pyrgeometer Housing Temperature
({∘}C): IRxHT}

The temperature of the modified pyrgeometer housing, measured by a
platinum resistance temperature sensor. The calibrated temperature
(IRxHT) is derived from the raw signal (IRxHTV) as described below:

\hypertarget{irxc}{%
\subsubsection*{\texorpdfstring{Calibrated Infrared Irradiance
(W~m\textsuperscript{-2}):
IRxC}{Calibrated Infrared Irradiance (W~m-2): IRxC}}\label{irxc}}
\addcontentsline{toc}{subsubsection}{Calibrated Infrared Irradiance
(W~m\textsuperscript{-2}): IRxC}

The infrared irradiance measured by a Kipp \& Zonen CGR4
instrument,\footnote{Prior to 2009, IRx and IRxC were used to denote
  measurements from Eppley pyrgeometers. Processing methods for these
  obsolete variables are described in Section~@ref(obsolete-variables).}
after application of a calibration function. The relationship between
IRxV (V) and IRxC (W m{ − 2}) is determined by a calibration in which
the CGR4 views a NIST-referenced source over a range of sensor
temperatures controlled by a cold bath. The processing algorithm is
described in the following box:

\protect\hyperlink{irxht}{IRxHTV} = voltage from a platinum resistance
thermometer attached to the housing of the pyrgeometer {[}V{]}\\
\{\(a_4,\,a_5\)\} = calibration coefficients {[}\(^\circ\)C{]}\\
\(V_1\) = 1 V (for consistency of units)

\begin{equation}
\mathrm{\{IRxHT\}} = a_4 + a_5\,\log_{10}(\mathrm{\{IRxHTV\}}\,/\,V_1)
(\#eq:IRxHTbox)
\end{equation}

\hypertarget{visxv}{%
\subsubsection*{Pyranometer Output (V): VISxV}\label{visxv}}
\addcontentsline{toc}{subsubsection}{Pyranometer Output (V): VISxV}

The voltage from a pyranometer, representing visible irradiance. On the
GV and C-130, Kipp \& Zonen CMP22 pyranometers measure visible
irradiance. A high-quality quartz dome allows for a wide spectral range,
improved directional response, and reduced thermal offsets. The spectral
range is 0.32 to 3.6 m. The pyranometers are usually flown in pairs, one
looking upward and one downward. On the C-130, these sensors are mounted
on stabilized platforms that remain level during aircraft pitch and roll
variations. They are calibrated pre- and post-project at the Naval
Research Lab (Bucholtz et al, 2008; see footnote {[}fn:Bucholtz-2008{]}
on page ) using a sun-tracking shadow device and diffuse sunlight as a
source. The letter 'x' denotes either bottom (B, nadir-viewing) or top
(T, zenith-viewing). The primary derived variable from this instrument
is VISxC, below.

\hypertarget{visxht}{%
\subsubsection*{\texorpdfstring{Pyranometer Housing Temperature ({∘}C):
VISxHT}{Pyranometer Housing Temperature (∘C): VISxHT}}\label{visxht}}
\addcontentsline{toc}{subsubsection}{Pyranometer Housing Temperature
({∘}C): VISxHT}

The temperature of the modified housing unit of a pyranometer, measured
by a platinum resistance temperature sensor. A calibrated temperature
(VISxHT) is derived from the raw signal, VISxHTV, which is normally not
included in archive netCDF files. The equation used for the calibration
is VISxHT = {a1 + a2log10}(\{VISxHTV\}/{V1}) where {V1}is 1~V and
{\{a1, a2\}} are calibration coefficients having dimensions of
{{[}{∘}C{]}}.

\hypertarget{visxc}{%
\subsubsection*{\texorpdfstring{Calibrated Visible Irradiance
(W~m\textsuperscript{-2}):
VISxC}{Calibrated Visible Irradiance (W~m-2): VISxC}}\label{visxc}}
\addcontentsline{toc}{subsubsection}{Calibrated Visible Irradiance
(W~m\textsuperscript{-2}): VISxC}

The visible irradiance measured by a Kipp \& Zonen CMP22 pyranometer.
The relationship between VISxV (V) and VISxC (W m{ − 2}) is determined
by calibration procedures in which the CMP22 views a clear sky source
while a sun-tracking device blocks direct solar radiation. The normal
processing algorithm is to apply a simple linear calibration, as
follows:

\protect\hyperlink{visxv}{VISxV} = voltage output by a pyranometer
{[}V{]}\\
\(a_{1}\) = linear calibration coefficient {[}W~m\(^{-2}\)~V\(^{-1}\){]}

\begin{equation}
\mathrm{\{VISxC\}}=a_{1}\mathrm{\{VISxV\}}
(\#eq:VISxC)
\end{equation}

\hypertarget{stabilized-platform-angles-spxpitch-spxroll--spx}{%
\subsubsection{\texorpdfstring{Stabilized Platform Angles ({∘}):
SPxPitch, SPxRoll
\{-spx\}}{Stabilized Platform Angles (∘): SPxPitch, SPxRoll \{-spx\}}}\label{stabilized-platform-angles-spxpitch-spxroll--spx}}

The pitch and roll angles of the stabilized platforms, relative to the
aircraft reference frame. Upward- and downward-looking pyrgeometers and
pyranometers on the C-130 are mounted on stabilized platforms that
compensate for aircraft pitch and roll. These variables record the
movement of the top (x=T) and bottom (x=B) platforms in response to
aircraft pitch and roll changes. The platforms are mounted with 2.85{∘}
downward pitch angle to compensate for the normal upward pitch of the
aircraft. The range of motion is { ± 5∘} in pitch and { ± 10∘} in roll.
The sign convention is that of the aircraft, for which nose-upward pitch
and right-wing-down roll are positive.

\hypertarget{harp}{%
\section{Spectral Irradiance and Actinic Flux}\label{harp}}

The HIAPER Atmospheric Radiation Package (HARP) includes separate
components that measure spectral irradiance (both upwelling and
downwelling) and actinic flux. The instrument is described at this URL.
Data are recorded on dedicated disk drives associated with the
instrument, not in the standard aircraft data-system files. This is an
ancillary data set, for which special Matlab and IDL analysis routines
have been developed, but the measurements are not merged into the netCDF
archives produced by EOL. For data access and assistance with analysis
routines, contact EOL/RAF data managers at mailto:raf-dm@eol.ucar.edu.

\hypertarget{solar-angles}{%
\section{Solar Angles}\label{solar-angles}}

The calculations described in this group are used primarily when
interpreting the calibrated visible irradiance (VISxC) but can be used
by themselves or in conjunction with other measurements that need them.
For additional documentation see Bannehr and Glover, 1991, NCAR
Technical Note NCAR/TN-364+STR and this NOAA web site.\footnote{The
  descriptions of SOLZE, SOLEL, and SOLAZ in Bulletin 9 were incorrect,
  but the code in use has been consistent and correct and continues to
  be used unchanged. For reference, that code is contained in the nimbus
  subroutine 'solang.c'.} The calculator at this link can also be used
to find these angles from the position and time in data files.

\hypertarget{solde}{%
\subsubsection*{Solar Declination Angle (radians): SOLDE}\label{solde}}
\addcontentsline{toc}{subsubsection}{Solar Declination Angle (radians):
SOLDE}

The solar declination angle, the angular distance of the sun north of
the earth's equator. (Negative values are south.) To obtain this, the
solar hour angle is calculated (taking leap years into account).

\(N\) = day number\\
\hspace*{0.333em}\hspace*{0.333em}\hspace*{0.333em}\hspace*{0.333em}=
number of days (corrected for leap years) since 1 January 1980\\
\hspace*{0.333em}\hspace*{0.333em}\hspace*{0.333em}\hspace*{0.333em}\hspace*{0.333em}\hspace*{0.333em}\hspace*{0.333em}(including
fractional day from UTC time)\\
\hspace*{0.333em}\hspace*{0.333em}\hspace*{0.333em}=
(year-1980)\emph{365+(int)(year-1980)/4+day\\
\hspace*{0.333em}\hspace*{0.333em}\hspace*{0.333em}\hspace*{0.333em}\hspace*{0.333em}\hspace*{0.333em}+
(hour+min/60.+sec/3600.)/24.+\(M\)\\
\hspace*{0.333em}\hspace*{0.333em}\hspace*{0.333em}where
\(M\)=(int)(k+(int)((month-i)}30.6+b)\\
\hspace*{0.333em}\hspace*{0.333em}\hspace*{0.333em}\hspace*{0.333em}\hspace*{0.333em}\hspace*{0.333em}with
\{i,b,k\}=\{1,0.5,0\} for month \textless= 2\\
\hspace*{0.333em}\hspace*{0.333em}\hspace*{0.333em}\hspace*{0.333em}\hspace*{0.333em}\hspace*{0.333em}and
otherwise \{3, 59.5, (1 for leap years, else 0)\}\\
\(\Theta_h\): UTC time expressed as radians after solar noon\\
\(f,\ \alpha.\ \epsilon\): internal-calculation variables defined
below\\
\{SOLDE\}: solar declination angle

\begin{equation}
\theta_{h}=2\pi\frac{N}{365.25}
(\#eq:SOLDEbox1)
\end{equation} \begin{equation}
f=-0.031271-4.53963\times10^{-7}N+\theta_{h}
(\#eq:SOLDEbox2)
\end{equation} \begin{align}
\alpha  &=  \theta_{h}+4.900968+0.000349\,\sin(2f)+3.67474\times10^{-7}N\notag \\
        &+(0.033434-2.3\times10^{-9}N)\,\sin(f)
(\#eq:SOLDEalpha)
\end{align} \begin{equation}
\epsilon=0.409140-6.2149\times10^{-9}N
(\#eq:SOLDEeps)
\end{equation} \begin{equation}
\mathrm{{\{SOLDE\}}=}\arcsin(\sin\alpha\sin\epsilon)
(\#eq:SOLDEbox)
\end{equation}

\hypertarget{solel}{%
\subsubsection*{Solar Elevation Angle (radians):SOLEL}\label{solel}}
\addcontentsline{toc}{subsubsection}{Solar Elevation Angle
(radians):SOLEL}

The solar elevation angle, describing how high the sun appears in the
sky. The angle is measured between a line from the observer to the sun
and the horizontal plane on which the observer is standing. The
elevation angle is negative when the sun drops below the horizon, and
the sum of the elevation angle and the zenith angle is {π/2.}

\(\theta_{G}\) = Greenwich hour
angle\sindex[lis]{thetaG@$\theta_{G}$=Greenwich hour angle}
{[}radians{]}\\
\(\theta_{L}\) = local hour
angle\sindex[lis]{thetaL@$\theta_{L}$=local hour angle} {[}radians{]}\\
\(N\) = day number {[}see SOLDE box above{]}\\
\(Y\) = year (format as in 1980) \(\lambda\) =
latitude\sindex[lis]{lambda@$\lambda$=latitude} {[}radians{]}\\
\(\psi\) = longitude\sindex[lis]{psi@$\psi$=longitude} {[}radians{]}
\(h\) = fractional hour = (hour + minute/60. + second/3600.)\\
\(\alpha\)~~~~~~see @ref(eq:SOLDEalpha) in the SOLDE box above\\
\(\epsilon\)~~~~~~see @ref(eq:SOLDEeps) in the SOLDE box\\
\{SOLDE\} = solar declination angle (radians) described above;
cf.~@ref(eq:SOLDEbox).

\begin{equation}
\theta_{G}=\arctan(\frac{\sin\alpha\cos\epsilon}{\cos\alpha})
(\#eq:SOLELbox1)
\end{equation} \begin{align}
\theta_{L} = & \theta_{G}+\psi-2\pi\frac{h}{24}-1.759335\notag \\
 - &2\pi(\frac{N}{365}-Y+1980)-3.694\times10^{-7}N
(\#eq:SOLELbox2)
\end{align} \begin{equation}
\mathrm{\mathrm{\{SOLEL\}}=\arcsin\left(\sin\lambda\sin\mathrm{\{SOLDE\}+\cos\lambda}\cos\mathrm{\{SOLDE\}}\cos\theta_{L}\right)}
(\#eq:SOLELbox3)
\end{equation}

\hypertarget{solze}{%
\subsubsection*{Solar Zenith Angle (radians): SOLZE}\label{solze}}
\addcontentsline{toc}{subsubsection}{Solar Zenith Angle (radians):
SOLZE}

The angle of the sun from the zenith, or the solar zenith angle. Cf.
also the discussion of the solar elevation angle, SOLEL.
{\{SOLZE\} = (π/2) − \{SOLEL\}} with \{SOLEL\} given by
@ref(eq:SOLELbox3) above.

\hypertarget{solaz}{%
\subsubsection*{Solar Azimuth Angle (radians): SOLAZ}\label{solaz}}
\addcontentsline{toc}{subsubsection}{Solar Azimuth Angle (radians):
SOLAZ}

The solar azimuth angle, the angular distance between due south and the
projection of the line of sight to the sun on the ground. A positive
solar azimuth angle indicates a position east of south (i.e., morning).

\(\theta_{L}\) = local hour angle (radians): see @ref(eq:SOLELbox2)\\
\{SOLDE\} = solar declination angle (radians): see @ref(eq:SOLDEbox)\\
\{SOLEL\} = solar elevation angle (radians): see @ref(eq:SOLELbox3)\\
\{SOLAZ\} = solar azimuth angle {[}radians{]}

\begin{equation}
\mathrm{\{SOLAZ\}=\arcsin\left(\frac{\cos\mathrm{\{SOLDE\}\sin\theta_{L}}}{\cos\mathrm{\{SOLEL\}}}\right)}
(\#eq:SOLAZbox1)
\end{equation}
If~sin(\{SOLAZ\})~\textless~sin(\{SOLDE\})/sin(\(\phi):\)\\
\begin{equation}
\mathrm{\{SOLAZ\}} = \pi/2-\mathrm{\{SOLAZ\}}
(\#eq:SOLAZbox2)
\end{equation}

\hypertarget{experimental-variables}{%
\chapter{Experimental Variables}\label{experimental-variables}}

This document does not include experimental variables, conventionally
denoted by variable names starting with 'X'. Project documentation
should be consulted for such variables. Many projects also include
measurements from instruments provided by investigators outside
NCAR/RAF. Identification of those variables, and processing algorithms,
are contained in the project documentation and/or the NETCDF headers.

\hypertarget{obsolete-variables}{%
\chapter{Obsolete Variables}\label{obsolete-variables}}

RAF retired the ``GENPRO'' processor, the software program previously
used to produce data sets, in 1993, but data files produced by that
processor are still retained and available for use. Also, there are some
instruments that are now retired but provided measurements in some
archived data files. Obsolete variable names that are associated only
with GENPRO or a retired instrument are discussed below, for reference
and to facilitate use of old data files.

\hypertarget{tptime}{%
\subsubsection*{Unaltered Tape Time (s): TPTIME}\label{tptime}}
\addcontentsline{toc}{subsubsection}{Unaltered Tape Time (s): TPTIME}

This variable is derived by converting the HOUR, MINUTE and SECOND to
elapsed seconds after midnight of the current day. If time increments to
the next day, its value is not reset to zero, but 86400 seconds are
added to produce ever-increasing values for the data set.

\hypertarget{ptime}{%
\subsubsection*{Processor Time (s): PTIME}\label{ptime}}
\addcontentsline{toc}{subsubsection}{Processor Time (s): PTIME}

This is an internal time variable created by the GENPRO processor. It
represents elapsed seconds after midnight. It differs from TPTIME in
that, after it has been set at the beginning of the data set, it is
incremented internally for each second of data processed. If duplicate
or missing raw data records exist, it can differ from TPTIME. It is
guaranteed to be a monotonically increasing and continuous series of
values.

\hypertarget{tmlag}{%
\subsubsection*{INS: Data System Time Lag (s): TMLAG}\label{tmlag}}
\addcontentsline{toc}{subsubsection}{INS: Data System Time Lag (s):
TMLAG}

TMLAG is the amount of time between the reference time of a Litton
LTN-5l Inertial Navigation System (INS) and the data system clock, in
seconds. TMLAG will always be greater than zero and less than 2.

\hypertarget{loranc}{%
\subsubsection*{LORAN-C Variables:}\label{loranc}}
\addcontentsline{toc}{subsubsection}{LORAN-C Variables:}

Latitude and Longitude ({º}): CLAT, CLON Circular Error of Probability
(n mi): CCEP Ground Speed (m/s): CGS Status: CSTAT Time (s): CSEC
Fractional Time (s): CFSEC

Before the advent of GPS, NCAR/RAF operated a LORAN-C receiver that
provided information on the position and groundspeed of the aircraft.
The measurements of latitude and longitude from this system are CLAT and
CLON, measured at 1 Hz and with positive values of longitude to the east
and positive values of latitude to the north. and CCEP provides an
estimate of the uncertainty in those measurements (in units of nautical
miles). A status word, CSTAT, was used to record a value of 15 when the
system was operational. The ground speed and reference times were also
recorded in the above corresponding variables. The sum of CSEC and CFSEC
represented the time of the measurement, which was not always the time
in the data file when the measurements were recorded.

\hypertarget{ltn51}{%
\subsubsection*{LTN-51 Variables:}\label{ltn51}}
\addcontentsline{toc}{subsubsection}{LTN-51 Variables:}

INS Latitude ({º}): \underline{ALAT} INS Longitude ({º}):
\underline{ALON} Raw INS Ground Speed X Component (m/s): \underline{XVI}
Raw INS Ground Speed Y Component (m/s): \underline{YVI} Raw INS True
Heading ({º}): \underline{THI} INS Wander Angle ({º}): \underline{ALPHA}
INS Platform Heading ({º}): \underline{PHDG}

These variables from the Litton LTN-51 Inertial Navigation System (INS)
are analogous to the modern variables discussed in
Section~@ref(inertial-reference-systems) The measurements of latitude
and longitude were provided with 1-Hz frequency and had a resolution of
0.0014{º}, while the ground speed components were provided at 10 Hz and
had resolution equal to 0.012 m/s. The X component of the ground speed
was along the longitudinal axis of the aircraft at the time of
alignment, and the Y axis was in the starboard direction at the time of
alignment. PHDG recorded the orientation of the platform relative to
true north, with resolution 0.0028{º}. THI was the true heading of the
aircraft, produced at 5 Hz with resolution of 0.0014{º}. The ``wander
angle'' is an INS-only variable that recorded the angle of the INS
platform x-axis relative to its original orientation; it ``wandered'' in
response to east-west motion of the aircraft on a spherical Earth.

\hypertarget{vzi}{%
\subsubsection*{Raw Aircraft Vertical Velocity (m/s): VZI}\label{vzi}}
\addcontentsline{toc}{subsubsection}{Raw Aircraft Vertical Velocity
(m/s): VZI}

This is an integrated output from an up/down binary counter connected to
the INS vertical accelerometer. Resolution is 0.012 m/s. Due to changes
in local gravity and accumulated errors, this often develops a
significant offset during flight.

\hypertarget{thf}{%
\subsubsection*{\texorpdfstring{Aircraft True Heading ({º}):
THF}{Aircraft True Heading (º): THF}}\label{thf}}
\addcontentsline{toc}{subsubsection}{Aircraft True Heading ({º}): THF}

This measurement of aircraft heading was derived from the angle between
the horizontal projection of the aircraft center and true north: THF =
PHDG + ALPHA. Resolution is 0.0028{º}.

\hypertarget{gsf-obsolete}{%
\subsubsection*{Aircraft Ground Speed and East/North Components (m/s):
GSF, VEW, VNS}\label{gsf-obsolete}}
\addcontentsline{toc}{subsubsection}{Aircraft Ground Speed and
East/North Components (m/s): GSF, VEW, VNS}

These variables have the same names as the modern variables for ground
speed. (Cf.~Section~@ref(the-state-of-the-aircraft).) GSF is the
magnitude of the ground speed determined by the INS, as derived from XVI
and YVI:\\

\begin{equation}
\mathrm{GSF=\sqrt{\{XVI\}^{2}+\{YVI\}^{2}}}
(\#eq:GSF)
\end{equation}

VEW and VNS are the east and north projections of this ground speed,
derived using THF for the aircraft heading.

\hypertarget{wspd}{%
\subsubsection*{\texorpdfstring{Wind Speed and Direction (m/s, {º}):
WSPD,
WDRCTN}{Wind Speed and Direction (m/s, º): WSPD, WDRCTN}}\label{wspd}}
\addcontentsline{toc}{subsubsection}{Wind Speed and Direction (m/s,
{º}): WSPD, WDRCTN}

These variables are calculated from UI and VI, the east and north
components of the wind determined as described in RAF Bulletin No.~23
and summarized in Section~@ref(wind):

\begin{align}
\mathrm{WS} = & \sqrt{\mathrm{\{UI\}^{2}+\{VI\}^{2}}} (\#eq:WSobs)\\
\mathrm{WD} = & \mathrm{\frac{180^{\circ}}{\pi}atan2(-\{UI\},}-\{VI\})+180^{\circ}
(\#eq:WDobs)
\end{align}

\hypertarget{vanes}{%
\subsubsection*{Raw Attack and Sideslip Force (Fixed Vane) (g): AFIXx,
BFIXx}\label{vanes}}
\addcontentsline{toc}{subsubsection}{Raw Attack and Sideslip Force
(Fixed Vane) (g): AFIXx, BFIXx}

AFIXx and BFIXx are amplified outputs from strain-gauges on fixed-vane
sensors mounted at the end of a gust boom in the horizontal or vertical
planes, respectively of the aircraft . The ``force'' on the vanes
(calibrated in ``equivalent grams'' at Jefferson County Airport gravity)
varies as a function of the aircraft attack or sideslip angles and
dynamic pressure. Here x refers to left or right for AFIXx or top or
bottom for BFIXx.

\hypertarget{akfxx}{%
\subsubsection*{\texorpdfstring{Attack Angle or Sideslip Angle (Fixed
Vanes) ({º}): AKFXx,
SSFXx}{Attack Angle or Sideslip Angle (Fixed Vanes) (º): AKFXx, SSFXx}}\label{akfxx}}
\addcontentsline{toc}{subsubsection}{Attack Angle or Sideslip Angle
(Fixed Vanes) ({º}): AKFXx, SSFXx}

AKFXx and SSFXx are the respective angles of attack and sideslip
computed from AFIXx or BFIXx and QCx (either boom or gust dynamic
pressure). An empirically derived function, HSSATK, is used to determine
the attack pr sideslip angle based on wind tunnel test data.

\hypertarget{qcb}{%
\subsubsection*{Dynamic Pressure (Boom or Gust Probe) (mb): QCB or QCBC;
QCG or QCGC}\label{qcb}}
\addcontentsline{toc}{subsubsection}{Dynamic Pressure (Boom or Gust
Probe) (mb): QCB or QCBC; QCG or QCGC}

These variables, measured by a differential pressure gauge, record the
difference between a pitot (total) pressure and a static pressure. The
QCBC and QCGC values are corrected for local flow-field distortion. The
boom and gust probe measurements referred to the same aircraft
structure. The different designations used for those measurements
specified the transducer used and its location. In the gust probe
dynamic pressure measurement (QCG), a Rosemount Model 1332 differential
pressure transducer was located closer to the sensor in the gust probe
itself, whereas in the boom measurement (QCB), a Rosemount Model 1221
pressure transducer was typically located in the aircraft nose.

\hypertarget{atc}{%
\subsubsection*{\texorpdfstring{Ambient Temperature ({∘C}):
ATC}{Ambient Temperature (∘C): ATC}}\label{atc}}
\addcontentsline{toc}{subsubsection}{Ambient Temperature ({∘C}): ATC}

A variable obtained by combining the avionics temperature on the GV,
AT\_A, with a Rosemount temperature, so that the absolute value tracked
AT\_A but faster response was provided by the Rosemount temperature.
This was used in some early GV projects because there were unresolved
problems with the data-system temperature sensors and it was thought
that AT\_A provided a more accurate result, but AT\_A was filtered to
have slow response to it was combined with the faster-response signal
from the Rosemount sensor.

\hypertarget{ttx}{%
\subsubsection*{\texorpdfstring{Total Temperature ({∘}C):
TTx}{Total Temperature (∘C): TTx}}\label{ttx}}
\addcontentsline{toc}{subsubsection}{Total Temperature ({∘}C): TTx}

This variable was used before 2014 for measurements of the recovery
temperature, for which the variable is now RTx. Because the quantity
measured is not the total temperature, the variables TTx were replaced
by RTx, but the meaning historically was the same as that now described
for RTX, apart from how humidity is now handled.

\hypertarget{ttrf}{%
\subsubsection*{\texorpdfstring{Total Temperature, Reverse Flow ({∘C}):
TTRF}{Total Temperature, Reverse Flow (∘C): TTRF}}\label{ttrf}}
\addcontentsline{toc}{subsubsection}{Total Temperature, Reverse Flow
({∘C}): TTRF}

TTRF is the recovery temperature from a calibrated NCAR reverse-flow
temperature sensor, for which the housing was designed to separate water
droplets and protect the element from wetting in cloud.

\hypertarget{ttkp}{%
\subsubsection*{\texorpdfstring{Total Temperature (Fast Response)
({∘C}):
TTKP}{Total Temperature (Fast Response) (∘C): TTKP}}\label{ttkp}}
\addcontentsline{toc}{subsubsection}{Total Temperature (Fast Response)
({∘C}): TTKP}

This is the output of recovery temperature from the NCAR fast-response
temperature probe, originally designed by Karl Danninger. (See the
discussion of total temperature in Section~@ref(temperature-section).)

\hypertarget{atrf}{%
\subsubsection*{\texorpdfstring{Ambient Temperature ({∘C}):
ATRF}{Ambient Temperature (∘C): ATRF}}\label{atrf}}
\addcontentsline{toc}{subsubsection}{Ambient Temperature ({∘C}): ATRF}

The ambient temperature computed using the NCAR reverse-flow temperature
sensor. (See the discussion in Section~@ref(temperature-section).)

\hypertarget{atkp}{%
\subsubsection*{\texorpdfstring{Ambient Temperature (Fast Response)
({∘C}):
ATKP}{Ambient Temperature (Fast Response) (∘C): ATKP}}\label{atkp}}
\addcontentsline{toc}{subsubsection}{Ambient Temperature (Fast Response)
({∘C}): ATKP}

The ambient temperature computed using the fast-response temperature
probe. (See the discussion of ambient temperature in
Section~@ref(temperature-section).)

\hypertarget{jwlwc}{%
\subsubsection*{\texorpdfstring{Raw Cloud Technology (Johnson-Williams)
Liquid Water Content ({g/m3}):
LWC}{Raw Cloud Technology (Johnson-Williams) Liquid Water Content (g/m3): LWC}}\label{jwlwc}}
\addcontentsline{toc}{subsubsection}{Raw Cloud Technology
(Johnson-Williams) Liquid Water Content ({g/m3}): LWC}

This is the raw output of a Johnson-Williams liquid water content sensor
converted to units of grams per cubic meter. The Johnson-Williams
indicator measures the evaporative cooling caused by the latent heat of
vaporization of droplets contacting the heated sensing element by
sensing changes in its resistance as it cools. Through calibration this
resistance is converted to a liquid water content. A ``compensation''
wire is also mounted in the J-W sensor, parallel to the droplet stream,
to compensate for cooling effects of the airstream. Typically the
instrument is set for a true airspeed of 200 knots. The instrument must
be zeroed in ``cloud-free air.'' The Johnson-Williams liquid water
content sensor is designed for the cloud droplet spectrum. There is some
evidence to indicate that droplets larger than 30 {μm} are shed before
completely vaporizing on the sensor element. This tends to underestimate
the liquid water content.

\hypertarget{jwlw-corrected}{%
\subsubsection*{\texorpdfstring{Corrected Cloud Technology
(Johnson-Williams) Liquid Water Content (g/m\textsuperscript{3}):
LWCC}{Corrected Cloud Technology (Johnson-Williams) Liquid Water Content (g/m3): LWCC}}\label{jwlw-corrected}}
\addcontentsline{toc}{subsubsection}{Corrected Cloud Technology
(Johnson-Williams) Liquid Water Content (g/m\textsuperscript{3}): LWCC}

This is the corrected liquid water content obtained by using the
aircraft's true airspeed after removing the zero offset:
LWCC=LWC{Ua/Uref} where {Ua} is the true airspeed of the aircraft and
{Uref} is the true airspeed set on the dial of the instrument. {Uref}
was normally 200 kts = 102.88889 m/s.

\hypertarget{ias}{%
\subsubsection*{Indicated Airspeed (knots): IAS}\label{ias}}
\addcontentsline{toc}{subsubsection}{Indicated Airspeed (knots): IAS}

In some old data files, a variable representing the indicated airspeed
was included because this was used for some derived variables. The
indicated airspeed is the airspeed that would produce the observed
difference between dynamic and static pressure under standard conditions
of 1013.25 mb and {15∘}C.

\hypertarget{edpc}{%
\subsubsection*{Water Vapor Pressure (mb): EDPC}\label{edpc}}
\addcontentsline{toc}{subsubsection}{Water Vapor Pressure (mb): EDPC}

This is a derived intermediate variable used in the calculation of
several derived thermodynamic variables. The vapor pressure over a plane
water surface is obtained by the method of Paul R. Lowe (1977), a
derived, sixth-order, Chebyshev polynomial fit to the Goff-Gratch
Formulation (1946) as a function of temperature expressed in {∘C}. The
error is much less than 1\% over the range -50{º}C to +50{º}C. EDPC was
calculated using this method for most RAF research projects between 1993
and 1996. This variable did not have the enhancement factor applied that
was discussed in Appendix C of Bulletin 9. A variable of the same name
but calculated differently replaced this in 1996, and with changes
described in Section~@ref(humidity) continues in use, recently replaced
by EWx.

For air temperature \(T < -50^\circ\)C:\\
\begin{align}
\mathrm{\{EDPC\}} &= 4.4685 + T(0.27347 + T (6.83811\times 10^{-3}\notag \\
&+ T(8.7094x10^{-5} + T(5.63513x10^{-7} + 1.47796 × 10^{-9}T)))
(\#eq:EDPC1)
\end{align} For air temperature \(T \ge -50^\circ\)C:\\
\begin{align}
\mathrm{\{EDPC\}} &= 6.107799961 + T(0.4436518521 + T(0.01428945805\notag \\
&+ T(2.650648471 × 10^{-4} + T(3.031240396 × 10^{-6}\notag \\
&+ T (2.034080948 × 10^{-8} + 6.136820929 × 10^{-11}T)))))
(\#eq:EDPC2)
\end{align}

\hypertarget{cryo-hygro}{%
\subsubsection*{Cryogenic Hygrometer Variables:}\label{cryo-hygro}}
\addcontentsline{toc}{subsubsection}{Cryogenic Hygrometer Variables:}

\textbf{Inlet Pressure (hPa): CRHP} \textbf{Frost Point Temperature
({∘C}): VCRH} \textbf{Corrected Frost Point Temperature ({∘C}): FPCRC}
\textbf{Corrected Dew Point Temperature ({∘C}): DPCRC}

The first two of these measurements were made directly in the chamber of
the cryogenic hygrometer, a now obsolete cabin-mounted instrument
connected to outside air by an inlet line. VCRH was determined from a
third-order calibration equation applied to the voltage measured by the
instrument. The corrected frost point and dew point temperatures are
those determined after corrections are applied to the direct
measurements from a cryogenic hygrometer. These measurements were from a
now obsolete instrument but the variables are included here because they
appear in some old data files. To obtain estimates of the ambient frost
point and dew point, the measurements made inside the chamber of the
cryogenic hygrometer (CVRH and CRHP) must be corrected for the
difference in water vapor pressure between that chamber and ambient
conditions. The ratio of the chamber pressure to the ambient pressure is
assumed to be the same as the ratio of the chamber vapor pressure to the
ambient vapor pressure. The vapor pressure in the chamber was determined
from the Goff-Gratch (1946) equation\footnote{Goff, J. A., and S. Gratch
  (1946) Low-pressure properties of water from 160 to 212 °F, referenced
  and used in the Smithsonian Tables (List, 1980).} for saturation vapor
pressure with respect to a plane ice surface. This vapor pressure was
then used with CRHP and a measure of the ambient pressure (PSXC) to
determine the vapor pressure in the outside air, and this was converted
to an equivalent dew-point. The instrument was only used for
measurements of frost point less than -15{∘}C because it did not
function well above that frost point. The steps are documented below:

\protect\hyperlink{cryo-hygro}{VCRH} = frost point inside the cryogenic
hygrometer {[}\(^{\circ}\)C{]}\\
\protect\hyperlink{cryo-hygro}{CRHP} = pressure inside the chamber of
the cryogenic hygrometer {[}hPa{]}\\
\href{./4-the-state-of-the-atmosphere.html\#psx}{PSXC} = reference
ambient pressure {[}hPa{]}\\
f\(_{i}\) = enhancement factor (see Appendix C of Bulletin 9)\\
\(F_{1}\)(\(T_{d}\)) =Goff-Gratch formula for vapor pressure at dew
point \(T_{d}\)\\
\(F_{2}(T_{f})\) = Goff-Gratch formula for vapor pressure at frost point
\(T_{f}\)\\
\(T_{3}\) = temperature at the triple point of water = 273.16 K

chamber vapor pressure \(e_{ic}\) (hPa):\\
\begin{equation}
e_{ic}=(6.1071\,\mathrm{mb})\times10^{A}  
(\#eq:DPCRCbox1)
\end{equation}\\
where \begin{align}  
A=&-9.09718\left(\frac{T_{3}}{\mathrm{\{VCRH\}}+T_{3}}-1\right) \notag \\
+ & 3.56654\log_{10}\left(\frac{T_{3}}{\mathrm{\{VCRH\}}+T_{3}}\right) \notag \\
+ & 0.876793\left(1-\frac{\mathrm{\{VCRH\}}+T_{3}}{T_{3}}\right)
(\#eq:DPCRCbox2)
\end{align}\\
The ambient vapor pressure \(e_{a}\) (hPa) then is:\\
\begin{equation}
e_{a}=e_{ic}\left(\frac{\mathrm{\{PSXC\}}}{\mathrm{\{CRHP\}}}\right)f_{i}
(\#eq:DPCRCbox3)
\end{equation}\\
The ambient dew and frost points DPCRC and FPCRC are found iteratively
by finding the values that lead to \(e_a\) in the Goff-Gratch
equations:\\
\begin{align}
e_{a} & = F_{1}\left(\mathrm{\{DPCRC\}}\right)\notag \\
 & = F_{2}\left(\mathrm{\{FPCRC\}}\right)
(\#eq:DPXRXbox4)
\end{align}

\hypertarget{vla}{%
\subsubsection*{Voltage Output From the Lyman-alpha Sensor (V):
VLA}\label{vla}}
\addcontentsline{toc}{subsubsection}{Voltage Output From the Lyman-alpha
Sensor (V): VLA}

The voltage output from the Lyman-alpha absorption hygrometer. This
instrument provided fast-response, high-resolution measurements of water
vapor density. (If a second sensor was used, a 1 was added to the
variable name associated with the second sensor.) The sensors are now
obsolete.

\hypertarget{xuvi}{%
\subsubsection*{Voltage Output from the UV Hygrometer (V):
XUVI}\label{xuvi}}
\addcontentsline{toc}{subsubsection}{Voltage Output from the UV
Hygrometer (V): XUVI}

The voltage from a modern (as of 2009) version of the Lyman-alpha
hygrometer, which provides a signal that represents water vapor density.
The instrument also provides measurements of pressure and temperature
inside the sensing cavity; they are, respectively, \underline{XUVP} and
\underline{XUVT}. These variables and the processing algorithm below
have now been replaced by XSIGV\_UVH and the algorithm discussed with
the variable EW\_UVH.

\newcommand{\pluseq}{\mathrel{+}\mathrel{\mkern-1mu}=}

\protect\hyperlink{xuvi}{XUVI} = output from the UV Hygrometer, after
application of calibration coefficients\\
\href{./4-the-state-of-the-atmosphere.html\#dewpt-corrected}{DPXC}\index{DPXC}
= corrected dewpoint from some preferred source, \(^{\circ}\)C\\
\href{./4-the-state-of-the-atmosphere.html\#ambient-t}{ATX}\index{ATX} =
preferred temperature, \(^{\circ}\)C\\
\href{./4-the-state-of-the-atmosphere.html\#rho}{RHODT}\index{RHODT}
=water vapor density determined by a chilled-mirror sensor\\
Tau = time constant for the exponential update (typically 300 s)

For valid measurements (i.e., when DPXC \(<\) ATX and XUVI and RHODT are
not missing\}:\\
\begin{equation}
\mathrm{Offset}\mathrel{+}\mathrel{\mkern-1mu}=(\mathrm{\{RHODT\}-\{XUVI\}-Offset})/Tau
(\#eq:RHOUVbox1)
\end{equation} \begin{equation}
\mathrm{\{RHOUV\} = \{XUVI\} + Offset}
(\#eq:RHOUBbox2)
\end{equation}

\hypertarget{irx}{%
\subsubsection*{\texorpdfstring{Raw Pyrgeometer Output
(W m\textsuperscript{-2}):
IRx}{Raw Pyrgeometer Output (W m-2): IRx}}\label{irx}}
\addcontentsline{toc}{subsubsection}{Raw Pyrgeometer Output
(W m\textsuperscript{-2}): IRx}

\protect\hypertarget{EppleyReference}{}{{[}EppleyReference{]}}A
pyrgeometer manufactured by Eppley Laboratory, Inc.~measures long-wave
irradiance using a calibrated thermopile. It has a coated glass
hemisphere that transmits radiation in a bandwidth between 3.5 {μm} and
50 {μm}. It is calibrated at RAF according to procedures specified by
Albrecht and Cox (1977). (See the reference in the next paragraph.) The
pyrgeometers are usually flown in pairs, one up-looking and one
down-looking. The letter 'x' denotes either bottom (B) or top (T).
Corrected Infrared Irradiance (W m{ − 2}): \underline{IRxC} Because the
pyrgeometer measures net radiation, IRx must be corrected for emission
from the dome covering the sensor and for emission from the thermopile
itself. IRxC is the corrected infrared irradiance, determined following
procedures of Albrecht and Cox, 1977.

\protect\hyperlink{irx}{IRx} = raw pyrgeometer output
{[}W\(\,\)m\(^{-2}\){]}\\
\(T_{D}\) = dome temperature {[}K{]}\\
\(T_{S}\) = ``sink'\,' temperature (approx.~the thermopile temperature)
{[}K{]}\\
\(\epsilon\) = emissivity of the thermopile (dimensionless) = 0.986\\
\(\beta\) = empirical constant dependent on the dome type = 5.5\\
\(\sigma\) = Stephan-Boltzmann constant = 5.6704\(\times10^{-8}\)
W\(\,\)m\(^{-2}\)K\(^{-4}\)

\begin{equation}
\mathrm{\{IRxC\}}=\mathrm{\{IRx\}}-\beta\sigma(T_{D}^{4}-T_{S}^{4})+\epsilon\sigma T_{S}^{4}
(\#eq:IRxTbox)
\end{equation}

\hypertarget{swx}{%
\subsubsection*{\texorpdfstring{Shortwave Irradiance
(W/m\textsuperscript{2}):
SWx}{Shortwave Irradiance (W/m2): SWx}}\label{swx}}
\addcontentsline{toc}{subsubsection}{Shortwave Irradiance
(W/m\textsuperscript{2}): SWx}

An Eppley Laboratory, Inc., pyranometer measures short-wave irradiance.
The dome normally used is UG295 glass, which gives wide coverage of the
solar spectrum (from 0.285 {μm} to 2.8 {μm}). Different bandwidths can
be obtained by use of different glass domes, available from RAF upon
request. (See Bulletin No.~25.) The pyranometers are usually flown in
pairs, one up-looking and one down-looking. They are calibrated
periodically at the NOAA Solar Radiation Facility in Boulder, Colorado.
The letter 'x' denotes either bottom (B) or top (T).

\hypertarget{swtc}{%
\subsubsection*{\texorpdfstring{Corrected Incoming Shortwave Irradiance
(W/m\textsuperscript{2}):
SWTC}{Corrected Incoming Shortwave Irradiance (W/m2): SWTC}}\label{swtc}}
\addcontentsline{toc}{subsubsection}{Corrected Incoming Shortwave
Irradiance (W/m\textsuperscript{2}): SWTC}

The down-welling shortwave irradiance measured by the difference between
SWT and SWB) is corrected to take into account the sun angle and small
variations in the aircraft attitude angles (pitch and roll). The
correction is limited to { ± 6∘} in either angle, so these measurements
should be considered invalid beyond these limits. This is the derived
output of incoming (down-welling) shortwave irradiance, taking into
account both solar position (sun angle) and modest variations in
aircraft attitude (at present, restricted to less than 6{º} in pitch
and/or roll). (For more information, refer to RAF Bulletin 25.)

\hypertarget{uvx}{%
\subsubsection*{\texorpdfstring{Ultraviolet Irradiance
(W/m\textsuperscript{2}):
UVx}{Ultraviolet Irradiance (W/m2): UVx}}\label{uvx}}
\addcontentsline{toc}{subsubsection}{Ultraviolet Irradiance
(W/m\textsuperscript{2}): UVx}

A pair of UV radiometer/photometers measure either down-welling (x=T) or
up-welling (x=B) irradiance in the ultraviolet, approximately from 0.295
{μm} to 0.385 {μm}. These units are periodically returned to the Eppley
Laboratories for recalibration.

\hypertarget{co}{%
\subsubsection*{Raw Carbon Monoxide Concentration (ppb): CO}\label{co}}
\addcontentsline{toc}{subsubsection}{Raw Carbon Monoxide Concentration
(ppb): CO}

CO is the uncorrected output of the TECO model 48 CO analyzer. This
instrument measures the concentration of CO by gas filter correlation.
The optics of the version operated by the RAF have been modified to
increase the light through the absorption cell, and a zero trap has been
added that periodically removes CO from the sample air stream to obtain
an accurate zero. This permits correction for the significant
temperature-dependent drift of the zero level of the measurement.

\hypertarget{co-vars}{%
\subsubsection*{Carbon Monoxide Analyzer Variables:}\label{co-vars}}
\addcontentsline{toc}{subsubsection}{Carbon Monoxide Analyzer
Variables:}

\textbf{Status (V): CMODE} \textbf{Baseline Zero Signal (V): COZRO}
**Raw Carbon Monoxide Signal, Baseline Corrected (V): COCOR*

\textbf{ }Corrected Carbon Monoxide Concentration (ppmv): COCAL**

CMODE records if the CO analyzer is supplied with air from which CO has
been removed and so is recording its zero level. When CMODE is less than
0.2 V, the instrument is in the normal operational mode, and when CMODE
is greater than 8.0 V the instrument is in the ``zero'' mode. When
measurements are processed, the zero-mode signals are represented by a
cubic spline to obtain a reference baseline for the signal (COZRO), and
this baseline is subtracted from the measured value (CO) to obtain
COCOR. This variable still jumps to zero periodically and does not
include the calibration that enters the corrected variable, COCAL, which
is the calibrated signal from the CO instrument after correction for
drift of the baseline and after application of the appropriate
calibration coefficients to produce units of ppmv. The quality of the
baseline fit can be judged by examining the offset at the zero points.
If there are relatively small changes in the baseline, the zero offset
will be only a few ppbv. If there have been rapid changes in the
baseline, the zero offset can be up to 50 ppbv. The magnitude of the
offset at the zero values gives a good measure of uncertainty in the
data set. The detection limit is 10 ppbv, with an uncertainty of
{ ± 15\%}. At 1 Hz, data will have considerable variability, so 10-s
averaging is often useful when the measurements are used for analysis.

\hypertarget{o3fs}{%
\subsubsection*{Raw Chemiluminescent Ozone Signal (V):
O3FS}\label{o3fs}}
\addcontentsline{toc}{subsubsection}{Raw Chemiluminescent Ozone Signal
(V): O3FS}

Voltage output from the chemiluminescence ozone instrument, which
operates on the basis of reacting nitric oxide with ozone and detecting
the resulting chemiluminescence.

\hypertarget{SCLWC}{%
\subsubsection*{\texorpdfstring{Derived Supercooled Liquid Water Content
(g/m\textsuperscript{3}):
SCLWC}{Derived Supercooled Liquid Water Content (g/m3): SCLWC}}\label{SCLWC}}
\addcontentsline{toc}{subsubsection}{Derived Supercooled Liquid Water
Content (g/m\textsuperscript{3}): SCLWC}

This variable is the supercooled liquid water content obtained from the
change in accreted mass on the Rosemount 871F ice-accretion probe over
one second. The output is not valid during the probe deicing cycle. This
cycle is apparent in the RICE output (a peak followed by a decrease to
near zero). Supercooled liquid water content is determined by first
calculating a water drop impingement rate which is a function of the
effective surface area, the collection efficiency, the true airspeed,
and the supercooled liquid water content. The impingement rate obtained
is equated to the accreted mass of ice collected by the probe in one
second (empirical voltage/mass relationship). The resulting equation is
solved for supercooled water content. This calculation is not included
in normal processing or special processing, but some users of the
instrument use an approach like the following to calculate supercooled
liquid water:

A = \sindex[lis]{A=area}effective surface area of the probe
{[}m\(^{2}\){]}\\
\(\Delta t\) = time
interval\sindex[lis]{Deltat@$\Delta t$=time interval}\sindex[lis]{t@$t$=time}
during which an increment of mass accretes {[}s{]}\\
\(\Delta m\) = mass\sindex[lis]{m@$m$=mass} of ice accreted on the probe
in the time interval \(\Delta t\) {[}g{]}\\
\(U_{a}\) = true airspeed {[}m/s{]}

\begin{equation}
\mathrm{\{SCLWC\}}=A\,U_{a}\frac{\Delta m}{\Delta t}
(\#eq:SCLWCbox)
\end{equation}

\hypertarget{freset}{%
\subsubsection*{FSSP-100 Fast Resets (number per sample interval): FRST,
FRESET}\label{freset}}
\addcontentsline{toc}{subsubsection}{FSSP-100 Fast Resets (number per
sample interval): FRST, FRESET}

The rate at which fast resets occur in an FSSP-100 probe. The FSSP-100
records events called ``fast resets'' that occur when a particle
traverses the beam outside the depth-of-field and therefore is not
accepted for sizing. To avoid the processing time associated with
sizing, the probe resets quickly in this case, but there is still some
dead time when the probe cannot record another event. Fast resets
consume a time determined by circuit characteristics, and that time has
been determined in laboratory tests of the FSSP circuitry. This variable
is needed in addition to the ``Total Stobes'' to determine what fraction
of the time the probe is unable to accept another particle, and this
``dead time'' enters calculation of the concentration for the original
(old) FSSP.

\hypertarget{fstrob}{%
\subsubsection*{FSSP-100 Total Strobes (number per sample interval):
FSTB, FSTROB}\label{fstrob}}
\addcontentsline{toc}{subsubsection}{FSSP-100 Total Strobes (number per
sample interval): FSTB, FSTROB}

The rate at which strobes are generated in an FSSP-100 probe. A
``strobe'' is generated in the FSSP-100 whenever a particle is detected
within its depth-of-field. Not all such particles are accepted for
inclusion in the size distribution, however, because some pass through
the outer regions of the illuminating laser beam and therefore produce
shorter and smaller-amplitude pulses than those passing through the
center of the beam. The probe maintains a running estimate of the
average transit time and rejects particles with transit times shorter
than this average. The total number of strobes recorded is therefore
more than the number of sized particles, and the ratio of strobes to
accepted particles can indicate quality of operation of the probe. Also,
the strobes require processing and so contribute to the dead time of the
probe, affecting the concentration unless a correction is made. See RAF
Bulletin 24 for more discussion on the operation of the ``old'' FSSP.

\hypertarget{fbmfr}{%
\subsubsection*{FSSP-100 Beam Fraction (dimensionless):
FBMFR}\label{fbmfr}}
\addcontentsline{toc}{subsubsection}{FSSP-100 Beam Fraction
(dimensionless): FBMFR}

The ratio of the number of velocity-accepted particles (particles that
pass through the effective beam diameter) to the total number of
particles detected in the depth-of-field of the beam (the total
strobes). See the discussion of Total Strobes for more information.

\{\href{./5-cloud-physics-variables.html\#CRPC}{AFSSP}\}\(_{i}\) = valid
particles sized in size interval i\\
\{\protect\hyperlink{fstrob}{FSTROB}\} = strobes generated by particles
in the depth-of-field,\\
\hspace*{0.333em}\hspace*{0.333em}\hspace*{0.333em}\hspace*{0.333em}\hspace*{0.333em}\hspace*{0.333em}\hspace*{0.333em}\hspace*{0.333em}\hspace*{0.333em}\hspace*{0.333em}\hspace*{0.333em}\hspace*{0.333em}\hspace*{0.333em}\hspace*{0.333em}per
sample interval

\begin{equation}
\mathrm{\{FBMFR\}=\{AFSSP\}/\{FSTROB\}}
(\#eq:FBMFRbox)
\end{equation}

\hypertarget{fact}{%
\subsubsection*{FSSP-100 Calculated Activity Fraction (dimensionless):
FACT}\label{fact}}
\addcontentsline{toc}{subsubsection}{FSSP-100 Calculated Activity
Fraction (dimensionless): FACT}

This variable represents the fraction of the time that the FSSP is
unable to count and size particles (its ``dead time''). The activity
fraction is not measured directly but is estimated from fast resets and
total strobes along with measurements of the dead times associated with
each (as determined in laboratory tests). The characteristic times are
in the NetCDF header (for recent projects).

\protect\hyperlink{fstrob}{FSTROB} = strobes generated by particles in
the depth-of-field,\\
\hspace*{0.333em}\hspace*{0.333em}\hspace*{0.333em}\hspace*{0.333em}\hspace*{0.333em}\hspace*{0.333em}\hspace*{0.333em}\hspace*{0.333em}\hspace*{0.333em}\hspace*{0.333em}per
sample interval\\
\protect\hyperlink{freset}{FRESET} = ``fast resets'\,' generated per
sample interval\\
\(t_{1}\) = slow reset time (for each strobe)\\
\(t_{2}\) - fast reset time (for each fast reset)

\begin{equation}
\mathrm{\{FACT\}=\{FSTROB\}}\times t_{1}+\mathrm{\{FRESET\}}\times t_{2}
(\#eq:FACTbox)
\end{equation}

\hypertarget{AACT}{%
\subsubsection*{PCAS Raw Activity (dimensionless): PACT,
AACT}\label{AACT}}
\addcontentsline{toc}{subsubsection}{PCAS Raw Activity (dimensionless):
PACT, AACT}

The PCAS probe provides this measure of dead time, the time that the
probe is unable to sample particles because the electronics are occupied
with processing particles. The manufacturer suggests that the actual
dead time (\(f_{PCAS}\)) is given by the following formula, which should
be used when determining concentrations for the PCAS:\\

\[F_{PCAS} = 1024\,\mathrm{s}^{-1}\]

\begin{equation}
f_{PCAS} = 0.52\frac{\mathrm{\{PACT\}}}{F_{PCAS}}
(\#eq:fPCAS)
\end{equation}

However, PACT (or AACT) is the variable archived in the data files.

\hypertarget{appendix-appendices}{%
\chapter*{(APPENDIX) Appendices}\label{appendix-appendices}}
\addcontentsline{toc}{chapter}{(APPENDIX) Appendices}

\hypertarget{appendix-a-list-of-symbols}{%
\chapter*{Appendix A: List of
Symbols}\label{appendix-a-list-of-symbols}}
\addcontentsline{toc}{chapter}{Appendix A: List of Symbols}

\hypertarget{appendix-b-variable-names}{%
\chapter*{Appendix B: Variable Names}\label{appendix-b-variable-names}}
\addcontentsline{toc}{chapter}{Appendix B: Variable Names}

Links are to the primary discussion of each variable. \(\ddagger\)
denotes an obsolete variable.

\begin{longtable}[]{@{}lllll@{}}
\toprule
Variable Name & Variable Name & Variable Name & Variable Name & Variable
Name \\
\midrule
\endhead
\href{./cloud-physics-variables.html\#CRPC}{A200X} &
\href{./cloud-physics-variables.html\#CRPC}{A200Y} &
\href{./cloud-physics-variables.html\#CRPC}{A260X} &
\href{./obsolete-variables.html\#AACT}{AACT} &
\href{./cloud-physics-variables.html\#CRPC}{ACDP} \\
\href{./the-state-of-the-aircraft.html\#ACINS}{ACINS} &
\href{./the-state-of-the-atmosphere.html\#adifr}{ADIFR} &
\href{./cloud-physics-variables.html\#CRPC}{AF300} &
\href{./obsolete-variables.html\#vanes}{AFIXx\(\ddagger\)} &
\href{./cloud-physics-variables.html\#CRPC}{AFSSP} \\
\href{./cloud-physics-variables.html\#a1dc-a1dp}{A1DC} &
\href{./cloud-physics-variables.html\#a1dc-a1dp}{A1DP} &
\href{./obsolete-variables.html\#akfxx}{AKFXx\(\ddagger\)} &
\href{./the-state-of-the-atmosphere.html\#akrd}{AKRD} &
\href{./obsolete-variables.html\#ltn51}{ALAT\(\ddagger\)} \\
\href{./obsolete-variables.html\#ltn51}{ALON\(\ddagger\)} &
\href{./obsolete-variables.html\#ltn51}{ALPHA\(\ddagger\)} &
\href{./the-state-of-the-aircraft.html\#alt}{ALT} &
\href{./the-state-of-the-aircraft.html\#altg}{ALTG} &
\href{./the-state-of-the-aircraft.html\#altx}{ALTX\(\ddagger\)} \\
\href{./aerosol-particle-measurements.html\#special-aerosol}{AMS} &
\href{./cloud-physics-variables.html\#CRPC}{APCAS} &
\href{./cloud-physics-variables.html\#CRPC}{AS100} &
\href{./cloud-physics-variables.html\#CRPC}{AS200} &
\href{./obsolete-variables.html\#atc}{ATC} \\
\href{./the-state-of-the-atmosphere.html\#at-itr}{AT\_ITR} &
\href{./obsolete-variables.html\#atkp}{ATKP\(\ddagger\)} &
\href{./obsolete-variables.html\#atrf}{ATRF\(\ddagger\)} &
\href{./the-state-of-the-atmosphere.html\#attack}{ATTACK} &
\href{./the-state-of-the-atmosphere.html\#ambient-t}{ATx} \\
\href{./the-state-of-the-atmosphere.html\#ambient-t}{ATX} &
\href{./the-state-of-the-atmosphere.html\#ambient-t}{ATHx} &
\href{./the-state-of-the-atmosphere.html\#ambient-t}{ATxD} &
\href{./cloud-physics-variables.html\#CRPC}{AUHSAS} &
\href{./the-state-of-the-atmosphere.html\#special-use-remote}{AVAPS} \\
\href{./air-chemistry-measurements.html\#awas-cims-qcls-toga}{AWAS} &
\href{./general-information-about-data-files.html\#base-time}{base\_time}
& \href{./the-state-of-the-atmosphere.html\#bdifr}{BDIFR} &
\href{./obsolete-variables.html\#vanes}{BFIXx\(\ddagger\)} &
\href{./the-state-of-the-aircraft.html\#blata}{BLATA} \\
\href{./the-state-of-the-aircraft.html\#blona}{BLONA} &
\href{./the-state-of-the-aircraft.html\#bnorma}{BNORMA} &
\href{./the-state-of-the-aircraft.html\#bpitchr}{BPITCHR} &
\href{./the-state-of-the-aircraft.html\#brollr}{BROLLR} &
\href{./the-state-of-the-aircraft.html\#byawr}{BYAWR} \\
\href{./cloud-physics-variables.html\#c1dc-c1dp}{C1DC} &
\href{./cloud-physics-variables.html\#c1dc-c1dp}{C1DP} &
\href{./cloud-physics-variables.html\#size-distribution}{C200X} &
\href{./cloud-physics-variables.html\#size-distribution}{C200Y} &
\href{./cloud-physics-variables.html\#size-distribution}{C260X} \\
\href{./the-state-of-the-atmosphere.html\#p-special}{CAVP\_x} &
\href{./cloud-physics-variables.html\#size-distribution}{CCDP} &
\href{./obsolete-variables.html\#loranc}{CCEP\(\ddagger\)} &
\href{./cloud-physics-variables.html\#size-distribution}{CF300} &
\href{./obsolete-variables.html\#loranc}{CFSEC\(\ddagger\)} \\
\href{./cloud-physics-variables.html\#size-distribution}{CFSSP} &
\href{./obsolete-variables.html\#loranc}{CGS\(\ddagger\)} &
\href{./air-chemistry-measurements.html\#co2-pic}{CH4\_PICx} &
\href{./air-chemistry-measurements.html\#awas-cims-qcls-toga}{CIMS} &
\href{./obsolete-variables.html\#loranc}{CLAT\(\ddagger\)} \\
\href{./obsolete-variables.html\#loranc}{CLON\(\ddagger\)} &
\href{./obsolete-variables.html\#co-vars}{CMODE} &
\href{./aerosol-particle-measurements.html\#cntemp}{CNTEMP} &
\href{./aerosol-particle-measurements.html\#cnts}{CNTS} &
\href{./obsolete-variables.html\#co}{CO} \\
\href{./air-chemistry-measurements.html\#co2-pic}{CO2\_PICx} &
\href{./obsolete-variables.html\#co-vars}{COCAL} &
\href{./obsolete-variables.html\#co-vars}{COCOR} &
\href{./air-chemistry-measurements.html\#comr-al}{COMR\_AL} &
\href{./cloud-physics-variables.html\#conc2d}{CONC1DC} \\
\href{./cloud-physics-variables.html\#conc2d}{CONC1DC100} &
\href{./cloud-physics-variables.html\#conc2d}{CONC1DC150} &
\href{./cloud-physics-variables.html\#conc2d}{CONC1DP} &
\href{./cloud-physics-variables.html\#concentration}{CONC3} &
\href{./cloud-physics-variables.html\#concentration}{CONC6} \\
\href{./cloud-physics-variables.html\#concentration}{CONCD} &
\href{./cloud-physics-variables.html\#concentration}{CONCF} &
\href{./the-state-of-the-atmosphere.html\#uvh-n}{CONCH\_UVH} &
\href{./aerosol-particle-measurements.html\#concn}{CONCN} &
\href{./aerosol-particle-measurements.html\#special-aerosol}{CONCP} \\
\href{./cloud-physics-variables.html\#concentration}{CONCP} &
\href{./aerosol-particle-measurements.html\#concu-concp}{CONCU} &
\href{./cloud-physics-variables.html\#concentration}{CONCU} &
\href{./cloud-physics-variables.html\#concentration}{CONCU100} &
\href{./cloud-physics-variables.html\#concentration}{CONCU500} \\
\href{./the-state-of-the-atmosphere.html\#vcsel-corr}{CONCV\_VXL} &
\href{./cloud-physics-variables.html\#concentration}{CONCX} &
\href{./cloud-physics-variables.html\#concentration}{CONCY} &
\href{./air-chemistry-measurements.html\#coraw-al}{CORAW\_AL} &
\href{./obsolete-variables.html\#co-vars}{COZRO} \\
\href{./cloud-physics-variables.html\#size-distribution}{CPCAS} &
\href{./obsolete-variables.html\#cryo-hygro}{CRHP} &
\href{./cloud-physics-variables.html\#size-distribution}{CS100} &
\href{./cloud-physics-variables.html\#size-distribution}{CS200} &
\href{./obsolete-variables.html\#loranc}{CSEC\(\ddagger\)} \\
\href{./obsolete-variables.html\#loranc}{CSTAT\(\ddagger\)} &
\href{./cloud-physics-variables.html\#size-distribution}{CUHSAS} &
\href{./general-information-about-data-files.html\#mdy}{DAY} &
\href{./cloud-physics-variables.html\#dbar2d}{DBAR1DC} &
\href{./cloud-physics-variables.html\#dbar2d}{DBAR1DP} \\
\href{./cloud-physics-variables.html\#mean-diameter}{DBAR3} &
\href{./cloud-physics-variables.html\#mean-diameter}{DBAR6} &
\href{./cloud-physics-variables.html\#mean-diameter}{DBARD} &
\href{./cloud-physics-variables.html\#mean-diameter}{DBARF} &
\href{./cloud-physics-variables.html\#mean-diameter}{DBARP} \\
\href{./cloud-physics-variables.html\#mean-diameter}{DBARU} &
\href{./cloud-physics-variables.html\#mean-diameter}{DBARX} &
\href{./cloud-physics-variables.html\#mean-diameter}{DBARY} &
\href{./cloud-physics-variables.html\#dbz2d}{DBZ1DC} &
\href{./cloud-physics-variables.html\#dbz2d}{DBZ1DP} \\
\href{./cloud-physics-variables.html\#DBZ}{DBZ6} &
\href{./cloud-physics-variables.html\#DBZ}{DBZD} &
\href{./cloud-physics-variables.html\#DBZ}{DBZF} &
\href{./cloud-physics-variables.html\#DBZ}{DBZX} &
\href{./cloud-physics-variables.html\#DBZ}{DBZY} \\
\href{./the-state-of-the-aircraft.html\#dei-dni}{DEI} &
\href{./cloud-physics-variables.html\#disp2d}{DISP1DC} &
\href{./cloud-physics-variables.html\#disp2d}{DISP1DP} &
\href{./cloud-physics-variables.html\#dispersion}{DISP3} &
\href{./cloud-physics-variables.html\#dispersion}{DISP6} \\
\href{./cloud-physics-variables.html\#dispersion}{DISPD} &
\href{./cloud-physics-variables.html\#dispersion}{DISPF} &
\href{./cloud-physics-variables.html\#dispersion}{DISPP} &
\href{./cloud-physics-variables.html\#dispersion}{DISPU} &
\href{./cloud-physics-variables.html\#dispersion}{DISPX} \\
\href{./cloud-physics-variables.html\#dispersion}{DISPY} &
\href{./the-state-of-the-aircraft.html\#dei-dni}{DNI} &
\href{./the-state-of-the-atmosphere.html\#dp-cr2}{DP\_CR2C} &
\href{./obsolete-variables.html\#cryo-hygro}{DPCRC} &
\href{./the-state-of-the-atmosphere.html\#vcsel-dp}{DP\_VXL} \\
\href{./the-state-of-the-atmosphere.html\#dew-point}{DPx} &
\href{./the-state-of-the-atmosphere.html\#dew-point}{DPx} &
\href{./the-state-of-the-atmosphere.html\#dewpt-corrected}{DPxC} &
\href{./the-state-of-the-atmosphere.html\#dewpt-corrected}{DPXC} &
\href{./the-state-of-the-atmosphere.html\#dew-point}{DP\_x} \\
\href{./the-state-of-the-atmosphere.html\#dew-point}{DP\_DPB} &
\href{./the-state-of-the-atmosphere.html\#dew-point}{DP\_DPT} &
\href{./the-state-of-the-atmosphere.html\#dew-point}{DP\_DPL} &
\href{./the-state-of-the-atmosphere.html\#dew-point}{DP\_DPR} & \\
\href{./cloud-physics-variables.html\#dt1dc}{DT1DC} &
\href{./the-state-of-the-atmosphere.html\#dvalue}{DVALUE} &
\href{./the-state-of-the-atmosphere.html\#ewx}{EDPC} &
\href{./obsolete-variables.html\#edpc}{EDPC\(\ddagger\)} &
\href{./the-state-of-the-atmosphere.html\#ewx}{EW\_UVH} \\
\href{./the-state-of-the-atmosphere.html\#ewx}{EWx} &
\href{./the-state-of-the-atmosphere.html\#ewx}{EWX} &
\href{./obsolete-variables.html\#fact}{FACT} &
\href{./obsolete-variables.html\#fbmfr}{FBMFR} &
\href{./aerosol-particle-measurements.html\#fcnc}{FCN} \\
\href{./aerosol-particle-measurements.html\#fcnc}{FCNC} &
\href{./air-chemistry-measurements.html\#fo3-acd}{FO3\_ACD} &
\href{./air-chemistry-measurements.html\#fo3-acd}{FO3\_CL} &
\href{./air-chemistry-measurements.html\#fo3-acd}{FO3\_x} &
\href{./the-state-of-the-atmosphere.html\#mirror-cr2}{FP\_CR2} \\
\href{./obsolete-variables.html\#cryo-hygro}{FPCRC} &
\href{./cloud-physics-variables.html\#fssp-range}{FRANGE} &
\href{./obsolete-variables.html\#freset}{FRESET} &
\href{./cloud-physics-variables.html\#fssp-range}{FRNG} &
\href{./obsolete-variables.html\#freset}{FRST} \\
\href{./obsolete-variables.html\#fstrob}{FSTB} &
\href{./obsolete-variables.html\#fstrob}{FSTROB} &
\href{./the-state-of-the-aircraft.html\#fxazim}{FXAZIM} &
\href{./the-state-of-the-aircraft.html\#fxazim}{FXDIST} &
\href{./the-state-of-the-aircraft.html\#ggalt}{GALT\_A} \\
\href{./the-state-of-the-aircraft.html\#geopth}{GEOPHT} &
\href{./the-state-of-the-aircraft.html\#ggalt}{GGALT} &
\href{./the-state-of-the-aircraft.html\#altx}{GGALTC\(\ddagger\)} &
\href{./the-state-of-the-aircraft.html\#ggeoidht}{GGEOIDHT} &
\href{./the-state-of-the-aircraft.html\#gghwgs}{GGHWGS} \\
\href{./the-state-of-the-aircraft.html\#gglat}{GGLAT} &
\href{./the-state-of-the-aircraft.html\#gglon}{GGLON} &
\href{./the-state-of-the-aircraft.html\#ggnsat}{GGNSAT} &
\href{./the-state-of-the-aircraft.html\#ggqual}{GGQUAL} &
\href{./the-state-of-the-aircraft.html\#ggspd}{GGSPD} \\
\href{./the-state-of-the-aircraft.html\#ggstatus}{GGSTATUS} &
\href{./the-state-of-the-aircraft.html\#ggtrk}{GGTRK} &
\href{./the-state-of-the-aircraft.html\#ggvew}{GGVEW} &
\href{./the-state-of-the-aircraft.html\#ggvns}{GGVNS} &
\href{./the-state-of-the-aircraft.html\#ggvspd}{GGVSPD} \\
\href{./the-state-of-the-aircraft.html\#gglat}{GLAT} &
\href{./the-state-of-the-aircraft.html\#gglon}{GLON} &
\href{./the-state-of-the-aircraft.html\#gmode}{GMODE\(\ddagger\)} &
\href{./aerosol-particle-measurements.html\#special-aerosol}{GNI} &
\href{./the-state-of-the-atmosphere.html\#special-use-remote}{GNSS} \\
\href{./the-state-of-the-aircraft.html\#gsf}{GSF} &
\href{./the-state-of-the-aircraft.html\#ggspd}{GSF\_G} &
\href{./obsolete-variables.html\#gsf-obsolete}{GSF\(\ddagger\)} &
\href{./the-state-of-the-aircraft.html\#ggstatus}{GSTAT\_G} &
\href{./the-state-of-the-aircraft.html\#ggstatus}{GSTAT} \\
\href{./the-state-of-the-aircraft.html\#ggvspd}{GVZI\(\ddagger\)} &
\href{./radiation-variables.html\#harp}{HARP} &
\href{./the-state-of-the-aircraft.html\#hgm}{HGM} &
\href{./the-state-of-the-aircraft.html\#hgm-232}{HGM232} &
\href{./the-state-of-the-aircraft.html\#hgme-159}{HGME} \\
\href{./the-state-of-the-aircraft.html\#hi3}{HI3\(\ddagger\)} &
\href{./general-information-about-data-files.html\#hms}{HOUR} &
\href{./obsolete-variables.html\#ias}{IAS} &
\href{./obsolete-variables.html\#irx}{IRx} &
\href{./radiation-variables.html\#irxc}{IRxC} \\
\href{./radiation-variables.html\#irxc}{IRXC} &
\href{./radiation-variables.html\#irxht}{IRxHT} &
\href{./radiation-variables.html\#irxht}{IRxHTV} &
\href{./radiation-variables.html\#irxv}{IRXV} &
\href{./the-state-of-the-aircraft.html\#latitude}{LAT} \\
\href{./the-state-of-the-aircraft.html\#latc-lonc}{LATC} &
\href{./the-state-of-the-aircraft.html\#gglat}{LAT\_G} &
\href{./the-state-of-the-aircraft.html\#longitude}{LON} &
\href{./the-state-of-the-aircraft.html\#latc-lonc}{LONC} &
\href{./the-state-of-the-aircraft.html\#gglon}{LON\_G} \\
\href{./obsolete-variables.html\#jwlwc}{LWC\(\ddagger\)} &
\href{./obsolete-variables.html\#jwlw-corrected}{LWCC\(\ddagger\)} &
\href{./the-state-of-the-atmosphere.html\#mach-number}{MACH} &
\href{./the-state-of-the-atmosphere.html\#mach-number}{MACHx} &
\href{./the-state-of-the-atmosphere.html\#mach-number}{MACHX}{]} \\
\href{./general-information-about-data-files.html\#hms}{MINUTE} &
\href{./the-state-of-the-atmosphere.html\#mirror-cr2}{MIRRORT\_CR2} &
\href{./the-state-of-the-atmosphere.html\#MR}{MR} &
\href{./general-information-about-data-files.html\#mdy}{MONTH} &
\href{./the-state-of-the-atmosphere.html\#MR}{MRCR} \\
\href{./the-state-of-the-atmosphere.html\#MR}{MRLA} &
\href{./the-state-of-the-atmosphere.html\#MR}{MRLH} &
\href{./air-chemistry-measurements.html\#no-noy}{NO} &
\href{./air-chemistry-measurements.html\#no-noy}{NO\textsubscript{y}}
& \\
\href{./obsolete-variables.html\#o3fs}{O3FS} &
\href{./air-chemistry-measurements.html\#f03-acd}{O3MR\_CL} &
\href{./the-state-of-the-atmosphere.html\#special-use-remote}{MTP} &
\href{./the-state-of-the-atmosphere.html\#oat}{OAT} &
\href{./obsolete-variables.html\#AACT}{PACT} \\
\href{./the-state-of-the-aircraft.html\#palt}{PALT} &
\href{./the-state-of-the-atmosphere.html\#p-special}{PCAB} &
\href{./aerosol-particle-measurements.html\#pcn}{PCN} &
\href{./the-state-of-the-atmosphere.html\#p-special}{P\_CR2} &
\href{./aerosol-particle-measurements.html\#pflw}{PFLW} \\
\href{./aerosol-particle-measurements.html\#pflw}{PFLWC} &
\href{./obsolete-variables.html\#ltn51}{PHDG\(\ddagger\)} &
\href{./the-state-of-the-aircraft.html\#pitch}{PITCH} &
\href{./cloud-physics-variables.html\#plwc}{PLWC} &
\href{./cloud-physics-variables.html\#plwc}{PLWC1} \\
\href{./cloud-physics-variables.html\#lwc2d}{PLWC1DC} &
\href{./cloud-physics-variables.html\#lwc2d}{PLWC1DP} &
\href{./cloud-physics-variables.html\#PSD-LWC}{PLWC6} &
\href{./cloud-physics-variables.html\#plwcc}{PLWCC} &
\href{./cloud-physics-variables.html\#plwcc}{PLWCC1} \\
\href{./cloud-physics-variables.html\#PSD-LWC}{PLWCD} &
\href{./cloud-physics-variables.html\#PSD-LWC}{PLWCF} &
\href{./cloud-physics-variables.html\#plwcg}{PLWCG} &
\href{./cloud-physics-variables.html\#PSD-LWC}{PLWCX} &
\href{./cloud-physics-variables.html\#PSD-LWC}{PLWCY} \\
\href{./the-state-of-the-atmosphere.html\#psx}{PS\_A} &
\href{./the-state-of-the-atmosphere.html\#p-special}{PSDPx} &
\href{./the-state-of-the-atmosphere.html\#psx}{PSFD} &
\href{./the-state-of-the-atmosphere.html\#psx}{PSFRD} &
\href{./the-state-of-the-atmosphere.html\#psx}{PSTF} \\
\href{./the-state-of-the-atmosphere.html\#p-special}{PSURF} & & & & \\
\href{./the-state-of-the-atmosphere.html\#psx}{PSX} &
\href{./the-state-of-the-atmosphere.html\#psx}{PSxC} &
\href{./the-state-of-the-atmosphere.html\#psx}{PSXC} &
\href{./the-state-of-the-atmosphere.html\#psx}{PSx} &
\href{./obsolete-variables.html\#ptime}{PTIME\(\ddagger\)} \\
\href{./the-state-of-the-atmosphere.html\#qcx}{QC\_A} &
\href{./obsolete-variables.html\#qcb}{QCBC\(\ddagger\)} &
\href{./obsolete-variables.html\#qcb}{QCB\(\ddagger\)} &
\href{./obsolete-variables.html\#qcb}{QCGC\(\ddagger\)} &
\href{./obsolete-variables.html\#qcb}{QCG\(\ddagger\)} \\
\href{./air-chemistry-measurements.html\#awas-cims-qcls-toga}{QCLS} &
\href{./the-state-of-the-atmosphere.html\#qcx}{QCR} &
\href{./the-state-of-the-atmosphere.html\#qcx}{QCRC} &
\href{./the-state-of-the-atmosphere.html\#qcx}{QCTF} &
\href{./the-state-of-the-atmosphere.html\#qcx}{QCTFC} \\
\href{./the-state-of-the-atmosphere.html\#qcx}{QCx} &
\href{./the-state-of-the-atmosphere.html\#qcx}{QCX} &
\href{./the-state-of-the-atmosphere.html\#qcx}{QCxC} &
\href{./the-state-of-the-atmosphere.html\#qcx}{QCXC} & \\
\href{./the-state-of-the-atmosphere.html\#vcsel-uncor}{RAWCONC\_VXL} &
\href{./cloud-physics-variables.html\#reff2d}{REFF2DC} &
\href{./cloud-physics-variables.html\#reff2d}{REFF2DP} &
\href{./cloud-physics-variables.html\#effective-radius}{REFFD} &
\href{./cloud-physics-variables.html\#effective-radius}{REFFF} \\
\href{./the-state-of-the-atmosphere.html\#rho}{RHOLA} &
\href{./the-state-of-the-atmosphere.html\#rho}{RHOUV} &
\href{./the-state-of-the-atmosphere.html\#rho}{RHOx} &
\href{./the-state-of-the-atmosphere.html\#rhumw}{RHUM} &
\href{./the-state-of-the-atmosphere.html\#rhumi}{RHUMI} \\
\href{./cloud-physics-variables.html\#rice}{RICE} &
\href{./the-state-of-the-aircraft.html\#roc}{ROC} &
\href{./the-state-of-the-aircraft.html\#roll}{ROLL} &
\href{./radiation-variables.html\#rstx}{RSTx} &
\href{./the-state-of-the-atmosphere.html\#recovery-t}{RTHRx} \\
\href{./the-state-of-the-atmosphere.html\#recovery-t}{RTx} &
\href{./the-state-of-the-atmosphere.html\#recovery-t}{RTX} &
\href{./the-state-of-the-atmosphere.html\#recovery-t}{RTHx} &
\href{./general-information-about-data-files.html\#hms}{SECOND} &
\href{./obsolete-variables.html\#sclwc}{SCLWC} \\
\href{./the-state-of-the-aircraft.html\#sfc}{SFC} &
\href{./aerosol-particle-measurements.html\#special-aerosol}{SMPS} &
\href{./radiation-variables.html\#solaz}{SOLAZ} &
\href{./radiation-variables.html\#solde}{SOLDE} &
\href{./radiation-variables.html\#solel}{SOLEL} \\
\href{./radiation-variables.html\#solze}{SOLZE} &
\href{./the-state-of-the-atmosphere.html\#sphum}{SPHUM} &
\href{./radiation-variables.html\#spx}{SPxPitch} &
\href{./radiation-variables.html\#spx}{SPxRoll} &
\href{./obsolete-variables.html\#akfxx}{SSFXx\(\ddagger\)} \\
\href{./the-state-of-the-atmosphere.html\#sslip}{SSLIP} &
\href{./the-state-of-the-atmosphere.html\#ssrd}{SSRD} &
\href{./obsolete-variables.html\#swtc}{SWTC} &
\href{./obsolete-variables.html\#swx}{SWx} &
\href{./the-state-of-the-atmosphere.html\#true-airspeed}{TASx} \\
\href{./the-state-of-the-atmosphere.html\#true-airspeed}{TASX} &
\href{./the-state-of-the-atmosphere.html\#tashc}{TASHC} &
\href{./the-state-of-the-atmosphere.html\#true-airspeed}{TASxD} &
\href{./radiation-variables.html\#rstx}{TCAVB} &
\href{./radiation-variables.html\#rstx}{TCAVT} \\
\href{./aerosol-particle-measurements.html\#tcntu-tcntp}{TCNTP} &
\href{./aerosol-particle-measurements.html\#tcntu-tcntp}{TCNTU} &
\href{./aerosol-particle-measurements.html\#cntemp}{TEMP1} &
\href{./aerosol-particle-measurements.html\#cntemp}{TEMP2} &
\href{./air-chemistry-measurements.html\#te03}{TEO3} \\
\href{./air-chemistry-measurements.html\#te03c}{TEO3C} &
\href{./air-chemistry-measurements.html\#tep}{TEO3P} &
\href{./air-chemistry-measurements.html\#tep}{TEP} &
\href{./air-chemistry-measurements.html\#tet}{TET} &
\href{./the-state-of-the-aircraft.html\#thdg}{THDG}{]} \\
\href{./the-state-of-the-atmosphere.html\#theta}{THETA} &
\href{./the-state-of-the-atmosphere.html\#thetae}{THETAE} &
\href{./the-state-of-the-atmosphere.html\#thetae}{THETAP} &
\href{./the-state-of-the-atmosphere.html\#thetaq}{THETAQ} &
\href{./the-state-of-the-atmosphere.html\#thetav}{THETAV} \\
\href{./obsolete-variables.html\#thf}{THF\(\ddagger\)} &
\href{./obsolete-variables.html\#ltn51}{THI\(\ddagger\)} &
\href{./general-information-about-data-files.html\#time}{Time} &
\href{./general-information-about-data-files.html\#time-offset}{time\_offset}
& \href{./the-state-of-the-aircraft.html\#ggtrk}{TKAT\_G} \\
\href{./obsolete-variables.html\#tmlag}{TMLAG\(\ddagger\)} &
\href{./air-chemistry-measurements.html\#awas-cims-qcls-toga}{TOGA} &
\href{./obsolete-variables.html\#tptime}{TPTIME\(\ddagger\)} &
\href{./radiation-variables.html\#trstx}{TRSTB} &
\href{./radiation-variables.html\#trstx}{TRSTT} \\
\href{./obsolete-variables.html\#ttkp}{TTKP\(\ddagger\)} &
\href{./obsolete-variables.html\#ttrf}{TTRF\(\ddagger\)} &
\href{./obsolete-variables.html\#ttx}{TTx\(\ddagger\)} &
\href{./the-state-of-the-atmosphere.html\#TVIR}{TVIR} &
\href{./the-state-of-the-atmosphere.html\#ui-vi-wi}{UI} \\
\href{./the-state-of-the-atmosphere.html\#uic-vic}{UIC} &
\href{./aerosol-particle-measurements.html\#upress}{UPRESS} &
\href{./aerosol-particle-measurements.html\#pflw}{USFLWC} &
\href{./aerosol-particle-measurements.html\#pflw}{USMPFLW} &
\href{./obsolete-variables.html\#uvx}{UVx} \\
\href{./the-state-of-the-atmosphere.html\#ux-vy}{UX} &
\href{./the-state-of-the-atmosphere.html\#uxc-vyc}{UXC} &
\href{./obsolete-variables.html\#cryo-hygro}{VCRH} &
\href{./obsolete-variables.html\#gsf-obsolete}{VEW\(\ddagger\)} &
\href{./the-state-of-the-aircraft.html\#vewc-vnsc}{VEWC} \\
\href{./the-state-of-the-aircraft.html\#vew}{VEW}\_ &
\href{./the-state-of-the-aircraft.html\#ggvew}{VEW\_G\(\ddagger\)} &
\href{./the-state-of-the-atmosphere.html\#ui-vi-wi}{VI} &
\href{./the-state-of-the-atmosphere.html\#uic-vic}{VIC} &
\href{./radiation-variables.html\#visxc}{VISxC} \\
\href{./radiation-variables.html\#visxht}{VISxHT} &
\href{./radiation-variables.html\#visxht}{VISxHTV} &
\href{./radiation-variables.html\#visxv}{VISxV} &
\href{./obsolete-variables.html\#vla}{VLA\(\ddagger\)} &
\href{./the-state-of-the-aircraft.html\#vns}{VNS} \\
\href{./the-state-of-the-aircraft.html\#vewc-vnsc}{VNSC} &
\href{./the-state-of-the-aircraft.html\#ggvns}{VNS\_G} &
\href{./obsolete-variables.html\#gsf-obsolete}{VNS\(\ddagger\)} &
\href{./the-state-of-the-aircraft.html\#vspd}{VSPD} &
\href{./the-state-of-the-aircraft.html\#ggvspd}{VSPD\_G} \\
\href{./the-state-of-the-atmosphere.html\#ux-vy}{VY} &
\href{./the-state-of-the-atmosphere.html\#uxc-vyc}{VYC} &
\href{./obsolete-variables.html\#vzi}{VZI\(\ddagger\)} &
\href{./the-state-of-the-atmosphere.html\#ws-wd}{WD} &
\href{./the-state-of-the-atmosphere.html\#wsc-wdc}{WDC} \\
\href{./obsolete-variables.html\#wspd}{WDRCTN\(\ddagger\)} &
\href{./the-state-of-the-atmosphere.html\#ui-vi-wi}{WI} &
\href{./the-state-of-the-atmosphere.html\#wic}{WIC} &
\href{./the-state-of-the-aircraft.html\#wp3}{WP3} &
\href{./the-state-of-the-atmosphere.html\#ws-wd}{WS} \\
\href{./the-state-of-the-atmosphere.html\#wsc-wdc}{WSC} &
\href{./obsolete-variables.html\#wspd}{WSPD\(\ddagger\)} &
\href{./air-chemistry-measurements.html\#comr-al}{XCOMR} &
\href{./air-chemistry-measurements.html\#comr-al}{XCOMR\_AL\(\ddagger\)}
& \href{./air-chemistry-measurements.html\#xf03fs}{XFO3FNO} \\
\href{./air-chemistry-measurements.html\#xf03fs}{XFO3FS} &
\href{./air-chemistry-measurements.html\#xf03p}{XFO3P} &
\href{./aerosol-particle-measurements.html\#xicnc}{XICN} &
\href{./aerosol-particle-measurements.html\#xicnc}{XICNC} &
\href{./the-state-of-the-atmosphere.html\#mach-number}{XMACH2} \\
\href{./air-chemistry-measurements.html\#no-noy}{XNCLF} &
\href{./air-chemistry-measurements.html\#no-noy}{XNMBT} &
\href{./air-chemistry-measurements.html\#no-noy}{XNO} &
\href{./air-chemistry-measurements.html\#mr-no-no2}{XNOCAL} &
\href{./air-chemistry-measurements.html\#mr-no-no2}{XNYCAL} \\
\href{./air-chemistry-measurements.html\#no-noy}{XNOCF} &
\href{./air-chemistry-measurements.html\#no-noy}{XNOSF} &
\href{./air-chemistry-measurements.html\#no-noy}{XNOY} &
\href{./air-chemistry-measurements.html\#no-noy}{XNOYP} &
\href{./air-chemistry-measurements.html\#no-noy}{XNOZA} \\
\href{./air-chemistry-measurements.html\#no-noy}{XNSAF} &
\href{./air-chemistry-measurements.html\#no-noy}{XNST} &
\href{./air-chemistry-measurements.html\#no-noy}{XNZAF} &
\href{./air-chemistry-measurements.html\#f03-acd}{XO3} &
\href{./the-state-of-the-atmosphere.html\#uvh-voltage}{XSIG\_UVH} \\
\href{./obsolete-variables.html\#xuvi}{XUVI} &
\href{./obsolete-variables.html\#xuvi}{XUVP} &
\href{./obsolete-variables.html\#xuvi}{XUVT} &
\href{./obsolete-variables.html\#ltn51}{XVI\(\ddagger\)} &
\href{./general-information-about-data-files.html\#mdy}{YEAR} \\
\href{./obsolete-variables.html\#ltn51}{YVI\(\ddagger\)} & & & & \\
\bottomrule
\end{longtable}

\hypertarget{appendix-c-editing-the-t.n.}{%
\chapter*{Appendix C: Editing The
T.N.}\label{appendix-c-editing-the-t.n.}}
\addcontentsline{toc}{chapter}{Appendix C: Editing The T.N.}

\hypertarget{notes-regarding-the-construction-and-structure-of-this-document}{%
\section*{Notes regarding the construction and structure of this
document:}\label{notes-regarding-the-construction-and-structure-of-this-document}}
\addcontentsline{toc}{section}{Notes regarding the construction and
structure of this document:}

\begin{itemize}
\tightlist
\item
  The original version is ProcessingAlgorithms.lyx, which needs 'LyX', a
  user interface to TeX. It is available on EOL machines like tikal.
  Start it with ``lyx ProcessingAlgorithms.lyx''. This version was last
  updated in Feb 2019. It can be obtained from the
  \href{https://github.com/WilliamCooper/ProcessingAlgorithms}{GitHub
  site}. Copy this to an RStudio project file to use this version.\\
\item
  A revised version is written in RMarkdown to facilitate editing by
  others. It provides an HTML version of the document suitable for
  hosting on web servers. To obtain this version, download
  \href{https://github.com/WilliamCooper/ProcessingAlgorithms/tree/Rmd}{this
  repository}. This works best if output: bookdown::gitbook is selected.
  Download this branch to an RStudio project directory and use the
  ``knit'' button to construct the web site. The output will be html
  files that can be moved from there to a web server. The directories
  ``assets'' and ``www'' should also be transferred; the former contains
  CSS files and the latter some memos that are referenced in the
  document. For example, if the web server delivers files from
  /var/www/html, make a new directory there called
  ``ProcessingAlgorithms'' and move all .html files and those two
  directories to ``ProcessingAlgorithms''. Then view the document via
  \url{http://URL/ProcessingAlgorithms} where URL is the appropriate
  reference for your web files.

  \begin{itemize}
  \tightlist
  \item
    The R Markdown files have a mix of conventions including HTML, LaTeX
    formulas, and R Markdown conventions. At some point, it may be
    useful to become more consistent, e.g., by changine HTML references
    to R Markdown reference, changing HTML italics to R Markdown
    italics, changing table structures from HTML to R Markdown kable
    format, etc.\\
  \item
    The R Markdown files can produce a PDF file, but many features
    available in the HTML files will not be available, at least at
    present. PDF files can always be produced by printing from a web
    browser.\\
  \end{itemize}
\item
  The document is broken into many sections, referenced by the above
  file, so they must be present also. In the lyx version they have names
  like Section3.lyx. In the Rmd version, they reside as, e.g.,
  Section3.Rmd in the ``Children'' directory.\\
\item
  The lyx document generates three indices: a regular index, a list of
  symbols, and a list of variables. The references for these are
  embedded in the .lyx files, and they can be modified or more can be
  added via the ``Insert -\textgreater{} Index Entry'' controls. These
  practices are useful when generating index entries:

  \begin{itemize}
  \item
    entries like 'wind!relative' will generate index entries as
    subordinate entries with 'relative' below 'wind'. In the Rmd
    version, the index is not generated automatically but the existing
    links should remain valid as the text is changed. To add a variable,
    use a subsection or subsubsection entry and identify a label by
    following the heading with \{\#newlabel\}. Then follow the pattern
    in the existing index to add an entry for the new term. Also add a
    similar link to the list of variables in Appendix B.
  \item
    I have tried to emphasize using nouns to start index entries, so for
    example I would favor ``coefficient!calibration'' over ``calibration
    coefficient.\\
  \item
    It is sometimes useful to generate ``see xxx'' entries, which can be
    done as follows in the LyX version: ``INS\textbar see \{Inertial
    Navigation System\}'' where the part in braces is also in LaTeX
    code, generated by pressing CNTL-L.\\
  \end{itemize}
\item
  Creating a PDF-format file in LyX usually will generate these lists
  also.\\
\item
  The LyX files have embedded notes with additional information that
  should be retained, and exporting to LaTeX will lose this information,
  so it will be useful to retain the LyX format. The suggested next
  steps in the table above, for example, almost all have associated
  notes that will appear in yellow and will help identify where the
  comment applies.\\
\item
  It is sometimes easiest to edit the PDF file directly. Some of the web
  references have been changed in this way and can be adjusted as the
  reference files are moved, e.g., from my Google Drive to the EOL web
  pages. For this purpose, I found master-pdf-editor useful. This will
  lose continuity, however, because then the links can't be re-generated
  by running LyX.\\
\item
  As of Feb 2019, many links formerly to google-drive addresses or eol
  system files have been changed to
  \url{https://github.com/NCAR/aircraft_ProcessingAlgorithms} links. In
  that directory there is a file ('links') with a list of all the links
  in the document. It is worthwhile when updating this document to check
  that all the links remain current. One way is to use these R
  statements and then check EURL to see that the links are all found:
\end{itemize}

\begin{verbatim}
    links <- readlines(’./links’); EURL <- rep(FALSE, length(links));
    for (i in 1:length(links)) {EURL<- RCurl::url.exists(links[i])}
\end{verbatim}

\begin{itemize}
\tightlist
\item
  For the Rmd version:

  \begin{itemize}
  \tightlist
  \item
    In place of the Table of Symbols, there is Appendix A. It is a .gif
    copy of the LyX-generated table, and there is no easy way to update
    it except vis LyX.
  \item
    In place of the ``Variable Names'' index, there is Appendix B that
    lists the variable names and includes links to appropriate
    references to variables in the technical note. This is not generated
    automatically, so when a new variable is added to the document a new
    entry with appropriate links should be added to Appendix B. Follow
    the style in ``Children/Appendices.Rmd'' and add an appropriate
    target for the link to the new discussion of the variable, which
    should be a 4th-level or higher heading.
  \end{itemize}
\end{itemize}

\hypertarget{referencing-specific-sections-or-pages-of-this-document-lyx-version}{%
\section*{Referencing Specific Sections or Pages of this Document (LyX
version):}\label{referencing-specific-sections-or-pages-of-this-document-lyx-version}}
\addcontentsline{toc}{section}{Referencing Specific Sections or Pages of
this Document (LyX version):}

\hypertarget{variables}{%
\subsection*{Variables}\label{variables}}
\addcontentsline{toc}{subsection}{Variables}

The document includes named destinations for each variable name, so when
used in a URL that destination can be reached. This is done differently
in different browsers or PDF viewers:

For a web browser like Chrome or Firefox, use the ``nameddest''
reference; e.g., for the discussion of variable ATX, use firefox
\url{http://www.eol.ucar.edu/system/files/ProcessingAlgorithms.pdf\#nameddest=ATX}

For a pdf viewer like evince, use this syntax: evince -n ATX
\url{http://www.eol.ucar.edu/system/files/ProcessingAlgorithms.pdf}

Most variable names can be used in these URL modifiers. Here is a list
of available targets by section in the report:

\underline{Section 1:} Time

\underline{Section 2:} {{[}none{]}}

\underline{Section 3:} ACINS ALT BLATA BLONA BNORMA BPITCHR BROLLR BYAWR
DEI DNI FXAZIM FXDIST GGALT GGLAT GGLON GGNSAT GGOIDHT GGSPD GGSTATUS
GGTRK GGVEW GGVNS GGVSPD GGWUAL GMODE GSF HGM HGM232 HGME HI3 LAT LATC
LON LONC PALT PITCH ROLL THDG VEW VEWC VNS VNSC VSPD

\underline{Section 4:} ADIFR AKRD AT\_ITR ATx ATX ATxD ATxJ BDIFR
CAVP\_x CONCH\_UVH CONCV\_VXL DP\_CR2C DP\_VXL DPx DP\_x DPxC DPXC
DVALUE EDPC EW\_UHV EWx EWX FP\_CR2 MACHx MACHX MIRRORT\_CR2
MIRRTMP\_DPX MR MRCR MRLA MRLH MRVCL OAT PCAB PS\_A PSDPx PSFD PSFRD
PSURF PSx PSX PSxC PSXC QCx QCX QCxC QCXC RAWCONC\_VXL RHOx RHUM RHUMI
RTHRx RTx RTX RTxH SPHUM SSLIP TASHC TASx TASX TASxD THETA THETAE THETAP
THETAQ THETAV TVIR UI UIC UX UXC VI VIC VY VYC WD WDC WI WIC WS WSC
XSIGV\_UHV

\underline{Section 5:} A1DC A1DP A200X A200Y A260X ACDP AF300 AFSSP
APCAS AS100 AUHSAS C1DC C1DP C200X C200Y C260X CCDP CF300 CFSSP COMCP
CONC1DC CONC1DC100 CONC1DC150 CONC1DP CONC3 CONC6 CONCD CONCF CONCU
CONCX CONCY CPCAS CS100 CUHSAS DBAR1DC DBAR1DP DBAR3 DBAR6 DBARD DBARF
DBARP DBARU DBARX DBARY DBZ DBZ1DC DBZ1DP DISP1DC DISP1DP DISP3 DISP6
DISPD DISPF DISPP DISPU DISPX DISPY DT1DC FRANGE FRNG PLWC1 PLWC1DC
PLWC1DP PLWC6 PLWCC PLWCC1 PLWCD PLWCF PLWCG PLWCX PLWCY REFF2DC REFF2DP
REFFD REFFF RICE

\underline{Section 6:} CO2\_PIC COMR\_AL CORAW\_AL FO3\_ACD FO3\_CL
O3MR\_CL TEO3 TEO3C TEO3P TEP TET XFO3FNO XFO3FS XFO3P XNCLF XNMBT XNO
XNOCAL XNOCF XNOSF XNOY XNOYP XNOZA XNSAF XNST XNYCAL XNZAF XO3

\underline{Section 7:} CNTEMP CNTS CONCN CONCP CONCU CONCU100 CONCU500
FCN FCNC PCN PFLW PFLWC TCHTP TCNTL TEMP1 TEMP2 UPRESS USFLWC USMPFLW
XICN XICNC

\underline{Section 8:} IRxHT IRxV RSTx SPxPitch SPxRoll TRSTx VISxC
VISxHT VISxV

\underline{Section 9:} {{[}none{]}}

\underline{Section 10:} OBSOLETE

\hypertarget{page-numbers-lyx-version}{%
\subsection*{Page Numbers (LyX
Version)}\label{page-numbers-lyx-version}}
\addcontentsline{toc}{subsection}{Page Numbers (LyX Version)}

To reference a specific page in the document, use a web reference like
this: ProcessingAlgorithms.pdf\#page=44

In evince, this syntax will work, or the page number can be specified in
this way:

evince -p 115 ProcessingAlgorithms.pdf

\hypertarget{sections-and-subsections}{%
\subsection*{Sections and Subsections}\label{sections-and-subsections}}
\addcontentsline{toc}{subsection}{Sections and Subsections}

Targets have not been provided for other parts of the document, but the
above method of referencing pages can be used to link to specific
sections and other components of the document.

\hypertarget{adding-new-targets}{%
\subsection*{Adding New Targets}\label{adding-new-targets}}
\addcontentsline{toc}{subsection}{Adding New Targets}

When a new variable is added, a new anchor point can be added by
inserting, in LaTeX mode, ``\textbackslash nop\{LAT\}'' at the
appropriate point in the LyX document. (``\textbackslash nop'' has been
defined to use \textbackslash hypertarget but displace the reference
upward one line.) In addition, when a new variable is added, entries
should be made in the index items and the variable-names list, following
the pattern used for existing variables, and if appropriate any new
symbols used in discussing the algorithm should be added to the similar
symbols list.

\hypertarget{referencing-specific-sections-of-this-document-rmarkdown-version}{%
\section*{Referencing Specific Sections of this Document (RMarkdown
version):}\label{referencing-specific-sections-of-this-document-rmarkdown-version}}
\addcontentsline{toc}{section}{Referencing Specific Sections of this
Document (RMarkdown version):}

Most discussions of variables are in subsubsubsections with appropriate
labels, and those labels can be referenced as in the model provided by
Appendix B.

\hypertarget{appendix-d-needing-attention}{%
\chapter*{Appendix D: Needing
Attention}\label{appendix-d-needing-attention}}
\addcontentsline{toc}{chapter}{Appendix D: Needing Attention}

page

suggested action

who?

Update MW table for current CO2 concentration

done, WAC 2019 -- keep updating

Add, to constants table, a reference to what has been in use previously.
(See Code.amlib as saved 2011, for examples)

WAC -- not sure if needed?

Ask Teresa and Mike R. to review the discussion re trace-gas units

done, MR

Get info from Chris W describing interpolation and time adjustments, for
inclusion as an addition to the section on times. Revise section.

Get place to put algorithm notes, and include links to those additional
discussions in this document.

done WAC

In algorithm boxes, when variables are referenced, make those references
active links to the discussion of the variable

WAC: mostly done now in Rmd version

Get description of the history of the C-130 INS, with characteristics
for the Litton at least.

Add history of GPS systems: What was used when (C-130 at least)

Add/clarify section on height-above-terrain; modify to ref. geoid. Need
to change HeightAboveTerrain() script.

partly done, WAC,

check/clarify discussion of height-above-geoid and, generally,
geopotential vs geometric vs geoid height

done - WAC

Add a variable representing geopotential height and change DVALUE to be
based on it minus PALT

WAC - done

Clarify meaning of mode and status for old GPS units, and if used
anymore

Add new section on ALTC? Info is there in comments. Implement?

should there be a vertical velocity of the AC based on data-system GPS?
ROC as used for reprocessing, and WIR as backup to WIC?

Check/update sensors used on both aircraft.

Add to historical description of PCORs, esp.~re subroutine references
(QCF, MACH\_A, ADIFR)

Add a discussion of the additional corrections to QCR that could make
this less sensitive to AOA? Algorithm is developed and documented;
implement? Coefficients in ProcessingAlgorithms.pdf are based on
ARISTO2016 flight 6.

WAC - done

Suggestion: consider ALT\_G and avoid ALT for GPS avionics variable

Consider change to spherical geometry for distance north and east of
reference point because range of GV is so great

need to explain how the two measures of longitude, with high and low
resolution, are used together.

Need to implement the discussion re correction for the displacement of
the GPS antenna from the INS. When done, need to add LG=-4.30 m to the
attributes for GGVSPD, GGVEW, GGVNS (GV) and get appropriate values for
the C-130

WAC - done

Get Dick Friesen or someone to review and update the discussion of GPS
GSTAT

Revise the values listed for the complementary-filter feedback to match
what is used now -- better with lower values than listed

WAC

It might be useful to disable the roll test in gpsc.c, now that GPS is
better than when this was implemented

Goodrich Technical Report 5755: should we get permission and post this?
(FAAM has it posted)

(It is posted now)

In-cloud air T radiometer: could use more detail re the processing
algorithm

Check all the complex M-K section, esp.~{Tk} and DP interp. function

done - WAC

Consider changing name to FP\_CR2 in preference to MIRRORT\_CR2?

For CONCV\_VXL, I think we need cal coefficients and equations used

For RHOx, the code now uses 216.68 instead of 100000/461.5228=216.674 as
specified here; change?

Check that current code uses the modified PCOR function with humidity
correction and early-error corrected

The PSURF definition references PSFDC; replace with PSXC?

``ATTACK'' and ``SSLIP'' differ from other ``preferred'' variables by
not having ``X'' at the end. Consider name change?

The variable WIC is described as ``GPS-corrected'' but that is
misleading because it is really based, for aircraft motion, solely on
GPS in recent usage (where dependence is on GGVSPD). Contrast to WDC/WSC
which are really GPS-corrected. Suggest a different name, like ``Wind
Vector, Vertical Component, using GPS''?

Re Gerber probe, I didn't find code for this; need to describe the
algorithm.

It would be useful to update Bulletin 24 re hydrometeor spectrometers,
with info from Bansemer and reference to work by Korolev, Strapp,
Jensen, etc.

Is ``PMS/CSIRO King'' correct? DMT?

OK - JBJ

Variables like AS200 have names with ``Raw Accumulation'' -- seems
awkward, consider name change? Maybe ``Count'' per channel?

add the variables for total counts?

some additions are needed here: RAFFD, PVOLU, TCNTD (total counts all
cells, CDP); housekeeping variables? All: first and last bins? UHSAS: T
and P in canister including UPRESS intensity, etc., better in sect 7

REFF2DC seems mis-named; all others based on 1D sizing from 2D have
names involving 2DC

need Teresa and/or Andy W to check this section

need to understand and document what te03c.c does, and perhaps move to
obsolete?

Is NO-related discussion OK? is this right: The one named NO2 is
actually for NOy?

In true measurement mode, XNOZA and XNZAF will be near zero -- is this
right?

I think the corrected-NO mixing ratio section may need revision?

``has the provision for the addition of water vapor'' -- does that mean
this is always done, or only sometimes?

to ``0.1-360 s'' add ``but is typically set to 0.1~s''?

UPRESS: the attributes for this variable say the units are kPa; is that
incorrect? Mike R lists it as hPa.

resolved

check signs here for WD and WS; this differs from the section-9 equation
\ldots{} ??

For EDPC, the \textless-50 branch looks suspicious and needs checking

For old cryogenic hygrometer, find and include the 3rd-order equation
referenced here

check Goff-Gratch formulas; there was some ambiguity in what was in B9

Should include basic equation for SWTC

re TECO CO: is the direct measurement (ppb) a mass ratio? Need
explanation here if so to describe difference between ppb and ppbv

COCAL: how does this differ from XCOMR? Why is this in the ``obsolete''
section? Same for O3FS?

For SCLWC, this is missing crucial information like how accreted mass is
obtained from voltage. Couldn't find the algorithm. Consider Mazin
version? Or old one for Wyo KA?

There are some additional notes regarding obsolete variables,
esp.~involving FSSP processing, that are not included here.

WIC, GGALT, etc.

Review and correct descriptive attributes (e.g., WIC, GGALT, *DGPS

StdSpeedofSound is wrong; enters ias.c -- obsolete now?

fix Rd in xlate/const.c: calculated with wrong Md, although right one is
listed later in routine. (trivial difference)

Lv defined in xlate/const.c is not latent heat but derivative of latent
heat vs T. Used correctly in thetap.c and plwcc.c, but deceptively
commented

Review and approve new AKRD description

\hypertarget{index-1}{%
\chapter{Index}\label{index-1}}

Page numbers refer to the PDF-format version. Links are to the
appropriate location in this html-format document, usually to the
appropriate section containing the referenced term. When viewed in a
browser, use the browser search function to find the exact location of
the referenced term. \twocolumn

\href{./5-cloud-physics-variables.html\#a1dc-a1dp}{A1DC}, 66\\
\href{./5-cloud-physics-variables.html\#a1dc-a1dp}{A1DP}, 66\\
\href{./5-cloud-physics-variables.html\#CRPC}{A200X}, 62\\
\href{./5-cloud-physics-variables.html\#CRPC}{A200Y}, 62\\
\href{./5-cloud-physics-variables.html\#CRPC}{A260X}, 62\\
\href{./10-obsolete-variables.html\#AACT}{AACT}, 94\\
abbreviations\\
\hspace*{0.333em}\hspace*{0.333em}\href{./2-general-information-about-data-files.html\#units-and-abbreviations}{non-standard},
6\\
acceleration\\
\hspace*{0.333em}\hspace*{0.333em}\href{./3-the-state-of-the-aircraft.html\#ACINS}{vertical},
11\\
\href{./5-cloud-physics-variables.html\#CRPC}{ACDP}, 62\\
\href{./3-the-state-of-the-aircraft.html\#ACINS}{ACINS}, 11\\
adiabatic compression, 36\\
\href{./4-the-state-of-the-atmosphere.html\#adifr}{ADIF}, 53\\
\href{./4-the-state-of-the-atmosphere.html\#adifr}{ADIFR}, 53, 54\\
ADS=aircraft data system, 10\\
aeros, 1\\
aerosol\\
\hspace*{0.333em}\hspace*{0.333em}ancillary datasets, 77\\
\hspace*{0.333em}\hspace*{0.333em}\href{./7-aerosol-particle-measurements.html\#aerosol-spec}{spectrometer},
76\\
\href{./5-cloud-physics-variables.html\#CRPC}{AF300}, 62\\
\href{./10-obsolete-variables.html\#vanes}{AFIXx}, 87\\
\href{./5-cloud-physics-variables.html\#CRPC}{AFSSP}, 62\\
airspeed, 31, 32, 34, 36, 53\\
\hspace*{0.333em}\hspace*{0.333em}\href{./10-obsolete-variables.html\#ias}{indicated},
88\\
\href{./10-obsolete-variables.html\#vanes}{AKFXx}, 87\\
\href{./4-the-state-of-the-atmosphere.html\#akrd}{AKRD}, 53, 54\\
\href{./10-obsolete-variables.html\#ltn51}{ALAT}, 85\\
\href{./10-obsolete-variables.html\#ltn51}{ALON}, 85\\
\href{./10-obsolete-variables.html\#ltn51}{ALPHA}, 85\\
\href{./3-the-state-of-the-aircraft.html\#alt}{ALT}, 15\\
\href{./3-the-state-of-the-aircraft.html\#altg}{ALTG}, 21\\
altimeter\\
\hspace*{0.333em}\hspace*{0.333em}\href{./3-the-state-of-the-aircraft.html\#hgme-159}{radar},
21\\
altitude\\
\hspace*{0.333em}\hspace*{0.333em}aircraft\\
\hspace*{0.333em}\hspace*{0.333em}\hspace*{0.333em}\hspace*{0.333em}\href{./3-the-state-of-the-aircraft.html\#ggalt}{GPS},
19\\
\hspace*{0.333em}\hspace*{0.333em}\hspace*{0.333em}\hspace*{0.333em}\href{./3-the-state-of-the-aircraft.html\#global-positioning-systems}{geometric},
16\\
\hspace*{0.333em}\hspace*{0.333em}\hspace*{0.333em}\hspace*{0.333em}\href{./3-the-state-of-the-aircraft.html\#geoph}{geopotential},
16, 32\\
\hspace*{0.333em}\hspace*{0.333em}\hspace*{0.333em}\hspace*{0.333em}\href{./3-the-state-of-the-aircraft.html\#alt}{inertial},
15\\
\hspace*{0.333em}\hspace*{0.333em}\hspace*{0.333em}\hspace*{0.333em}\href{./3-the-state-of-the-aircraft.html\#palt}{pressure},
13, 16, 22, 32\\
\href{/3-the-state-of-the-aircraft.html\#altx}{ALTX}, 23\\
angle\\
\hspace*{0.333em}\hspace*{0.333em}\href{./8-radiation-variables.html\#solaz}{solar
azimuth}, 82\\
\hspace*{0.333em}\hspace*{0.333em}\href{./8-radiation-variables.html\#solde}{solar
declination}, 81\\
\hspace*{0.333em}\hspace*{0.333em}\href{./8-radiation-variables.html\#wolel}{solar
elevation}, 81\\
\hspace*{0.333em}\hspace*{0.333em}\href{./8-radiation-variables.html\#solze}{solar
zenith}, 82\\
angles\\
\hspace*{0.333em}\hspace*{0.333em}\href{./8-radiation-variables.html\#solar-angles}{solar},
81\\
\href{./5-cloud-physics-variables.html\#CRPC}{APCAS}, 62\\
\href{./3-the-state-of-the-aircraft.html\#hgme-159}{APN-159}, 21\\
\href{./5-cloud-physics-variables.html\#CRPC}{AS100}, 62\\
\href{./5-cloud-physics-variables.html\#CRPC}{AS200}, 62\\
\href{./4-the-state-of-the-atmosphere.html\#AT=ITR}{AT\_ITR}, 37\\
\href{./10-obsolete-variables.html\#atc}{ATC}, 87\\
\href{./10-obsolete-variables.html\#atkp}{ATKP}, 88\\
\href{./10-obsolete-variables.html\#atrf}{ATRF}, 88\\
\href{./4-the-state-of-the-atmosphere.html\#attack}{ATTACK}, 54\\
\href{./4-the-state-of-the-atmosphere.html\#attack}{attack, angle of},
53\\
\href{./4-the-state-of-the-atmosphere.html\#ambient-t}{ATX}, 33, 41,
44--46, 48, 50, 53, 75, 91\\
\href{./4-the-state-of-the-atmosphere.html\#ambient-t}{ATx}, 33, 53\\
\href{./4-the-state-of-the-atmosphere.html\#ambient-t}{ATxD}, 33\\
\href{./4-the-state-of-the-atmosphere.html\#ambient-t}{ATxH}, 33\\
\href{./5-cloud-physics-variables.html\#CRPC}{AUHSAS}, 62\\
\href{./4-the-state-of-the-atmosphere.html\#psx}{avionics}, 29\\
\href{./1-introduction.html\#constants-and-symbols}{Avogadro constant},
4, 42\\
\href{./6-air-chemistry-measurements.html\#awas-cims-qcls-toga}{AWAS},
72\\
\href{./3-the-state-of-the-aircraft.html\#wp3}{baro-inertial loop: see},
12\\
\href{./3-the-state-of-the-aircraft.html\#wp3}{baro-loop: see}, 12\\
\href{./2-general-information-about-data-files.html\#base-time}{base
time}, 7\\
\href{./4-the-state-of-the-atmosphere.html\#bdifr}{BDIF}, 53\\
\href{./4-the-state-of-the-atmosphere.html\#bdifr}{BDIFR}, 53, 55\\
\href{./10-obsolete-variables.html\#vanes}{BFIXx}, 87\\
\href{./3-the-state-of-the-aircraft.html\#blata}{BLATA}, 16\\
\href{./3-the-state-of-the-aircraft.html\#blona}{BLONA}, 16\\
\href{./3-the-state-of-the-aircraft.html\#bnorma}{BNORMA}, 16\\
\href{./3-the-state-of-the-aircraft.html\#thetae}{Bolton}, 47\\
Boltzmann constant, see\\
\hspace*{0.333em}\hspace*{0.333em}\href{./1-introduction.html\#constants-and-symbols}{Constants
and Symbols}, 4\\
\href{./3-the-state-of-the-aircraft.html\#bpitchr}{BPITCHR}, 16\\
\href{./3-the-state-of-the-aircraft.html\#brollr}{BROLLR}, 16\\
\href{./4-the-state-of-the-atmosphere.html\#qcx}{Bulletin 21}, 31\\
\href{./4-the-state-of-the-atmosphere.html\#wind}{Bulletin 23}, 53\\
\href{./5-cloud-physics-variables.html\#plwc}{Bulletin 24}, 93\\
\href{./10-obsolete-variables.html\#swtc}{Bulletin 25}, 92\\
\href{./4-the-state-of-the-atmosphere.html\#humidity}{Bulletin 9}, 2\\
\href{./3-the-state-of-the-aircraft.html\#byawr}{BYAWR}, 16\\
\href{./5-cloud-physics-variables.html\#c1dc-c1dp}{C1DC}, 66
\href{./5-cloud-physics-variables.html\#c1dc-c1dp}{C1DP}, 66\\
\href{./5-cloud-physics-variables.html\#size-distribution}{C200X}, 62\\
\href{./5-cloud-physics-variables.html\#size-distribution}{C200Y}, 62\\
\href{./5-cloud-physics-variables.html\#size-distribution}{C260X}, 62\\
calibration, numerous references throughout, 29\\
\hspace*{0.333em}\hspace*{0.333em}\href{./1-general-information-about-data-files.html\#background-information}{coefficients},
9\\
\hspace*{0.333em}\hspace*{0.333em}\href{./6-air-chemistry-measurements.html\#comr-al}{gas},
68\\
\href{./4-the-state-of-the-atmosphere.html\#p-special}{CAVP\_x}, 32\\
\href{./5-cloud-physics-variables.html\#size-distribution}{CCDP}, 62\\
\href{./10-obsolete-variables.html\#loranc}{CCEP}, 85\\
\href{./3-the-state-of-the-aircraft.html\#wp3}{centrifugal
acceleration}, 14\\
\href{./5-cloud-physics-variables.html\#size-distribution}{CF300}, 62\\
\href{./10-obsolete-variables.html\#loranc}{CFSEC}, 85\\
\href{./5-cloud-physics-variables.html\#size-distribution}{CFSSP}, 62\\
\href{./10-obsolete-variables.html\#loranc}{CGS}, 85\\
\href{./6-air-chemistry-measurements.html\#co2-pic}{CH4\_PICx}, 68\\
\href{./4-the-state-of-the-atmosphere.html\#dew-point}{chilled-mirror},
44\\
\href{./6-air-chemistry-measurements.html\#awas-cims-qcls-toga}{CIMS},
72\\
\href{./10-obsolete-variables.html\#loranc\%7C}{CLAT}, 85\\
\href{./10-obsolete-variables.html\#loranc}{CLON}, 85\\
\href{./10-obsolete-variables.html\#co-vars}{CMODE}, 92\\
\href{./7-aerosol-particle-measurements.html\#condensation-nucleus-counter}{CN
counter}, 73\\
\hspace*{0.333em}\hspace*{0.333em}3760A, 73\\
\hspace*{0.333em}\hspace*{0.333em}3786, 73\\
\hspace*{0.333em}\hspace*{0.333em}\href{./7-aerosol-particle-measurements.html\#concn}{coincidence
in}, 74\\
\hspace*{0.333em}\hspace*{0.333em}\href{./7-aerosol-particle-measurements.html\#concn}{dead
time}, 74\\
\hspace*{0.333em}\hspace*{0.333em}\href{./7-aerosol-particle-measurements.html\#fcnc}{flow
rate}, 73\\
\hspace*{0.333em}\hspace*{0.333em}\href{./7-aerosol-particle-measurements.html\#xicnc}{side
flow}, 74\\
\href{./7-aerosol-particle-measurements.html\#cntemp}{CNTEMP}, 75\\
\href{./7-aerosol-particle-measurements.html\#cnts}{CNTS}, 75\\
\href{./6-air-chemistry-measurements.html\#comr-al}{CO}, 92
\href{./6-air-chemistry-measurements.html\#co2-pic}{CO2\_PICx}, 68\\
\href{./10-obsolete-variables.html\#co-vars}{COCAL}, 92\\
\href{./10-obsolete-variables.html\#co-vars}{COCOR}, 92\\
coefficients\\
\hspace*{0.333em}\hspace*{0.333em}\href{./1-general-information-about-data-files.html\#background-information}{calibration},
29\\
\hspace*{0.333em}\hspace*{0.333em}\href{./4-the-state-of-the-atmosphere.html\#akrd}{sensitivity},
30\\
compression\\
\hspace*{0.333em}\hspace*{0.333em}\href{./4-the-state-of-the-atmosphere.html\#recovery-t}{adiabatic},
33\\
\href{./6-air-chemistry-measurements.html\#comr-al}{COMR\_AL}, 68
\href{./5-cloud-physics-variables.html\#conc2d}{CONC1DC}, 66\\
\href{./5-cloud-physics-variables.html\#conc2d}{CONC1DP}, 66\\
\href{./5-cloud-physics-variables.html\#concentration}{CONC3}, 63\\
\href{./5-cloud-physics-variables.html\#concentration}{CONC6}, 63\\
\href{./5-cloud-physics-variables.html\#concentration}{CONCD}, 63\\
concentration\\
\hspace*{0.333em}\hspace*{0.333em}\href{./5-cloud-physics-variables.html\#conc2d}{2D},
66\\
\hspace*{0.333em}\hspace*{0.333em}\href{./7-aerosol-particle-measurements.html\#concu-concp}{aerosol},
73\\
\hspace*{0.333em}\hspace*{0.333em}\href{./7-aerosol-particle-measurements.html\#concn}{ambient},
73\\
\hspace*{0.333em}\hspace*{0.333em}\href{./6-air-chemistry-measurements.html\#comr-al}{calibration
gas}, 68\\
\hspace*{0.333em}\hspace*{0.333em}\href{./2-general-information-about-data-files.html\#units-and-abbreviations}{chemical
species}, 5, 6\\
\hspace*{0.333em}\hspace*{0.333em}\href{./7-aerosol-particle-measurements.html\#concn}{CN},
74\\
\hspace*{0.333em}\hspace*{0.333em}\href{./5-cloud-physics-variables.html\#concentration}{droplet},
5\\
\hspace*{0.333em}\hspace*{0.333em}\href{./5-cloud-physics-variables.html\#concentration}{FSSP},
63\\
\hspace*{0.333em}\hspace*{0.333em}\href{./5-cloud-physics-variables.html\#concentration}{hydrometeor},
62\\
\hspace*{0.333em}\hspace*{0.333em}hydrometeor,
\href{./5-cloud-physics-variables.html\#size-distribution}{size
distribution}, 62\\
\hspace*{0.333em}\hspace*{0.333em}\href{./5-cloud-physics-variables.html\#size-distribution}{particle},
77\\
\hspace*{0.333em}\hspace*{0.333em}\href{./7-aerosol-particle-measurements.html\#concu-concp}{PCAS},
94\\
\hspace*{0.333em}\hspace*{0.333em}\href{./7-aerosol-particle-measurements.html\#condensation-nucleus-counter}{ultrafine
particles}, 73\\
\hspace*{0.333em}\hspace*{0.333em}\href{./4-the-state-of-the-atmosphere.html\#vxcel-corr}{water
vapor}, 42\\
\href{./5-cloud-physics-variables.html\#concentration}{CONCF}, 63\\
\href{./4-the-state-of-the-atmosphere.html\#uvh-n}{CONCH\_UVH}, 43\\
\href{./7-aerosol-particle-measurements.html\#concn}{CONCN}, 75\\
\href{./7-aerosol-particle-measurements.html\#concu-concp}{CONCP}, 76\\
\href{./7-aerosol-particle-measurements.html\#concu-concp}{CONCU}, 76\\
\href{./5-cloud-physics-variables.html\#concentration}{CONCU100}, 76\\
\href{./5-cloud-physics-variables.html\#concentration}{CONCU500}, 76\\
\href{./4-the-state-of-the-atmosphere.html\#vcsel-corr}{CONCV\_VXL},
42\\
\href{./5-cloud-physics-variables.html\#concentration}{CONCX}, 63\\
\href{./5-cloud-physics-variables.html\#concentration}{CONCY}, 63\\
\href{./7-aerosol-particle-measurements.html\#condensation-nucleus-counter}{condensation
nucleus counter}, see CN counter\\
conductivity\\
\hspace*{0.333em}\hspace*{0.333em}\href{./5-cloud-physics-variables.html\#plwcc}{thermal},
59\\
\href{./4-the-state-of-the-atmosphere.html\#ATX}{conservation of
energy}, 34\\
constants, see symbols\\
\hspace*{0.333em}\hspace*{0.333em}\href{./1-introduction.html\#constants-and-symbols}{table},
4\\
\href{./6-air-chemistry-measurements.html\#coraw-al}{CORAW\_AL}, 68\\
\href{./3-the-state-of-the-aircraft.html\#wp3}{Coriolis acceleration},
14\\
correction\\
\hspace*{0.333em}\hspace*{0.333em}\href{./4-the-state-of-the-atmosphere.html\#qcx}{dynamic
pressure}, 31\\
\hspace*{0.333em}\hspace*{0.333em}\href{./4-the-state-of-the-atmosphere.html\#ambient-t}{moist
air}, 33\\
\hspace*{0.333em}\hspace*{0.333em}\href{./4-the-state-of-the-atmosphere.html\#psx}{pressure},
30\\
\hspace*{0.333em}\hspace*{0.333em}vertical wind\\
\hspace*{0.333em}\hspace*{0.333em}\hspace*{0.333em}\hspace*{0.333em}\href{./4-the-state-of-the-atmosphere.html\#wind}{rotation
in pitch}, 19\\
count rate\\
\hspace*{0.333em}\hspace*{0.333em}\href{./5-cloud-physics-variables.html\#a1dc-a1dp}{2D},
66\\
\href{./10-obsolete-variables.html\#co-vars}{COZRO}, 92\\
\href{./5-cloud-physics-variables.html\#size-distribution}{CPCAS}, 62\\
\href{./5-cloud-physics-variables.html\#a1dc-a1dp}{CPS}=counts per
second, 68, 70\\
\href{./10-obsolete-variables.html\#cryo-hygro}{CRHP}, 89\\
\href{./5-cloud-physics-variables.html\#size-distribution}{CS100}, 62\\
\href{./5-cloud-physics-variables.html\#size-distribution}{CS200}, 62\\
\href{./10-obsolete-variables.html\#loranc}{CSEC}, 85\\
\href{./5-cloud-physics-variables.html\#size-distribution}{CUHSAS}, 62\\
\href{./4-the-state-of-the-atmosphere.html\#dvalue}{d-value}, 32\\
data\\
\hspace*{0.333em}\hspace*{0.333em}\href{./1-introduction.html}{acquisition},
1\\
\hspace*{0.333em}\hspace*{0.333em}\href{./1-introduction.html}{display},
1\\
\hspace*{0.333em}\hspace*{0.333em}\href{./1-introduction.html}{processing},
1\\
\hspace*{0.333em}\hspace*{0.333em}\href{./1-introduction.html}{sample
rate}, 1\\
\href{./1-introduction.html}{data rates}, 8\\
\href{./4-the-state-of-the-atmosphere.html\#thetae}{Davies-Jones}, 46\\
\href{./2-general-information-about-data-files.html\#mdy\%7C}{DAY}, 7\\
\href{./5-cloud-physics-variables.html\#dbar2d}{DBAR1DC}, 67\\
\href{./5-cloud-physics-variables.html\#dbar2d}{DBAR1DP}, 67\\
\href{./5-cloud-physics-variables.html\#mean-diameter}{DBAR3}, 63\\
\href{./5-cloud-physics-variables.html\#mean-diameter}{DBAR6}, 63\\
\href{./5-cloud-physics-variables.html\#mean-diameter}{DBARD}, 63\\
\href{./5-cloud-physics-variables.html\#mean-diameter}{DBARF}, 63\\
\href{./5-cloud-physics-variables.html\#mean-diameter}{DBARP}, 63\\
\href{./5-cloud-physics-variables.html\#mean-diameter}{DBARU}, 63\\
\href{./5-cloud-physics-variables.html\#mean-diameter}{DBARX}, 63\\
\href{./5-cloud-physics-variables.html\#mean-diameter}{DBARY}, 63\\
\href{./5-cloud-physics-variables.html\#DBZ}{dBz}, 64\\
\href{./5-cloud-physics-variables.html\#dbz2d}{DBZ1DC}, 67\\
\href{./5-cloud-physics-variables.html\#dbz2d}{DBZ1DP}, 67\\
\href{./5-cloud-physics-variables.html\#DBZ}{DBZ6}, 64\\
\href{./5-cloud-physics-variables.html\#DBZ}{DBZD}, 64\\
\href{./5-cloud-physics-variables.html\#DBZ}{DBZF}, 64\\
\href{./5-cloud-physics-variables.html\#DBZ}{DBZX}, 64\\
\href{./5-cloud-physics-variables.html\#DBZ}{DBZY}, 64\\
\href{./7-aerosol-particle-measurements.html\#concn}{dead time}, 66\\
\hspace*{0.333em}\hspace*{0.333em}\href{./10-obsolete-variables.html\#fstrob}{FSSP},
94\\
defect\\
\hspace*{0.333em}\hspace*{0.333em}\href{4-the-state-of-the-atmosphere.html$psx}{static},
30\\
\href{./3-the-state-of-the-aircraft.html\#dei-dni}{DEI}, 15\\
density\\
\hspace*{0.333em}\hspace*{0.333em}\href{4-the-state-of-the-atmosphere.html$rho}{water
vapor}, 44\\
\href{4-the-state-of-the-atmosphere.html$derived-thermodynamic-variables}{derived
variables}, 46\\
dew point, 38\\
\hspace*{0.333em}\hspace*{0.333em}\href{./4-the-state-of-the-atmosphere.html\#dewpt-corrected}{corrected},
38, 40\\
diameter\\
\hspace*{0.333em}\hspace*{0.333em}\href{./5-cloud-physics-variables.html\#effective-radius}{effective},
64\\
\hspace*{0.333em}\hspace*{0.333em}\href{./5-cloud-physics-variables.html\#PSD-LWC}{equivalent},
64\\
\hspace*{0.333em}\hspace*{0.333em}\href{./5-cloud-physics-variables.html\#mean-diameter}{mean,
1D probes}, 63\\
\hspace*{0.333em}\hspace*{0.333em}\href{./5-cloud-physics-variables.html\#dbar2d}{mean,
2D probes}, 67\\
\href{./2-general-information-about-data-files.html\#dimensions-in-equations}{dimensions
in equations}, 64\\
\href{./5-cloud-physics-variables.html\#disp2d}{DISP1DC}, 67\\
\href{./5-cloud-physics-variables.html\#disp2d}{DISP1DP}, 67\\
\href{./5-cloud-physics-variables.html\#dispersion}{DISP3}, 63\\
\href{./5-cloud-physics-variables.html\#dispersion}{DISP6}, 63\\
\href{./5-cloud-physics-variables.html\#dispersion}{DISPD}, 63\\
dispersion, 67\\
\hspace*{0.333em}\hspace*{0.333em}\href{./5-cloud-physics-variables.html\#dispersion}{1D
probes}, 63\\
\href{./5-cloud-physics-variables.html\#dispersion}{DISPF}, 63\\
\href{./5-cloud-physics-variables.html\#dispersion}{DISPP}, 63\\
\href{./5-cloud-physics-variables.html\#dispersion}{DISPU}, 63\\
\href{./5-cloud-physics-variables.html\#dispersion}{DISPX}, 63\\
\href{./5-cloud-physics-variables.html\#dispersion}{DISPY}, 63\\
\href{./3-the-state-of-the-aircraft.html\#dei-dni}{DNI}, 15\\
\href{./4-the-state-of-the-atmosphere.html\#dp-cr2}{DP\_CR2C}, 42\\
\href{./4-the-state-of-the-atmosphere.html\#vcsel-dp}{DP\_VXL}, 41\\
\href{./4-the-state-of-the-atmosphere.html\#dew-point}{DP\_x}, 38\\
\href{./10-obsolete-variables.html\#cryo-hygro}{DPCRC}, 89\\
\href{./4-the-state-of-the-atmosphere.html\#dew-point}{DPx}, 41\\
\href{./4-the-state-of-the-atmosphere.html\#dewpt-corrected}{DPXC}, 91\\
\href{./4-the-state-of-the-atmosphere.html\#dewpt-corrected}{DPxC}, 40\\
\href{./4-the-state-of-the-atmosphere.html\#special-use-remote}{dropsonde},
57\\
\href{./5-cloud-physics-variables.html\#dt1dc}{DT1DC}, 66\\
\href{./4-the-state-of-the-atmosphere.html\#dvalue}{DVALUE}, 32\\
\href{./4-the-state-of-the-atmosphere.html\#qcx}{dynamic pressure}, 52\\
\href{./10-obsolete-variables.html\#edpc}{EDPC}, 88\\
\href{./4-the-state-of-the-atmosphere.html\#humidity}{enhancement
factor}, 44\\
equation\\
\hspace*{0.333em}\hspace*{0.333em}\href{./4-the-state-of-the-atmosphere.html\#p-special}{thickness},
32\\
equations\\
\hspace*{0.333em}\hspace*{0.333em}\href{./2-general-information-about-data-files.html\#dimensions-in-equations}{dimensionless},
9\\
\hspace*{0.333em}\hspace*{0.333em}\href{./2-general-information-about-data-files.html\#dimensions-in-equations}{scale
factors}, 9\\
\href{./4-the-state-of-the-atmosphere.html\#ewx}{EW\_UVH}, 43\\
\href{./4-the-state-of-the-atmosphere.html\#ewx}{EWX}, 43--46\\
\href{./4-the-state-of-the-atmosphere.html\#ewx}{EWx}, 43\\
\href{./10-obsolete-variables.html\#fact}{FACT}, 94\\
\href{./10-obsolete-variables.html\#fbmfr}{FBMFR}, 94\\
\href{./7-aerosol-particle-measurements.html\#fcnc}{FCN}, 73\\
\href{./7-aerosol-particle-measurements.html\#fcnc}{FCN}, 73\\
\href{./7-aerosol-particle-measurements.html\#fcnc}{FCNC}, 75\\
filter\\
~~\href{./3-the-state-of-the-aircraft.html\#vewc-vnsc}{Butterworth},
25\\
\hspace*{0.333em}\hspace*{0.333em}\href{./3-the-state-of-the-aircraft.html\#compFilter}{complementary}
(for wind), 24\\
\hspace*{0.333em}\hspace*{0.333em}\href{./2-general-information-about-data-files.html\#elecFilter}{electronic},
8\\
~~\href{./2-general-information-about-data-files.html\#elecFilter}{FIR},
8\\
\href{./3-the-state-of-the-aircraft.html\#vewc-vnsc}{fit matrix}, 27\\
\href{./4-the-state-of-the-atmosphere.html\#true-airspeed}{flight
speed}, see true airspeed\\
flow\\
\hspace*{0.333em}\hspace*{0.333em}\href{./2-general-information-about-data-files.html\#units-and-abbreviations}{conversions},
5\\
\hspace*{0.333em}\hspace*{0.333em}\href{./7-aerosol-particle-measurements.html\#pflw}{PCASP},
76\\
\hspace*{0.333em}\hspace*{0.333em}\href{./2-general-information-about-data-files.html\#units-and-abbreviations}{SLPM},
5\\
\hspace*{0.333em}\hspace*{0.333em}\href{./2-general-information-about-data-files.html\#units-and-abbreviations}{volumetric},
5\\
\href{./4-the-state-of-the-atmosphere.html\#psx}{flow distortion}, 29\\
\href{./2-general-information-about-data-files.html\#units-and-abbreviations}{flow
rates}, 5\\
flux\\
\hspace*{0.333em}\hspace*{0.333em}\href{8-radiation-variables.html\#harp}{actinic},
80\\
\href{./6-air-chemistry-measurements.html\#fo3-acd}{FO3\_ACD}, 69\\
\href{./6-air-chemistry-measurements.html\#fo3-acd}{FO3\_CL}, 69\\
\href{./6-air-chemistry-measurements.html\#fo3-acd}{FO3\_x}, 69\\
\href{./2-general-information-about-data-files.html\#bpitchr}{foot}, 6\\
\href{./4-the-state-of-the-atmosphere.html\#mirror-cr2}{FP\_CR2}, 42\\
\href{./10-obsolete-variables.html\#cryo-hygro}{FPCRC}, 89\\
\href{./5-cloud-physics-variables.html\#fssp-range}{FRANGE}, 65\\
\href{./10-obsolete-variables.html\#freset}{FRESET}, 93\\
\href{./5-cloud-physics-variables.html\#fssp-range}{FRNG}, 65\\
\href{./3-the-state-of-the-aircraft.html\#humidity}{frost point}, 89\\
\href{./10-obsolete-variables.html\#freset}{FRST}, 93\\
\href{./5-cloud-physics-variables.html\#VariableNames1DProbes}{FSSP-100}\\
\hspace*{0.333em}\hspace*{0.333em}\href{./10-obsolete-variables.html\#fact}{activity},
94\\
\hspace*{0.333em}\hspace*{0.333em}\href{./10-obsolete-variables.html\#fbmfr}{beam
fraction}, 94\\
\hspace*{0.333em}\hspace*{0.333em}\href{./5-cloud-physics-variables.html\#concentration}{concentration},
63\\
\hspace*{0.333em}\hspace*{0.333em}\href{./10-obsolete-variables.html\#fstrob}{dead
time}, 93, 94\\
\hspace*{0.333em}\hspace*{0.333em}\href{./5-cloud-physics-variables.html\#dispersion}{dispersion},
63\\
\hspace*{0.333em}\hspace*{0.333em}\href{./10-obsolete-variables.html\#freset}{fast
resets}, 93\\
\hspace*{0.333em}\hspace*{0.333em}\href{./5-cloud-physics-variables.html\#PSD-LWC}{liquid
water content}, 63\\
\hspace*{0.333em}\hspace*{0.333em}\href{./5-cloud-physics-variables.html\#mean-diameter}{mean
diameter}, 63\\
\hspace*{0.333em}\hspace*{0.333em}\href{./5-cloud-physics-variables.html\#fssp-range}{range}
65\\
\hspace*{0.333em}\hspace*{0.333em}\href{./5-cloud-physics-variables.html\#size-distribution}{size
distribution}, 62\\
\hspace*{0.333em}\hspace*{0.333em}\href{./10-obsolete-variables.html\#fstrob}{total
strobes}, 93\\
\href{./5-cloud-physics-variables.html\#VariableNames1DProbes}{FSSP-300},
63\\
\href{./10-obsolete-variables.html\#fstrob}{FSTB}, 93\\
\href{./10-obsolete-variables.html\#fstrob}{FSTROB}, 93\\
\href{./3-the-state-of-the-aircraft.html\#fxazim}{FXAZIM}, 15\\
\href{./3-the-state-of-the-aircraft.html\#fxazim}{FXDIST}, 15\\
\href{./3-the-state-of-the-aircraft.html\#ggalt}{GALT\_A}, 19\\
\href{./1-introduction.html\#constants-and-symbols}{gas constant}, 4\\
\hspace*{0.333em}\hspace*{0.333em}\href{./4-the-state-of-the-atmosphere.html\#moist-air}{dry
air}, 4, 46\\
\hspace*{0.333em}\hspace*{0.333em}\href{./4-the-state-of-the-atmosphere.html\#moist-air}{moist
air}, 33, 34\\
\hspace*{0.333em}\hspace*{0.333em}\href{./1-introduction.html\#constants-and-symbols}{universal},
4\\
\hspace*{0.333em}\hspace*{0.333em}\href{./1-introduction.html\#constants-and-symbols}{water
vapor}, 4\\
\href{./1-introduction\#background-information}{GENPRO}, 2\\
\href{./3-the-state-of-the-aircraft.html\#geopth}{GEOPHT}, 19\\
\href{./3-the-state-of-the-aircraft.html\#ggalt}{GGALT}, 19\\
\href{./3-the-state-of-the-aircraft.html\#altx}{GGALTC}, 23\\
\href{./3-the-state-of-the-aircraft.html\#ggeoidht}{GGEOIDHT}, 20\\
\href{./3-the-state-of-the-aircraft.html\#gghwgs}{GGHWGS}, 19\\
\href{./3-the-state-of-the-aircraft.html\#gglat}{GGLAT}, 28\\
\href{./3-the-state-of-the-aircraft.html\#gglon}{GGLON}, 28\\
\href{./3-the-state-of-the-aircraft.html\#ggnsat}{GGNSAT}, 20\\
\href{./3-the-state-of-the-aircraft.html\#ggqual}{GGQUAL}, 20\\
\href{./3-the-state-of-the-aircraft.html\#ggspd}{GGSPD}, 18\\
\href{./3-the-state-of-the-aircraft.html\#ggstatus\%7C}{GGSTATUS}, 20\\
\href{./3-the-state-of-the-aircraft.html\#ggtrk}{GGTRK}, 20\\
\href{./3-the-state-of-the-aircraft.html\#ggvew}{GGVEW}, 26\\
\href{./3-the-state-of-the-aircraft.html\#ggvns}{GGVNS}, 26\\
\href{./3-the-state-of-the-aircraft.html\#ggvspd}{GGVSPD}, 19\\
\href{./3-the-state-of-the-aircraft.html\#special-use-remote}{GISMOS},
57\\
\href{./3-the-state-of-the-aircraft.html\#gglat}{GLAT}, 18\\
\href{./3-the-state-of-the-aircraft.html\#global-positioning-systems}{global
positioning system}, see GPS\\
\href{./3-the-state-of-the-aircraft.html\#gglon}{GLON}, 18\\
\href{./3-the-state-of-the-aircraft.html\#gglon}{GLON}, 18\\
\href{./3-the-state-of-the-aircraft.html\#gmode}{GMODE}, 20\\
\href{./7-aerosol-particle-measurements.html\#special-aerosol}{GNI},
77\\
\href{./3-the-state-of-the-aircraft.html\#global-positioning-systems}{GPS},
27\\
\hspace*{0.333em}\hspace*{0.333em}\href{./3-the-state-of-the-aircraft.html\#gglat}{aircraft
avionics unit}, 17\\
\hspace*{0.333em}\hspace*{0.333em}\href{./3-the-state-of-the-aircraft.html\#gglat}{Garmin},
17\\
\hspace*{0.333em}\hspace*{0.333em}\href{./3-the-state-of-the-aircraft.html\#gglat}{NovAtel},
16\\
\hspace*{0.333em}\hspace*{0.333em}\href{./3-the-state-of-the-aircraft.html\#gglat}{Trimble
TANS-III}, 17\\
\href{./3-the-state-of-the-aircraft.html\#global-positioning-systems}{GPS
receivers}, 16\\
gravity, see Eqn. (\href{./1-introduction.html\#eq:gsublambda}{1.1})\\
\hspace*{0.333em}\hspace*{0.333em}\href{./3-the-state-of-the-aircraft.html\#palt}{standard},
16\\
\href{./3-the-state-of-the-aircraft.html\#ggspd}{ground speed}, 15\\
\href{./3-the-state-of-the-aircraft.html\#gsf}{GSF}, 86\\
\href{./3-the-state-of-the-aircraft.html\#ggspd}{GSF\_G}, 18\\
\href{./3-the-state-of-the-aircraft.html\#ggstatus\%7C}{GSTAT}, 20\\
\href{./3-the-state-of-the-aircraft.html\#ggstatus}{GSTAT}, 20\\
\href{./3-the-state-of-the-aircraft.html\#ggstatus\%7C}{GSTAT\_G}, 20\\
\href{./4-the-state-of-the-atmosphere.html\#variable-names}{gust probe}
~~\href{./4-the-state-of-the-atmosphere.html\#relative-wind}{858AJ},
52\\
\hspace*{0.333em}\hspace*{0.333em}\href{./4-the-state-of-the-atmosphere.html\#relative-wind}{radome},
52, 53\\
\hspace*{0.333em}\hspace*{0.333em}\href{./4-the-state-of-the-atmosphere.html\#relative-wind}{Rosemount
858AJ}, 53\\
\href{./3-the-state-of-the-aircraft.html\#ggvspd}{GVZI}, 19\\
\href{./8-radiation-variables.html\#harp}{HARP},see HIAPER Atmospheric
Radiation Package\\
heading\\
\hspace*{0.333em}\hspace*{0.333em}\href{./3-the-state-of-the-aircraft.html\#thdg}{magnetic},
11\\
\hspace*{0.333em}\hspace*{0.333em}\href{./3-the-state-of-the-aircraft.html\#thdg}{true},
11\\
\href{./8-radiation-variables.html\#rstx}{Heimann radiometer}, 78\\
\href{./3-the-state-of-the-aircraft.html\#hgm}{HGM}, 21\\
\href{./3-the-state-of-the-aircraft.html\#hgme-159}{HGM}, 21\\
\href{./3-the-state-of-the-aircraft.html\#hgm-232}{HGM232}, 21\\
\href{./3-the-state-of-the-aircraft.html\#hgme-159}{HGME}, 32\\
\href{./3-the-state-of-the-aircraft.html\#hi3}{HI3}, 21\\
\href{./7-aerosol-particle-measurements.html\#harp}{HIAPER Atmospheric
Radiation Package}, 80\\
\href{./2-general-information-about-data-files.html\#hms\%7C}{HOUR}, 7\\
\href{./5-cloud-physics-variables.html\#sensors-1-D-probes}{housekeeping
variables}, 61\\
\href{./4-the-state-of-the-atmosphere.html\#humidity}{humidity}, 38\\
\hspace*{0.333em}\hspace*{0.333em}\href{./4-the-state-of-the-atmosphere.html\#rho}{absolute},
44\\
\hspace*{0.333em}\hspace*{0.333em}\href{./4-the-state-of-the-atmosphere.html\#rhumw}{relative},
44\\
\hspace*{0.333em}\hspace*{0.333em}\href{./4-the-state-of-the-atmosphere.html\#rhumi}{relative
to ice}, 44\\
\hspace*{0.333em}\hspace*{0.333em}\href{./4-the-state-of-the-atmosphere.html\#sphum}{specific},
45, 53\\
\href{./5-cloud-physics-variables.html\#VariableNames1DProbes}{hydrometeor
detector}, 60\\
hydrometeor probes\\
\hspace*{0.333em}\hspace*{0.333em}\href{./5-cloud-physics-variables.html\#VariableNames1DProbes}{table
of}, 62\\
\href{./5-cloud-physics-variables.html\#size-distribution}{hydrometeor
spectrometer}, 63\\
\href{./4-the-state-of-the-atmosphere.html\#dew-point}{hygrometer}, 38\\
\hspace*{0.333em}\hspace*{0.333em}\href{./4-the-state-of-the-atmosphere.html\#dew-point}{chilled-mirror},
44\\
\hspace*{0.333em}\hspace*{0.333em}\href{./4-the-state-of-the-atmosphere.html\#mirror-cr2}{CR2},
42\\
\hspace*{0.333em}\hspace*{0.333em}\href{10-obsolete-variables.html\#cryo-hygro}{cryogenic},
45, 89\\
\hspace*{0.333em}\hspace*{0.333em}\href{./4-the-state-of-the-atmosphere.html\#dew-point}{dew
point}, 38, 45\\
\hspace*{0.333em}\hspace*{0.333em}\href{./4-the-state-of-the-atmosphere.html\#dew-point}{housing},
38\\
\hspace*{0.333em}\hspace*{0.333em}\href{10-obsolete-variables.html\#vla}{Lyman-alpha},
45, 90\\
\hspace*{0.333em}\hspace*{0.333em}\href{./4-the-state-of-the-atmosphere.html\#MR}{tunable
diode laser}, 40, 45\\
\hspace*{0.333em}\hspace*{0.333em}\href{./4-the-state-of-the-atmosphere.html\#uvh-n}{UV},
42, 45, 90\\
\hspace*{0.333em}\hspace*{0.333em}\href{./4-the-state-of-the-atmosphere.html\#vcsel-dp}{VCSEL},
37, 40, 41, 45\\
\href{./5-cloud-physics-variables.html\#PSD-LWC}{ice water content},
64\\
impactor\\
\hspace*{0.333em}\hspace*{0.333em}\href{./7-aerosol-particle-measurements.html\#special-aerosol}{giant
nucleus}, 77\\
\href{./10-obsolete-variables\#ias}{indicated airspeed}, 88\\
\href{./3-the-state-of-the-aircraft.html\#inertial-reference-systems}{inertial
navigation system}, see INS\\
\href{./10-obsolete-variables.html\#mdy}{inertial reference unit}, see
IRU\\
\href{./3-the-state-of-the-aircraft.html\#inertial-reference-systems}{INS},
10\\
\hspace*{0.333em}\hspace*{0.333em}alignment, 10\\
\hspace*{0.333em}\hspace*{0.333em}C-MIGITS, 10\\
\hspace*{0.333em}\hspace*{0.333em}C130, 10\\
\hspace*{0.333em}\hspace*{0.333em}GV, 10\\
\hspace*{0.333em}\hspace*{0.333em}Honeywell, 13\\
\hspace*{0.333em}\hspace*{0.333em}Honeywell Laseref, 10\\
\hspace*{0.333em}\hspace*{0.333em}Litton LTN-51, 12\\
\hspace*{0.333em}\hspace*{0.333em}measurements not in normal data files,
16\\
\hspace*{0.333em}\hspace*{0.333em}specifications, 11\\
instrument descriptions\\
\hspace*{0.333em}\hspace*{0.333em}EOL web site, 29\\
instruments\\
\hspace*{0.333em}\hspace*{0.333em}\href{./8-radiation-variables.html}{radiation},
78\\
\hspace*{0.333em}\hspace*{0.333em}\href{./6-air-chemistry-measurements.html\#awas-cims-qcls-toga}{user},
1, 71\\
\href{./3-the-state-of-the-aircraft.html\#palt}{International Standard
Atmosphere}, 22\\
\hspace*{0.333em}\hspace*{0.333em}\href{./3-the-state-of-the-aircraft.html\#palt}{lapse
rate}, 23\\
\hspace*{0.333em}\hspace*{0.333em}\href{./3-the-state-of-the-aircraft.html\#palt}{tropopause},
23\\
\href{./2-general-information-about-data-files.html\#synchronization-of-measurements}{interpolation},
8\\
\href{./8-radiation-measurements.html\#measurements-of-irradiance-and-radiometric-temperature}{irradiance},
78\\
\hspace*{0.333em}\hspace*{0.333em}long-wave, 79\\
\hspace*{0.333em}\hspace*{0.333em}visible, 80\\
\href{./3-the-state-of-the-aircraft.html\#inertial-reference-systems}{IRU},
11\\
\hspace*{0.333em}\hspace*{0.333em}\href{./10-obsolete-variables.html\#ltn51}{Litton
LTN-51}, 86\\
\hspace*{0.333em}\hspace*{0.333em}\href{./3-the-state-of-the-aircraft\#special-use-irs}{raw
variables}, 16\\
\href{./10-obsolete-variables.html\#irx}{IRx}, 91\\
\href{./8-radiation-variables.html\#irxc}{IRxC}, 91\\
\href{./8-radiation-variables.html\#irxht}{IRxHT}, 79\\
\href{./7-aerosol-particle-measurements.html\#trstx}{IRxV}, 79\\
\href{./3-the-state-of-the-aircraft.html\#palt}{ISA}, see International
Standard Atmosphere\\
\href{./10-obsolete-variables.html\#jwlwc}{Johnson-Williams sensor},
88\\
\href{./5-cloud-physics-variables\#plwcc}{King probe}, 58\\
\hspace*{0.333em}\hspace*{0.333em}element dimensions, 59\\
\hspace*{0.333em}\hspace*{0.333em}power dissipated, 59\\
\hspace*{0.333em}\hspace*{0.333em}sensor temperature, 59\\
\href{./2-general-information-about-data-files.html\#units-and-abbreviations}{knot},
6\\
\href{./2-general-information-about-data-files\#synchronization-of-measurements}{lags
in sampling}\\
\hspace*{0.333em}\hspace*{0.333em}static, 8\\
\hspace*{0.333em}\hspace*{0.333em}dynamic, 8
\href{./4-the-state-of-the-atmosphere\#psx}{LAMS}, 31
\href{./3-the-state-of-the-aircraft.html\#latitude}{LAT}, 28\\
\href{./3-the-state-of-the-aircraft.html\#gglat}{LAT\_G}, 18\\
\href{./3-the-state-of-the-aircraft.html\#latc-lonc}{LATC}, 27\\
\href{./3-the-state-of-the-aircraft.html\#thetae}{latent heat of
vaporization}, 59\\
\href{./3-the-state-of-the-aircraft.html\#latitude}{latitude}, 27\\
\href{./3-the-state-of-the-aircraft.html\#thetae}{lifted condensation
level}, 47\\
\href{./4-the-state-of-the-atmosphere.html\#atx}{linkage. temperature
and airspeed}, 33\\
\href{./5-cloud-physics-variables.html\#plwcc}{liquid water content},
67\\
\hspace*{0.333em}\hspace*{0.333em}\href{./5-cloud-physics-variables.html\#PSD-LWC}{1D
probes}, 63\\
\hspace*{0.333em}\hspace*{0.333em}\href{./5-cloud-physics-variables.html\#PSD-LWC}{FSSP-100},
63\\
\hspace*{0.333em}\hspace*{0.333em}\href{./5-cloud-physics-variables.html\#plwcc}{King
probe}, 58\\
\hspace*{0.333em}\hspace*{0.333em}\href{./5-cloud-physics-variables.html\#rice}{supercooled},
60\\
\href{./3-the-state-of-the-aircraft.html\#longitude}{LON}, 28
\href{./3-the-state-of-the-aircraft.html\#gglon}{LON\_G}, 18\\
\href{./3-the-state-of-the-aircraft.html\#latc-lonc}{LONC}, 27\\
\href{./3-the-state-of-the-aircraft.html\#longitude}{longitude}, 27\\
\href{./5-cloud-physics-variables.html\#plwcc}{LWC}, 88\\
\href{./10-obsolete-variables.html\#jwlw-corrected\%7C}{LWCC}, 88\\
\href{./4-the-state-of-the-atmosphere.html\#humidity}{Lyman-alpha
hygrometer}, 40\\
\href{./3-the-state-of-the-aircraft.html\#psx}{Mach number}, 36\\
\hspace*{0.333em}\hspace*{0.333em}uncorrected, 31\\
\href{./4-the-state-of-the-atmosphere.html\#mach-number}{MACHX}, 54\\
\href{./4-the-state-of-the-atmosphere.html\#mach-number}{MACHx}, 53\\
\href{./2-general-information-about-data-files\#distinction-between-original-measurements\%20and-derived-variables}{measurement}\\
\hspace*{0.333em}\hspace*{0.333em}derived, 8\\
\hspace*{0.333em}\hspace*{0.333em}original, 8\\
\hspace*{0.333em}\hspace*{0.333em}\href{./4-the-state-of-the-atmosphere.html\#variable-names}{preferred},
29\\
\hspace*{0.333em}\hspace*{0.333em}raw, 8, 33\\
meter\\
\hspace*{0.333em}\hspace*{0.333em}\href{2-general-information-about-data-files.html\#units-and-abbreviations}{mass
flow}, 5\\
\href{2-general-information-about-data-files.html\#units-and-abbreviations}{millibar},
5\\
\href{./2-general-information-about-data-files.html\#hms\%7C}{MINUTE},
7\\
\href{./4-the-state-of-the-atmosphere.html\#mirror-cr2}{MIRRORT\_CR2},
42\\
\href{./3-the-state-of-the-aircraft.html\#humidity}{MIRRTMP\_DPx}, 38\\
\href{./4-the-state-of-the-atmosphere.html\#MR}{mixing ratio}, 50\\
\hspace*{0.333em}\hspace*{0.333em}\href{2-general-information-about-data-files.html\#units-and-abbreviations}{conversion},
6\\
\href{./4-the-state-of-the-atmosphere.html\#moist-air}{moist-air
properties}, 34\\
molecular weight\\
\hspace*{0.333em}\hspace*{0.333em}\href{./1-introduction.html\#constants-and-symbols}{dry
air}, 4\\
\hspace*{0.333em}\hspace*{0.333em}\href{./1-introduction.html\#constants-and-symbols}{water},
4\\
\href{./2-general-information-about-data-files.html\#mdy\%7C}{MONTH},
7\\
motion\\
\hspace*{0.333em}\hspace*{0.333em}\href{./4-the-state-of-the-atmosphere.html\#wind}{GPS
antenna}, 51\\
\href{./4-the-state-of-the-atmosphere.html\#MR}{MR}, 50\\
\href{./4-the-state-of-the-atmosphere.html\#MR}{MRCR}, 45\\
\href{./4-the-state-of-the-atmosphere.html\#MR}{MRLA}, 45\\
\href{./4-the-state-of-the-atmosphere.html\#MR}{MRLH}, 45\\
\href{./4-the-state-of-the-atmosphere.html\#MR}{MRVXL}, 45\\
\href{./4-the-state-of-the-atmosphere.html\#special-use-remote}{MTP},
57\\
\href{./4-the-state-of-the-atmosphere.html\#humidity}{Murphy and Koop},
39\\
\href{./4-the-state-of-the-atmosphere.html\#variable-names}{names}\\
\hspace*{0.333em}\hspace*{0.333em}variable, 29\\
\hspace*{0.333em}\hspace*{0.333em}\hspace*{0.333em}\hspace*{0.333em}location
in, 29\\
\hspace*{0.333em}\hspace*{0.333em}\hspace*{0.333em}\hspace*{0.333em}suffixes,
29\\
\href{./2-general-information-about-data-files.html\#units-and-abbreviations}{nautical
mile}, 6\\
NetCDF\\
\hspace*{0.333em}\hspace*{0.333em}\href{./introduction.html}{format},
1\\
\hspace*{0.333em}\hspace*{0.333em}\href{./introduction.html}{header}, 1,
10\\
\hspace*{0.333em}\hspace*{0.333em}\href{./5-cloud-physics-variables.html\#CRPC}{vector},
62\\
\href{./1-introduction.html}{NIDAS}, 7\\
\href{./1-introduction.html}{nimbus}, 1\\
nomenclature\\
\hspace*{0.333em}\hspace*{0.333em}\href{./2-general-information-about-data-files.html\#variable-names-in-equations}{code},
8\\
\hspace*{0.333em}\hspace*{0.333em}\href{./2-general-information-about-data-files.html\#dimensions-in-equations}{dimensionless
equations}, 8, 9\\
\href{./10-obsolete-variables.html\#o3fs}{O3FS}, 92\\
\href{./6-air-chemistry-measurements.html\#f03-acd\%7C}{O3MR\_CL}, 69\\
\href{./4-the-state-of-the-atmosphere.html\#oat\%7C}{OAT}, 37\\
\href{./2-general-information-about-data-files.html\#synchronization-of-measurements}{sample
time}, 7\\
\href{./2-general-information-about-data-files.html\#bpitchr}{oscillation},
10\\
\href{./10-obsolete-variables.html\#AACT}{PACT}, 94\\
\href{./3-the-state-of-the-aircraft.html\#palt}{PALT}, 21\\
particles\\
\hspace*{0.333em}\hspace*{0.333em}\href{./7-aerosol-particle-measurements.html\#condensation-nucleus-counter}{ultrafine},
73\\
\href{./4-the-state-of-the-atmosphere.html\#p-special}{PCAB}, 32\\
\href{./7-aerosol-particle-measurements.html\#aerosol-spec}{PCASP}, 77\\
\href{./7-aerosol-particle-measurements.html\#pcn}{PCN}, 75\\
PCORS, see pressure corrections\\
\href{./3-the-state-of-the-aircraft.html\#ATX}{perfect gas}, 33\\
\href{./7-aerosol-particle-measurements.html\#pflw}{PFLW}, 76\\
\href{./7-aerosol-particle-measurements.html\#pflw}{PFLWC}, 76\\
\href{./10-obsolete-variables.html\#ltn51}{PHDG}, 85\\
\href{./3-the-state-of-the-aircraft.html\#pitch}{PITCH}, 11\\
\href{./3-the-state-of-the-aircraft.html\#pitch}{pitch}, 11\\
platform\\
\hspace*{0.333em}\hspace*{0.333em}\href{./8-radiation-variables.html\#spx}{stabilized},
79, 80\\
\href{./5-cloud-physics-variables.html\#plwc}{PLWC}, 58\\
\href{./5-cloud-physics-variables.html\#plwc}{PLWC1}, 58\\
\href{./5-cloud-physics-variables.html\#lwc2d}{PLWC1DC}, 67\\
\href{./5-cloud-physics-variables.html\#PSD-LWC}{PLWC6}, 63\\
\href{./5-cloud-physics-variables.html\#plwcc}{PLWCC}, 58\\
\href{./5-cloud-physics-variables.html\#plwcc}{PLWCC1}, 58\\
\href{./5-cloud-physics-variables.html\#PSD-LWC}{PLWCD}, 63\\
\href{./5-cloud-physics-variables.html\#PSD-LWC}{PLWCF}, 63\\
\href{./5-cloud-physics-variables.html\#plwcg}{PLWCG}, 60\\
\href{./5-cloud-physics-variables.html\#PSD-LWC}{PLWCX}, 63\\
\href{./5-cloud-physics-variables.html\#PSD-LWC}{PLWCY}, 63\\
\href{./4-the-state-of-the-atmosphere.html\#thetae}{potential
temperature}, 46\\
\href{./2-general-information-about-data-files.html\#units-and-abbreviations}{ppbv},
6\\
\href{./2-general-information-about-data-files.html\#units-and-abbreviations}{ppmv},
6\\
\href{./2-general-information-about-data-files.html\#units-and-abbreviations}{pptv},
5, 6\\
\href{./4-the-state-of-the-atmosphere.html\#psx}{pressure}, 29\\
\hspace*{0.333em}\hspace*{0.333em}\href{./4-the-state-of-the-atmosphere.html\#psx}{ambient},
29, 31, 33, 36, 52\\
\hspace*{0.333em}\hspace*{0.333em}\href{./4-the-state-of-the-atmosphere.html\#p-special}{cabin},
32\\
\hspace*{0.333em}\hspace*{0.333em}\href{./4-the-state-of-the-atmosphere.html\#psx}{corrections},
30\\
\hspace*{0.333em}\hspace*{0.333em}\href{./4-the-state-of-the-atmosphere.html\#dew-point}{dew
point housing}, 38\\
\hspace*{0.333em}\hspace*{0.333em}\href{./4-the-state-of-the-atmosphere.html\#p-special}{dew-point
cavity}, 32\\
\hspace*{0.333em}\hspace*{0.333em}\href{./4-the-state-of-the-atmosphere.html\#qcx}{dynamic},
30, 31, 33, 55\\
\hspace*{0.333em}\hspace*{0.333em}\hspace*{0.333em}\hspace*{0.333em}\href{./4-the-state-of-the-atmosphere.html\#qcx}{corrected},
31\\
\hspace*{0.333em}\hspace*{0.333em}\href{./7-aerosol-particle-measurements.html\#pcn}{inlet},
73\\
\hspace*{0.333em}\hspace*{0.333em}\href{./4-the-state-of-the-atmosphere.html\#dewpt-corrected}{partial,
water vapor}, 38\\
\hspace*{0.333em}\hspace*{0.333em}pitot, see pressure, total\\
\hspace*{0.333em}\hspace*{0.333em}static, see pressure ambient\\
\hspace*{0.333em}\hspace*{0.333em}\href{./4-the-state-of-the-atmosphere.html\#p-special}{surface},
32\\
\hspace*{0.333em}\hspace*{0.333em}\href{./4-the-state-of-the-atmosphere.html\#qcx}{total},
31, 35, 36\\
\hspace*{0.333em}\hspace*{0.333em}transducer,\\
\hspace*{0.333em}\hspace*{0.333em}\hspace*{0.333em}\hspace*{0.333em}\href{./4-the-state-of-the-atmosphere.html\#psx}{ambient
pressure}, 30\\
\hspace*{0.333em}\hspace*{0.333em}\hspace*{0.333em}\hspace*{0.333em}\href{./4-the-state-of-the-atmosphere.html\#qcx}{dynamic
pressure}, 31 ~~\href{./4-the-state-of-the-atmosphere.html\#ewx}{water
vapor}, 43\\
\hspace*{0.333em}\hspace*{0.333em}\hspace*{0.333em}\hspace*{0.333em}\href{./4-the-state-of-the-atmosphere.html\#dewpt-corrected}{equilibrium},
38, 39, 44\\
pressure damping\\
\hspace*{0.333em}\hspace*{0.333em}\href{./3-the-state-of-the-aircraft.html\#acins}{ACINS},
11\\
\hspace*{0.333em}\hspace*{0.333em}\href{./3-the-state-of-the-aircraft.html\#vspd}{VSPD},
12\\
\hspace*{0.333em}\hspace*{0.333em}\href{./3-the-state-of-the-aircraft.html\#wp3}{WP3},
12\\
\href{./4-the-state-of-the-atmosphere.html\#moist-air}{properties of
moist air}, 34\\
\href{./4-the-state-of-the-atmosphere.html\#psx}{PS\_A}, 29\\
\href{./4-the-state-of-the-atmosphere.html\#p-special}{PSDPx}, 32\\
\href{./4-the-state-of-the-atmosphere.html\#psx}{PSFD}, 29\\
\href{./4-the-state-of-the-atmosphere.html\#psx}{PSFRD}, 29\\
\href{./4-the-state-of-the-atmosphere.html\#p-special}{PSURF}, 32\\
\href{./4-the-state-of-the-atmosphere.html\#psx}{PSX}, 33\\
\href{./4-the-state-of-the-atmosphere.html\#psx}{PSx}, 29\\
\href{./4-the-state-of-the-atmosphere.html\#psx}{PSxC}, 29\\
\href{./10-obsolete-variables.html\#ptime\%7C}{PTIME}, 85\\
\href{./8-radiation-variables.html\#visxv}{pyranometer}, 79, 91\\
\hspace*{0.333em}\hspace*{0.333em}\href{./8-radiation-variables.html\#visxv}{calibration},
79\\
\href{./8-radiation-variables.html\#irxv}{pyrgeometer}, 78, 91\\
\hspace*{0.333em}\hspace*{0.333em}\href{./8-radiation-variables.html\#irxv}{calibration},
78\\
\href{./10-obsolete-variables.html\#qcb}{QCB}, 87\\
\href{./10-obsolete-variables.html\#qcb}{QCBC}, 87\\
\href{./3-the-state-of-the-aircraft.html\#psx}{QCF}, 54\\
\href{./10-obsolete-variables.html\#qcb}{QCG}, 87\\
\href{./10-obsolete-variables.html\#qcb}{QCG}, 87\\
\href{./10-obsolete-variables.html\#qcb}{QCGC}, 87\\
\href{./6-air-chemistry-measurements.html\#awas-cims-qcls-toga}{QCLS},
72\\
\href{./4-the-state-of-the-atmosphere.html\#qcx}{QCR}, 31\\
\hspace*{0.333em}\hspace*{0.333em}\href{./4-the-state-of-the-atmosphere.html\#qcx}{correction},
31\\
\href{./4-the-state-of-the-atmosphere.html\#qcx}{QCRC}, 31\\
\hspace*{0.333em}\hspace*{0.333em}\href{./4-the-state-of-the-atmosphere.html\#qcx}{flow-angle
correction}, 31\\
\href{./4-the-state-of-the-atmosphere.html\#qcx}{QCX}, 33\\
\href{./4-the-state-of-the-atmosphere.html\#qcx}{QCx}, 31\\
\href{./4-the-state-of-the-atmosphere.html\#qcx}{QCXC}, 36\\
\href{./4-the-state-of-the-atmosphere.html\#qcx}{QCxC}, 36\\
radar reflectivity factor\\
\hspace*{0.333em}\hspace*{0.333em}\href{./5-cloud-physics-variables.html\#DBZ}{1D
probes}, 64\\
\hspace*{0.333em}\hspace*{0.333em}\href{./5-cloud-physics-variables.html\#dbz2d}{2D
probes}, 67\\
\href{./8-radiation-variables.html}{radiation}, 78\\
\hspace*{0.333em}\hspace*{0.333em}\href{./8-radiation-variables.html\#irxv}{long-wave},
91\\
\hspace*{0.333em}\hspace*{0.333em}\href{./8-radiation-variables.html\#visxv}{short
wave}, 91\\
radiometer\\
\hspace*{0.333em}\hspace*{0.333em}\href{./8-radiation-variables.html\#rstx}{Heimann},
78\\
\hspace*{0.333em}\hspace*{0.333em}\href{./8-radiation-variables.html\#irxc}{Kipp
\& Zonen IR}, 79\\
\hspace*{0.333em}\hspace*{0.333em}\href{./8-radiation-variables.html\#visxc}{Kipp
\& Zonen visible}, 79\\
radius\\
\hspace*{0.333em}\hspace*{0.333em}\href{./1-introduction.html\#constants-and-symbols}{of
the Earth}, 4\\
\hspace*{0.333em}\hspace*{0.333em}\href{./5-cloud-physics-variables.html\#effective-radius}{effective},
67\\
\href{./4-the-state-of-the-atmosphere.html\#relative-wind}{radome gust
probe}, 52, 53\\
range\\
\hspace*{0.333em}\hspace*{0.333em}\href{./5-cloud-physics-variables.html\#fssp-range}{FSSP},
65\\
\href{./3-the-state-of-the-aircraft.html\#roc}{rate of climb}, 12\\
\href{./4-the-state-of-the-atmosphere.html\#vcsel-uncor}{RAWCONC\_VXL},
42\\
\href{./4-the-state-of-the-atmosphere.html\#ATX}{recovery factor}, 35\\
\hspace*{0.333em}\hspace*{0.333em}Rosemount sensors, 35\\
\href{./5-cloud-physics-variables.html\#reff2d}{REFF2DC}, 67\\
\href{./5-cloud-physics-variables.html\#reff2d}{REFF2DP}, 67\\
\href{./5-cloud-physics-variables.html\#effective-radius}{REFFD}, 64\\
\href{./5-cloud-physics-variables.html\#effective-radius}{REFFF}, 64\\
reflectivity factor, 64\\
\hspace*{0.333em}\hspace*{0.333em}\href{./5-cloud-physics-variables.html\#DBZ}{1D
probes}, 64\\
\hspace*{0.333em}\hspace*{0.333em}\href{./5-cloud-physics-variables.html\#dbz2d}{2D
probes}, 67 \href{./4-the-state-of-the-atmosphere.html\#rhumw}{relative
humidity}, 44 relative humidity wrt ice, see humidity, relative to ice\\
resets\\
\hspace*{0.333em}\hspace*{0.333em}\href{./10-obsolete-variables.html\#freset}{FSSP-100},
93\\
\href{./10-obsolete-variables.html\#ttrf}{reverse-flow temperature
sensor}, 88\\
\href{./4-the-state-of-the-atmosphere.html\#rho}{RHOLA}, 45\\
\href{./4-the-state-of-the-atmosphere.html\#rho}{RHOUV}, 45\\
\href{./4-the-state-of-the-atmosphere.html\#rho}{RHOx}, 44\\
\href{./4-the-state-of-the-atmosphere.html\#rhumw}{RHUM}, 44\\
\href{./4-the-state-of-the-atmosphere.html\#rhumi}{RHUMI}, 44\\
\href{./5-cloud-physics-variables.html\#rice}{RICE}, 60\\
\href{./3-the-state-of-the-aircraft.html\#roll}{ROLL}, 11\\
\href{./3-the-state-of-the-aircraft.html\#roll}{roll}, 11\\
\href{./5-cloud-physics-variables.html\#rice}{Rosemount 871F icing
probe}, 60\\
\href{./8-radiation-variables.html\#rstx}{RSTx}, 78\\
\href{./4-the-state-of-the-atmosphere.html\#recovery-t}{RTHRx}, 32\\
\href{./4-the-state-of-the-atmosphere.html\#recovery-t}{RTX}, 37\\
\href{./4-the-state-of-the-atmosphere.html\#recovery-t\%7C}{RTx}, 33\\
\href{./4-the-state-of-the-atmosphere.html\#recovery-t}{RTxH}, 32\\
\href{./2-general-information-about-data-files.html\#units-and-abbreviations}{samples
per second}, 6\\
\href{./2-general-information-about-data-files.html\#synchronization-of-measurements}{sampling
rates}, 7\\
\href{./3-the-state-of-the-aircraft.html\#inertial-reference-systems}{Schuler
oscillation}, 27\\
\href{./10-obsolete-variables.html\#sclwc}{SCLWC}, 93\\
\href{./2-general-information-about-data-files.html\#hms\%7C}{SECOND},
7\\
sensitivity coefficient\\
\hspace*{0.333em}\hspace*{0.333em}\href{./4-the-state-of-the-atmosphere.html\#akrd}{AKRD},
53\\
\hspace*{0.333em}\hspace*{0.333em}\href{./4-the-state-of-the-atmosphere.html\#ssrd}{SSRD},
54\\
sensor\\
\hspace*{0.333em}\hspace*{0.333em}\href{./4-the-state-of-the-atmosphere.html\#recovery-t}{temperature}\\
\hspace*{0.333em}\hspace*{0.333em}\hspace*{0.333em}\hspace*{0.333em}anti-iced,
33\\
\hspace*{0.333em}\hspace*{0.333em}\hspace*{0.333em}\hspace*{0.333em}heated,
33\\
\hspace*{0.333em}\hspace*{0.333em}\hspace*{0.333em}\hspace*{0.333em}K-probe,
33\\
\href{./3-the-state-of-the-aircraft.html\#sfc}{SFC}, 21\\
\href{./4-the-state-of-the-atmosphere.html\#ssrd}{sideslip}, 55\\
size distribution\\
\hspace*{0.333em}\hspace*{0.333em}\href{./5-cloud-physics-variables.html\#size-distribution}{1D
probes}\\
\hspace*{0.333em}\hspace*{0.333em}\href{./5-cloud-physics-variables.html\#a1dc-a1dp}{2D
probes}, 66\\
\href{./7-aerosol-particle-measurements.html\#special-aerosol}{SMPS},
77\\
\href{./8-radiation-variables.html\#solaz\%7C}{SOLAZ}, 82\\
\href{./8-radiation-variables.html\#solde}{SOLDE}, 81\\
\href{./8-radiation-variables.html\#solel}{SOLEL}, 81\\
\href{./8-radiation-variables.html\#solze}{SOLZE}, 82\\
\href{./3-the-state-of-the-aircraft.html\#ATX}{specific heat}, 34\\
\hspace*{0.333em}\hspace*{0.333em}\href{./1-introduction.html\#constants-and-symbols}{dry
air}\\
\hspace*{0.333em}\hspace*{0.333em}\hspace*{0.333em}\hspace*{0.333em}constant
pressure, 4\\
\hspace*{0.333em}\hspace*{0.333em}\hspace*{0.333em}\hspace*{0.333em}constant
volume, 4\\
\hspace*{0.333em}\hspace*{0.333em}\hspace*{0.333em}\hspace*{0.333em}ratio,
dry air, 4\\
\hspace*{0.333em}\hspace*{0.333em}\href{./4-the-state-of-the-atmosphere.html\#moist-air}{moist
air}, 33, 34\\
spectrometer\\
\hspace*{0.333em}\hspace*{0.333em}\href{./7-aerosol-particle-measurements.html\#special-aerosol}{aerosol
mass}, 77\\
\hspace*{0.333em}\hspace*{0.333em}\href{./5-cloud-physics-variables.html\#size-distribution}{hydrometeor},
63\\
speed\\
\hspace*{0.333em}\hspace*{0.333em}\href{./3-the-state-of-the-aircraft.html\#ggspd}{ground},
15\\
\hspace*{0.333em}\hspace*{0.333em}airspeed\\
\hspace*{0.333em}\hspace*{0.333em}\hspace*{0.333em}\hspace*{0.333em}\href{./10-obsolete-variables.html\#ias}{indicated},
88\\
\hspace*{0.333em}\hspace*{0.333em}\hspace*{0.333em}\hspace*{0.333em}\href{./4-the-state-of-the-atmosphere.html\#true-airspeed}{true},
53\\
\href{./4-the-state-of-the-atmosphere.html\#ATX}{speed of sound}, 35\\
\href{./4-the-state-of-the-atmosphere.html\#sphum}{SPHUM}, 53\\
\href{./8-radiation-variables.html\#spx}{SPxPitch}, 80\\
\href{./8-radiation-variables.html\#spx}{SPxRoll}, 80\\
\href{./10-obsolete-variables.html\#akfxx}{SSFXx}, 87\\
\href{./4-the-state-of-the-atmosphere.html\#sslip}{SSLIP}, 55\\
\href{./4-the-state-of-the-atmosphere.html\#ssrd}{SSRD}, 55\\
Stephan-Boltzmann Constant, , see\\
\hspace*{0.333em}\hspace*{0.333em}\href{./1-introduction.html\#constants-and-symbols}{Constants
and Symbols}, 4\\
\href{./2-general-information-about-data-files.html\#units-and-abbreviations}{STP
(standard conditions)}, 5\\
strobes\\
\hspace*{0.333em}\hspace*{0.333em}\href{./10-obsolete-variables.html\#fstrob}{FSSP},
93\\
\href{./10-obsolete-variables.html\#swtc}{SWTC}, 92\\
\href{./10-obsolete-variables.html\#swx}{SWx}, 91\\
\href{./2-general-information-about-data-files.html\#synchronization-of-measurements}{synchronization},
7\\
\href{./2-general-information-about-data-files.html\#units-and-abbreviations}{system
of units}, 5\\
\href{./4-the-state-of-the-atmosphere.html\#tashc}{TASHC}, 53\\
\href{./4-the-state-of-the-atmosphere.html\#true-airspeed\%7C}{TASx},
53\\
\href{./4-the-state-of-the-atmosphere.html\#true-airspeed}{TASx}, 53\\
\href{./4-the-state-of-the-atmosphere.html\#true-airspeed}{TASxD}, 52\\
\href{./8-radiation-variables.html\#rstx}{TCAVB}, 78\\
\href{./8-radiation-variables.html\#rstx}{TCAVT}, 78\\
\href{./7-aerosol-particle-measurements.html\#tcntu-tcntp}{TCNTP}, 76\\
\href{./7-aerosol-particle-measurements.html\#tcntu-tcntp}{TCNTU}, 76\\
\href{./7-aerosol-particle-measurements.html\#cntemp}{TEMP1}, 73\\
\href{./7-aerosol-particle-measurements.html\#cntemp}{TEMP2}, 73\\
\href{./4-the-state-of-the-atmosphere.html\#ambient-t}{temperature},
32\\
\hspace*{0.333em}\hspace*{0.333em}ambient, 32, 33, 41\\
\hspace*{0.333em}\hspace*{0.333em}calculation, 33\\
\hspace*{0.333em}\hspace*{0.333em}\href{./4-the-state-of-the-atmosphere.html\#thetae}{equivalent
potential}, 47\\
\hspace*{0.333em}\hspace*{0.333em}\href{./4-the-state-of-the-atmosphere.html\#AT-ITR}{in-cloud},
37\\
\hspace*{0.333em}\hspace*{0.333em}\href{./7-aerosol-particle-measurements.html\#cntemp}{inlet},
73\\
\hspace*{0.333em}\hspace*{0.333em}\href{./4-the-state-of-the-atmosphere.html\#theta}{potential},
46, 47\\
\hspace*{0.333em}\hspace*{0.333em}\href{./4-the-state-of-the-atmosphere.html\#thetae}{pseudo-adiabatic
equivalent potential}, 46\\
\hspace*{0.333em}\hspace*{0.333em}\href{./4-the-state-of-the-atmosphere.html\#AT-ITR}{radiometric},
37, 78\\
\hspace*{0.333em}\hspace*{0.333em}\href{./4-the-state-of-the-atmosphere.html\#recovery-t}{recovery},
32, 34\\
\hspace*{0.333em}\hspace*{0.333em}\href{./4-the-state-of-the-atmosphere.html\#recovery-t}{recovery
vs.~total}, 33\\
\hspace*{0.333em}\hspace*{0.333em}\href{./10-obsolete-variables.html\#ttrf}{reverse-flow},
88\\
\hspace*{0.333em}\hspace*{0.333em}\href{./4-the-state-of-the-atmosphere.html\#recovery-t}{sensor},
33\\
\hspace*{0.333em}\hspace*{0.333em}static air, 33\\
\hspace*{0.333em}\hspace*{0.333em}total, 33, 87\\
\hspace*{0.333em}\hspace*{0.333em}virtual, 48\\
\hspace*{0.333em}\hspace*{0.333em}virtual potential, 48\\
\href{./6-air-chemistry-measurements.html\#te03}{TEO3}, 70\\
\href{./6-air-chemistry-measurements.html\#te03c}{TEO3C}, 70\\
\href{./6-air-chemistry-measurements.html\#tep}{TEO3P}, 70\\
\href{./6-air-chemistry-measurements.html\#tep}{TEP}, 70\\
\href{./6-air-chemistry-measurements.html\#tet}{TET}, 70\\
\href{./3-the-state-of-the-aircraft.html\#thdg}{THDG}, 11\\
thermal conductivity, see conductivity, thermal\\
\href{./4-the-state-of-the-atmosphere.html\#theta}{THETA}, 48\\
\href{./4-the-state-of-the-atmosphere.html\#thetae}{THETAE}, 46\\
\href{./4-the-state-of-the-atmosphere.html\#thetaq}{THETAQ}, 49\\
\href{./4-the-state-of-the-atmosphere.html\#thetav}{THETAV}, 48\\
\href{./10-obsolete-variables.html\#thf}{THF}, 86\\
\href{./10-obsolete-variables.html\#ltn51}{THI}, 85\\
\href{./2-general-information-about-data-files.html\#time}{time}, 7\\
\hspace*{0.333em}\hspace*{0.333em}\href{./2-general-information-about-data-files.html\#base_time}{base\_time},
7\\
\hspace*{0.333em}\hspace*{0.333em}\href{./2-general-information-about-data-files.html\#time}{example
of usage}, 7\\
\hspace*{0.333em}\hspace*{0.333em}\href{./2-general-information-about-data-files.html\#synchronization-of-measurements}{interpolation},
8\\
\hspace*{0.333em}\hspace*{0.333em}\href{./2-general-information-about-data-files.html\#synchronization-of-measurements}{lags},
8\\
\hspace*{0.333em}\hspace*{0.333em}\href{./2-general-information-about-data-files.html\#synchronization-of-measurements}{offset},
7\\
\hspace*{0.333em}\hspace*{0.333em}\href{./2-general-information-about-data-files.html\#synchronization-of-measurements}{shifts},
8\\
\hspace*{0.333em}\hspace*{0.333em}\href{./2-general-information-about-data-files.html\#time}{variable
Time}, 7\\
\hspace*{0.333em}\hspace*{0.333em}\href{./2-general-information-about-data-files.html\#variables-used-to-denote-time}{variables},
6\\
\href{./3-the-state-of-the-aircraft.html\#ggtrk}{TKAT\_G}, 20\\
\href{./10-obsolete-variables.html\#tmlag}{TMLAG}, 85\\
\href{./6-air-chemistry-measurements.html\#awas-cims-qcls-toga}{TOGA},
72\\
\href{./10-obsolete-variables.html\#tptime}{TPTIME}, 85\\
\href{./4-the-state-of-the-atmosphere.html\#psx}{trailing cone}, 30\\
transducer\\
\hspace*{0.333em}\hspace*{0.333em}barometric, 29\\
\hspace*{0.333em}\hspace*{0.333em}Paroscientific, 30\\
\href{./1-introduction.html\#constants-and-symbols}{triple point of
water}, 4\\
\href{./8-radiation-variables.html\#trstx}{TRSTB}, 78\\
\href{./4-the-state-of-the-atmosphere.html\#true-airspeed}{true
airspeed}, see airspeed\\
\href{./10-obsolete-variables.html\#ttkp}{TTKP}, 88\\
\href{./10-obsolete-variables.html\#ttrf}{TTRF}, 87\\
\href{./10-obsolete-variables.html\#ttx}{TTx}, 87\\
\href{./4-the-state-of-the-atmosphere.html\#TVIR}{TVIR}, 48\\
\href{./5-cloud-physics-variables.html\#size-distribution}{UHSAS}, 77\\
\hspace*{0.333em}\hspace*{0.333em}\href{./7-aerosol-particle-measurements.html\#pflw}{flow},
76\\
\hspace*{0.333em}\hspace*{0.333em}\href{./7-aerosol-particle-measurements.html\#pflw}{STP},
76\\
\href{./4-the-state-of-the-atmosphere.html\#ui-vi-wi}{UI}, 55\\
\href{./4-the-state-of-the-atmosphere.html\#uic-vic}{UIC}, 56\\
uncertainty\\
\hspace*{0.333em}\hspace*{0.333em}wind\\
\hspace*{0.333em}\hspace*{0.333em}\hspace*{0.333em}\hspace*{0.333em}\href{./4-the-state-of-the-atmosphere.html\#wind}{Technical
Note}, 51\\
\href{./2-general-information-about-data-files.html\#units-and-abbreviations}{units},
5\\
\hspace*{0.333em}\hspace*{0.333em}exceptions to SI, 5\\
\hspace*{0.333em}\hspace*{0.333em}hertz, 6\\
\hspace*{0.333em}\hspace*{0.333em}ppmb, 5\\
\href{./1-introduction.html\#constants-and-symbols}{universal gas
constant}, 4\\
\href{./7-aerosol-particle-measurements.html\#upress}{UPRESS}, 76\\
\href{./7-aerosol-particle-measurements.html\#pflw}{USFLWC}, 76\\
\href{./7-aerosol-particle-measurements.html\#pflw}{USMPFLW}, 76\\
\href{./10-obsolete-variables.html\#uvx}{UVx}, 92\\
\href{./4-the-state-of-the-atmosphere.html\#ux-vy}{UX}, 56\\
\href{./4-the-state-of-the-atmosphere.html\#uxc-vyc}{UXC}, 57\\
variable\\
\hspace*{0.333em}\hspace*{0.333em}\href{./2-general-information-about-data-files\#distinction-between-original-measurements\%20and-derived-variables}{derived},
9\\
\hspace*{0.333em}\hspace*{0.333em}\href{./2-general-information-about-data-files\#variable-names-in-equations}{names
in brackets}, 8\\
\hspace*{0.333em}\hspace*{0.333em}\href{./4-the-state-of-the-atmosphere.html\#variable-names}{preferred
choice}, 29\\
variable names\\
\hspace*{0.333em}\hspace*{0.333em}\href{./4-the-state-of-the-atmosphere.html\#variable-names}{conventions},
29\\
\hspace*{0.333em}\hspace*{0.333em}\href{./5-cloud-physics-variables.html\#sensors-1-D-probes}{hydrometeor
probes}, 60\\
\href{./10-obsolete-variables.html\#cryo-hygro}{VCRH}, 89\\
\href{./3-the-state-of-the-aircraft.html\#vew}{VEW}, 15\\
\href{./3-the-state-of-the-aircraft.html\#ggvew}{VEW\_G}, 18\\
\href{./3-the-state-of-the-aircraft.html\#vewc-vnsc}{VEWC}, 28\\
\href{./4-the-state-of-the-atmosphere.html\#ui-vi-wi}{VI}, 55\\
\href{./4-the-state-of-the-atmosphere.html\#uic-vic}{VIC}, 56\\
\href{./5-cloud-physics-variables.html\#plwcc}{viscosity}, 59\\
\href{./8-radiation-variables.html\#visxc}{VISxC}, 80\\
\href{./8-radiation-variables.html\#visxht}{VISxHT}, 80\\
\href{./8-radiation-variables.html\#visxht}{VISxHTV}, 80\\
\href{./8-radiation-variables.html\#visxv}{VISxV}, 79\\
\href{./10-obsolete-variables.html\#vla}{VLA}, 90\\
\href{./10-obsolete-variables.html\#vla}{VLA1}, 90\\
\href{./3-the-state-of-the-aircraft.html\#vns}{VNS}, 15\\
\href{./3-the-state-of-the-aircraft.html\#ggvns}{VNS\_G}, 18\\
\href{./3-the-state-of-the-aircraft.html\#vewc-vnsc}{VNSC}, 28\\
\href{./3-the-state-of-the-aircraft.html\#vspd}{VSPD}, 13\\
\href{./3-the-state-of-the-aircraft.html\#ggvspd}{VSPD\_G}, 19\\
\href{./4-the-state-of-the-atmosphere.html\#ux-vy}{VY}, 56\\
\href{./4-the-state-of-the-atmosphere.html\#uxc-vyc}{VY}, 56\\
\href{./4-the-state-of-the-atmosphere.html\#uxc-vyc}{VYC}, 57\\
\href{./3-the-state-of-the-aircraft.html\#ggvspd}{VZI}, 86\\
\href{./10-obsolete-variables.html\#vzi}{VZI}, 86\\
water\\
\hspace*{0.333em}\hspace*{0.333em}\href{./5-cloud-physics-variables.html\#plwcc}{boiling
point}, 58\\
water vapor\\
\hspace*{0.333em}\hspace*{0.333em}\href{./4-the-state-of-the-atmosphere.html\#rho}{density},
41, 42, 44\\
\hspace*{0.333em}\hspace*{0.333em}\href{./4-the-state-of-the-atmosphere.html\#MR}{mixing
ratio}, 45\\
\href{./4-the-state-of-the-atmosphere.html\#ws-wd}{WD}, 56\\
\href{./4-the-state-of-the-atmosphere.html\#wsc-wdc}{WDC}, 56\\
\href{./10-obsolete-variables.html\#wspd}{WDRCTN}, 86\\
wetting\\
\hspace*{0.333em}\hspace*{0.333em}\href{./4-the-state-of-the-atmosphere.html\#AT-ITR}{of
thermometers}, 37\\
\href{./1-introduction.html\#constants-and-symbols}{WGS-84 geoid}, 4\\
\href{./3-the-state-of-the-aircraft.html\#ggalt}{WGS84}, 19\\
\href{./4-the-state-of-the-atmosphere.html\#ui-vi-wi}{WI}, 55\\
\href{./4-the-state-of-the-atmosphere.html\#wic}{WIC}, 56\\
\href{./4-the-state-of-the-atmosphere.html\#wind}{wind}, 55\\
\hspace*{0.333em}\hspace*{0.333em}\href{./4-the-state-of-the-atmosphere.html\#wind-components-and-the-wind-vector}{components},
55\\
\hspace*{0.333em}\hspace*{0.333em}\href{./4-the-state-of-the-atmosphere.html\#wsc-wdc}{direction},
56\\
\hspace*{0.333em}\hspace*{0.333em}\href{./4-the-state-of-the-atmosphere.html\#wsc-wdc}{GPS-corrected},
56\\
\hspace*{0.333em}\hspace*{0.333em}\href{./4-the-state-of-the-atmosphere.html\#uxc-vyc}{lateral
component}, 57\\
\hspace*{0.333em}\hspace*{0.333em}\href{./4-the-state-of-the-atmosphere.html\#uxc-vyc}{longitudinal
component}, 57\\
\hspace*{0.333em}\hspace*{0.333em}\href{./4-the-state-of-the-atmosphere.html\#relative-wind}{relative},
52\\
\hspace*{0.333em}\hspace*{0.333em}\href{./4-the-state-of-the-atmosphere.html\#ws-wd}{sign
convention}, 55\\
\hspace*{0.333em}\hspace*{0.333em}\href{./4-the-state-of-the-atmosphere.html\#wsc-wdc}{speed},
56\\
\hspace*{0.333em}\hspace*{0.333em}uncertainty\\
\hspace*{0.333em}\hspace*{0.333em}\hspace*{0.333em}\hspace*{0.333em}\href{./4-the-state-of-the-atmosphere.html\#wind}{Tech
Note}, 10, 51\\
\hspace*{0.333em}\hspace*{0.333em}\href{./4-the-state-of-the-atmosphere.html\#wsc-wdc}{vector},
55, 56\\
\hspace*{0.333em}\hspace*{0.333em}\href{./4-the-state-of-the-atmosphere.html\#wic}{vertical},
56\\
\href{./3-the-state-of-the-aircraft\#global-positioning-systems}{World
Geodetic System}, 19\\
\href{./3-the-state-of-the-aircraft.html\#wp3}{WP3}, 12\\
\href{./4-the-state-of-the-atmosphere.html\#ws-wd}{WS}, 56\\
\href{./4-the-state-of-the-atmosphere.html\#wsc-wdc}{WSC}, 56\\
\href{./10-obsolete-variables.html\#wspd}{WSPD}, 86\\
\href{./6-air-chemistry-measurements.html\#xf03fs}{XFO3FNO}, 69\\
\href{./6-air-chemistry-measurements.html\#xf03fs}{XFO3FS}, 69\\
\href{./6-air-chemistry-measurements.html\#xf03p}{XFO3P}, 69\\
\href{./7-aerosol-particle-measurements.html\#xicnc}{XICN}, 74\\
\href{./7-aerosol-particle-measurements.html\#xicnc}{XICNC}, 74\\
\href{./4-the-state-of-the-atmosphere.html\#mach-number}{XMACH2}, 52\\
\href{./6-air-chemistry-measurements.html\#no-noy}{XNCLF}, 71\\
\href{./6-air-chemistry-measurements.html\#no-noy}{XNMBT}, 71\\
\href{./6-air-chemistry-measurements.html\#no-noy}{XNO}, 71\\
\href{./6-air-chemistry-measurements.html\#mr-no-no2}{XNOCAL}, 71\\
\href{./6-air-chemistry-measurements.html\#no-noy}{XNOCF}, 71\\
\href{./6-air-chemistry-measurements.html\#no-noy}{XNOSF}, 71\\
\href{./6-air-chemistry-measurements.html\#no-noy}{XNOY}, 71\\
\href{./6-air-chemistry-measurements.html\#no-noy}{XNOYP}, 71\\
\href{./6-air-chemistry-measurements.html\#no-noy}{XNOZA}, 71\\
\href{./6-air-chemistry-measurements.html\#no-noy}{XNSAF}, 71\\
\href{./6-air-chemistry-measurements.html\#no-noy}{XNST}, 71\\
\href{./6-air-chemistry-measurements.html\#mr-no-no2}{XNYCAL}, 71\\
\href{./6-air-chemistry-measurements.html\#no-noy}{XNZAF}, 71\\
\href{./6-air-chemistry-measurements.html\#f03-acd}{XO3}, 69\\
\href{./3-the-state-of-the-aircraft.html\#humidity}{XSIGV\_UVH}, 42\\
\href{./10-obsolete-variables.html\#xuvi}{XUVI}, 90\\
\href{./10-obsolete-variables.html\#xuvi}{XUVP}, 90\\
\href{./10-obsolete-variables.html\#xuvi}{XUVT}, 90\\
\href{./10-obsolete-variables.html\#ltn51}{XVI}, 85\\
\href{./2-general-information-about-data-files.html\#mdy\%7C}{YEAR}, 7\\
\href{./10-obsolete-variables.html\#ltn51\%7C}{YVI}, 85

\backmatter
\end{document}
