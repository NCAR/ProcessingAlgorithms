
\section{OBSOLETE VARIABLES\label{sec:OBSOLETE-VARIABLES}}

\nop{OBSOLETE}{}RAF retired the ``GENPRO'' processor, the software
program previously used to produce data sets, in 1993, but data files
produced by that processor are still retained and available for use.
Also, there are some instruments that are now retired but provided
measurements in some archived data files. Obsolete variable names
that are associated only with GENPRO or a retired instrument are discussed
below, for reference and to facilitate use of old data files. 
\begin{hangparagraphs}
\textbf{Unaltered Tape Time (s): }\textbf{\uline{TPTIME}}\sindex[var]{TPTIME (obsolete)}\index{TPTIME}\\
This variable is derived by converting the HOUR, MINUTE and SECOND
to elapsed seconds after midnight of the current day. If time increments
to the next day, its value is not reset to zero, but 86400 seconds
are added to produce ever-increasing values for the data set.

\textbf{Processor Time (s): }\textbf{\uline{PTIME}}\sindex[var]{PTIME (obsolete)}\index{PTIME}\\
This is an internal time variable created by the GENPRO processor.
It represents elapsed seconds after midnight. It differs from TPTIME
in that, after it has been set at the beginning of the data set, it
is incremented internally for each second of data processed. If duplicate
or missing raw data records exist, it can differ from TPTIME. It is
guaranteed to be a monotonically increasing and continuous series
of values.

\textbf{INS: Data System Time Lag (s): }\textbf{\uline{TMLAG}}\sindex[var]{TMLAG (obsolete)}\index{TMLAG}\\
TMLAG is the amount of time between the reference time of a Litton
LTN-5l Inertial Navigation System (INS) and the data system clock,
in seconds. TMLAG will always be greater than zero and less than 2.

\textbf{}%
\noindent\begin{minipage}[t]{1\columnwidth}%
\textbf{LORAN-C Latitude (}$\text{�}$\textbf{): }\textbf{\uline{CLAT}}\textbf{\sindex[var]{CLAT (obsolete)}\index{CLAT}}\\
\textbf{LORAN-C Longitude (}$\text{�}$\textbf{): }\textbf{\uline{CLON}}\textbf{\sindex[var]{CLON (obsolete)}\index{CLON}}\\
\textbf{LORAN-C Circular Error of Probability (n mi): }\textbf{\uline{CCEP}}\textbf{\sindex[var]{CCEP (obsolete)}\index{CCEP}}\\
\textbf{LORAN-C Ground Speed (m/s): }\textbf{\uline{CGS}}\textbf{\sindex[var]{CGS (obsolete)}\index{CGS}}\\
\textbf{LORAN-C Time (s): }\textbf{\uline{CSEC}}\textbf{\sindex[var]{CSEC (obsolete)}\index{CSEC}}\\
\textbf{LORAN-C Fractional Time (s): }\textbf{\uline{CFSEC\sindex[var]{CFSEC (obsolete)}\index{CFSEC}}}%
\end{minipage}\\
Before the advent of GPS, NCAR/RAF operated a LORAN-C receiver that
provided information on the position and groundspeed of the aircraft.
The measurements of latitude and longitude from this system are CLAT
and CLON, measured at 1 Hz and with positive values of longitude to
the east and positive values of latitude to the north. and CCEP provides
an estimate of the uncertainty in those measurements (in units of
nautical miles). A status word, CSTAT, was used to record a value
of 15 when the system was operational. The ground speed and reference
times were also recorded in the above corresponding variables. The
sum of CSEC and CFSEC represented the time of the measurement, which
was not always the time in the data file when the measurements were
recorded, 

\noindent\begin{minipage}[t]{1\columnwidth}%
\textbf{INS Latitude (}$\text{�}$\textbf{): }\textbf{\uline{ALAT}}\textbf{\sindex[var]{ALAT (obsolete)}\index{ALAT}}\\
\textbf{INS Longitude (}$\text{�}$\textbf{): }\textbf{\uline{ALON}}\textbf{\sindex[var]{ALON (obsolete)}\index{ALON}}\\
\textbf{Raw INS Ground Speed X Component (m/s): }\textbf{\uline{XVI}}\sindex[var]{XVI (obsolete)}\textbf{\index{XVI}}\\
\textbf{Raw INS Ground Speed Y Component (m/s): }\textbf{\uline{YVI}}\textbf{\sindex[var]{YVI (obsolete)}\index{YVI}}\\
\textbf{Raw INS True Heading (}$\text{�}$\textbf{): }\textbf{\uline{THI}}\textbf{\sindex[var]{THI (obsolete)}\index{THI}}\\
\textbf{INS Wander Angle (}$\text{�}$\textbf{): }\textbf{\uline{ALPHA}}\textbf{\sindex[var]{ALPHA (obsolete)}\index{ALPHA}}\\
\textbf{INS Platform Heading (}$\text{�}$\textbf{): }\textbf{\uline{PHDG\sindex[var]{PHDG (obsolete)}\index{PHDG}}}%
\end{minipage}

~~~~~~~~~\label{LTN51}These variables from the Litton LTN-51
Inertial Navigation System\index{IRU!Litton LTN-51} (INS) are analogous
to the modern variables discussed in section \ref{sec:INS}. The measurements
of latitude and longitude were provided with 1-Hz frequency and had
a resolution of 0.0014$\text{�}$, while the ground speed components
were provided at 10 Hz and had resolution equal to 0.012 m/s. The
X component of the ground speed was along the longitudinal axis of
the aircraft \emph{at the time of alignment,}  and the Y axis was
in the starboard direction at the time of alignment. PHDG recorded
the orientation of the platform relative to true north, with resolution
0.0028$\text{�}$. THI was the true heading of the aircraft, produced
at 5 Hz with resolution of 0.0014$\text{�}$. The ``wander angle''
is an INS-only variable that recorded the angle of the INS platform
x-axis relative to its original orientation; it ``wandered'' in
response to east-west motion of the aircraft on a spherical Earth. 

\textbf{Raw Aircraft Vertical Velocity (m/s): }\textbf{\uline{VZI}}\sindex[var]{VZI (obsolete)}\index{VZI}\\
This is an integrated output from an up/down binary counter connected
to the INS vertical accelerometer. Resolution is 0.012 m/s. Due to
changes in local gravity and accumulated errors, this often develops
a significant offset during flight. 

\textbf{Aircraft True Heading (}$\text{�}$\textbf{): }\textbf{\uline{THF}}\sindex[var]{THF (obsolete)}\index{THF}\\
This measurement of aircraft heading was derived from the angle between
the horizontal projection of the aircraft center and true north: THF
= PHDG + ALPHA. Resolution is 0.0028$\text{�}$. 

\noindent\begin{minipage}[t]{1\columnwidth}%
\textbf{Aircraft Ground Speed (m/s): }\textbf{\uline{GSF}}\textbf{\sindex[var]{GSF (obsolete)}\index{GSF}}\\
\textbf{Aircraft Ground Speed East Component (m/s): }\textbf{\uline{VEW}}\textbf{\sindex[var]{VEW (obsolete)}\index{VEW}}\\
\textbf{Aircraft Ground Speed North Component (m/s): }\textbf{\uline{VNS\sindex[var]{VNS (obsolete)}\index{VNS}}}%
\end{minipage}

~~~~~~~~~These variables have the same names as the modern
variables for ground speed. (Cf.~section \ref{sec:INS}.) GSF is
the magnitude of the ground speed determined by the INS, as derived
from XVI and YVI: \\
\[
\mathrm{GSF=\sqrt{\{XVI\}^{2}+\{YVI\}^{2}}}
\]
\\
VEW and VNS are the east and north projections of this ground speed,
derived using THF for the aircraft heading.

\noindent\begin{minipage}[t]{1\columnwidth}%
\textbf{Wind Speed (m/s): }\textbf{\uline{WSPD}}\textbf{\sindex[var]{WSPD (obsolete)}\index{WSPD}}\\
\textbf{Wind Direction (}$\text{�}$\textbf{): }\textbf{\uline{WDRCTN\sindex[var]{WDRCTN (obsolete)}\index{WDRCTN}}}%
\end{minipage}

~~~~~~~~~These variables are calculated from UI and VI, the
east and north components of the wind determined as described in RAF
Bulletin No.~23 and summarized in section \ref{sec:WIND}:\label{punch:10-1}

\begin{eqnarray*}
\mathrm{WS} & = & \sqrt{\mathrm{\{UI\}^{2}+\{VI\}^{2}}}\\
\mathrm{WD} & = & \mathrm{\frac{180^{\circ}}{\pi}atan2(-\{UI\},}-\{VI\})+180^{\circ}
\end{eqnarray*}
\\

\textbf{Raw Attack Force (Fixed Vane) (g): }\textbf{\uline{AFIXx}}\sindex[var]{AFIXx (obsolete)}\index{AFIXx}\\
AFIXx is an amplified output from a strain-gauge, fixed-vane sensor
mounted in the horizontal plane of the aircraft at the end of a gust
boom. The ``force'' on the vane (calibrated in ``equivalent grams''
at Jefferson County Airport gravity) varies as a function of the aircraft
attack angle and dynamic pressure. Here x refers to left or right.

\textbf{Raw Sideslip Force(Fixed Vane) (g): }\textbf{\uline{BFIXx}}\sindex[var]{BFIXx (obsolete)}\index{BFIXx}\\
BFIXx is an amplified output from a strain-gauge, fixed-vane sensor
mounted in the vertical plane of the aircraft at the end of a gust
boom. The ``force'' on the vane (calibrated in ``equivalent grams''
at Jefferson County Airport gravity) varies as a function of the aircraft
sideslip angle and dynamic pressure. Here x refers to top or bottom.

\textbf{Attack Angle (Fixed Vane) (}$\text{�}$\textbf{): }\textbf{\uline{AKFXx}}\sindex[var]{AKFXx (obsolete)}\index{AKFXx}\\
AKFXx is the angle of attack, computed from AFIXx and QCx (either
boom or gust dynamic pressure). An empirically derived function, HSSATK,
is used to determine the attack angle based upon wind tunnel test
data.

\textbf{Sideslip Angle (Fixed Vane) (}$\text{�}$\textbf{): }\textbf{\uline{SSFXx}}\sindex[var]{SSFXx (obsolete)}\index{SSFXx}\\
SSFXx is the sideslip angle, computed from BFIXx, and QCx (either
boom or gust dynamic pressure). An empirically derived function, HSSATK,
is used to determine the sideslip angle based upon wind tunnel test
data.

\noindent\begin{minipage}[t]{1\columnwidth}%
\textbf{Dynamic Pressure (Boom) (mb): }\textbf{\uline{QCB}}\textbf{\sindex[var]{QCB (obsolete)}\index{QCB},
}\textbf{\uline{QCBC}}\textbf{\sindex[var]{QCBC (obsolete)}\index{QCBC}}\\
\textbf{Dynamic Pressure (Gust Probe) (mb): }\textbf{\uline{QCG}}\textbf{\sindex[var]{QCG (obsolete)}\index{QCG},
}\textbf{\uline{QCGC\sindex[var]{QCGC (obsolete)}\index{QCGC}}}%
\end{minipage}

~~~~~~~~~These variables, measured by a differential pressure
gauge, record the difference between a pitot (total) pressure and
a static pressure. The QCBC and QCGC values are corrected for local
flow-field distortion. The boom and gust probe measurements referred
to the same aircraft structure. The different designations used for
those measurements specified the transducer used and its location.
In the gust probe dynamic pressure measurement (QCG), a Rosemount
Model 1332 differential pressure transducer was located closer to
the sensor in the gust probe itself, whereas in the boom measurement
(QCB), a Rosemount Model 1221 pressure transducer was typically located
in the aircraft nose.

\textbf{Ambient Temperature }($^{\circ}C$):\textbf{\uline{ ATC}}\textbf{\index{ATC}}\sindex[var]{ATC}\\
A variable obtained by combining the avionics temperature on the GV,
AT\_A, with a Rosemount temperature, so that the absolute value tracked
AT\_A but faster response was provided by the Rosemount temperature.
This was used in some early GV projects because there were unresolved
problems with the data-system temperature sensors and it was thought
that AT\_A provided a more accurate result, but AT\_A was filtered
to have slow response to it was combined with the faster-response
signal from the Rosemount sensor.

\textbf{Total Temperature ($^{\circ}$C): TTx\index{TTx}\sindex[var]{TTx}}\\
This variable was used before 2014 for measurements of the recovery
temperature, for which the variable is now \textbf{RTx}. Because the
quantity measured is not the total temperature,\index{temperature!total}
the variables TTx were replaced by RTx, but the meaning historically
was the same as that now described for \textbf{RTX}, apart from how
humidity is now handled. 

\textbf{Total Temperature, Reverse Flow ($^{\circ}C$): }\textbf{\uline{TTRF}}\sindex[var]{TTRF (obsolete)}\index{TTRF}\\
TTRF is the recovery temperature from a calibrated NCAR reverse-flow
temperature sensor\index{reverse-flow temperature sensor}, for which
the housing was designed to separate water droplets and protect the
element from wetting in cloud. 

\textbf{Total Temperature (Fast Response) ($^{\circ}C$): }\textbf{\uline{TTKP}}\sindex[var]{TTKP (obsolete)}\index{TTKP}\\
This is the output of recovery temperature from the NCAR fast-response
temperature probe, originally designed by Karl Danninger. (See discussion
of total temperature in section \ref{subsec:PTq}.)

\textbf{Ambient Temperature ($^{\circ}C$): }\textbf{\uline{ATRF}}\sindex[var]{ATRF (obsolete)}\index{ATRF}\\
The ambient temperature computed using the NCAR reverse-flow temperature
sensor. (See discussion in Section \ref{subsec:PTq} above.)

\textbf{Ambient Temperature (Fast Response) ($^{\circ}C$): }\textbf{\uline{ATKP}}\sindex[var]{ATKP (obsolete)}\index{ATKP}\\
The ambient temperature computed using the fast-response temperature
probe. (See discussion of ambient temperature in section \ref{subsec:PTq}.)

\textbf{Raw Cloud Technology (Johnson-Williams) }\\
\textbf{Liquid Water Content ($g/m^{3}$): }\textbf{\uline{LWC}}\sindex[var]{LWC (obsolete)}\index{LWC}\\
This is the raw output of a Johnson-Williams\index{Johnson-Williams sensor}
liquid water content sensor converted to units of grams per cubic
meter. The Johnson-Williams indicator measures the evaporative cooling
caused by the latent heat of vaporization of droplets contacting the
heated sensing element by sensing changes in its resistance as it
cools. Through calibration this resistance is converted to a liquid
water content. A ``compensation'' wire is also mounted in the J-W
sensor, parallel to the droplet stream, to compensate for cooling
effects of the airstream. Typically the instrument is set for a true
airspeed of 200 knots. The instrument must be zeroed in ``cloud-free
air.'' The Johnson-Williams liquid water content sensor is designed
for the cloud droplet spectrum. There is some evidence to indicate
that droplets larger than 30 $\mu m$ are shed before completely vaporizing
on the sensor element. This tends to underestimate the liquid water
content.

\textbf{Corrected Cloud Technology (Johnson-Williams) }\\
\textbf{Liquid Water Content (g/M3): }\textbf{\uline{LWCC}}\sindex[var]{LWCC (obsolete)}\index{LWCC}\\
This is the corrected liquid water content obtained by using the aircraft's
true airspeed after removing the zero offset: LWCC=LWC$U_{a}/U_{ref}$
where $U_{a}$ is the true airspeed of the aircraft and $U_{ref}$
is the true airspeed set on the dial of the instrument. $U_{ref}$
was normally 200 kts = 102.88889 m/s.

\textbf{Indicated Airspeed (knots): }\textbf{\uline{IAS\index{indicated airspeed}\index{airspeed!indicated}\sindex[var]{IAS}}}\\
In some old data files, a variable representing the indicated airspeed
was included because this was used for some derived variables. The
indicated airspeed is the airspeed that would produce the observed
difference between dynamic and static pressure under standard conditions
of 1013.25 mb and $15^{\circ}$C.

\textbf{Water Vapor Pressure (mb): }\textbf{\uline{EDPC}}\sindex[var]{EDPC (obsolete)}\index{EDPC}\\
This is a derived intermediate variable used in the calculation of
several derived thermodynamic variables. The vapor pressure over a
plane water surface is obtained by the method of Paul R. Lowe (1977),
a derived, sixth-order, Chebyshev polynomial fit to the Goff-Gratch
Formulation (1946) as a function of temperature expressed in $^{\circ}C$.
The error is much less than 1\% over the range -50$\text{�}$C to
+50$\text{�}$C. EDPC was calculated using this method for most RAF
research projects between 1993 and 1996. This variable did not have
the enhancement factor applied that was discussed in Appendix C of
Bulletin 9. A variable of the same name but calculated differently
replaced this in 1996, and with changes described in Section \ref{sec:State Variables}
continues in use, recently replaced by EWx.\label{punch:10-2}\\
\\
\fbox{\begin{minipage}[t]{0.9\textwidth}%
A. T $<$ -50 C:\\
\begin{eqnarray*}
\mathrm{EDPC} & = & 4.4685+T(0.27347+T\{6.83811\times10^{-3}\\
 & + & T[8.7094x10^{-5}+T(5.63513x10^{-7}+T\,1.47796\times10^{\mbox{-9}})]\})
\end{eqnarray*}
\\
B. T $>$= -50$\text{�}$C:\\
\begin{eqnarray*}
\mathrm{EDPC} & = & 6.107799961+T\,[0.4436518521+T(0.01428945805\\
 & + & T\{2.650648471\times10^{-4}+T\,[3.031240396\times10^{-6}\\
 & + & T(2.034080948\times10^{-8}+T\,6.136820929\times10^{-11})]\})]
\end{eqnarray*}
%
\end{minipage}}

\textbf{Cryogenic Hygrometer Inlet Pressure (hPa) and Frost Point
Temperature ($^{\circ}C$): }\textbf{\uline{CRHP\sindex[var]{CRHP}\index{CRHP}}}
\textbf{and }\textbf{\uline{VCRH}}\sindex[var]{VCRH}\index{VCRH}
(obsolete)\\
These are measurements made directly in the chamber of the cryogenic
hygrometer\index{hygrometer!cryogenic}, a now obsolete cabin-mounted
instrument connected to outside air by an inlet line. CRHP is the
pressure and VCRH is the frost-point temperature measured inside that
chamber. VCRH is determined from a third-order calibration equation
applied to the voltage measured by the instrument. \label{punch:10-3}

\textbf{Corrected Cryogenic Frost Point Temperature and Dew Point
Temperature ($\text{�}$C): }\textbf{\uline{FPCRC\sindex[var]{FPCRC}\index{FPCRC}}}\textbf{
and }\textbf{\uline{DPCRC}}\sindex[var]{DPCRC}\index{DPCRC}\\
\emph{The frost point or dew point determined after corrections are
applied to the direct measurements from a cryogenic hygrometer. }These
measurements were from a now obsolete instrument but the variables
are included here because they appear in some old data files. To obtain
estimates of the ambient frost point and dew point, the measurements
made inside the chamber of the cryogenic hygrometer (CVRH and CRHP)
must be corrected for the difference in water vapor pressure between
that chamber and ambient conditions. The ratio of the chamber pressure
to the ambient pressure is assumed to be the same as the ratio of
the chamber vapor pressure to the ambient vapor pressure. The vapor
pressure in the chamber was determined from the Goff-Gratch (1946)
equation\footnote{Goff, J. A., and S. Gratch (1946) Low-pressure properties of water
from \textminus 160 to 212 \textdegree F, referenced and used in the
Smithsonian Tables (List, 1980).} for saturation vapor pressure with respect to a plane ice surface.
This vapor pressure was then used with CRHP and a measure of the ambient
pressure (PSXC) to determine the vapor pressure in the outside air,
and this was converted to an equivalent dew-point. The instrument
was only used for measurements of frost point\index{frost point}
less than -15$^{\circ}$C because it did not function well above that
frost point. \label{punch:10-4}The steps are documented below:\\
\fbox{\begin{minipage}[t]{0.9\textwidth}%
VCRH = frost point inside the cryogenic hygrometer ($^{\circ}C$)\\
CRHP = pressure inside the chamber of the cryogenic hygrometer (hPa)\\
PSXC = reference ambient pressure (hPa)\\
f$_{i}$ = enhancement factor (see Appendix C of Bulletin 9)\\
$F_{1}$($T_{d}$) =Goff-Gratch formula for vapor pressure at dew
point $T_{d}$\\
$F_{2}(T_{f})$ = Goff-Gratch formula for vapor pressure at frost
point $T_{f}$ \\
$T_{3}$ = temperature at the triple point of water = 273.16 K\\
\\
\\
\rule[0.5ex]{1\linewidth}{1pt}

chamber vapor pressure $e_{ic}$ (hPa):

\[
e_{ic}=(6.1071\,\mathrm{mb})\times10^{A}
\]

\begin{eqnarray*}
\mathrm{where}\,\,\,A & = & -9.09718\left(\frac{T_{3}}{\mathrm{VCRH}+T_{3}}-1\right)\\
 & + & 3.56654\log_{10}\left(\frac{T_{3}}{\mathrm{VCRH}+T_{3}}\right)\\
 & + & 0.876793\left(1-\frac{\mathrm{VCRH}+T_{3}}{T_{3}}\right)
\end{eqnarray*}

ambient vapor pressure $e_{a}$ (hPa):

\[
e_{a}=e_{ic}\left(\frac{\mathrm{PSXC}}{\mathrm{CRHP}}\right)f_{i}
\]

ambient dew and frost point DPCRC and FPCRC: (iterative solution)

\begin{eqnarray*}
e_{a} & = & F_{1}\left(\mathrm{DPCRC}\right)\\
 & = & F_{2}\left(\mathrm{FPCRC}\right)
\end{eqnarray*}

\begin{lyxcode}
\end{lyxcode}
%
\end{minipage}}

\textbf{Voltage Output From the Lyman-alpha Sensor (V): }\textbf{\uline{VLA}}\sindex[var]{VLA}\textbf{\index{VLA},
}\textbf{\uline{VLA1}}\index{VLA1} (obsolete)\\
\emph{The voltage output from the Lyman-alpha absorption hygrometer}.\index{hygrometer!Lyman-alpha}
This instrument provided fast-response, high-resolution measurements
of water vapor density. (If a second sensor was used, a 1 was added
to the variable name associated with the second sensor.) The sensors
are now obsolete.

\textbf{Voltage Output from the UV Hygrometer (V): }\textbf{\uline{XUVI}}\sindex[var]{XUVI}\textbf{\index{XUVI}}\\
\index{hygrometer!UV}\emph{The voltage from a modern (as of 2009)
version of the Lyman-alpha hygrometer, which provides a signal that
represents water vapor density.} The instrument also provides measurements
of pressure and temperature inside the sensing cavity; they are, respectively,
\textbf{\uline{XUVP}}\sindex[var]{XUVP}\textbf{\index{XUVP}}
and \textbf{\uline{XUVT}}\sindex[var]{XUVT}\textbf{\index{XUVT}}.
These variables and the processing algorithm below have now been replaced
by XSIGV\_UVH and the algorithm discussed  with the variable EW\_UVH.\\
\fbox{\begin{minipage}[t]{0.95\columnwidth}%
XUVI = output from the UV Hygrometer, after application of calibration
coefficients\\
DPXC\index{DPXC} = corrected dewpoint from some preferred source,
$^{\circ}$C\\
ATX\index{ATX} = preferred temperature, $^{\circ}$C\\
RHODT\index{RHODT} =water vapor density determined by a chilled-mirror
sensor\\
Tau = time constant for the exponential update (typically 300 s)\\
\rule[0.5ex]{1\columnwidth}{1pt}
\begin{lyxcode}
For~valid~measurements:\footnote{i.e., DPXC<ATX and XUVI and RHODT are not missing}~

~~~~Offset~+=~(RHODT-XUVI-Offset)/Tau~\\
RHOUV~=~XUVI~+~Offset
\end{lyxcode}
%
\end{minipage}}

\textbf{Raw Pyrgeometer Output (W\,m$^{-2}$): }\textbf{\uline{IRx}}\sindex[var]{IRx}\index{IRx}\\
\label{EppleyReference}A pyrgeometer\index{pyrgeometer} manufactured
by Eppley Laboratory, Inc. measures \index{radiation!long-wave}long-wave
irradiance using a calibrated thermopile. It has a coated glass hemisphere
that transmits radiation in a bandwidth between 3.5 $\mu m$ and 50
$\mu m$. It is calibrated at RAF according to procedures specified
by Albrecht and Cox (1977). (See the reference in the next paragraph.)
The pyrgeometers are usually flown in pairs, one up-looking and one
down-looking. The letter 'x' denotes either bottom (B) or top (T).\\

\textbf{Corrected Infrared Irradiance (W\,m$^{-2}$): }\textbf{\uline{IRxC}}\sindex[var]{IRxC}\index{IRxC}\\
Because the pyrgeometer measures net radiation, IRx must be corrected
for emission from the dome covering the sensor and for emission from
the thermopile itself. IRxC is the corrected infrared irradiance,
determined following procedures of \href{http://journals.ametsoc.org/doi/pdf/10.1175/1520-0450\%281977\%29016\%3C0190\%3APFIPP\%3E2.0.CO\%3B2}{Albrecht and Cox, 1977}.
. \\
\fbox{\begin{minipage}[t]{0.9\textwidth}%
IRx = raw pyrgeometer output {[}W\,m$^{-2}${]}\\
$T_{D}$ = dome temperature {[}K{]}\\
$T_{S}$ = ``sink'' temperature (approx.~the thermopile temperature)
{[}K{]}\\
$\epsilon$ = emissivity of the thermopile (dimensionless) = 0.986\\
$\beta$ = empirical constant dependent on the dome type = 5.5\\
$\sigma$ = Stephan-Boltzmann constant = 5.6704$\times10^{-8}$ W\,m$^{-2}$K$^{-4}$\\
\\
\rule[0.5ex]{1\linewidth}{1pt}
\[
\mathrm{IRxC}=\mathrm{IRx}-\beta\sigma(T_{D}^{4}-T_{S}^{4})+\epsilon\sigma T_{S}^{4}
\]
%
\end{minipage}}\\

\textbf{Shortwave Irradiance (W/m$^{2}$): }\textbf{\uline{SWx}}\sindex[var]{SWx}\index{SWx}\\
An Eppley Laboratory, Inc., pyranometer\index{pyranometer} measures
\index{radiation!short wave}short-wave irradiance. The dome normally
used is UG295 glass, which gives wide coverage of the solar spectrum
(from 0.285 $\mu m$ to 2.8 $\mu m$). Different bandwidths can be
obtained by use of different glass domes, available from RAF upon
request. (See Bulletin No. 25.) The pyranometers are usually flown
in pairs, one up-looking and one down-looking. They are calibrated
periodically at the NOAA Solar Radiation Facility in Boulder, Colorado.
The letter 'x' denotes either bottom (B) or top (T).

\textbf{Corrected Incoming Shortwave Irradiance (W/m$^{2}$): }\textbf{\uline{SWTC}}\sindex[var]{SWTC}\index{SWTC}\\
The down-welling shortwave irradiance measured by the difference between
SWT and SWB) is corrected to take into account the sun angle and small
variations in the aircraft attitude angles (pitch and roll). The correction
is limited to $\pm6^{\circ}$ in either angle, so these measurements
should be considered invalid beyond these limits. This is the derived
output of incoming (down-welling) shortwave irradiance, taking into
account both solar position (sun angle) and modest variations in aircraft
attitude (at present, restricted to less than 6$\text{�}$ in pitch
and/or roll). (For more information, refer to \href{http://www.eol.ucar.edu/raf/Bulletins/bulletin25.html}{RAF Bulletin 25}.)\index{Bulletin 25}\label{punch:10-5}

\textbf{Ultraviolet Irradiance (W/m$^{2}$): UVx}\sindex[var]{UVx}\index{UVx}\\
A pair of UV radiometer/photometers measure either down-welling (x=T)
or up-welling (x=B) irradiance in the ultraviolet, approximately from
0.295 $\mu m$ to 0.385 $\mu m$. These units are periodically returned
to the Eppley Laboratories for recalibration. 

\textbf{Raw Carbon Monoxide Concentration (ppb): }\textbf{\uline{CO}}\index{CO}\sindex[var]{CO}\\
CO is the uncorrected output of the TECO model 48 CO analyzer. \label{punch:10-6}This
instrument measures the concentration of CO by gas filter correlation.
The optics of the version operated by the RAF have been modified to
increase the light through the absorption cell, and a zero trap has
been added that periodically removes CO from the sample air stream
to obtain an accurate zero. This permits correction for the significant
temperature-dependent drift of the zero level of the measurement.

\textbf{}%
\noindent\begin{minipage}[t]{1\columnwidth}%
\textbf{Carbon Monoxide Analyzer Status (V): }\textbf{\uline{CMODE}}\index{CMODE}\sindex[var]{CMODE}\\
\textbf{Carbon Monoxide Baseline Zero Signal (V): }\textbf{\uline{COZRO}}\textbf{\index{COZRO}}\sindex[var]{COZRO}\textbf{}\\
\textbf{Raw Carbon Monoxide, Baseline Corrected (V): }\textbf{\uline{COCOR\index{COCOR}}}\sindex[var]{COCOR}%
\end{minipage}\\
CMODE records if the CO analyzer is supplied with air from which CO
has been removed and so is recording its zero level. When CMODE is
less than 0.2 V, the instrument is in the normal operational mode,
and when CMODE is greater than 8.0 V the instrument is in the ``zero''
mode. When measurements are processed, the zero-mode signals are represented
by a cubic spline to obtain a reference baseline for the signal (COZRO),
and this baseline is subtracted from the measured value (CO) to obtain
COCOR. This variable still jumps to zero periodically and does not
include the calibration that enters the following variable, COCAL. 

\textbf{Corrected Carbon Monoxide Concentration (ppmv): }\textbf{\uline{COCAL}}\index{COCAL}\sindex[var]{COCAL}\\
\label{punch:10-7}The calibrated signal from the CO instrument after
correction for drift of the baseline and after application of the
appropriate calibration coefficients to produce units of ppmv. The
quality of the baseline fit can be judged by examining the offset
at the zero points. If there are relatively small changes in the baseline,
the zero offset will be only a few ppbv. If there have been rapid
changes in the baseline, the zero offset can be up to 50 ppbv. The
magnitude of the offset at the zero values gives a good measure of
uncertainty in the data set. The detection limit is 10 ppbv, with
an uncertainty of $\pm15\%$. At 1 Hz, data will have considerable
variability, so 10-s averaging is often useful when the measurements
are used for analysis. 

\textbf{Raw Chemiluminescent Ozone Signal (V): }\textbf{\uline{O3FS}}\index{O3FS}\sindex[var]{O3FS}\\
\emph{Voltage output from the chemiluminescence ozone instrument,}
which operates on the basis of reacting nitric oxide with ozone and
detecting the resulting chemiluminescence.

\textbf{Derived Supercooled Liquid Water Content (g/m$^{3}$): }\textbf{\uline{SCLWC}}\sindex[var]{SCLWC}\index{SCLWC}\label{SCLWC}\\
This variable is the supercooled liquid water content obtained from
the change in accreted mass on the Rosemount 871F ice-accretion probe
over one second. The output is not valid during the probe deicing
cycle. This cycle is apparent in the RICE output (a peak followed
by a decrease to near zero). Supercooled liquid water content is determined
by first calculating a water drop impingement rate which is a function
of the effective surface area, the collection efficiency, the true
airspeed, and the supercooled liquid water content. The impingement
rate obtained is equated to the accreted mass of ice collected by
the probe in one second (empirical voltage/mass relationship). The
resulting equation is solved for supercooled water content. This calculation
is not included in normal processing or special processing, but some
users of the instrument use an approach like the following to calculate
supercooled liquid water:\label{punch:10-8}\\
\fbox{\begin{minipage}[t]{0.9\textwidth}%
A = \sindex[lis]{A=area}effective surface area of the probe (m$^{2}$)\\
$\Delta t$ = time interval\sindex[lis]{Deltat@$\Delta t$=time interval}\sindex[lis]{t@$t$=time}
during which an increment of mass accretes (s)\\
$\Delta m$ = mass\sindex[lis]{m@$m$=mass} of ice accreted on the
probe in the time interval $\Delta t$ (g)\\
$U_{a}$ = true airspeed (m/s)\\
\\
\\
\rule[0.5ex]{1\linewidth}{1pt}

\[
\mathrm{SCLWC}=AU_{a}\frac{\Delta m}{\Delta t}
\]
%
\end{minipage}}\\

\textbf{FSSP-100 Fast Resets }(number per sample interval):\textbf{
}\textbf{\uline{FRST}}\textbf{\sindex[var]{FRST}\index{FRST},
}\textbf{\uline{FRESET}}\textbf{\sindex[var]{FRESET}\index{FRESET}}\\
\emph{The rate at which fast resets occur in an FSSP-100 probe. }The
FSSP-100\index{FSSP-100!fast resets}\index{resets!FSSP-100} records
events called ``fast resets'' that occur when a particle traverses
the beam outside the depth-of-field and therefore is not accepted
for sizing. To avoid the processing time associated with sizing, the
probe resets quickly in this case, but there is still some dead time\index{FSSP-100!dead time}
when the probe cannot record another event. Fast resets consume a
time determined by circuit characteristics, and that time has been
determined in laboratory tests of the FSSP circuitry. This variable
is needed in addition to the ``Total Stobes'' to determine what
fraction of the time the probe is unable to accept another particle,
and this ``dead time'' enters calculation of the concentration for
the original (old) FSSP. \\

\textbf{FSSP-100 Total Strobes }(number per sample interval)\textbf{:
}\textbf{\uline{FSTB}}\textbf{\sindex[var]{FSTB}\index{FSTB},
}\textbf{\uline{FSTROB}}\sindex[var]{FSTROB}\index{FSTROB}\\
\emph{The rate at which strobes are generated in an FSSP-100 probe.
}A ``strobe'' is generated in the FSSP-100\index{FSSP-100!total strobes}\index{strobes!FSSP}
whenever a particle is detected within its depth-of-field. Not all
such particles are accepted for inclusion in the size distribution,
however, because some pass through the outer regions of the illuminating
laser beam and therefore produce shorter and smaller-amplitude pulses
than those passing through the center of the beam. The probe maintains
a running estimate of the average transit time and rejects particles
with transit times shorter than this average. The total number of
strobes recorded is therefore more than the number of sized particles,
and the ratio of strobes to accepted particles can indicate quality
of operation of the probe. Also, the strobes require processing and
so contribute to the dead time of the probe, affecting the concentration
unless a correction is made. See \href{http://www.eol.ucar.edu/raf/Bulletins/bulletin24.html}{RAF Bulletin 24}\index{Bulletin 24}
for more discussion on the operation of the ``old'' FSSP.\\

\textbf{FSSP-100 Beam Fraction }(dimensionless)\textbf{: }\textbf{\uline{FBMFR}}\sindex[var]{FBMFR}\index{FBMFR}\\
\index{FSSP-100!beam fraction}\emph{The ratio of the number of velocity-accepted
particles (particles that pass through the effective beam diameter)
to the total number of particles detected in the depth-of-field of
the beam (the total strobes).} See the discussion of Total Strobes
for more information.\\
 \\
\fbox{\begin{minipage}[t]{0.9\textwidth}%
\{AFSSP\}$_{i}$ = valid particles sized in size interval i\\
\{FSTROB\} = strobes generated by particles in the depth-of-field,
\\
\hspace*{0.7in}per sample interval\\
\\
\rule[0.5ex]{1\linewidth}{1pt}
\[
\mathrm{FBMFR=\{AFSSP\}/\{FSTROB\}}
\]
%
\end{minipage}}\\

\textbf{FSSP-100 Calculated Activity Fraction} (dimensionless):\textbf{
}\textbf{\uline{FACT}}\index{FACT}\sindex[var]{FACT}\\
This variable\index{FSSP-100!activity} represents the fraction of
the time that the FSSP is unable to count and size particles (its
``dead time\index{dead time!FSSP}\index{FSSP-100!dead time}'').
The activity fraction is not measured directly but is estimated from
fast resets and total strobes along with measurements of the dead
times associated with each (as determined in laboratory tests). The
characteristic times are in the NetCDF header (for recent projects).
.\\
\noindent\fbox{\begin{minipage}[t]{1\columnwidth - 2\fboxsep - 2\fboxrule}%
FSTROB = strobes generated by particles in the depth-of-field, \\
\hspace*{0.7in}per sample interval\\
FRESET = ``fast resets'' generated per sample interval\\
$t_{1}$ = slow reset time (for each strobe)\\
$t_{2}$ - fast reset time (for each fast reset)\\
\\
\rule[0.5ex]{1\linewidth}{1pt}

\[
FACT=\{FSTROB\}\,t_{1}+\mathrm{\{FRESET\}}\,t_{2}
\]
%
\end{minipage}}\\
\\

\textbf{PCAS Raw Activity }(dimensionless):\textbf{ }\textbf{\uline{AACT}}\uline{\sindex[var]{AACT}\index{AACT},
}\textbf{\uline{PACT}}\sindex[var]{PACT}\index{PACT}\\
The PCAS probe provides this measure of dead time, the time that the
probe is unable to sample particles because the electronics are occupied
with processing particles. The manufacturer suggests that the actual
dead time ($f_{PCAS}$) is given by the following formula, which is
used in determining concentrations\index{concentration!PCAS} for
the PCAS:\\
\[
f_{PCAS}=0.52\frac{\mathrm{\{PACT\}}}{F_{PCAS}}
\]
\\
where $F_{PCAS}=1024\,s^{-1}$. However, PACT (or AACT) is the variable
archived in the data files. \\
\end{hangparagraphs}


