
\subsection{Wind\label{sec:WIND}}

\href{http://www.eol.ucar.edu/raf/Bulletins/bulletin23.html}{RAF Bulletin 23}\index{Bulletin 23}
documents the calculation of wind components, both with respect to
the earth (UI, VI, WI, WS and WD) and with respect to the aircraft
(UX and VY). In data processing, a separate function (GUSTO in GENPRO,
gust.c in NIMBUS) is used to derive these wind components. That function
uses the measurements from an Inertial Navigation System (INS\index{INS})
as well as aircraft true airspeed, aircraft angle of attack, and aircraft
sideslip angle. The wind components calculated in GUSTO/gust.c are
used to derive the wind direction (WD) and wind speed (WS). Additional
variables UIC, VIC, WSC, WDC, UXC, and VYC are also calculated based
on the variables VNSC, VEWC discussed in section \ref{subsec:IRS/GPS},
which combine INS and GPS information to obtain improved measurements
of the aircraft motion. Those are usually the highest-quality measurements
of wind because the merged INS/GPS variables combine the high-frequency
response of the INS with the long-term accuracy of the GPS.

There is an extensive discussion of the wind-sensing system and the
uncertainties associated with measurements of wind in this \href{http://dx.doi.org/10.5065/D60G3HJ8}{Technical Note}.\index{uncertainty!wind!Technical Note}
The details contained therein and in Bulletin 23 will not be repeated
here, so those documents should be consulted for additional information.
There are two exceptions that are discussed in more detail here:
\begin{enumerate}
\item The calculation of vertical wind is described in more detail below
for the variables WI and WIC. 
\item Because measurements obtained by a GPS receiver are often used, the
motion of the GPS receiving antenna\index{motion!GPS antenna} relative
to the IRU must be considered. Standard processing corrects for the
motion of the gust system relative to the IRU arising from aircraft
rotation, but a similar correction is needed because the GPS antenna
is displaced from the IRU. The displacement is almost entirely along
the longitudinal axis of the aircraft, so GPS-measured velocities
like GGVNS, GGVEW, and GGVSPD (denoted here $v_{n}$, $v_{e}$, and
$v_{u}$) are corrected as follows to give measurements that apply
at the location of the IRU. Then these variables can be used in place
of or to complement similar measurements from the IRU in the processing
algorithms. The equations are:\\
\begin{equation}
\delta v_{u}{\rm =-L_{G}\dot{\theta}}\label{eq:pitchr}
\end{equation}
\begin{eqnarray}
\delta v_{e} & = & -L_{G}\dot{\psi}\,{\rm \cos\psi}\label{eq:GGVEW}\\
\delta v_{n} & = & L_{G}\dot{\psi}{\rm \,\sin\psi}\label{eq:GGVNS}
\end{eqnarray}
where $L_{G}$ is the distance\sindex[lis]{Lg@$L_{G}$=distance from IRU to GPS antenna}
forward along the longitudinal axis from the IRU to the GPS antenna
($-4.30$\,m for the GV, where the GPS receiver is behind the IRU)
and where $\theta$ and $\psi$ respectively represent the pitch and
heading angle. The dots over the attitude-angle symbols represent
time derivatives, so for example $\dot{\theta}$ is the rate of change
of the pitch angle and all angles are expressed in radians. The correction
terms are added to the GPS-measured velocity components so that they
represent the motion of the IRU relative to the Earth.
\end{enumerate}
The variables pertaining to the relative wind are described in the
next subsection, and the variables characterizing the wind are then
described briefly in the last subsection. Some additional detail is
included in cases where procedures are not documented in that earlier
bulletin.

\subsubsection{Relative Wind}

Wind\index{wind!relative} is measured by adding two vectors: (1)
the measured air motion relative to the aircraft (called the relative
wind), and (2) the motion of the aircraft relative to the Earth. The
following are the measurements used to determine the relative wind.
The motion of the aircraft relative to the ground was discussed in
Section \ref{sec:INS}, and the combination of these two vectors to
measure the wind is described in \href{http://www.eol.ucar.edu/raf/Bulletins/bulletin23.html}{RAF Bulletin 23}. 

RAF uses the radome\label{radome gust-sensing system} gust-sensing\index{radome gust probe}\index{gust probe!radome}
technique\footnote{Brown, E.~N, C.~A.~Friehe, and D.~H.~Lenschow, 1983:\emph{ Journal
of Climate and Applied Meteorology,} \textbf{22, }171\textendash 180} to measure incidence angles of the relative wind (i.e., angles of
attack and sideslip). The pressure difference between sensing ports
above and below the center line of the radome is used, along with
the dynamic pressure\index{dynamic pressure} measured at a pitot
tube and referenced to the static pressure source, to determine the
angle of attack. The sideslip angle is determined similarly using
the pressure ports on the starboard and port sides of the radome.
A Rosemount Model 858AJ gust probe\index{gust probe!858AJ} has occasionally
been used for specialized measurements. The radome measurements are
made by differential pressure sensors located in the nose area of
the aircraft and connected to the radome by semi-rigid tubing.
\begin{hangparagraphs}
\textbf{Mach Number (dimensionless): }\textbf{\uline{MACHx}}\textbf{\sindex[var]{MACHx}}\index{MACHx}\\
\emph{The Mach Number that characterizes the flight speed.} The Mach
number is defined as the ratio of the flight speed (or the magnitude
of the relative wind) to the speed of sound. See Eq.~(\ref{eq:8.8-1})
in Section \ref{sec:State Variables} for the equation used. Many
archived data files have a variable XMACH2\index{XMACH2}\sindex[var]{XMACH2},
which is the square of MACHx.

\textbf{Aircraft True Airspeed (m/s): }\textbf{\uline{TASx}}\index{TASx}\textbf{,
}\textbf{\uline{TASxD}}\textbf{\sindex[var]{TASxD}}\index{TASxD}\\
\emph{The flight speed of the aircraft relative to the atmosphere.}
This derived measurement of the flight speed\index{flight speed|see {true airspeed}}
of the aircraft relative to the atmosphere is based on the Mach number
calculated from both the dynamic pressure at location x and the static
pressure\index{pressure!ambient}. See the derivation for ATx \vpageref{ambient temperature and TAS calculation}.
The different variables for TASx (TASF, TASR, etc) use different measurements
of QCxC in the calculation of Mach number. The variable TASxD is the
result of calculations for which the Mach number, air temperature,
and true airspeed are determined for dry instead of humid air. See
the discussion of ATX on page \ref{ATX discussion} for an explanation
of how humidity is handled in the calculation of true airspeed.\\
\\
\fbox{\begin{minipage}[t]{0.95\textwidth}%
(see box for ATx\index{ATx} and MACHx\index{MACHx})\\
Note dependence of MACHx on choices for QCxC\index{QCxC} and PSXC\index{PSXC}\\
TASx\index{TASx} depends on QCxC, PSXC\index{PSXC}, ATX\index{ATX}\\
~~~~~where PSXC and ATX are the preferred choices\\
$\gamma^{\prime}$, $R^{\prime}$, and $T_{0}$: See the List of Symbols\\
\rule[0.5ex]{1\columnwidth}{1pt}

\begin{equation}
\mathrm{TASx}=\mathrm{\{MACHx\}}\sqrt{\gamma^{\prime}R^{\prime}\mathrm{\,(\{ATX\}}+T_{0})}\label{eq:8.10-1}
\end{equation}
%
\end{minipage}}\\
\\

\textbf{Aircraft True Airspeed (Humidity Corrected) (m/s): }\textbf{\uline{TASHC}}\sindex[var]{TASHC}\index{TASHC}
\textendash{} \emph{obsolete}\\
 This derived measurement of \index{airspeed}true airspeed accounted
for deviations of specific heats of moist air from those of dry air.
See List, 1971, pp 295, 331-339, and Khelif, et al., 1999. The equation
used for this variable, given by Khelif et al.~1999,\footnote{Khelif, D., S.P. Burns, and C.A. Friehe, 1999: Improved wind measurements
on research aircraft. \emph{Journal of Atmospheric and Oceanic Technology,}
\textbf{16,} 860\textendash 875.} added a moisture correction to the true airspeed derived for dry
air, as follows:\\
\\
 %
\fbox{\begin{minipage}[t]{0.95\columnwidth}%
$q$ = specific humidity (dimensionless) = SPHUM/1000.\index{SPHUM}
\index{humidity!specific} for SPHUM expressed in g/kg\\
$c=0.000304\,\mathrm{kg\,g^{-1}}=0.304$ (dimensionless)\\
\rule[0.5ex]{1\columnwidth}{1pt}
\begin{lyxcode}
TASHC~=~TASX~{*}~(1.0~+~$c$~{*}~$q$)
\end{lyxcode}
%
\end{minipage}}\\

\textbf{Attack Angle Differential Pressure (mb): }\textbf{\uline{ADIFR}}\sindex[var]{ADIFR}\index{ADIFR}\\
\emph{The pressure difference between the top and bottom pressure
ports of a radome gust-sensing system. }\index{attack, angle of}This
measurement is used to determine the angle of attack; see AKRD below.\index{radome gust probe}\index{gust probe!radome}\textbf{
}Obsolete variable \uline{ADIF}\index{ADIF} is a similar variable
used for old gust-boom systems or for Rosemount Model 858AJ flow-angle
sensors.\index{gust probe!Rosemount 858AJ}

\textbf{Sideslip Angle Differential Pressure (mb): }\textbf{\uline{BDIFR}}\sindex[var]{BDIFR}\index{BDIFR}\\
\emph{The pressure difference between starboard and port pressure
inlets of a radome gust-sensing system. }\index{sideslip, angle of}\index{radome gust probe}
This measurement is used to determine the sideslip angle; see SSRD
below. Obsolete variable \uline{BDIF}\index{BDIF} is a similar
variable used for old gust-boom systems or for Rosemount Model 858AJ
flow-angle sensors.

\textbf{Attack Angle, Radome (}$\text{�}$\textbf{): }\textbf{\uline{AKRD}}\sindex[var]{AKRD}\index{AKRD}\\
\emph{The angle of attack of the aircraft. }This derived measurement
represents the angle between the relative wind vector and the plane
determined by the longitudinal and lateral axes of the aircraft. Positive
values indicate flow moving upward relative to that plane.\index{attack, angle of}
The calculation is based on ADIFR\index{ADIFR} and a measurement
of dynamic pressure, and so is the measurement produced by a radome
gust-sensing system. Empirical sensitivity coefficients for each aircraft,
determined from special flight maneuvers, are used; see \href{http://www.eol.ucar.edu/raf/Bulletins/bulletin23.html}{RAF Bulletin 23}\index{Bulletin 23}
and this \href{http://dx.doi.org/10.5065/D60G3HJ8}{Technical Note}
for more information.\index{sensitivity coefficient!ADIFR} The sensitivity
coefficients listed below have changed when the radomes were changed
or refurbished, so the project documentation should be consulted for
the values used in a particular project. For more information on the
latest C-130 calibration, see \href{https://drive.google.com/open?id=0B1kIUH45ca5AN2VYdFF2V0tHcVU}{this note}.\\
\fbox{\begin{minipage}[t]{0.95\textwidth}%
ADIFR\index{ADIFR} = attack differential pressure, radome (hPa)\\
QCXC\index{QCXC} = reference dynamic pressure (hPa)\\
MACHX\index{MACHX} = reference Mach number\\
$e_{0},\,e{}_{1},\,e_{2}$ = \sindex[lis]{e02@$e_{0-2}$=sensitivity coefficients, angle of attack}sensitivity
coefficients determined empirically; typically:\\
~~~~~~~~~~%
\begin{minipage}[t]{0.8\textwidth}%
\begin{quote}
\{5.776$\,[^{\circ}]$, 15.031$\,[^{\circ}]$, 0\} for the C-130\\
\footnote{prior to Jan 2012, when the radome was changed: \{5.516, 19.07, 2.08\}}
\{4.605$\,[^{\circ}]$, 18.44$\,[^{\circ}]$, 6.75$\,[^{\circ}]$\}
for the GV\\
\end{quote}
%
\end{minipage}

\rule[0.5ex]{1\linewidth}{1pt}

\begin{equation}
\mathrm{AKRD}=e_{0}+\frac{\{\mathrm{ADIFR}\}}{\{\mathrm{QCF}\}}\left(e_{1}+e{}_{2}\mathrm{\{MACHX\}}\right)\label{eq:AKRDeq}
\end{equation}
%
\end{minipage}}

\textbf{}%
\begin{comment}
Old Bulletin-9 code: omitted now

\textbf{Attack Angle, 858 (}$\text{�}$\textbf{): }\textbf{\uline{AKDF}}\index{AKDF}\\
This is a derived output of the aircraft's angle of attack obtained
from ADIF (the vertical differential pressure measured by a Rosemount
858AJ flow-angle sensor) and a dynamic pressure (Qc) using an empirically
determined sensitivity function. For the Rosemount 858AJ, the sensitivity
function (GR) is a constant 0.079 for Mach numbers less than 0.515.
For larger Mach numbers the sensitivity decreases as a function of
Mach number.\\
\fbox{\begin{minipage}[t]{0.9\textwidth}%
MACH = Aircraft Mach Number\\
GR = sensitivity function (given below)\\
Qc = dynamic pressure, mb

\begin{eqnarray*}
\mathrm{for\,MACH} & \leq & 0.515:\\
 &  & \mathrm{GR}=0.079\\
\mathrm{otherwise}:\\
 & \mathrm{GR} & =0.086577797-0.03560256\,MACH\\
 &  & +0.00006143\,MACH^{2}
\end{eqnarray*}

\rule[[0.5ex]]{1.0\linewidth}{1pt}

\[
\mathrm{AKDF=\frac{ADIF}{GR\,Qc}}
\]
%
\end{minipage}}
\end{comment}

\textbf{Reference Attack Angle (}$\text{�}$\textbf{): }\textbf{\uline{ATTACK}}\sindex[var]{ATTACK}\index{ATTACK}\label{punch:4-12}\\
\emph{The reference angle of attack used to calculate derived variables.
}This variable is the reference selected from other measurements of
angle of attack in the data set. In most projects, it is equal to
AKRD\index{AKRD}. It is used where attack angle is needed for other
derived calculations (e.g., wind measurements).

\textbf{Sideslip Angle (Differential Pressure) (}$\text{�}$\textbf{):
}\textbf{\uline{SSRD}}\sindex[var]{SSRD}\index{SSRD}\\
\emph{The angle of sideslip of the aircraft. }This derived measurement
represents the angle between the longitudinal axis of the aircraft
and the projection of the relative wind onto the plane determined
by the longitudinal and lateral axes. Positive values indicate airflow
from the starboard side. This variable is derived from BDIFR\index{BDIFR}
and a dynamic pressure\index{pressure!dynamic} using a sensitivity
function that has been determined empirically for each aircraft. \\
\fbox{\begin{minipage}[t]{0.9\textwidth}%
BDIFR = differential pressure between sideslip pressure ports, radome
(mb)\\
QCXC = dynamic pressure (mb)\\
$s_{0},\,s{}_{1}$ = \sindex[lis]{s0@$s_{0,1}$=sensitivity coefficients, sideslip}empirical
coefficients dependent on the aircraft and radome configuration\\
\hspace*{1cm} ~ = \{-0.000983, (1/0.08189) $\text{�}$ \} for the
C-130\\
\hspace*{1cm} ~ = \{-0.0025, (1/0.04727) $\text{�}$\} for the GV

\rule[0.5ex]{1\linewidth}{1pt}

\[
\mathrm{SSRD=}s_{1}(\frac{\{BDIFR\}}{\{\mathrm{QCXC}\}}+s_{0})
\]
%
\end{minipage}}

\textbf{}%
\begin{comment}
The following is from old Bulletin 9:

\textbf{Sideslip Angle (Differential Pressure) (}$\text{�}$\textbf{):
}\textbf{\uline{SSDF}}\index{SSDF}\\
This variable is derived from BDIF and a dynamic pressure using an
empirically determined sensitivity function, as expressed in the following:\\
\fbox{\begin{minipage}[t]{0.9\textwidth}%
$M$ = Aircraft Mach Number\\
BDIF = differential pressure between sideslip pressure ports, 858
probe (mb)\\
QC = dynamic pressure from the 858 probe (mb)\\
$g_{r}$ = dimensionless sensitivity function (given below)
\begin{eqnarray*}
\mathrm{for\,M} & \leq & 0.515:\\
 & g_{r} & =0.079\\
\mathrm{otherwise}:\\
 & g_{r} & =0.086577797-0.03560256\,M\\
 &  & +0.00006143\,M^{2}
\end{eqnarray*}

\rule[0.5ex]{1\linewidth}{1pt}

\[
\mathrm{SSDF=}\frac{\mathrm{\{BDIF\}}}{g_{r}\mathrm{\{QC\}}}
\]
%
\end{minipage}}
\end{comment}

\textbf{Reference Sideslip Angle (}$\text{�}$\textbf{): }\textbf{\uline{SSLIP}}\sindex[var]{SSLIP}\index{SSLIP}\\
\emph{The reference sideslip angle used to calculate derived variables}.
This variable is the reference selected from other measurements of
sideslip angle\index{sideslip, angle of} in the data set. In most
projects, it is equal to SSRD\index{SSRD}. It is used where sideslip
angle is needed for other derived calculations (e.g., wind measurements). 

\end{hangparagraphs}


\subsubsection{Wind Components and the Wind Vector}
\begin{hangparagraphs}
\textbf{}%
\noindent\begin{minipage}[t]{1\columnwidth}%
\begin{hangparagraphs}
\textbf{Wind Vector East Component (m/s): }\textbf{\uline{UI\sindex[var]{UI}\index{UI}}}

\textbf{Wind Vector North Component (m/s): }\textbf{\uline{VI\sindex[var]{VI}\index{VI}}}

\textbf{Wind Vector Vertical Component (m/s): }\textbf{\uline{WI\sindex[var]{WI}\index{WI}}}
\end{hangparagraphs}

%
\end{minipage}\textbf{\uline{}}\\
\textbf{\uline{}}\\
\emph{The three-dimensional wind\index{wind}\index{wind!vector}\index{wind!components}
vector with respect to the earth, }as determined from the inertial
reference systems. UI is the east-west component with positive values
\uline{toward} the east, VI is the north-south component with positive
values \uline{toward} the north,\index{wind!sign convention} and
WI is the vertical component with positive values toward the zenith.\\
\\
The calculation of WI differs from the description in Bulletin 23
because the output from the inertial reference system is different
for the modern units now in use. The vertical wind is the sum of the
vertical gust component (represented approximately by TASX\,sin(ATTACK-PITCH))
and the motion of the aircraft as measured by VSPD (discussed in Section
\mbox{\pageref{sec:INS}}). Bulletin 23 describes the historical calculation
of the vertical motion of the aircraft via a barometric-inertial feedback
loop, but equivalent calculations (including pressure damping to the
pressure altitude) are incorporated into current IRS units so VSPD
already is the product of such a calculation. To calculate WI, VSPD
is therefore used in place of the obsolete variable WP3 that was discussed
in Bulletin 23.\\
\\
WIC should usually be used instead of WI because VSPD, entering WI,
is updated to the pressure altitude and so can have false variations
in baroclinic conditions. \\

\textbf{}%
\noindent\begin{minipage}[t]{1\columnwidth}%
\begin{hangparagraphs}
\textbf{Wind Speed (m/s): }\textbf{\uline{WS}}\textbf{\sindex[var]{WS}\index{WS}}

\textbf{Wind Direction (}$\text{�}$\textbf{): }\textbf{\uline{WD}}\sindex[var]{WD}\index{WD}
\end{hangparagraphs}

%
\end{minipage}\\
\\
\emph{The magnitude and direction of the horizontal wind. }These variables\index{wind!speed}\index{wind!direction}
are obtained in a straightforward manner from UI and VI. The resulting
wind direction is relative to true north and represents the direction
\uline{from which} the wind blows. That is the reason that 180$^{\circ}$
appears in the following algorithm.\\
\\
\fbox{\begin{minipage}[t]{0.95\textwidth}%
UI = easterly component of the horizontal wind\\
VI = northerly component of the horizontal wind\\
atan2 = 4-quadrant arc-tangent function producing output in radians
from -$\pi$ to $\pi$\\
$C_{rd}$\sindex[con]{Crd@$C_{rd}$= conversion factor, radians to degrees (180/$\pi$)}
= conversion factor, radians to degrees, = 180/$\pi$ {[}units: $^{\circ}/radian${]}\\
\rule[0.5ex]{1\linewidth}{1pt} 
\begin{eqnarray}
\mathrm{WS} & = & \sqrt{\mathrm{\{UI\}}^{2}+\{\mathrm{VI\}}^{2}}\nonumber \\
\mathrm{WD} & = & C_{rd}\mathrm{\,atan2(\{UI\},\,\{VI\})}+180^{\circ}\label{eq:9.1}
\end{eqnarray}
%
\end{minipage}}\\

\textbf{}%
\noindent\begin{minipage}[t]{1\columnwidth}%
\begin{hangparagraphs}
\textbf{Wind Vector Longitudinal Component (m/s): }\textbf{\uline{UX\sindex[var]{UX}\index{UX}}}

\textbf{Wind Vector Lateral Component (m/s): }\textbf{\uline{VY}}\sindex[var]{VY}\index{VY}
\end{hangparagraphs}

%
\end{minipage}\textbf{}\\
\textbf{}\\
\emph{The horizontal wind}\index{wind!vector}\emph{ vector relative
to the frame of reference attached to the aircraft.} UX is parallel
to the longitudinal axis and positive toward the nose.\index{wind!longitudinal}\index{wind!lateral}
VY is along the lateral axis and normal to the longitudinal axis;
positive is toward the port (or left) wing.\\

\textbf{}%
\noindent\begin{minipage}[t]{1\columnwidth}%
\begin{hangparagraphs}
\textbf{GPS-Corrected Wind Vector, East Component (m/s): }\textbf{\uline{UIC\sindex[var]{UIC}\index{UIC}}}

\textbf{GPS-Corrected Wind Vector, North Component (m/s): }\textbf{\uline{VIC}}\sindex[var]{VIC}\index{VIC}
\end{hangparagraphs}

%
\end{minipage}\\
\\
\index{wind!GPS-corrected} \emph{The horizontal wind}\index{wind!vector}
c\emph{omponents respectively }\emph{\uline{toward}}\emph{ the
east and }\emph{\uline{toward}}\emph{ the north.} They are derived
from measurements from an inertial reference unit (IRU) and a Global
Positioning System (GPS), as described in the discussion of VEW and
VNS above. They are calculated just as for UX and VY except that the
GPS-corrected values for the aircraft groundspeed are used in place
of the IRU-based values. They are considered ``corrected'' from
the original measurements from the IRU or GPS, as described in section
\ref{subsec:IRS/GPS}.\\

\textbf{Wind Vector, Vertical Component (m/s): }\textbf{\uline{WIC}}\sindex[var]{WIC}\index{WIC}\\
\emph{The component of the wind in the vertical direction. }This is
the standard calculation of vertical wind\index{wind!vertical}, obtained
from the difference between the measured vertical component of the
relative wind and the vertical motion of the aircraft (usually GGVSPD
in recent projects).\label{punch:4-13}\textbf{} This should be used
in preference to WI if the latter is present; see the discussion of
WP3 in section \ref{sec:INS}. Positive values are toward the zenith.\\

\textbf{}%
\noindent\begin{minipage}[t]{1\columnwidth}%
\begin{hangparagraphs}
\textbf{GPS-Corrected Wind Direction (}$\text{�}$\textbf{): }\textbf{\uline{WDC}}\sindex[var]{WDC}\index{WDC}

\textbf{GPS-Corrected Wind Speed (m/s): }\textbf{\uline{WSC\sindex[var]{WSC}}}\index{WSC}
\end{hangparagraphs}

%
\end{minipage}\\
\\
\emph{The direction and magnitude of the wind vector, }obtained by
combining measurements from GPS and IRU units. These variables\index{wind!GPS-corrected}
are obtained in a straightforward manner from UIC and VIC, using equations
analogous to (\ref{eq:9.1}) but with UIC and VIC as input measurements.
They are expected to be the preferred measurements of wind because
they combine the best features of the IRU and GPS measurements.\\

\textbf{}%
\noindent\begin{minipage}[t]{1\columnwidth}%
\begin{hangparagraphs}
\textbf{GPS-Corrected Wind Vector, Longitudinal Component (m/s): }\textbf{\uline{UXC}}\sindex[var]{UXC}\textbf{\uline{\index{UXC}}}

\textbf{GPS-Corrected Wind Vector, Lateral Component (m/s): }\textbf{\uline{VYC}}\sindex[var]{VYC}\index{VYC}
\end{hangparagraphs}

%
\end{minipage}\\
\\
\emph{The longitudinal and lateral components of the three-dimensional
wind}\index{wind!lateral component}\index{wind!longitudinal component}\emph{,
similar to UX and VY, but corrected by the complementary-filter algorithm
that combines IRU and GPS measurements}. See the discussion in Section
\ref{subsec:IRS/GPS}. The components UXC and VYC are toward the front
of the aircraft and toward the port (left) wing, respectively. 
\end{hangparagraphs}


\subsection{Special-Use Remote Sensors}

The above variables are normally included in the archived netCDF files
from projects, but there are a few remote sensors that provide additional
state-parameter measurements in some projects. These include:\label{subsec:MTP}

\noindent\fbox{\begin{minipage}[t]{1\columnwidth - 2\fboxsep - 2\fboxrule}%
\begin{itemize}
\item Microwave Temperature Profiler (\href{http://www.eol.ucar.edu/instruments/microwave-temperature-profiler}{MTP})
\textendash{} remotely sensed temperature profiles\index{MTP}
\item Dropsonde System (\href{http://www.eol.ucar.edu/instruments/avaps}{AVAPS})
\textendash{} profiles of temperature, humidity, and wind vs pressure.\index{dropsonde}
\item GPS-Occultation Sensor (\href{http://www.eol.ucar.edu/instruments/gnss-instrument-system-multi-static-and-occultation-sensing}{GISMOS})
\textendash{} atmospheric soundings of refractivity via GPS occultation.\index{GISMOS}
\end{itemize}
%
\end{minipage}} 

The links provided connect to descriptions of these instruments on
the EOL web site, and each provides a summary of how data are acquired
and processed. These measurements are not normally part of the archived
netCDF project files. Those interested in using these measurements
should contact EOL data management (\url{mailto:raf-dm@eol.ucar.edu})
for access to the measurements and for information on how the measurements
are processed.
