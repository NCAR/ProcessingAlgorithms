
\section{CLOUD PHYSICS VARIABLES}

\subsection{Measurements of Liquid Water Content}
\begin{hangparagraphs}
\textbf{Raw Output PMS/CSIRO (King) Liquid Water Content} (W):\textbf{
}\textbf{\uline{PLWC}}\sindex[var]{PLWC}\textbf{\index{PLWC},
}\textbf{\uline{PLWC1}}\index{PLWC1}\sindex[var]{PLWC1}\label{punch:5-3}\\
\emph{The power\index{King probe} dissipated by the sensor of a PMS/CSIRO
(King) liquid water\index{liquid water content} probe (in watts).
}PLWC is the power required to maintain constant temperature in a
heated element as that element is cooled by convection and evaporation
of impinging liquid water. The convective heat losses are determined
by calibration in dry air over a range of airspeeds and temperatures,
so that the remaining power can be related to the liquid water content.
The instrument is described in \href{http://www.eol.ucar.edu/raf/Bulletins/bulletin24.html}{RAF Bulletin 24}\index{Bulletin 24}
and at \href{http://www.eol.ucar.edu/instruments/king-csiro-liquid-water-sensor}{this URL}.
See \nameref{PLWCC} (which follows) for processing.\\

\textbf{\label{PLWCC}PMS/CSIRO (King) Liquid Water Content} (g/m$^{3}$):\textbf{
}\textbf{\uline{PLWCC}}\sindex[var]{PLWCC}\textbf{\index{PLWCC},
}\textbf{\uline{PLWCC1}}\index{PLWCC1}\sindex[var]{PLWC1}\\
\emph{The liquid water content}\index{liquid water content!King probe}
\emph{measured by a King probe.} This is calculated by relating the
power consumption required to maintain a constant temperature to the
liquid water content, taking into account the effect of convective
heat losses. The instrument and processing are described by King et
al. (1978)\footnote{King, W. D., D. A. Parkin and R. J. Handsworth, 1978 A hot-wire liquid
water device having fully calculable response characteristics. J.
Appl. Meteorol., 17, 1809\textendash 1813. See also Bradley, S. G.,
and W. D. King, 1979 Frequency response of the CSIRO Liquid Water
Probe. J. Appl. Meteorol., 18, 361\textendash 366.} and in a note available at \href{https://drive.google.com/open?id=0B1kIUH45ca5AUkFDWmRwci12eVU}{this URL}.
Because the temperature of the sensing wire is typically well above
the boiling point of water\index{water!boiling point}, the assumption
made in processing is that the water collected on the sensing wire
is vaporized at the boiling point $T_{b}$. The boiling point is represented
as a function of pressure as described below. \\
\\
\\
\\
{[}see the algorithmn box on the next page{]}\\
\vfill

~~\\
\fbox{\begin{minipage}[t]{0.95\textwidth}%
PLWC = total power dissipated by the probe (W)\\
$P_{D}$ = power dissipated\index{King probe!power dissipated}\sindex[lis]{Power@$P$=power}
by the cooling effect of dry air alone\\
$P_{W}$ = power needed to heat and vaporize the liquid water that
hits the probe element\\
$L$ = length\sindex[lis]{L@$L$ =length (of a King-probe element)}
of the probe sensitive element\index{King probe!element dimensions},
typically 0.021\,m\\
$d$= diameter\sindex[lis]{d@$d$=diameter} of the probe sensitive
element, typically 1.805$\times10^{-3}$m\\
$T_{s}$= sensor temperature\index{King probe!sensor temperature}\sindex[lis]{Ts@$T_{s}$=temperature of a sensor}
($^{\circ}$C)\\
$T_{a}$= ambient temperature ($^{\circ}$C) = ATX\\
$T_{b}$\sindex[lis]{Tb@$T_{b}$= boiling temperature of water} =
boiling temperature of water (dependent on pressure): 

~~~~~~~~~with $x=\log_{10}(p/(1$hPa)), $B=1^{\circ}C$,
and \{$b_{0},$ $b_{1}$, $b_{2}$, $b_{3}$\} = \{0.03366503, 1.34236135,
-0.33479451, 0.0351934\}: $T_{b}=B\times10^{(b_{0}+b_{1}x+b_{2}x^{2}+b_{3}x^{3})}$\\
$T_{m}=(T_{a}+T_{s})/2$ = mean temperature for air properties\\
$L_{v}(T_{b})$ = latent heat of vaporization of water\sindex[lis]{Lv@$L_{v}$=latent heat of vaporization of water}\index{latent heat of vaporization}
= (2.501-0.00237$T_{b}$)$\times10^{6}$J\,kg$^{-1}$\\
$c_{w}$\sindex[lis]{cw@$c_{w}$= specific heat of liquid water} =
specific heat of water = 4190~J\,kg$^{-1}$K$^{-1}$ (mean value,
0\textendash 90$^{\circ}$C)\\
$U_{a}$ = true airspeed (m/s) = TASX\\
$\lambda_{c}$\sindex[lis]{lambdac@$\lambda_{c}$= thermal conductivity, dry air}
= thermal conductivity\index{thermal conductivity|see {conductivity, thermal}}\index{conductivity!thermal}
of dry air (2.38+0.0071$T_{m}$)$\times10^{-2}$J\,m$^{-1}$s$^{-1}$K$^{-1}$\\
$\mu$\sindex[lis]{mu_{a}= dynamic viscosity of air@$\mu_{a}$= dynamic viscosity of air}
=viscosity of air =\index{viscosity} (1.718+0.0049$T_{m})\times10^{-5}$
kg\,~m$^{-1}$s$^{-1}$\\
$\rho_{a}$\sindex[lis]{rhoa@$\rho_{a}$= density of air} = density
of air = $p/(R_{d}(T_{a}+T_{0}))$\\
Re\sindex[lis]{Re= Reynolds number} = Reynolds number = $\rho_{a}U_{a}d/\mu_{a}$\\
Nu\sindex[lis]{Nu= Nusselt number} = Nusselt Number relating conduction
heat loss to the total heat loss for dry air: 

~~~~~~~~~~%
\fbox{\begin{minipage}[t]{0.9\textwidth}%
typically Nu=$a_{0}\mathrm{Re}^{a_{1}}$ where, for the GV, $\{a_{0},\,a_{1}\}=1.868,\,0.343$
for Re<7244 and $\{0.135,\,0.638\}$ otherwise, except when TASX <
150 m/s; then use $\{0.133,\,0.382\}$. For the C-130 $\{a_{0},\,a_{1}\}=\{0.118,\,0.675\}$.%
\end{minipage}}, \\
$C_{kg2g}=1000$\sindex[lis]{Ckg2g@$C_{kg2g}$= conversion factor, g to kg}
= grams per kilogram 

~~~~~~~~~~(unit conversion to conventional units for liquid
water content)\\
$\chi$ \sindex[lis]{chi@$\chi$=liquid water content}= liquid water
content (g/m$^{3}$) = PLWCC

\rule[0.5ex]{1\linewidth}{1pt}
\[
\mathrm{PLWC}=P_{D}+P_{W}
\]
where
\[
P_{D}=\pi\mathrm{Nu}\,L\lambda_{c}(T_{s}-T_{a})
\]
\[
P_{W}=L\,d[L_{v}(T_{b})+c_{w}(T_{b}-T_{a})]\,U_{a}\chi
\]
Result:
\[
\mathrm{PLWCC}=\chi=\frac{C_{kg2g}(\mathrm{\{PLWC\}}-P_{D})}{L\,d\,U_{a}[L_{v}(T_{b})+c_{w}(T_{b}-T_{a})]}
\]
%
\end{minipage}}\\
\\
In addition, a processing step is used to remove drift by calculating
the offset required to zero measurements obtained outside cloud. This
is done by adjusting the coefficient $a_{0}$ by nudging toward the
value required to give zero liquid water content outside cloud (as
indicated by another instrument, often a CDP showing droplet concentration
of <1 cm$^{-3}$). Specifically, when out-of-cloud, Nu$^{\prime}$
is calculated from Nu$^{\prime}=\mathrm{\{PLWC\}}/(\pi L\lambda_{c}(T_{s}-T_{a}))$.
Then the value of $a_{0}$ is updated via $a_{0}$ += (Nu$^{\prime}/Re^{a_{1}}-a_{0})/\tau$
(using, for the GV, separate coefficients for each of the three branches).
In this formula, $\tau$ should be the number of updates in a fixed
period, e.g., for a 100 s time constant and for 25-Hz processing,
$\tau=100\times25$. In addition, to avoid jumps when switching among
the branches, the linear coefficients \{$a_{0}$\} are adjusted with
each transition between branches to provide a continuous estimate
of the zero value.

\textbf{PVM-100 Liquid Water Content} (g/m$^{-3}$):\textbf{ }\textbf{\uline{PLWCG}}\textbf{\sindex[var]{PLWCG}}\index{PLWCG}\\
\emph{Cloud liquid water content for cloud droplets in the approximate
size range from 3\textendash 50 $\mu$m. }The PVM produces a measure
of the liquid water content directly, but a baseline value is sometimes
subtracted by reference to another cloud droplet instrument such as
an FSSP or CDP, such that when the other instrument measures a very
low droplet concentration the baseline value for the PVM-100 is updated
at the corresponding time and that average is then subtracted from
the measurements directly produced by the PVM-100.\label{punch:5-1}

\textbf{Rosemount Icing Detector Signal} (V):\textbf{ }\textbf{\uline{RICE}}\sindex[var]{RICE}\index{RICE}\\
\emph{The voltage related to loading on the element of a Rosemount
871F\index{Rosemount 871F icing probe} ice-accretion probe.} This
instrument (see \href{http://www.eol.ucar.edu/instruments/rosemount-icing-probe}{this URL})
consists of a rod set in vibration by a piezoelectric crystal. The
oscillation frequency of the probe changes with ice loading, so in
supercooled cloud ice accumulates on the sensor and the change in
oscillation frequency is transmitted as a DC voltage. When the rod
loads to a trigger point, the probe heats the rod to remove the ice.
The rate of voltage change can be converted to an estimate of the
supercooled liquid water content\index{liquid water content!supercooled},
as described in connection with the obsolete variable SCLWC. This
calculation is no longer provided routinely but can be duplicated
by a user on the basis of the SCLWC algorithm (see page \pageref{SCLWC})
or one of several other published algorithms.
\end{hangparagraphs}


\subsection{Sensors of Individual Particles (1-D Probes)\label{subsec:1DProbes}}

The RAF operates a set of hydrometeor detectors\index{hydrometeor detector}
that provide single-dimension measurements (i.e., not images) of individual
particle sizes. \href{http://www.eol.ucar.edu/raf/Bulletins/bulletin24.html}{RAF Bulletin 24}
contains extensive information on the operating principles and characteristics
of some of the older instruments. Here the focus will be on the meanings
of the variables in the archived data files.\index{Bulletin 24}\label{punch:5-2}

\label{VariableNames1DProbes}Four- and five-character variable names\index{variable names!hydrometeor probes}
shown in this section are generic. The actual names appearing in NIMBUS-generated
production output data sets have appended to them an underscore (\_)
and three or four more characters that indicate a probe's specific
aircraft mounting location. For example, AFSSP\_RPI refers to a variable
from an FSSP-100 probe mounted on the inboard, right-side pod. The
codes presently in use are given in the following table. For the GV,
there are 12 locations available, characterized by three letters.
The first is the wing (\{L,R\} for \{port,starboard\}), the second
is the pylon (\{I,M,O\} for inboard, middle, outboard\}), the third
is which of the two possible canister locations at the pylon is used
(\{I,O\} for \{inboard, outboard\}).

\begin{center}
\begin{tabular}{|c|c|c|}
\hline 
\textbf{Code} & \textbf{Location} & \textbf{Aircraft}\tabularnewline
\hline 
\hline 
OBL  & Outboard Left  & C-130Q \tabularnewline
\hline 
IBL  & Inboard Left  & C-130Q \tabularnewline
\hline 
OBR  & Outboard Right  & C-130Q \tabularnewline
\hline 
IBR  & Inboard Right  & C-130Q \tabularnewline
\hline 
LPO  & Left Pod Outboard  & C-130Q \tabularnewline
\hline 
LPI  & Left Pod Inboard  & C-130Q \tabularnewline
\hline 
LPC  & Left Pod Center  & C-130Q \tabularnewline
\hline 
RPO  & Right Pod Outboard  & C-130Q \tabularnewline
\hline 
RPI  & Right Pod Inboard  & C-130Q \tabularnewline
\hline 
RPC  & Right Pod Center  & C-130Q \tabularnewline
\hline 
OBL  & Left Wing  & Electra \tabularnewline
\hline 
IBL  & Left Pylon  & Electra \tabularnewline
\hline 
WDL  & Window Left  & Electra \tabularnewline
\hline 
OBR  & Right Wing  & Electra \tabularnewline
\hline 
IBR  & Right Pylon  & Electra \tabularnewline
\hline 
WDR  & Window Right  & Electra \tabularnewline
\hline 
\{L,R\}W\{I,M,O\}\{I,O\} & see discussion above & GV\tabularnewline
\hline 
\end{tabular}
\par\end{center}

The probe type also is coded into each variable's name, sometimes
using four characters, sometimes only one: FSSP-100 (FSSP or F), FSSP-300
(F300 or 3), CDP (CDP or D), UHSAS (UHSAS or U), PCASP (PCAS or P),
OAP-200X (200X or X), OAP-260X (260X or 6) and OAP-200Y (200Y or Y).
Prefix letters are used to identify the type of measurement (A=accumulated
particle counts per time interval per channel, C = concentration per
channel, CONC = Concentration from all channels, DBAR = mean diameter,
DISP = dispersion, PLWC =liquid water content, DBZ = radar reflectivity
factor).

Some of the probes discussed in this section are primarily aerosol
spectrometers but are described here, rather than in Section \ref{sec:AEROSOL-PARTICLE-MEASUREMENTS},
because they are similar to the hydrometeor spectrometers and so are
most economically discussed here. However, see Sect.~\ref{sec:AEROSOL-PARTICLE-MEASUREMENTS}
for the processing algorithms that lead to concentrations from the
UHSAS, PCASP, and F300. The following table and discussion includes
some obsolete variables (for the 200X and 200Y) for the same reason.
The table also includes some variables derived from imaging spectrometers
(the 2DC and 2DP probes) to highlight that the primary variables are
similar to those discussed in this sub-section. Those variables are
discussed in the next sub-section. In two cases, the FSSP and PCASP,
two versions are listed, an obsolete version and a current version
with a revised processing package (SPP-100 for the FSSP, SPP-200 for
the PCASP). Both are included for historical completeness, but algorithms
in this document discuss the current versions.

The archived data files sometimes have ``housekeeping'' variables\index{housekeeping variables}
included that provide information on the operating state and data
quality from the probes. For example, the CDP provides information
on the average transit time, the voltage from the nominal 5-V source,
the control board temperature, the laser block temperature, the laser
current, the laser power monitor, the qualifier bandwidth, the qualifier
baseline, the qualifier threshold, the sizer baseline, the wing-board
temperature, an A-to-D overflow flag, and a count of particles rejected
as being outside the depth of field. The netCDF variables and attributes
should be consulted for this housekeeping information. The large number
of housekeeping variables will not be included in this document, so
appropriate manuals and the netCDF files should be consulted when
interpreting this information.

\index{hydrometeor probes!table of}

\begin{center}
\noindent\begin{minipage}[t]{1\columnwidth}%
\begin{center}
\textbf{Probes that produce size distributions of particles (with
links to descriptions):\label{TableOfProbes}}
\par\end{center}
\begin{center}
\begin{tabular}{|c|c|c|c|c|c|c|}
\hline 
\textbf{\small{}Generic Name} &  & \textbf{\small{}Probe} & \textbf{\small{}Channels} & \textbf{\small{}Usable}\footnote{Channels may be unusable because the first channel is a historical
carry-over and should be ignored, or because in the case of 2D probes
the entire-in sizing technique reduces the number of bins where particles
can be sized. Also, when some channels have been considered unreliable
the netCDF header may specify that the usable bins are smaller than
indicated here.} & \textbf{\small{}Diameter Range} & \textbf{\small{}Bin Width}\tabularnewline
\hline 
\hline 
{\small{}FSSP-100/original} & {\small{}F} & {\small{}\href{http://www.eol.ucar.edu/raf/Bulletins/B24/fssp100.html}{FSSP-100}}\footnote{Now obsolete but present in many archived data sets} & {\small{}0\textendash 15} & {\small{}1-{}-15} & \multicolumn{2}{c|}{{\small{}(See FRNG below)}}\tabularnewline
\hline 
FSSP/SPP-100 & F & \href{http://www.eol.ucar.edu/instruments/forward-scattering-spectrometer-probe-model-100}{SPP-100} & 0\textendash 30 & 1-{}-30 & \multicolumn{1}{c||}{3\textendash 45 $\mu$m } & 3 $\mu$m (typ.)\tabularnewline
\hline 
UHSAS & U & \href{https://www.eol.ucar.edu/instruments/ultra-high-sensitivity-aerosol-spectrometer}{UHSAS} & 0\textendash 99 & 1-{}-99 & \multicolumn{1}{c||}{0.06\textendash 1.0 $\mu$m} & variable\tabularnewline
\hline 
CDP & D & \href{https://www.eol.ucar.edu/instruments/cloud-droplet-probe}{CDP} & 0\textendash 30 & 1-\textendash 30 & \multicolumn{1}{c||}{2.0\textendash 50} & variable\tabularnewline
\hline 
{\small{}F300} & {\small{}3} & {\small{}\href{https://www.eol.ucar.edu/instruments/forward-scattering-spectrometer-probe-model-300}{FSSP-300}$^{b}$} & {\small{}0\textendash 30} & {\small{}1-{}-30} & {\small{}0.3\textendash 20.0 $\mu m$} & {\small{}variable}\tabularnewline
\hline 
PCASP/original & P & \href{http://www.eol.ucar.edu/raf/Bulletins/B24/pcasp100.html}{PCASP}$^{b}$ & 0\textendash 15 & 1-{}-15 & 0.1\textendash 3.0 $\mu$m & variable\tabularnewline
\hline 
PCASP/SPP-200 & P & {\small{}\href{https://www.eol.ucar.edu/instruments/signal-processing-package-200-passive-cavity-aerosol-spectrometer-probe}{SPP-200}} & {\small{}0\textendash 31} & {\small{}1-{}-31} & {\small{}0.1\textendash 3.0 $\mu m$} & {\small{}variable}\tabularnewline
\hline 
{\small{}200X} & {\small{}X} & OAP-200X{\small{}$^{b}$} & {\small{}0\textendash 15} & {\small{}1-{}-15} & {\small{}40\textendash 280 $\mu m$} & {\small{}10 $\mu m$}\tabularnewline
\hline 
{\small{}260X} & {\small{}6} & \href{http://www.eol.ucar.edu/raf/Bulletins/B24/260X.html}{OAP-260X} & {\small{}0-63} & {\small{}3-{}-62} & {\small{}40-620 $\mu m$} & {\small{}10 $\mu m$}\tabularnewline
\hline 
{\small{}200Y} & {\small{}Y} & {\small{}OAP-200Y$^{b}$} & {\small{}0-15} & {\small{}1-{}-15} & {\small{}300\textendash 4500 $\mu m$} & {\small{}300 $\mu m$}\tabularnewline
\hline 
1DC\footnote{See p.~\pageref{Despite-the-'1D'} for an explanation of this name
convention}  &  & \href{http://www.eol.ucar.edu/raf/Bulletins/B24/2dProbes.html}{2DC}$^{b,}$\footnote{Measurements from this and the next three 2D probes are discussed
in section \vref{subsec:Hydrometeor-Imaging-Probes}} (old) & 0-32 & 1-30$^{e}$ & 25\textendash 800 $\mu$m  & 25 $\mu$m\tabularnewline
\hline 
1DP &  & \href{http://www.eol.ucar.edu/raf/Bulletins/B24/2dProbes.html}{2DP}$^{b}$
(old) & 0-32 & 1-30 & 200\textendash 6400 $\mu$m & 200 $\mu$m\tabularnewline
\hline 
1DC  &  & \href{https://www.eol.ucar.edu/instruments/two-dimensional-optical-array-cloud-probe}{2DC}
(fast) & 0-63 & 1-62\footnote{Some of the lowest channels are often considered unreliable and excluded
in processing} & 25\textendash 1600 $\mu$m & 25 $\mu$m\tabularnewline
\hline 
1DP &  & \href{https://www.eol.ucar.edu/instruments/two-dimensional-optical-array-precipitation-probe}{2DP}
(new) & 0-63 & 1-62 & 100\textendash 6400 $\mu$m & 100 $\mu$m\tabularnewline
\hline 
\end{tabular}
\par\end{center}%
\end{minipage}\\
\par\end{center}
\begin{hangparagraphs}
\textbf{Count Rate Per Channel }(number per time interval):\textbf{\uline{\label{punch:5-4}\label{punch:5-7}}}\\
\textbf{\uline{AFSSP}}\textbf{\sindex[var]{AFSSP}\index{AFSSP},
}\textbf{\uline{AS100\index{AS100}}}\sindex[var]{AS100}\textbf{\uline{,
AF300}}\sindex[var]{AF300}\textbf{\index{AF300}, }\textbf{\uline{APCAS}}\textbf{\sindex[var]{APCAS}\index{APCAS},
}\textbf{\uline{A200X}}\textbf{\sindex[var]{A200X}\index{A200X},
}\textbf{\uline{A260X}}\textbf{\sindex[var]{A260X}\index{A260X},
}\textbf{\uline{A200Y}}\sindex[var]{A200Y}\index{A200Y}, \textbf{\uline{ACDP}}\sindex[var]{ACDP}\index{ACDP},
\textbf{\uline{AUHSAS\sindex[var]{AUHSAS}\index{AUHSAS}}}\\
\emph{The size distribution of the number of particles detected by
a 1D hydrometeor probe per unit time.} These measurements have ``vector''
character in the NetCDF\index{NetCDF!vector} output files, with dimension
equal to the number of channels in the table above and with one entry
per channel. The first element in the vector is a historical remnant
from a time when housekeeping information was stored here and should
be ignored. For the size limits of the channels, see the netCDF attributes
of the following variables for ``Size Distribution''.

\textbf{}%
\noindent\begin{minipage}[t]{1\columnwidth}%
\textbf{Size Distribution} ($\mathrm{cm}{}^{-3}$channel$^{-1}$)\textbf{:
}\textbf{\uline{CFSSP}}\textbf{\index{CFSSP}\sindex[var]{CFSSP},
}\textbf{\uline{CS100}}\textbf{\index{CS100}}\sindex[var]{CS100}\textbf{,
}\textbf{\uline{CF300}}\textbf{\index{CF300}\sindex[var]{CF300},
}\textbf{\uline{CPCAS\index{CPCAS}}}\textbf{\sindex[var]{CPCAS},
}\textbf{\uline{CCDP\index{CCDP}\sindex[var]{CCDP}}}\textbf{,
}\textbf{\uline{CUHSAS\index{CUHSAS}\sindex[var]{CUHSAS}}}\\
\textbf{Size Distribution }(L\textbf{$^{-1}$}channel$^{-1}$):\textbf{
}\textbf{\uline{C200X}}\textbf{\index{C200X}\sindex[var]{C200X},
}\textbf{\uline{C260X}}\textbf{\index{C260X}\sindex[var]{C260X},
}\textbf{\uline{C200Y\index{C200Y}\sindex[var]{C200Y}}}%
\end{minipage}\textbf{\label{CUHSAS}}\\
\emph{The particle concentrations}\index{concentration!hydrometeor}\index{concentration!hydrometeor, size distribution}\index{FSSP-100!size distribution}
\emph{in each usable bin of the probe.} These netCDF variables have
``vector'' character with dimension equal to the number of channels
in the table above. The first vector member should be ignored. For
some scattering spectrometer\index{spectrometer!hydrometeor}\index{hydrometeor spectrometer}
probes (FSSP-100, FSSP-300\index{FSSP-300}, PCASP\index{PCASP})
the concentration value is modified by the probe activity (FACT, PACT)
as described below. The concentration is obtained from the total number
of particles detected and a calculated, probe-dependent sample volume
that is specified in recent projects by attributes (e.g., depth of
field and beam diameter) of this variable in the netCDF file. For
additional details, see the links in the table \vpageref{TableOfProbes}
or, for older probes, \href{http://www.eol.ucar.edu/raf/Bulletins/bulletin24.html}{RAF Bulletin 24}\index{Bulletin 24}.

\begin{singlespace}
\textbf{}%
\noindent\begin{minipage}[t]{1\columnwidth}%
\begin{hangparagraphs}
\begin{singlespace}
\textbf{Concentration }(cm$^{-3}$):\textbf{ }\textbf{\uline{CONCF}}\textbf{\sindex[var]{CONCF}\index{CONCF},
}\textbf{\uline{CONC3}}\sindex[var]{CONC3}\textbf{\index{CONC3},
}\textbf{\uline{CONCP\sindex[var]{CONCP}\index{CONCP}, CONCD\sindex[var]{CONCD}\index{CONCD}}}\textbf{,
}\textbf{\uline{CONCU\sindex[var]{CONCU}\index{CONCU}}}

\textbf{Concentration (}L\textbf{$^{-1}$): }\textbf{\uline{CONCX}}\textbf{\sindex[var]{CONCX}\index{CONCX},
}\textbf{\uline{CONC6}}\sindex[var]{CONC6}\textbf{\index{CONC6},
}\textbf{\uline{CONCY\sindex[var]{CONCY}\index{CONCY}}}
\end{singlespace}
\end{hangparagraphs}

%
\end{minipage}\textbf{}\\
\emph{The particle concentrations}\index{FSSP-100!concentration}\index{concentration!FSSP}
\emph{summed over all channels to give the total concentration in
the size range of the probe.} For example, \{CONCF\} = $\sum_{i}\mathrm{\{CFSSP\}_{i}}$.
For additional details, see \href{http://www.eol.ucar.edu/raf/Bulletins/bulletin24.html}{RAF Bulletin 24}.\\

\end{singlespace}

\textbf{Mean Diameter }($\mu m$):\textbf{ }\textbf{\uline{DBARF}}\textbf{\index{DBARF},
}\textbf{\uline{DBAR3}}\textbf{\index{DBAR3}, }\textbf{\uline{DBARP}}\textbf{\index{DBARP},
}\textbf{\uline{DBARX}}\textbf{\index{DBARX}, }\textbf{\uline{DBAR6}}\textbf{\index{DBAR6},
}\textbf{\uline{DBARY}}\index{DBARY}, \textbf{\uline{DBARD}}\index{DBARD},
\textbf{\uline{}}\\
\textbf{\uline{DBARU}}\index{DBARU}\\
\emph{The arithmetic average of all measured particle diameters from
a particular probe.}\index{diameter!mean, 1D probes}\index{FSSP-100!mean diameter}
This mean is calculated as follows: \\
\\
\fbox{\begin{minipage}[t]{0.9\textwidth}%
\{Cy$_{i}$\} = concentration\sindex[lis]{Cyi@$Cy_{i}$= concentration from hydrometeor probe y in channel i}
from probe y in channel i\label{MeanDiameter}\\
\hspace*{0.6cm}(e.g., y=FSSP to calculate DBARF)\\
i1 = lowest usable channel for the probe\\
i2 = highest usable channel for the probe\\
$d_{i}$\sindex[lis]{di@$d_{i}$= diameter of hydrometeor in channel $i$}
= diameter of particles in channel i for this probe ($\mu m$)\\
\hspace*{0.6cm}(calculated as the average of the lower and upper
size limits for the channel)\\
\rule[0.5ex]{1\linewidth}{1pt}

\begin{equation}
\mathrm{\{DBARx\}}=\frac{{\textstyle \sum_{i=i1}^{i2}}{\displaystyle {\displaystyle \left\{ \mathrm{Cy}_{i}\right\} d_{i}}}}{\sum_{i=i1}^{i2}\left\{ \mathrm{Cy}_{i}\right\} }\label{eq:dbar1d}
\end{equation}
%
\end{minipage}}\\

\textbf{Dispersion }(dimensionless)\textbf{: }\textbf{\uline{DISPF}}\textbf{\sindex[var]{DISPF}\index{DISPF},
}\textbf{\uline{DISP3}}\textbf{\sindex[var]{DISP3}\index{DISP3},
}\textbf{\uline{DISPP}}\textbf{\sindex[var]{DISPP}\index{DISPP},
}\textbf{\uline{DISPX}}\textbf{\sindex[var]{DISPX}\index{DISPX},
}\textbf{\uline{DISP6}}\textbf{\sindex[var]{DISP6}\index{DISP6},
}\textbf{\uline{DISPY}}\sindex[var]{DISPY}\index{DISPY}, \textbf{\uline{DISPD}}\sindex[var]{DISPD}\index{DISPD},
\textbf{\uline{}}\\
\textbf{\uline{DISPU}}\sindex[var]{DISPU}\index{DISPU}\\
\index{dispersion}\emph{The ratio of the standard deviation of particle
diameters to the mean particle diameter.}\index{FSSP-100!dispersion}\index{dispersion!1D probes}
\\
\\
\fbox{\begin{minipage}[t]{0.9\textwidth}%
\{DBARx\} = mean particle diameter ($\mu m$)\\
\{Cy$_{i}$\}, i1, i2, $d_{i}$ as for mean diameter \mbox{\vpageref{MeanDiameter}}\\
\rule[0.5ex]{1\linewidth}{1pt}

\begin{equation}
\mathrm{\{DISPx\}}==\frac{1}{\{\mathrm{DBARx}\}}\,\left\{ \frac{{\textstyle \sum_{i=i1}^{i2}}{\displaystyle {\displaystyle \left\{ \mathrm{Cy}_{i}\right\} d_{i}^{2}}}}{\sum_{i=i1}^{i2}\left\{ \mathrm{Cy}_{i}\right\} }-\{\mathrm{DBARx}\}^{2}\right\} ^{1/2}\label{eq:disp1d}
\end{equation}
%
\end{minipage}}

\textbf{Liquid Water Content }(g\,m\textbf{$^{-3}$}):\textbf{ }\textbf{\uline{PLWCF}}\textbf{\sindex[var]{PLWCF}\index{PLWCF},
}\textbf{\uline{PLWCX}}\textbf{\sindex[var]{PLWCX}\index{PLWCX},
}\textbf{\uline{PLWC6}}\textbf{\sindex[var]{PLWC6}\index{PLWC6},
}\textbf{\uline{PLWCY}}\sindex[var]{PLWCY}\index{PLWCY}, \textbf{\uline{PLWCD}}\sindex[var]{PLWCD}\index{PLWCD}\\
\index{liquid water content!1D probes}\emph{The density of liquid
water represented by the size distribution measured by a hydrometeor
probe.}\index{liquid water content}\index{FSSP-100!liquid water content}\index{liquid water content!FSSP-100}
These variables are calculated from the measured concentration (CONCx)
and the third moment of the particle diameter, with the assumption
that the particle is a water drop. The following box describes the
calculation in terms of an equivalent droplet diameter\index{diameter!equivalent},
the diameter that represents the mass in the detected particle. The
equivalent droplet diameter is normally the measured diameter for
liquid hydrometeors, but some processing has used other assumptions
and this is a choice that can be made based on project needs. Using
this definition allows for the approximate estimation of \index{ice water content}ice
water content in cases where it is known that all hydrometeors are
ice. \\
:\\
\fbox{\begin{minipage}[t]{0.9\textwidth}%
$d_{e,i}$\sindex[lis]{dei@$d_{e,i}$= equivalent melted diameter for channel i of a hydrometeor
spectrometer} = equivalent melted diameter for channel $i$ of probe x\\
\{Cy$_{i}$\}, i1, i2 as for mean diameter \vpageref{MeanDiameter}\\
$\varrho_{w}$\sindex[lis]{rhow@$\rho_{w}$= density of liquid water}
= density of water ( $10^{3}kg/m^{3}$)\\
\rule[0.5ex]{1\linewidth}{1pt}
\begin{equation}
\mathrm{\{PLWCx\}}=\frac{\pi\varrho_{w}}{6}{\textstyle \sum_{i=i1}^{i2}}{\displaystyle {\displaystyle \left\{ \mathrm{Cy}_{i}\right\} d_{e,i}^{3}}}\label{eq:lwc1d}
\end{equation}
(units and a scale factor are selected so that the output variable
is in units of g\,m$^{-3}$)%
\end{minipage}}\\

\textbf{Radar Reflectivity Factor }(dbZ)\textbf{: }\textbf{\uline{DBZF}}\textbf{\sindex[var]{DBZF}\index{DBZF},
}\textbf{\uline{DBZX}}\textbf{\sindex[var]{DBZX}\index{DBZX},
}\textbf{\uline{DBZ6}}\textbf{\sindex[var]{DBZ6}\index{DBZ6},
}\textbf{\uline{DBZY}}\sindex[var]{DBZY}\index{DBZY}, \textbf{\uline{DBZD\sindex[var]{DBZD}\index{DBZD}}}\\
\emph{The radar reflectivity factor}\index{reflectivity factor}\index{dBz}\index{radar reflectivity factor!1D probes}
\emph{calculated from the measured size distribution from a hydrometeor
probe.} This is calculated from the measured concentration and the
sixth moment of the size distribution, with the assumption that the
particles are water drops. An equivalent radar reflectivity factor
can be calculated from the hydrometeor size distribution if another
assumption is made about composition of the particles, but this variable
is not part of normal data files. The radar reflectivity factor is
a characteristic only of the hydrometeor size distribution; it is
\emph{not }a measure of radar reflectivity, because the latter also
depends on wavelength, dielectric constant, and other characteristics
of the hydrometeors. The normally used radar reflectivity factor is
measured on a logarithmic scale that depends on a particular choice
of units, so (although it is not conventional) an appropriate scale
factor $Z_{r}$ is included in the following equation to satisfy the
convention that arguments of logarithms should be dimensionless\index{dimensions in equations}.
\\
\\
\fbox{\begin{minipage}[t]{0.9\textwidth}%
$d_{i}$ = diameter for channel $i$ of probe x\\
\{Cy$_{i}$\}, i1, and i2 as for mean diameter \vpageref{MeanDiameter}\\
$Z_{r}$ \sindex[lis]{Zr@$Z_{r}$= scale factor for calculation of the radar reflectivity
factor}= reference factor for units = 1~mm$^{6}$m$^{-3}$\\
\rule[0.5ex]{1\linewidth}{1pt}

\begin{equation}
\mathrm{\{DBZx\}}=10\log_{10}\left({\textstyle \frac{1}{Z_{r}}\sum_{i=i1}^{i2}}{\displaystyle {\displaystyle \left\{ \mathrm{Cy}_{i}\right\} d_{i}^{6}}}\right)\label{eq:dbz1d}
\end{equation}
%
\end{minipage}}

\label{punch:5-5}

\textbf{Effective Radius }($\mu$m): \textbf{\uline{REFFD}}\index{REFFD}\sindex[var]{REFFD}\textbf{\uline{,
REFFF}}\index{REFFF}\sindex[var]{REFFF}\\
\index{radius, effective}\emph{One-half the ratio of the third moment
of the diameter measurements to the second moment.} This variable
is useful in some calculations that relate the liquid water content
of a cloud layer to its optical properties.\\
\fbox{\begin{minipage}[t]{0.9\textwidth}%
$d_{i}$ = diameter for channel $i$ of probe x\\
\{Cy$_{i}$\}, i1, and i2 as for mean diameter \vpageref{MeanDiameter}\\
\rule[0.5ex]{1\linewidth}{1pt}

\begin{equation}
\mathrm{\{REFFx\}}=\frac{1}{2}\frac{\sum{\displaystyle {\displaystyle \left\{ \mathrm{Cy}_{i}\right\} d_{i}^{3}}}}{\sum{\displaystyle {\displaystyle \left\{ \mathrm{Cy}_{i}\right\} d_{i}^{2}}}}\label{eq:reff1d}
\end{equation}
%
\end{minipage}}

\textbf{FSSP-100 Range }(dimensionless):\textbf{ }\textbf{\uline{FRNG}}\textbf{\sindex[var]{FRNG}\index{FRNG},
}\textbf{\uline{FRANGE}}\sindex[var]{FRANGE}\index{FRANGE}\\
\emph{The size range in use for the FSSP-100}\index{FSSP-100!range}\index{range!FSSP}
\emph{probe}.

\begin{minipage}[t]{0.9\textwidth}%
\hspace*{0.7in}%
\begin{tabular}{|c|c|c|}
\hline 
Range & \textbf{Nominal Size Range} & \textbf{Nominal Bin Width}\tabularnewline
\hline 
\hline 
0 & 2\textendash 47 $\mu m$ & 3 $\mu m$\tabularnewline
\hline 
1 & 2\textendash 32 $\mu m$ & 2 $\mu m$\tabularnewline
\hline 
2 & 1\textendash 15 $\mu m$ & 1 $\mu m$\tabularnewline
\hline 
3 & 0.5\textendash 7.5 $\mu m$ & 0.5 $\mu m$\tabularnewline
\hline 
\end{tabular}%
\end{minipage}\\
\\
In recent NETCDF data files, the actual bin boundaries used for processing
are recorded in the header. That header should be consulted because
processing sometimes uses non-standard sizes selected to adjust for
Mie scattering, which causes departures from the nominal linear bins.
Recent projects have all used range 0, but other choices have been
made in some older projects and other ranges are still available to
future projects.
\end{hangparagraphs}


\subsection{Hydrometeor Imaging Probes\label{subsec:Hydrometeor-Imaging-Probes}}

Instruments used to obtain hydrometeor images include the two-dimensional
imaging probes (\href{https://www.eol.ucar.edu/instruments/two-dimensional-optical-array-cloud-probe}{2DC}
and \href{https://www.eol.ucar.edu/instruments/two-dimensional-optical-array-precipitation-probe}{2DP})
and some others that require special processing and separate data
records. The former are described in this subsection. The latter include
a three-view cloud particle imager (\href{https://www.eol.ucar.edu/instruments/three-view-cloud-particle-imager}{3V-CPI}),
a small ice detector (\href{https://www.eol.ucar.edu/instruments/small-ice-detector-version-2-hiaper}{SID-2H}),
and a holographic imager (\href{https://www.eol.ucar.edu/instruments/holographic-detector-clouds}{HOLODEC}).
For information regarding use of data from the latter set of instruments,
consult EOL/RAF data management via \href{mailto:mailto:raf-dm@eol.ucar.edu}{this email address}.

In addition to the standard processing that produces the variables
in this subsection, an alternate processor is available that provides
some additional options and capabilities, including the production
of two sets of variables that include either all particles or all
particles that pass a roundness test. Additional options include different
ways of defining the particle size (including circle fitting or sizing
based on the dimension along the direction of flight. Corrections
to sizing are made to account for diffraction, and a shattering correction
can be applied based on interarrival times. Some categories of spurious
images (e.g., ``streakers'') can be recognized and rejected. This
processing is described in \href{https://drive.google.com/open?id=0B1kIUH45ca5AOFpZUGxGVUg3VWc}{this document}
and at \href{https://www.eol.ucar.edu/software/process2d}{this web page}
and is made available by special arrangement. 

Measurements based on the two 2D probes will be discussed together
in this section because the 2DC and 2DP probes function similarly,
differing primarily in the size resolution (typically 25 $\mu$m or
less for the 2DC and 100 or 200 $\mu$m for the 2DP). The following
variables have names like CONC1DC or CONC1DP to designate the two
types of hydrometeor imagers. In addition, variables normally have
location designations like '\_LWIO' as described at the beginning
of section \prettyref{subsec:1DProbes}; see page \pageref{VariableNames1DProbes}.
In the following 'y' is sometimes used to designate either 'C' or
'P'.

For the images from the 2D probes, separate data files need to be
used. RAF provides a routine ``\href{https://www.eol.ucar.edu/software/xpms2d}{XPMS2D}''
that can be used to view the images and calculate various properties
of the hydrometeor population based on these separate  files.

\label{Despite-the-'1D'}Despite the '1D' nomenclature, the following
variables are measured by 2D instruments; the '1D' designation is
used to indicate that this is the dimension that would be sized by
an equivalent 1D probe using a test that requires unshadowed end diodes
so that the full dimension of the particle can be determined. As a
consequence, the effective sample volume becomes smaller as the measured
dimension increases.
\begin{hangparagraphs}
\textbf{2D Count Rate Per Channel }(count per time interval)\textbf{:
}\textbf{\uline{A1DC}}\index{A1DC}\sindex[var]{AIDC}, \textbf{\uline{A1DP}}\index{A1DP}\sindex[var]{AIDP}\\
\index{count rate!2D}\emph{The number of particles counted by a 2D
probe in each of 62 size bins in a specified time interval, usually
1 s.} These are used to calculate the derived variables like CONC1DC,
C1DC, and others that follow, but are provided to allow re-calculation
if a user wants to use different sample volumes or sizing assumptions.\\

\textbf{2D Size Distribution }(L$^{-1}$channel$^{-1}$):\textbf{
}\textbf{\uline{C1DC}}\index{C1DC}\sindex[var]{C1DC}, \textbf{\uline{C1DP}}\index{C1DP}\sindex[var]{C1DP}\\
\index{size distribution!2D}\emph{The concentration of particles
measured by a 2D probe in each of 62 bins in a specified time interval,
usually 1 s.} These are calculated from A1DC by application of an
assumed size-dependent sample volume based on probe characteristics
and the flight speed. These are provided in a 64-element array for
historical convention; the first element should be ignored, and the
technique requires that the end elements be unshadowed and so precludes
any measurement with width of 63 bins, so the 64-element vector has
valid information only in bins 1\textendash 63. The cell boundaries
are specified in the netCDF header as an attribute of C1DC or C1DP,
and they specify the end points of the bin; e.g,, in the 64-element
array of provided cell boundaries, the first element is the lower
size limit of the first data cell which is the second element in C1DC.
For a typical 2DC with 25-$\mu$m size resolution, the cell sizes
increase by 25 $\mu$m per bin for each of the C1DC bins. Also included
as attributes with the netCDF variable C1DC or C1DP are the size-dependent
depth of field (mm) and effective sample area\footnote{commonly called ``EffectiveAreaWidth'' in the netCDF files}
(mm), the latter having values of zero for the first and last elements
in the 64-value vector. \\

\textbf{2D Concentration }(L$^{-1}$):\textbf{ }\textbf{\uline{CONC1DC}}\textbf{\index{CONC1DC}\sindex[var]{CONC1DC},
}\textbf{\uline{CONC1DC100}}\textbf{\sindex[var]{CONC1DC100},
}\textbf{\uline{CONC1DC150\sindex[var]{CONC1DC150},}}\textbf{
}\textbf{\uline{CONC1DP}}\textbf{\index{CONC1DP}\sindex[var]{CONC1DP}}\textbf{\uline{}}\\
\index{concentration!2D}\emph{The total concentration of all particles
detected by a 2D hydrometeor imager,} or in the case of CONC1DC100
or CONC1DC150, the concentration of all particles sized to be at least
xxx $\mu$m in the dimension perpendicular to the direction of flight,
where xxx may be 100 150. These concentrations are the sum of the
particle size distribution given below (C1DC or C1DP), with appropriate
channels excluded for CONC1DC100 and CONC1DC150.\\

\textbf{2D Dead Time }(ms):\textbf{ }\textbf{\uline{DT1DC}}\index{DT1DC}\sindex[var]{DT1DC}\\
\index{dead time}\emph{The time in the sample interval during which
the data rate exceeded the recording capability of a 2DC probe.} This
is used as a correction factor when concentrations like CONC1DC or
C1DC are calculated. The variable does not apply to measurements from
a 2DP probe.\\

\textbf{2D Mean Diameter }($\mu$m):\textbf{ }\textbf{\uline{DBAR1DC}}\index{DBAR1DC}\sindex[var]{DBAR1DC},
\textbf{\uline{DBAR1DP}}\index{DBAR1DP}\sindex[var]{DBAR1DP}\\
\index{diameter!mean, 2D probes}\emph{The mean diameter calculated
from the measured size distribution. }In this calculation, the bin
sizes are taken to be the averages of the lower and upper limits of
the size bins\emph{. }The calculation is as described by \eqref{eq:dbar1d}.\emph{
}\\

\textbf{2D Dispersion (dimensionless): }\textbf{\uline{DISP1DC}}\index{DISP1DC}\sindex[var]{DISP1DC},
\textbf{\uline{DISP1DP}}\index{DISP1DP}\sindex[var]{DISP1DP}\\
\index{dispersion}\emph{The standard deviation in particle diameter
divided by the mean diameter.} The formula used is given by \eqref{eq:disp1d}.\\

\textbf{2D Liquid Water Content} (g\,m$^{-3}$):\textbf{ }\textbf{\uline{PLWC1DC}}\index{PLWC1DC}\sindex[var]{PLWC1DC},
\textbf{\uline{PLWC1DP}}\index{PLWC1DP}\sindex[var]{PLWC1DP}\\
\emph{\index{liquid water content}The liquid water content (mass
per volume) calculated from C1DC or C1DP.} The calculation is as described
by \eqref{eq:lwc1d}, To conform to common usage, the liquid water
content is expressed in non-MKS units of g\,m$^{-3}$. \\

\textbf{2D Radar Reflectivity Factor (dBZ): }\textbf{\uline{DBZ1DC}}\index{DBZ1DC}\sindex[var]{DBZ1DC},
\textbf{\uline{DBZ1DP}}\index{DBZ1DP}\sindex[var]{DBZ1DP}\\
\index{radar reflectivity factor}\emph{The radar reflectivity factor
calculated from the measured size distribution under the assumption
that all particles are spherical water drops. }The calculation is
as described by\emph{ }\eqref{eq:dbz1d}.\\

\textbf{2D Effective Radius ($\mu$m): }\textbf{\uline{REFF2DC}}\index{REFF2DC}\sindex[var]{REFF2DC},
\textbf{\uline{REFF2DP}}\index{REFF2DP}\sindex[var]{REFF2DP}\\
\index{radius, effective}\emph{One-half the ratio of the third moment
of the particle diameter to the second moment. }The formula used is
given by \eqref{eq:reff1d}.\label{punch:5-6}\\
 
\end{hangparagraphs}


